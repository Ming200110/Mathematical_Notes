\usepackage{amsmath}                                % 数学公式
\usepackage{amsthm}                                 % 数学公式
\usepackage{amssymb}                                % 数学公式
\usepackage{mathrsfs}
\usepackage{appendix}                               % 添加附录
\usepackage{pdfpages}                               % 添加pdf
\usepackage{pgfplots}                               % 画图包
\usepackage{geometry}                               % 页边距包
\usepackage{fancyhdr}                               % 页眉包
\usepackage{float}                                  % 提供 float 浮动环境
\usepackage{booktabs}                               % 三线表
\usepackage{subfigure}                              % 子图
\usepackage{hyperref}                               % 引用跳转
\usepackage{extarrows}                              % 长等号
\usepackage{esint}                                  % 特殊的数学符号
\usepackage[inline]{enumitem}                       % 高级编号
\usepackage{multirow}                               % 表格单元占据多行/列
\usepackage{multicol}                               % 表格单元占据多行/列
\usepackage{arydshln}                               % 矩阵虚线
\usepackage{fancyvrb}
\usepackage{afterpage}
\usepackage{caption}
\usepackage{makecell}
\usepackage{physics}
\usepackage{rotating}
\usepackage{amscd}
\usepackage{tasks}
\usepackage{color}
\usepackage{framed}
\usepackage{cutwin}
\usepackage{tikz-3dplot}
\usepackage{nicematrix}
\usepackage{ocgx}
\usepackage{tablefootnote}
\usepackage{etoc}
\usepackage{titlesec}
\usepackage{makeidx}
\usepackage{bookmark}
\usepackage{epigraph}
% \usepackage[adobe-utopia]{mathdesign}
\makeindex

\titleformat
{\chapter} % command
[display] % shape
{\color{cyan}\bfseries\Large} % format
{第\ \thechapter\ 章} % label
{0.25ex} % sep
{
    \rule{\textwidth}{1pt}
    \vspace{1ex}
    \centering
} % before-code
[
\vspace{-2ex}%
\rule{\textwidth}{0.3pt}
] % after-code

\newcommand*\cdotfill{%
	\leavevmode
	\leaders \hbox to .5em {\hss.\hss}\hfill
}

\etocsetstyle{part}
{\parindent 0pt
	\nobreak
	\etocskipfirstprefix}
{\pagebreak[3]\bigskip}
{\bfseries\centering
	\etocifnumbered{\etocnumber{} – }{}\etocname\par}
{}

\titleclass{\part}{top}
\titleformat{\part}[display]
    {\normalfont\sffamily\Large\bfseries\color{cyan}}
    {\filleft\Huge\thepart\enspace }
    {1ex}{\titlerule[2pt]{\hfill\raisebox{1ex}[0pt]{\rule{60mm}{8pt}}}\vskip1ex\Huge}

\etocsetstyle{chapter}
{%
	\setlength\rightskip{4\ccwd}%
	\addtolength\parfillskip{-\rightskip}%
	\etocskipfirstprefix
}
{\medskip}
{%
	\bfseries
	\noindent
	\etocifnumbered
	{%
		\sbox0{\etocthenumber\unskip\quad}%
		\setlength\leftskip{\wd0}%
		\etoclink{\llap{\box0}\etocthename}%
	}
	{%
		\setlength\leftskip{\rightskip}%
		\hskip-\leftskip
		\etocname
	}%
	\nobreak\cdotfill\etocpage\par
}
{}

\etocsetstyle{section}
{\nopagebreak\normalfont}
{\smallskip}
{%
	\noindent
	\etocifnumbered
	{\etoclink{\llap{\etocthenumber\quad}\etocthename}}
	{\etocname}%
	\nobreak\cdotfill\etocpage\par
}
{}

\etocsetstyle{subsection}
{%
	\nopagebreak
	\begingroup
	\addtolength\parfillskip{\rightskip}%
	\addtolength\rightskip{\fill}%
	\etocskipfirstprefix
	\noindent
}
{\quad}
{%
	\hbox{%
		\etoclink{%
			\etocifnumbered{\etocthenumber\enskip}{}%
			\etocthename\ (\etocthepage)%
		}%
	}%
}
{\par\endgroup}

% \NiceMatrixOptions{rules/color=cyan}
\definecolor{shadecolor}{rgb}{0.92,0.92,0.92}
\numberwithin{figure}{section}
\usetikzlibrary{calc,decorations.markings}
\newcommand\myemptypage{
	\null
	\thispagestyle{empty}
	\addtocounter{page}{-1}
	\newpage
}
% \usepackage{pgffor}
% \makeatletter
% \newcommand{\twocolumntoc}{%
% 	\chapter*{\contentsname
% 	  \@mkboth{%
% 		  \MakeUppercase\contentsname}{\MakeUppercase\contentsname}}%
% 	\@starttoc{toc}%
% }
% \makeatother
% \setcounter{tocdepth}{3}

\newcommand{\e}{\mathrm{e}}
\newcommand{\vp}{\mathrm{v.p.}}
\newcommand{\B}{\mathrm{B}}
\newcommand{\C}{\mathrm{C}}
\newcommand{\D}{\mathrm{D}}
\newcommand{\sgn}{\mathrm{sgn}}
\newcommand{\cov}{\mathrm{cov}}
\newcommand{\adj}{\mathrm{adj}}
\newcommand{\diag}{\mathrm{diag}}
\renewcommand{\i}{\mathrm{i}}

\newcommand\sj[1]{\mathsf{#1}}
\settasks{
	label={\Alph*.},
	% label-align = left,
	% label-offset = {1em},
	% label-width = 0.5em,
	% item-indent = {1.5em}, % indent = label-width + label-offset
	% column-sep = {1em},
	% before-skip = {-1em},
	% after-skip = {-1em},
	% after-item-skip = 0em
}
% \renewcommand{\proofname}{\indent\bf 证明}
% \renewcommand{\qedsymbol}{$\blacksquare$}

\newcounter{magicrownumbers}
% stepcounter 对应计数器自增,arabic 表示输出阿拉伯数字
\newcommand\rownumber{\stepcounter{magicrownumbers}\arabic{magicrownumbers}}
\theoremstyle{definition}
\newtheorem*{solution}{{\songti \textbf{解}}}
\newtheorem{theorem}{\color{magenta}{{\songti 定理}}}[section]
\newtheorem{lemma}{\color{purple}{{\songti 引理}}}[section]
\newtheorem{inference}{\color{pink}{{\songti 推论}}}[section]
\newtheorem{definition}{\color{orange}{{\songti 定义}}}[section]
\newtheorem{example}{\color{cyan}{{\songti 例}}}[section]
\pgfplotsset{compat=1.18}
\usetikzlibrary{intersections}

\geometry{scale=0.85, inner=20mm}

\title{{\kaishu \Huge{\textbf{${\text{考}_\text{研}^{\text{数}_\text{学}}}_{\underline{\text{典型问题}}}^{_{\overset{\overline{\text{方}}\text{法}|}{|\text{技}\underline{\text{巧}}}}}$}}}}
\author{{\kaishu 黄国铭}}
\date{}

\linespread{1.5}

\renewenvironment{solution}[1][]
{\noindent\scriptsize\linespread{0.8}
	\itshape #1}
{}

\newenvironment{errorSolution}[1][]
{\noindent\scriptsize\linespread{0.8}\color{red}
	\itshape \textbf{错解.} #1}
{}

\renewenvironment{proof}[1][]
{\noindent\scriptsize\linespread{0.8}
	\itshape #1}
{\par}

% \renewenvironment{example}[2][]
% {#1\noindent\footnotesize\linespread{1}
% \itshape \color{cyan} #2}
% {}

\allowdisplaybreaks