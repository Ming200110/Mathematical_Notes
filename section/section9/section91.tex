\section{行列式的计算}

\subsection{行列式的定义及性质}

\begin{definition}
    $n $ 级排列由 $ 1,2, \cdots, n $ 组成的一个有序数组称为一个 $ n $ 级排列, 通常记为 $ i_{1} i_{2} \cdots i_{n} $.
\end{definition}
\begin{definition}
    在一个排列中, 如果一对数的前后位置与大小顺序相反, 即前面的数大于后面的数, 则它们构成一个逆序.
\end{definition}
\begin{definition}
    一个排列中的逆序总数, 称为这个排列的逆序数, 通常记为 $ \tau\left(i_{1} i_{2} \cdots i_{n}\right) $.
\end{definition}
\begin{definition}
    逆序数为奇数 (偶数) 的排列, 称为奇 (偶) 排列.
\end{definition}
\begin{definition}
    把一个排列中某两个数的位置互换, 而其余的数不动, 就得到另一个排列, 这样的一个变换称为一次对换.
\end{definition}
\begin{theorem}
    任意一个排列经过一次对换后, 奇偶性改变.
\end{theorem}
\begin{theorem}
    $ n $ 级排列共有 $ n! $ 种, 奇偶排列各占一半.
\end{theorem}

\begin{example}
    选择 $i$ 与 $k$, 使 $a_{1i}a_{32}a_{4k}a_{25}a_{53}$ 成为五阶行列式中一个带负号的项.
\end{example}
\begin{solution}
    将给定的项改写成行标自然顺序, 即 $$a_{1i}a_{25}a_{32}a_{4k}a_{53}$$
    列标构成的排列 $i52k4$ 中缺 $1$ 和 $4$, 令 $i=1,k=4,\tau(15243)=3+1=4$, 故该项带正号;
    令 $i=4,k=1,\tau(45213)=3+3+1=7$, 故该项带负号.
\end{solution}

\begin{example}
    求 $\displaystyle f(x)=\begin{vmatrix}
            2x & x & 1 & 2  \\
            1  & x & 1 & -1 \\
            3  & 2 & x & 1  \\
            1  & 1 & 1 & x
        \end{vmatrix}$ 中 $x^4$ 与 $x^3$ 的系数.
\end{example}
\begin{solution}
    只有行列式的对角线上的元素相乘才出现 $x^4$, 并且带正号, 于是 $x^4$ 的系数为 $2$;
    同理, 含 $x^3$ 的项只有一项, 即 $$x\cdot 1\cdot x\cdot x=x^3$$
    而且其列标所构成的排列为 $2134$, 于是 $\tau(2134)=1$, 即 $x^3$ 的系数为 $-1.$
\end{solution}

\begin{definition}
    $~ \left|\begin{array}{cccc}a_{11} & a_{12} & \cdots & a_{1 n} \\ a_{21} & a_{22} & \cdots & a_{2 n} \\ \vdots & \vdots & & \vdots \\ a_{n 1} & a_{n 2} & \cdots & a_{n n}\end{array}\right| $
    是所有取自不同行不同列的 $ n $ 个元素的乘积 $ a_{1_{1}} a_{2 j_{2}} \cdots a_{n j_{n}} $ 的代数和, 这里 $ j_{1} j_{2} \cdots j_{n} $ 是一个 $ n $ 级排列.
    当 $ j_{1} j_{2} \cdots j_{n} $ 是偶排列时, 该项前面带正号; 当 $ j_{1} j_{2} \cdots j_{n} $ 是奇排列时, 该项前面带负号, 即
    $$\left|\begin{array}{cccc}
            a_{11}  & a_{12}  & \cdots & a_{1 n} \\
            a_{21}  & a_{22}  & \cdots & a_{2 n} \\
            \vdots  & \vdots  &        & \vdots  \\
            a_{n 1} & a_{n 2} & \cdots & a_{n n}
        \end{array}\right|=\sum_{j_{1} j_{2} \cdots j_{n}}(-1)^{\tau\left(j_{j_{2}} \cdots j_{n}\right)} a_{1 j_{1}} a_{2 j_{2}} \cdots a_{n j_{n}}$$
    其中 $ \displaystyle\sum_{j_{1} j_{2} \cdots j_{n}} $ 表示对所有 $ n $ 级排列求和. $ n $ 阶行列式有时简记为 $ \left|a_{i j}\right|_{n} $, 而且有如下另外两种类似的定义:
    $$\left|a_{i j}\right|_{n}=\sum_{i_{1} i_{2} \cdots i_{n}}(-1)^{\tau\left(i_{1} i_{2} \cdots i_{n}\right)} a_{i_{1} 1} a_{i_{2} 2} \cdots a_{i_{n} n}$$
    和 $$\left|a_{i j}\right|_{n}=\sum_{\substack{i_{i} i_{2} \cdots i_{n} \\  j_{1} j_{2} \cdots j_{n}}}(-1)^{\tau\left(i_{1} i_{2} \cdots i_{n}\right)+\tau\left(j_{j_{2}} \cdots \cdot j_{n}\right)} a_{i_{1} j_{1}} a_{i_{2} j_{2}} \cdots a_{i_{n_{n}} j_{n}} .$$
\end{definition}

由 $ n $ 级排列的性质可知, $n $ 阶行列式共有 $ n! $ 项, 其中冠以正号的项和冠以负号的项 (不算元素本身所带的负号) 各占一半.

\begin{example}
    \label{n jie hang lie shi wei fen fa}
    证明 $n$ 阶行列式微分法:
    $$\frac{\dd  }{\dd  x} \begin{vmatrix}
            f_{11} & f_{12} & \cdots & f_{1n} \\
            \vdots & \vdots &        & \vdots \\
            f_{i1} & f_{i2} & \cdots & f_{in} \\
            \vdots & \vdots &        & \vdots \\
            f_{n1} & f_{n2} & \cdots & f_{nn}
        \end{vmatrix}=\sum_{i=1}^{n}
        \begin{vmatrix}
            f_{11}                       & f_{12}                      & \cdots & f_{1n}                       \\
            \vdots                       & \vdots                      &        & \vdots                       \\
            \dfrac{\dd  }{\dd x } f_{i1} & \dfrac{\dd  }{\dd x }f_{i2} & \cdots & \dfrac{\dd  }{\dd x } f_{in} \\
            \vdots                       & \vdots                      &        & \vdots                       \\
            f_{n1}                       & f_{n2}                      & \cdots & f_{nn}
        \end{vmatrix}.$$
\end{example}
\begin{proof}[{\songti \textbf{证}}]
    从行列式的定义出发予以证明, 
    \begin{flalign*}
         & \frac{\dd  }{\dd  x}
        \begin{vmatrix}
            f_{11} & f_{12} & \cdots & f_{1n} \\
            \vdots & \vdots &        & \vdots \\
            f_{i1} & f_{i2} & \cdots & f_{in} \\
            \vdots & \vdots &        & \vdots \\
            f_{n1} & f_{n2} & \cdots & f_{nn}
        \end{vmatrix}
        =\dfrac{\dd }{\dd x}\sum_{j_1~j_2~\cdots ~j_n}(-1)^{\tau (j_1~j_2~\cdots ~j_n)}\prod_{i=1}^{n}f_{i~j_i}                                                                                                                                                  \\
         & =\sum_{j_1~j_2~\cdots ~j_n}(-1)^{\tau (j_1~j_2~\cdots ~j_n)}\dfrac{\dd }{\dd x}\prod_{i=1}^{n}f_{i~j_i}=\sum_{j_1~j_2~\cdots ~j_n}(-1)^{\tau (j_1~j_2~\cdots ~j_n)}\sum_{i=1}^{n}f_{1~j_1}f_{2~j_2}\cdots\dfrac{\dd }{\dd x}f_{i~j_i}\cdots f_{n~j_n} \\
         & =\sum_{i=1}^{n}\sum_{j_1~j_2~\cdots ~j_n}(-1)^{\tau (j_1~j_2~\cdots ~j_n)}f_{1~j_1}f_{2~j_2}\cdots\dfrac{\dd }{\dd x}f_{i~j_i}\cdots f_{n~j_n}=
        \begin{vmatrix}
            f_{11}                       & f_{12}                      & \cdots & f_{1n}                       \\
            \vdots                       & \vdots                      &        & \vdots                       \\
            \dfrac{\dd  }{\dd x } f_{i1} & \dfrac{\dd  }{\dd x }f_{i2} & \cdots & \dfrac{\dd  }{\dd x } f_{in} \\
            \vdots                       & \vdots                      &        & \vdots                       \\
            f_{n1}                       & f_{n2}                      & \cdots & f_{nn}
        \end{vmatrix}.
    \end{flalign*}
\end{proof}

\begin{theorem}
    行列式 $ D $ 与它的转置行列式 $ D^{\top} $ (将 $ D $ 行的项转为列的项, 如第 1 行转为第 1 列, $ \cdots $, 第 $ n $ 行转为第 $ n $ 列) 相等.
\end{theorem}
\begin{theorem}
    互换行列式的两行 (列), 行列式变号 (例如, 交换第 1 行和第 2 行, 记为 $ r_{1} \leftrightarrow r_{2} $; 交换第 1 列和第 2 列, 
    记为 $ c_{1} \leftrightarrow c_{2} $).
\end{theorem}
\begin{theorem}
    行列式的某一行(列) 中所有的元素都乘以同一数 $ k $, 等于用数 $ k $ 乘此行列式.
\end{theorem}
\begin{inference}
    行列式中某一行 (列) 的所有元素的公因子可以提到整个行列式的外面.
\end{inference}
\begin{theorem}
    行列式中如果有两行 (列) 元素成比例, 则此行列式等于零.
\end{theorem}
\begin{theorem}
    若行列式的某一列 (行) 的所有元素都是两数之和, 例如第 $ i $ 列的元素都是两数之和, 即
    $$\begin{vmatrix}
            a_{11} & a_{12} & \cdots & a_{1i}+a_{1i}' & \cdots & a_{1n} \\
            a_{21} & a_{22} & \cdots & a_{2i}+a_{2i}' & \cdots & a_{2n} \\
            \vdots & \vdots &        & \vdots         &        & \vdots \\
            a_{n1} & a_{22} & \cdots & a_{ni}+a_{ni}' & \cdots & a_{2n} \\
        \end{vmatrix}$$
    则 $D$ 等于下列两个行列式之和
    $$D=\begin{vmatrix}
            a_{11} & a_{12} & \cdots & a_{1i} & \cdots & a_{1n} \\
            a_{21} & a_{22} & \cdots & a_{2i} & \cdots & a_{2n} \\
            \vdots & \vdots &        & \vdots &        & \vdots \\
            a_{n1} & a_{22} & \cdots & a_{2i} & \cdots & a_{nn} \\
        \end{vmatrix}
        =\begin{vmatrix}
            a_{11} & a_{12} & \cdots & a_{1i}' & \cdots & a_{1n} \\
            a_{21} & a_{22} & \cdots & a_{2i}' & \cdots & a_{2n} \\
            \vdots & \vdots &        & \vdots  &        & \vdots \\
            a_{n1} & a_{22} & \cdots & a_{ni}' & \cdots & a_{nn} \\
        \end{vmatrix}$$
\end{theorem}
\begin{theorem}
    把行列式的某一行 (列) 的各元素乘以同一数然后加到另一行 (列) 对应的元素上, 行列式不变.
\end{theorem}

常见的行列式展开.
\setcounter{magicrownumbers}{0}
\begin{table}[H]
    \centering
    \begin{tabular}{l}
        二阶行列式                                                                                                                                                 \\
        (\rownumber) $\begin{vmatrix}
                              a_{11} & a_{12} \\
                              a_{21} & a_{22}
                          \end{vmatrix}=a_{11}a_{22}-a_{12}a_{21}$                                                                                                     \\
        \midrule
        三阶行列式                                                                                                                                                 \\
        (\rownumber) $\begin{vmatrix}
                              a_{11} & a_{12} & a_{13} \\
                              a_{21} & a_{22} & a_{23} \\
                              a_{31} & a_{32} & a_{33}
                          \end{vmatrix}=a_{11} a_{22} a_{33}+a_{12} a_{23} a_{31}+a_{13} a_{21} a_{32}-a_{13} a_{22} a_{31}-a_{12} a_{21} a_{33}-a_{11} a_{23} a_{32}$ \\
        \midrule
        上三角形、下三角形、对角行列式                                                                                                                             \\
        (\rownumber) $\left|\begin{array}{cccc}
                                    a_{11}         &        &        & *       \\
                                                   & a_{22} &        &         \\
                                                   &        & \ddots &         \\
                                    \boldsymbol{O} &        &        & a_{n n}
                                \end{array}\right|
            =\left|\begin{array}{cccc}
                       a_{11} &        &        & \boldsymbol{O} \\
                              & a_{22} &        &                \\
                              &        & \ddots &                \\
                       *      &        &        & a_{n n}
                   \end{array}\right|
        =\left|\begin{array}{llll}
                       a_{11}         &        &        & \boldsymbol{O} \\
                                      & a_{22} &        &                \\
                                      &        & \ddots &                \\
                       \boldsymbol{O} &        &        & a_{n n}
                   \end{array}\right|=a_{11} a_{22} \cdots a_{n n}$                                                                                                   \\
        \midrule
        副对角线方向的行列式                                                                                                                                       \\
        (\rownumber) $\begin{vmatrix}
                              *      &                                        &           & a_{1n}         \\
                                     &                                        & a_{2,n-1} &                \\
                                     & \begin{sideways}$\ddots$\end{sideways} &           &                \\
                              a_{n1} &                                        &           & \boldsymbol{O}
                          \end{vmatrix}
            =\begin{vmatrix}
                 \boldsymbol{O} &                                        &           & a_{1n} \\
                                &                                        & a_{2,n-1} &        \\
                                & \begin{sideways}$\ddots$\end{sideways} &           &        \\
                 a_{n1}         &                                        &           & *
             \end{vmatrix}
        =\begin{vmatrix}
                 \boldsymbol{O} &                                        &           & a_{1n}         \\
                                &                                        & a_{2,n-1} &                \\
                                & \begin{sideways}$\ddots$\end{sideways} &           &                \\
                 a_{n1}         &                                        &           & \boldsymbol{O}
             \end{vmatrix}$                                                                      \\
        $=(-1)^{\frac{n(n-1)}{2}}\displaystyle\prod_{i=1}^{n}a_{i,n-i+1}$
    \end{tabular}
\end{table}

\subsection{行列式按行 (列) 展开定理}

\begin{theorem}[Laplace 展开定理]
    取定行指标 $i_1,i_2,\cdots,i_p,1\leqslant i_1<i_2<\cdots<i_p\leqslant n$, 遍取行列式 $\det\vb*{A}$ 中第 $i_1,i_2,\cdots,i_p$ 行上的 $p$ 阶子式, 
    并分别乘以相应的代数余子式, 其和即为 $\det\vb*{A}$, 具体地说, 有
    $$\det\vb*{A} =\sum_{1\leqslant j_1<j_2<\cdots<j_p\leqslant n}\vb*{A}
        \mqty(i_1i_2\cdots i_p\\j_1j_2\cdots j_p)
        \qty[(-1)^{\sum\limits_{k=1}^pi_k+\sum\limits_{k=1}^pj_k}\vb*{A}
            \mqty(i_{p+1}i_{p+2}\cdots i_n\\
            j_{p+1}j_{p+2}\cdots j_n)]$$
    其中 $i_1i_2\cdots i_pi_{p+1}\cdots i_n$ 和 $j_1j_2\cdots j_pj_{p+1}\cdots j_n$ 都是 $1,2,\cdots,n$ 的排列, 
    并且 $1\leqslant i_{p+1}<i_{p+2}<\cdots\leqslant i_n,1\leqslant j_{p+1}<j_{p+2}<\cdots\leqslant j_n$.
\end{theorem}

\begin{example}
    利用 Laplace 展开定理计算下列行列式:
    \setcounter{magicrownumbers}{0}
    \begin{table}[H]
        \centering
        \begin{tabular}{l || l}
            (\rownumber{}) $D=\begin{vmatrix}
                                      -4 & 1  & 2 & -2 & 1  \\
                                      0  & 3  & 0 & 1  & -5 \\
                                      2  & -3 & 1 & -3 & 1  \\
                                      -1 & -1 & 3 & -1 & 0  \\
                                      0  & 4  & 0 & 2  & 5
                                  \end{vmatrix}.$
             & (\rownumber{}) $\displaystyle
                D=\begin{vmatrix}
                      1     & 1     & 0     & 0     & 0     & 1     \\
                      x_1   & x_2   & 0     & 0     & 0     & x_3   \\
                      a_1   & b_1   & 1     & 1     & 1     & c_1   \\
                      a_2   & b_2   & x_1   & x_2   & x_3   & c_2   \\
                      a_3   & b_3   & x_1^2 & x_2^2 & x_3^2 & c_3   \\
                      x_1^2 & x_2^2 & 0     & 0     & 0     & x_3^2
                  \end{vmatrix}.$
        \end{tabular}
    \end{table}
\end{example}
\begin{solution}
    \begin{enumerate}[label=(\arabic{*})]
        \item 将行列式 $D$ 按第 1 列、第 3 列作 Laplace 展开, 得到
              \begin{flalign*}
                  D & =\sum_{1\leqslant i_1<i_2\leqslant 5}\vb*{A} \mqty(i_1 & i_2 \\1&3)\qty[(-1)^{i_1+i_2+1+3}\vb*{A}\mqty(i_3&i_4&i_5\\2&4&5)]\\
                    & =(-1)^{1+3+1+3}\mqty|-4                                & 2   \\2&1|\mqty|3&1&-5\\-1&-1&0\\4&2&5|+(-1)^{1+4+1+3}\mqty|-4&2\\-1&3|\mqty|3&1&-5\\-3&-3&1\\4&2&5|\\
                    & ~  +(-1)^{3+4+1+3}\mqty|2                              & 1   \\-1&3|\mqty|1&-2&1\\3&1&-5\\4&2&5|\\
                    & =-8\times(-20)-(-10)\times(-62)-7\times87=-1069.
              \end{flalign*}
        \item 将行列式 $D$ 按第 3 列、第 4 列和第 5 列作 Laplace 展开, 得到
              \begin{flalign*}
                  D & =\sum_{1\leqslant i_1<i_2<i_3\leqslant 6}\vb*{A}\mqty(i_1 & i_2 & i_3 \\3&4&5)\qty[(-1)^{i_1+i_2+i_3+3+4+5}\vb*{A}\mqty(i_4&i_5&i_6\\1&2&6)]\\
                    & =(-1)^{3+4+5+3+4+5}\mqty|1                                & 1   & 1   \\x_1&x_2&x_3\\x_1^2&x_2^2&x_3^2|\cdot\mqty|1&1&1\\x_1&x_2&x_3\\x_1^2&x_2^2&x_3^2|=\prod_{1\leqslant j<i\leqslant 3}(x_i-x_j)^2.
              \end{flalign*}
    \end{enumerate}
\end{solution}

\subsection{具象行列式的计算}

\subsubsection{化为三角形矩阵}

\begin{example}[2016 复旦大学]
    设 $\boldsymbol A=(a_{ij})_{n\times n}$, 其中 $a_{ij}=\max\{i,j\}$, 求行列式 $\det \boldsymbol{A}.$
\end{example}
\begin{solution}
    对于 $i=2,\cdots,n$, 依次将第 $i$ 行的 $-1$ 倍加到第 $i-1$ 行, 得
    \begin{flalign*}
        \det\boldsymbol{A} & =
        \begin{vmatrix}
            1      & 2      & 3      & \cdots & n      \\
            2      & 2      & 3      & \cdots & n      \\
            3      & 3      & 3      & \cdots & n      \\
            \vdots & \vdots & \vdots &        & \vdots \\
            n      & n      & n      & n      & n
        \end{vmatrix}
        \xlongequal[i=2,\cdots,n]{r_{i-1}-r_i}
        \begin{vmatrix}
            -1     & 0      & 0      & \cdots & 0      & 0      \\
            -1     & -1     & 0      & \cdots & 0      & 0      \\
            -1     & -1     & -1     & \cdots & 0      & 0      \\
            \vdots & \vdots & \vdots &        & \vdots & \vdots \\
            -1     & -1     & -1     & \cdots & -1     & 0      \\
            n      & n      & n      & \cdots & n      & n
        \end{vmatrix} =(-1)^{n-1}n.
    \end{flalign*}
\end{solution}

\begin{example}
    设 $n\geqslant 2$, 计算 $n$ 阶行列式 $D_n=\det (a_{ij})$, 其中 $a_{ij}=|i-j|$, $i,j=1,2,\cdots,n.$
\end{example}
\begin{solution}
    从最后一行起, 依次将后一行减去前一行, 再将所得行列式的最后一行加到其他行, 得
    \begin{flalign*}
        D_n & =\begin{vmatrix}
                   0      & 1      & 2      & \cdots & n-2    & n-1    \\
                   1      & 0      & 1      & \cdots & n-3    & n-2    \\
                   2      & 1      & 0      & \cdots & n-4    & n-3    \\
                   \vdots & \vdots & \vdots &        & \vdots & \vdots \\
                   n-2    & n-3    & n-4    & \cdots & 0      & 1      \\
                   n-1    & n-2    & n-3    & \cdots & 1      & 0
               \end{vmatrix}
        \xlongequal[i=2,\cdots,n]{r_i-r_{i-1}}
        \begin{vmatrix}
            0      & 1      & 2      & \cdots & n-2    & n-1    \\
            1      & -1     & -1     & \cdots & -1     & -1     \\
            1      & 1      & -1     & \cdots & -1     & -1     \\
            \vdots & \vdots & \vdots &        & \vdots & \vdots \\
            1      & 1      & 1      & \cdots & -1     & -1     \\
            1      & 1      & 1      & \cdots & 1      & -1
        \end{vmatrix}        \\
            & \xlongequal[j=1,\cdots,n-1]{c_j+c_n}
        \begin{vmatrix}
            n-1    & n      & n+1    & \cdots & 2n-3   & n-1    \\
            0      & -2     & -2     & \cdots & -2     & -1     \\
            0      & 0      & -2     & \cdots & -2     & -1     \\
            \vdots & \vdots & \vdots &        & \vdots & \vdots \\
            0      & 0      & 0      & \cdots & -2     & -1     \\
            0      & 0      & 0      & \cdots & 0      & -1
        \end{vmatrix}
        =(-1)^{n-1}2^{n-2}(n-1).
    \end{flalign*}
\end{solution}

\begin{example}
    计算行列式 $\displaystyle D_n=\begin{vmatrix}
            1      & 2      & 3      & \cdots & n      \\
            2      & 3      & 4      & \cdots & 1      \\
            \vdots & \vdots & \vdots &        & \vdots \\
            n-1    & n      & 1      & \cdots & n-2    \\
            n      & 1      & 2      & \cdots & n-1
        \end{vmatrix}.$
\end{example}
\begin{solution}
    依次将第 $n$ 行减去第 $n-1$ 行, 第 $n-1$ 行减去第 $n-2$ 行, $\cdots$, 第 $2$ 行减去第 $1$ 行, 再将各列加到第 $1$ 列, 得
    \begin{flalign*}
        D_n &
        \xlongequal[i=n,\cdots,1]{r_i-r_{i-1}}
        \begin{vmatrix}
            1      & 2      & 3      & \cdots & n      \\
            1      & 1      & 1      & \cdots & 1-n    \\
            \vdots & \vdots & \vdots &        & \vdots \\
            1      & 1      & 1-n    & \cdots & 1      \\
            1      & 1-n    & 1      & \cdots & 1
        \end{vmatrix}
        \xlongequal[j=2,\cdots,n]{c_1+c_j}
        \begin{vNiceArray}{c:cccc}
            \dfrac{n(n+1)}{2} & 2      & 3      & \cdots & n      \\ \hdottedline
            0                 & 1      & 1      & \cdots & 1-n    \\
            \vdots            & \vdots & \vdots &        & \vdots \\
            0                 & 1      & 1-n    & \cdots & 1      \\
            0                 & 1-n    & 1      & \cdots & 1
        \end{vNiceArray} \\
            & =\dfrac{n(n+1)}{2}\begin{vmatrix}
                                    1      & 1      & \cdots & 1-n    \\[6pt]
                                    \vdots & \vdots &        & \vdots \\
                                    1      & 1-n    & \cdots & 1      \\
                                    1-n    & 1      & \cdots & 1
                                \end{vmatrix}_{n-1}
        \xlongequal[j=2,\cdots,n-1]{c_1+c_j}
        \dfrac{n(n+1)}{2}\begin{vmatrix}
                             -1     & 1      & \cdots & 1-n    \\[6pt]
                             \vdots & \vdots &        & \vdots \\
                             -1     & 1-n    & \cdots & 1      \\
                             -1     & 1      & \cdots & 1
                         \end{vmatrix}_{n-1}            \\
            & \xlongequal[j=2,\cdots,n-1]{c_j+c_1}
        \dfrac{n(n+1)}{2}\begin{vmatrix}
                             -1     & 0      & \cdots & -n     \\[6pt]
                             \vdots & \vdots &        & \vdots \\
                             -1     & -n     & \cdots & 0      \\
                             -1     & 0      & \cdots & 0
                         \end{vmatrix}_{n-1}=(-1)^{\frac{n(n-1)}{2}}\dfrac{n^{n-1}(n+1)}{2}.
    \end{flalign*}
\end{solution}

\begin{example}[2008 山西师范大学]
    计算 $f(x+1)-f(x)$, 其中
    $$f(x)=\begin{vmatrix}
            1      & 0      & 0                  & 0                  & \cdots & 0                      & x       \\
            1      & 2      & 0                  & 0                  & \cdots & 0                      & x^2     \\
            1      & 3      & 3                  & 0                  & \cdots & 0                      & x^3     \\
            \vdots & \vdots & \vdots             & \vdots             &        & \vdots                 & \vdots  \\
            1      & n      & \mathrm{C}_n^2     & \mathrm{C}_n^3     & \cdots & \mathrm{C}_n^{n-1}     & x^n     \\
            1      & n+1    & \mathrm{C}_{n+1}^2 & \mathrm{C}_{n+1}^3 & \cdots & \mathrm{C}_{n+1}^{n-1} & x^{n+1}
        \end{vmatrix}.$$
\end{example}
\begin{solution}由二项式定理 $\displaystyle (a+b)^n=\sum_{k=0}^{n}\mathrm{C}_n^k a^{n-k}b^n$, 
    \begin{flalign*}
        f(x+1)-f(x)=
        \begin{vmatrix}
            1      & 0      & 0                  & 0                  & \cdots & 0                      & 1                                             \\
            1      & 2      & 0                  & 0                  & \cdots & 0                      & 2x+1                                          \\
            1      & 3      & 3                  & 0                  & \cdots & 0                      & 3x^2+3x+1                                     \\
            \vdots & \vdots & \vdots             & \vdots             &        & \vdots                 & \vdots                                        \\
            1      & n      & \mathrm{C}_n^2     & \mathrm{C}_n^3     & \cdots & \mathrm{C}_n^{n-1}     & nx^{n-1}+\mathrm{C}_n^2x^{n-2}+\cdots+1       \\
            1      & n+1    & \mathrm{C}_{n+1}^2 & \mathrm{C}_{n+1}^3 & \cdots & \mathrm{C}_{n+1}^{n-1} & (n+1)x^{n}+\mathrm{C}_{n+1}^2x^{n-1}+\cdots+1
        \end{vmatrix}
    \end{flalign*}
    将第一列乘 $-1$, 第二列乘 $-x$, 第二列乘 $-x^2$, $\cdots$, 第 $n$ 列乘 $-x^{n-1}$, 都加到最后一列, 得
    \begin{flalign*}
        f(x+1)-f(x)\xlongequal[j=1,\cdots,n]{c_{n+1}-x^{j-1}c_j}
        \left|\begin{array}{ccccccc}
                  1      & 0      & 0                    & 0                    & \cdots & 0                      & 0           \\
                  1      & 2      & 0                    & 0                    & \cdots & 0                      & 0           \\
                  1      & 3      & 3                    & 0                    & \cdots & 0                      & 0           \\
                  \vdots & \vdots & \vdots               & \vdots               &        & \vdots                 & \vdots      \\
                  1      & n      & \mathrm{C}_{n}^{2}   & \mathrm{C}_{n}^{3}   & \cdots & \mathrm{C}_{n}^{n-1}   & 0           \\
                  1      & n+1    & \mathrm{C}_{n+1}^{2} & \mathrm{C}_{n+1}^{3} & \cdots & \mathrm{C}_{n+1}^{n-1} & (n+1) x^{n}
              \end{array}\right| =(n+1) ! x^{n} .
    \end{flalign*}
\end{solution}

\subsubsection{化为对角型矩阵}

\begin{example}[2004 大连理工大学]
    计算 $n$ 阶行列式 $\displaystyle D_n=\begin{vmatrix}
            1      & 1      & \cdots & 1      & 2-n    \\
            1      & 1      & \cdots & 2-n    & 1      \\
            \vdots & \vdots &        & \vdots & \vdots \\
            1      & 2-n    & \cdots & 1      & 1      \\
            2-n    & 1      & \cdots & 1      & 1
        \end{vmatrix}.$
\end{example}
\begin{solution}
    第二列至第 $n$ 列都加到第一列, 再将第一列的 $-1$ 倍加到其他各列, 得
    \begin{flalign*}
        D_n & \xlongequal[j=2,\cdots,n]{c_1+c_j}
        \begin{vmatrix}
            1      & 1      & \cdots & 1      & 2-n    \\
            1      & 1      & \cdots & 2-n    & 1      \\
            \vdots & \vdots &        & \vdots & \vdots \\
            1      & 2-n    & \cdots & 1      & 1      \\
            1      & 1      & \cdots & 1      & 1
        \end{vmatrix}
        \xlongequal[j=2,\cdots,n]{c_j-c_1}
        \begin{vmatrix}
            1      & 0      & \cdots & 0      & 1-n    \\
            1      & 0      & \cdots & 1-n    & 0      \\
            \vdots & \vdots &        & \vdots & \vdots \\
            1      & 1-n    & \cdots & 0      & 0      \\
            1      & 0      & \cdots & 0      & 0
        \end{vmatrix}  \\
            & =(-1)^{\frac{n^2+n-2}{2}}(n-1)^{n-1}.
    \end{flalign*}
\end{solution}

\begin{example}[2008 北京科技大学]
    计算行列式 $|\boldsymbol{A}|$, 
    其中 $$\boldsymbol{A}=\begin{pmatrix}
            1      & 2      & \cdots & n-1     & n+x    \\
            1      & 2      & \cdots & (n-1)+x & n      \\
            \vdots & \vdots &        & \vdots  & \vdots \\
            1      & 2+x    & \cdots & n-1     & n      \\
            1+x    & 2      & \cdots & n-1     & n
        \end{pmatrix}.$$
\end{example}
\begin{solution}
    从第二行至第 $n$ 行都减去第一行, 再把所有列都加到最后一列, 得对角型行列式
    \begin{flalign*}
        |\boldsymbol{A}| & \xlongequal[i=2,\cdots,n]{r_i-r_1}
        \begin{vmatrix}
            1      & 2      & \cdots & n-1    & n+x    \\
            0      & 0      & \cdots & x      & -x     \\
            \vdots & \vdots &        & \vdots & \vdots \\
            0      & x      & \cdots & 0      & -x     \\
            x      & 0      & \cdots & 0      & -x
        \end{vmatrix}
        \xlongequal[j=1,\cdots,n-1]{c_n+c_j}
        \begin{vmatrix}
            1      & 2      & \cdots & n-1    & \displaystyle x+\frac{n(n+1)}{2} \\[6pt]
            0      & 0      & \cdots & x      & 0                                \\
            \vdots & \vdots &        & \vdots & \vdots                           \\
            0      & x      & \cdots & 0      & 0                                \\
            x      & 0      & \cdots & 0      & 0
        \end{vmatrix}       \\
                         & =(-1)^{\frac{n(n-1)}{2}}\left(x+\frac{n^2+n}{2}\right)x^{n-1}.
    \end{flalign*}
\end{solution}

\begin{example}[2002 华东师范大学]
    计算 $n$ 阶行列式
    $\displaystyle D_n=\begin{vmatrix}
            x      & 4      & 4      & \cdots & 4      \\
            1      & x      & 2      & \cdots & 2      \\
            1      & 2      & x      & \cdots & 2      \\
            \vdots & \vdots & \vdots &        & \vdots \\
            1      & 2      & 2      & \cdots & x
        \end{vmatrix}.$
\end{example}
\begin{solution}
    先把第一列乘二, 再把第一行除以二, 得
    \begin{flalign*}
        D_n & \xlongequal[r_1/2]{c_1\times2}
        \begin{vmatrix}
            x      & 2      & 2      & \cdots & 2      \\
            2      & x      & 2      & \cdots & 2      \\
            2      & 2      & x      & \cdots & 2      \\
            \vdots & \vdots & \vdots &        & \vdots \\
            2      & 2      & 2      & \cdots & x
        \end{vmatrix}
        \xlongequal[i=2,\cdots,n]{r_i-r_1}
        \begin{vmatrix}
            x      & 2      & 2      & \cdots & 2      \\
            2-x    & x-2    & 0      & \cdots & 0      \\
            2-x    & 0      & x-2    & \cdots & 0      \\
            \vdots & \vdots & \vdots &        & \vdots \\
            2-x    & 0      & 0      & \cdots & x-2
        \end{vmatrix} \\
            & \xlongequal[j=2,\cdots,n]{c_1+c_j}
        \begin{vmatrix}
            x+2(n-1) & 2      & 2      & \cdots & 2      \\
            0        & x-2    & 0      & \cdots & 0      \\
            0        & 0      & x-2    & \cdots & 0      \\
            \vdots   & \vdots & \vdots & \ddots & \vdots \\
            0        & 0      & 0      & \cdots & x-2
        \end{vmatrix}
        =(x+2n-2)(x-2)^{n-1}
    \end{flalign*}
\end{solution}

\begin{example}[2005 北京工业大学]
    计算 $n$ 阶行列式
    $$\begin{vmatrix}
            a+1    & a+2    & a+3    & \cdots & a+n    \\
            a+2    & a+3    & a+4    & \cdots & a+1    \\
            a+3    & a+4    & a+5    & \cdots & a+2    \\
            \vdots & \vdots & \vdots &        & \vdots \\
            a+n    & a+1    & a+2    & \cdots & a+n-1  \\
        \end{vmatrix}.$$
\end{example}
\begin{solution}
    从最后一行开始, 每行都减去其前一行, 然后各列都加到第一列, 并按第一列展开, 得
    \begin{flalign*}
        D & \xlongequal[i=n,\cdots,2]{r_i-r_{i-1}}
        \begin{vmatrix}
            a+1    & a+2    & a+3    & \cdots & a+n-1  & a+n    \\
            1      & 1      & 1      & \cdots & 1      & 1-n    \\
            1      & 1      & 1      & \cdots & 1-n    & 1      \\
            \vdots & \vdots & \vdots &        & \vdots & \vdots \\
            1      & 1-n    & 1      & \cdots & 1      & 1      \\
        \end{vmatrix}                       \\
          & \xlongequal[j=2,\cdots,n]{c_1+c_j}
        \begin{vmatrix}
            \dfrac{1}{2}n(2a+n+1) & a+2    & a+3    & \cdots & a+n-1  & a+n    \\[6pt]
            0                     & 1      & 1      & \cdots & 1      & 1-n    \\
            0                     & 1      & 1      & \cdots & 1-n    & 1      \\
            \vdots                & \vdots & \vdots &        & \vdots & \vdots \\
            0                     & 1-n    & 1      & \cdots & 1      & 1      \\
        \end{vmatrix} \\
          & =\dfrac{1}{2}n(2a+n+1)\begin{vmatrix}
                                      1      & 1      & \cdots & 1      & 1-n    \\
                                      1      & 1      & \cdots & 1-n    & 1      \\
                                      \vdots & \vdots &        & \vdots & \vdots \\
                                      1-n    & 1      & \cdots & 1      & 1      \\
                                  \end{vmatrix}_{n-1}
        =(-1)^{\frac{n(n-1)}{2}}\qty[a+\dfrac{n(n+1)}{2}]n^{n-1}.
    \end{flalign*}
\end{solution}

% \begin{example}
%     计算行列式 $\displaystyle D=\begin{vmatrix}
%             (a+b)^2 & c^2     & c^2     \\
%             a^2     & (b+c)^2 & a^2     \\
%             b^2     & b^2     & (c+a)^2
%         \end{vmatrix}$.
% \end{example}
% \begin{solution}
%     第一列与第二列都减去第三列, 并提取前两列的公因子 $(a+b+c)$, 再将第三列加上第一列的 $c$ 倍, 最后利用对角线法则展开, 得
%     \begin{flalign*}
%         D & \xlongequal[j=1,2]{c_j-c_3}
%         \begin{vmatrix}
%             (a+b)^2-c^2 & 0           & c^2     \\
%             0           & (b+c)^2-a^2 & a^2     \\
%             b^2-(c+a)^2 & b^2-(c+a)^2 & (c+a)^2
%         \end{vmatrix}
%         =(a+b+c)^2\begin{vmatrix}
%                       a+b-c & 0     & c^2     \\
%                       0     & b+c-a & a^2     \\
%                       b-a-c & b-a-c & (c+a)^2
%                   \end{vmatrix}            \\
%           & \xlongequal[c_3+c\cdot c_1]{r_3-r_1-r_2}
%         (a+b+c)^2\begin{vmatrix}
%                      a+b+c & 0     & ac+bc \\
%                      0     & b+c-a & a^2   \\
%                      -2a   & -2c   & 0
%                  \end{vmatrix}
%         =2abc(a+b+c)^3.
%     \end{flalign*}
% \end{solution}

\subsubsection{建立递推公式}

\begin{example}
    求下列 $n$ 阶行列式
    \setcounter{magicrownumbers}{0}
    \begin{table}[H]
        \centering
        \begin{tabular}{l || l}
            (\rownumber{}) $\displaystyle
                D_n=\begin{vmatrix}
                        2a & a^2    &        &        &     \\
                        1  & 2a     & a^2    &        &     \\
                           & \ddots & \ddots & \ddots &     \\
                           &        & 1      & 2a     & a^2 \\
                           &        &        & 1      & 2a
                    \end{vmatrix}.$
             & (\rownumber{}) $\displaystyle
                D_n=\begin{vmatrix}
                        a+b & ab     &        &        &     \\
                        1   & a+b    & ab     &        &     \\
                            & \ddots & \ddots & \ddots &     \\
                            &        & 1      & a+b    & ab  \\
                            &        &        & 1      & a+b
                    \end{vmatrix}.$
        \end{tabular}
    \end{table}
\end{example}
\begin{solution}
    \begin{enumerate}[label=(\arabic{*})]
        \item 把 $D_n$ 第 $1$ 列乘 $\displaystyle-\frac{1}{2}a$ 加到第 $2$ 列, 再把第 $2$ 列乘 $\displaystyle-\frac{2}{3}a$ 加到第 $3$ 列, 
              如此下去, 直至把第 $n-1$ 列乘 $\displaystyle-\frac{n-1}{n}a$ 加到第 $n$ 列, 得
              \begin{flalign*}
                  D_n\xlongequal[j=1,\cdots,n-1]{c_{j+1}-c_j\times\frac{j}{j+1}a}
                  \begin{vmatrix}
                      2a & 0                          &                            &        &                              \\[6pt]
                      1  & \displaystyle \frac{3}{2}a & 0                          &        &                              \\[6pt]
                         & 1                          & \displaystyle \frac{4}{3}a & \ddots &                              \\
                         &                            & \ddots                     & \ddots & 0                            \\[6pt]
                         &                            &                            & 1      & \displaystyle \frac{n+1}{n}a
                  \end{vmatrix}
                  =(n+1)a^n.
              \end{flalign*}
        \item 把 $D_n$ 按第一行展开, 得
              \begin{flalign*}
                  D_n=(a+b)D_{n-1}-ab\begin{vmatrix}
                                         1 & ab  &        &        &     \\
                                         0 & a+b & ab     &        &     \\
                                           & 1   & a+b    & \ddots &     \\
                                           &     & \ddots & \ddots & ab  \\
                                           &     &        & 1      & a+b
                                     \end{vmatrix}_{n-1}=(a+b)D_{n-1}-abD_{n-2}.
              \end{flalign*}
              所以有递推公式 $$D_n-aD_{n-1}=b(D_{n-1}-aD_{n-2})$$
              由此逐次递推, 得 $$D_n-aD_{n-1}=b^2(D_{n-2}-aD_{n-3})=\cdots=b^{n-2}(D_2-aD_1)$$
              注意到 $$D_1=a+b,~D_2=\begin{vmatrix}
                      a+b & ab  \\
                      1   & a+b
                  \end{vmatrix}=a^2+ab+b^2$$
              所以 $$D_n-aD_{n-1}=b^n$$
              根据对称性, 有 $$D_n-bD_{n-1}=a^n$$
              若 $a=b$ (可见上题), 则 $D_n=a^n+aD_{n-1}$, 由此递推, 得 $$D_n=(n+1)a^n$$
              若 $a\not=b$, 可得 $\displaystyle D_n=\frac{b^{n+1}-a^{n+1}}{b-a}.$
    \end{enumerate}
\end{solution}

\begin{example}[1994 华中师范大学]
    计算 $n+1$ 阶行列式
    $$D_{n+1}=\begin{vmatrix}
            a        & -1       &          &        &        &    \\
            ax       & a        & -1       &        &        &    \\
            ax^2     & ax       & a        & -1     &        &    \\
            \vdots   & \vdots   & \vdots   & \ddots & \ddots &    \\
            ax^{n-1} & ax^{n-2} & ax^{n-3} & \cdots & a      & -1 \\
            ax^n     & ax^{n-1} & ax^{n-2} & \cdots & ax     & a
        \end{vmatrix}.$$
\end{example}
\begin{solution}
    \textbf{法一: }将第二列的 $-x$ 倍加到第一列, 第三列的 $-x$ 倍加到第二列, 以此类推, 
    $$D_{n+2}\xlongequal[j=1,\cdots,n]{c_j-xc_{j+1}}\begin{vmatrix}
            a+x    & -1     &        &        &        &    \\
            0      & a+x    & -1     &        &        &    \\
            0      & 0      & a+x    & -1     &        &    \\
            \vdots & \vdots & \vdots & \ddots & \ddots &    \\
            0      & 0      & 0      & \cdots & a+x    & -1 \\
            0      & 0      & 0      & \cdots & 0      & a
        \end{vmatrix}=a(a+x)^n.$$
    \textbf{法二: }先把第二列乘 $a$ 加到第一列, 并提出第一列的公因子 $x+a$, 再按第一行展开, 得
    \begin{flalign*}
        D_{n+1}=(x+a)D_n=(x+a)^2D_{n-1}=\cdots=(x+a)^{n-1}D_2=(x+a)^{n-1}
        \begin{vmatrix}
            a  & -1 \\
            ax & a
        \end{vmatrix}=a(x+a)^n.
    \end{flalign*}
\end{solution}

\begin{example}[2006 中国科学院]
    已知 $\alpha,\beta,\gamma$ 为实数, 求
    $\det\boldsymbol{A}=\begin{pmatrix}
            \alpha & \beta  &        &        &        \\
            \gamma & \alpha & \beta  &        &        \\
                   & \ddots & \ddots & \ddots &        \\
                   &        & \gamma & \alpha & \beta  \\
                   &        &        & \gamma & \alpha
        \end{pmatrix}\in \mathbb{R}^{n\times n}$ 的值.
\end{example}
\begin{solution}
    显然, 若 $\beta=0$ 或 $\gamma=0$, 则 $\det\boldsymbol{A}=\alpha^n$. 因此, 下设 $\beta\gamma\not=0$
    把 $D_n=\det\boldsymbol{A}$ 按第一行展开, 得 $$D_a=\alpha D_{n-1}-\beta\gamma D_{n-2}$$
    令 $p+q=\alpha,pq=\beta\gamma$, 则 $p,q$ 是二次方程 $x^2-\alpha x+\beta\gamma=0$ 的根. 故有如下递推公式
    $$D_n-pD_{n-1}=q(D_{n-1}-pD_{n-2})$$
    由此逐次递推, 得 $$D_n-pD_{n-1}=q^2(D_{n-2}-pD_{n-3})=\cdots=q^{n-2}(D_2-pD_1)$$
    注意到 $$D_1=\alpha=p+q,~D_2=\begin{vmatrix}
            \alpha & \beta  \\
            \gamma & \alpha
        \end{vmatrix}=\alpha^2-\beta\gamma=p^2+pq+q^2$$
    所以 $$D_n-pD_{n-1}=q^{n-2}[p^2+pq+q^2-p(p+q)]=q^n$$
    根据对称性 $$D_n-qD_{n-1}=p^n$$
    若 $\alpha^2=4\beta\gamma$, 则 $p=q$, 所以 $D_n=p^p+pD_{n-1}$. 由此递推, 得
    $$D_n=(n+1)p^n=(n+1)\left(\frac{\alpha}{2}\right)^n$$
    若 $\alpha^2\not=4\beta\gamma$, 则 $p\not=q$, 则有
    $$D_n=\frac{p^{n+1}-q^{n+1}}{p-q}=\frac{\left(\alpha+\sqrt{\alpha^2-4\beta\gamma}\right)^{n+1}-\left(\alpha-\sqrt{\alpha^2-4\beta\gamma}\right)^{n+1}}{2^{n+1}\sqrt{\alpha^2-4\beta\gamma}}.$$
\end{solution}

\subsubsection{升阶法}

\begin{example}
    计算下列行列式 $|\boldsymbol{A}|$, $a_i\neq0~~(i=1,2,\cdots,n)$, 
    \label{ascendingMethod}
    \setcounter{magicrownumbers}{0}
    \begin{table}[H]
        \centering
        \begin{tabular}{l || l}
            (\rownumber) $\displaystyle
                \boldsymbol{A}=
            \begin{pmatrix}
                    1+a_1^2 & a_1a_2  & \cdots & a_1a_n  \\
                    a_2a_1  & 1+a_2^2 & \cdots & a_2a_n  \\
                    \vdots  & \vdots  &        & \vdots  \\
                    a_na_1  & a_na_2  & \cdots & 1+a_n^2
                \end{pmatrix}.$ & (\rownumber) $\displaystyle
                \boldsymbol{A}=
                \begin{pmatrix}
                    1+a_1  & 1      & \cdots & 1      \\
                    1      & 1+a_2  & \cdots & 1      \\
                    \vdots & \vdots &        & \vdots \\
                    1      & 1      & \cdots & 1+a_n  \\
                \end{pmatrix}.$
        \end{tabular}
    \end{table}
\end{example}
\begin{solution}
    \begin{enumerate}[label=(\arabic{*})]
        \item 把 $|\boldsymbol{A}|$ 添加一行、一列且保持值不变, 
              \begin{flalign*}
                  |\boldsymbol{A}| & =
                  \begin{vNiceArray}{c:cccc}
                      1      & a_1     & a_2     & \cdots & a_n     \\ \hdottedline
                      0      & 1+a_1^2 & a_1a_2  & \cdots & a_1a_n  \\
                      0      & a_2a_1  & 1+a_2^2 & \cdots & a_2a_n  \\
                      \vdots & \vdots  & \vdots  &        & \vdots  \\
                      0      & a_na_1  & a_na_2  & \cdots & 1+a_n^2
                  \end{vNiceArray}_{n+1}
                  \xlongequal[i=1,\cdots,n]{r_{i+1}-a_ir_1}
                  \begin{vmatrix}
                      1      & a_1    & a_2    & \cdots & a_n    \\
                      -a_1   & 1      & 0      & \cdots & 0      \\
                      -a_2   & 0      & 1      & \cdots & 0      \\
                      \vdots & \vdots & \vdots &        & \vdots \\
                      -a_n   & 0      & 0      & \cdots & 1
                  \end{vmatrix}                   \\
                                   & \xlongequal[j=1,\cdots,n]{c_1+a_jc_{j+1}}
                  \begin{vmatrix}
                      \displaystyle 1+\sum_{k=1}^{n}a_k^2 & a_1    & a_2    & \cdots & a_n    \\
                      0                                   & 1      & 0      & \cdots & 0      \\
                      0                                   & 0      & 1      & \cdots & 0      \\
                      \vdots                              & \vdots & \vdots &        & \vdots \\
                      0                                   & 0      & 0      & \cdots & 1
                  \end{vmatrix}
                  =1+\sum_{k=1}^{n}a_k^2.
              \end{flalign*}
        \item 同样地, 把 $|\boldsymbol{A}|$ 添加一行、一列且保持值不变, 
              \begin{flalign*}
                  |\boldsymbol{A}| & =
                  \begin{vNiceArray}{c:cccc}
                      1      & 1      & 1      & \cdots & 1      \\ \hdottedline
                      0      & 1+a_1  & 1      & \cdots & 1      \\
                      0      & 1      & 1+a_2  & \cdots & 1      \\
                      \vdots & \vdots & \vdots &        & \vdots \\
                      0      & 1      & 1      & \cdots & 1+a_n
                  \end{vNiceArray}_{n+1}
                  \xlongequal[i=1,\cdots,n]{r_{i+1}-r_1}
                  \begin{vmatrix}
                      1      & 1      & 1      & \cdots & 1      \\
                      -1     & a_1    & 0      & \cdots & 0      \\
                      -1     & 0      & a_2    & \cdots & 0      \\
                      \vdots & \vdots & \vdots &        & \vdots \\
                      -1     & 0      & 0      & \cdots & a_n    \\
                  \end{vmatrix}                             \\
                                   & \xlongequal[j=1,\cdots,n]{c_1+\frac{1}{a_j}c_{j+1}}
                  \begin{vmatrix}
                      \displaystyle 1+\sum_{k=1}^{n}\frac{1}{a_k} & 1      & 1      & \cdots & 1      \\
                      0                                           & a_1    & 0      & \cdots & 0      \\
                      0                                           & 0      & a_2    & \cdots & 0      \\
                      \vdots                                      & \vdots & \vdots &        & \vdots \\
                      0                                           & 0      & 0      & \cdots & a_n    \\
                  \end{vmatrix}
                  =\prod_{k=1}^{n}a_k\left(1+\sum_{k=1}^{n}\frac{1}{a_k}\right).
              \end{flalign*}
    \end{enumerate}
\end{solution}

\begin{example}[2003 南开大学]
    计算下列行列式的值:
    $$D=
        \begin{vmatrix}
            a_1+b_1c_1 & a_2+b_1c_2 & \cdots & a_n+b_1c_n \\
            a_1+b_2c_1 & a_2+b_2c_2 & \cdots & a_n+b_2c_n \\
            \vdots     & \vdots     &        & \vdots     \\
            a_1+b_nc_1 & a_2+b_nc_2 & \cdots & a_n+b_nc_n \\
        \end{vmatrix}$$
    其中 $n\geqslant3.$
\end{example}
\begin{solution}
    \textbf{法一: }利用升阶法, 
    \begin{flalign*}
        D & =
        \begin{vNiceArray}{c:cccc}
            1      & c_1        & c_2        & \cdots & c_n        \\ \hdottedline
            0      & a_1+b_1c_1 & a_2+b_1c_2 & \cdots & a_n+b_1c_n \\
            0      & a_1+b_2c_1 & a_2+b_2c_2 & \cdots & a_n+b_2c_n \\
            \vdots & \vdots     & \vdots     &        & \vdots     \\
            0      & a_1+b_nc_1 & a_2+b_nc_2 & \cdots & a_n+b_nc_n \\
        \end{vNiceArray}_{n+1}
        \xlongequal[i=1,...,n]{r_{i+1}-b_ir_1}
        \begin{vmatrix}
            1      & c_1    & c_2    & \cdots & c_n    \\
            -b_1   & a_1    & a_2    & \cdots & a_n    \\
            -b_2   & a_1    & a_2    & \cdots & a_n    \\
            \vdots & \vdots & \vdots &        & \vdots \\
            -b_n   & a_1    & a_2    & \cdots & a_n    \\
        \end{vmatrix} \\
          & =
        \begin{vNiceArray}{c:ccccc}
            1      & 0      & 0      & 0      & \cdots & 0      \\ \hdottedline
            0      & 1      & c_1    & c_2    & \cdots & c_n    \\
            -1     & -b_1   & a_1    & a_2    & \cdots & a_n    \\
            -1     & -b_2   & a_1    & a_2    & \cdots & a_n    \\
            \vdots & \vdots & \vdots & \vdots &        & \vdots \\
            -1     & -b_n   & a_1    & a_2    & \cdots & a_n
        \end{vNiceArray}_{n+2}
        \xlongequal[j=1,\cdots,n]{c_{j+2}+a_jc_1}
        \begin{vNiceArray}{cc:cccc}
            1      & 0      & a_1    & a_2    & \cdots & a_n    \\
            0      & 1      & c_1    & c_2    & \cdots & c_n    \\ \hdottedline
            -1     & -b_1   & 0      & 0      & \cdots & 0      \\
            -1     & -b_2   & 0      & 0      & \cdots & 0      \\
            \vdots & \vdots & \vdots & \vdots &        & \vdots \\
            -1     & -b_n   & 0      & 0      & \cdots & 0
        \end{vNiceArray}\xlongequal[n\geqslant3]{\text{Laplace}}0.
    \end{flalign*}
    \textbf{法二: }由行列式的乘法规则, 并注意到 $n\geqslant3$, 得
    $$D=
        \begin{vmatrix}
            1      & b_1    & 0      & \cdots & 0      \\
            1      & b_2    & 0      & \cdots & 0      \\
            \vdots & \vdots & \vdots &        & \vdots \\
            1      & b_n    & 0      & \cdots & 0      \\
        \end{vmatrix}\cdot
        \begin{vmatrix}
            a_1    & a_2    & \cdots & a_n    \\
            c_1    & c_2    & \cdots & c_n    \\
            0      & 0      & \cdots & 0      \\
            \vdots & \vdots &        & \vdots \\
            0      & 0      & \cdots & 0      \\
        \end{vmatrix}=0.$$
\end{solution}

\begin{example}[2017 中国科学院大学]
    计算行列式
    $$D=\begin{vmatrix}
            1-a_1 & a_2    &        &           &       \\
            -1    & 1-a_2  & a_3    &           &       \\
                  & \ddots & \ddots & \ddots    &       \\
                  &        & -1     & 1-a_{n-1} & a_n   \\
                  &        &        & -1        & 1-a_n
        \end{vmatrix}.$$
\end{example}
\begin{solution}
    利用升阶法, 考虑 $n+1$ 阶行列式
    $$\Delta=\begin{vNiceArray}{c:ccccc}
            1  & a_1   &        &        &           &       \\ \hdottedline
            -1 & 1-a_1 & a_2    &        &           &       \\
            & -1    & 1-a_2  & a_3    &           &       \\
            &       & \ddots & \ddots & \ddots    &       \\
            &       &        & -1     & 1-a_{n-1} & a_n   \\
            &       &        &        & -1        & 1-a_n
        \end{vNiceArray}_{n+1}\xlongequal[i=1,\cdots,n]{r_{i+1}+r_i}1$$
    另一方面, 将 $\Delta$ 按第一行展开, 得 $\Delta =D+a_1D'$, 即 $D=1-a_1D'$, 其中 $D'$ 是 $\Delta$ 右下角的 $n-1$ 阶子式, 由此递推, 可得
    $$D=1-a_1+a_1a_2-a_1a_2a_3+\cdots+(-1)^na_1a_2\cdots a_n.$$
\end{solution}

% \subsubsection{降阶法}

\subsection{抽象行列式的计算}

\begin{theorem}[行列式的乘法公式]
    $|k\boldsymbol{A}|=k^n|\boldsymbol{A}|.$
\end{theorem}

\begin{example}[2006 数一]
    设矩阵 $\boldsymbol{A}=\begin{pmatrix}
            2  & 1 \\
            -1 & 2
        \end{pmatrix}$, $\boldsymbol{E}$ 为 $2$ 阶单位矩阵, 矩阵 $\boldsymbol{B}$ 满足 $\boldsymbol{BA}=\boldsymbol{B}+2\boldsymbol{E}$, 求 $|\boldsymbol{B}|$.
\end{example}
\begin{solution}
    由 $\boldsymbol{BA}=\boldsymbol{B}+2\boldsymbol{E}$ 得 $\boldsymbol{B}(\boldsymbol{A}-\boldsymbol{E})=2\boldsymbol{E}$, 两边取行列式, 有
    $$|\boldsymbol{B}|\cdot|\boldsymbol{A}-\boldsymbol{E}|=|2\boldsymbol{E}|=4$$
    因为 $|\boldsymbol{A}-\boldsymbol{E}|=\begin{vmatrix}
            1  & 1 \\
            -1 & 1
        \end{vmatrix}=2$, 所以 $|\boldsymbol{B}|=2.$
\end{solution}

\begin{example}
    已知 $\vb*{\alpha}_1,\vb*{\alpha}_2,\vb*{\alpha}_3,\vb*{\beta},\vb*{\gamma}$ 均为 4 维列向量, 又 $\vb*{A}=(\vb*{\alpha}_1,\vb*{\alpha}_2,\vb*{\alpha}_3,\vb*{\beta}),~\vb*{B}=(\vb*{\alpha}_1,\vb*{\alpha}_2,\vb*{\alpha}_3,\vb*{\gamma})$, 
    若 $|\vb*{A}|=3,|\vb*{B}|=2$, 求 $|\vb*{A}+2\vb*{B}|.$
\end{example}
\begin{solution}
    $|\vb*{A}+2\vb*{B}|=|3\vb*{\alpha}_1,3\vb*{\alpha}_2,3\vb*{\alpha}_3,\vb*{\beta}+2\vb*{\gamma}|=3^3|\vb*{\alpha}_1,\vb*{\alpha}_2,\vb*{\alpha}_3,\vb*{\beta}+2\vb*{\gamma}|=3^3\qty(|\vb*{\alpha}_1,\vb*{\alpha}_2,\vb*{\alpha}_3,\vb*{\beta}|+2|\vb*{\alpha}_1,\vb*{\alpha}_2,\vb*{\alpha}_3,\vb*{\gamma}|)=27\times 7=187.$
\end{solution}
