\section{行列式的定义及其性质}

\subsection{逆序数}

\begin{definition}[排列]
    $n $ 级排列由 $ 1,2, \cdots, n $ 组成的一个有序数组称为一个 $ n $ 级\textit{排列}, 通常记为 $ i_{1} i_{2} \cdots i_{n} $.
\end{definition}
\begin{definition}[逆序数]
    在一个排列中, 如果一对数的前后位置与大小顺序相反, 即前面的数大于后面的数, 则它们构成一个\textit{逆序}.
    一个排列中的逆序总数, 称为这个排列的逆序数, 通常记为 $ \tau\left(i_{1} i_{2} \cdots i_{n}\right) $.
\end{definition}
\begin{definition}[奇 (偶) 排列]
    逆序数为奇数 (偶数) 的排列, 称为奇 (偶) 排列.
\end{definition}
\begin{definition}[对换]
    把一个排列中某两个数的位置互换, 而其余的数不动, 就得到另一个排列, 这样的一个变换称为一次对换.
\end{definition}
\begin{theorem}
    任意一个排列经过一次对换后, 奇偶性改变.
\end{theorem}
\begin{theorem}
    $ n ~(n>1)$ 级排列共有 $ n! $ 种, 奇偶排列各占一半.
\end{theorem}

\begin{example}
    选择 $i$ 与 $k$, 使 $a_{1i}a_{32}a_{4k}a_{25}a_{53}$ 成为五阶行列式中一个带负号的项.
\end{example}
\begin{solution}
    将给定的项改写成行标自然顺序, 即 $$a_{1i}a_{25}a_{32}a_{4k}a_{53}$$
    列标构成的排列 $i52k4$ 中缺 $1$ 和 $4$, 令 $i=1,k=4,\tau(15243)=3+1=4$, 故该项带正号;
    令 $i=4,k=1,\tau(45213)=3+3+1=7$, 故该项带负号.
\end{solution}

\subsection{行列式的概念}

\begin{definition}[$n$ 阶行列式]
    $\mqty|a_{11} & a_{12} & \cdots & a_{1 n} \\ a_{21} & a_{22} & \cdots & a_{2 n} \\ \vdots & \vdots & & \vdots \\ a_{n 1} & a_{n 2} & \cdots & a_{n n}|$
    是所有取自不同行不同列的 $ n $ 个元素的乘积 $$ a_{1_{1}} a_{2 j_{2}} \cdots a_{n j_{n}} $$ 的代数和, 这里 $ j_{1} j_{2} \cdots j_{n} $ 是一个 $ n $ 级排列.
    当 $ j_{1} j_{2} \cdots j_{n} $ 是偶排列时, 该项前面带正号; 当 $ j_{1} j_{2} \cdots j_{n} $ 是奇排列时, 该项前面带负号, 即
    $$\mqty|a_{11}  & a_{12}  & \cdots & a_{1 n} \\
            a_{21}  & a_{22}  & \cdots & a_{2 n} \\
            \vdots  & \vdots  &        & \vdots  \\
            a_{n 1} & a_{n 2} & \cdots & a_{n n}|=\sum_{j_{1} j_{2} \cdots j_{n}}(-1)^{\tau\left(j_{j_{2}} \cdots j_{n}\right)} a_{1 j_{1}} a_{2 j_{2}} \cdots a_{n j_{n}}$$
    其中 $ \displaystyle\sum_{j_{1} j_{2} \cdots j_{n}} $ 表示对所有 $ n $ 级排列求和. $ n $ 阶行列式有时简记为 $ \left|a_{i j}\right|_{n} $, 而且有如下另外两种类似的定义:
    $$\left|a_{i j}\right|_{n}=\sum_{i_{1} i_{2} \cdots i_{n}}(-1)^{\tau\left(i_{1} i_{2} \cdots i_{n}\right)} a_{i_{1} 1} a_{i_{2} 2} \cdots a_{i_{n} n}$$
    和 $$\left|a_{i j}\right|_{n}=\sum_{\substack{i_{i} i_{2} \cdots i_{n} \\  j_{1} j_{2} \cdots j_{n}}}(-1)^{\tau\left(i_{1} i_{2} \cdots i_{n}\right)+\tau\left(j_{j_{2}} \cdots \cdot j_{n}\right)} a_{i_{1} j_{1}} a_{i_{2} j_{2}} \cdots a_{i_{n_{n}} j_{n}} .$$
\end{definition}

由 $ n $ 级排列的性质可知, $n $ 阶行列式共有 $ n! $ 项, 其中冠以正号的项和冠以负号的项 (不算元素本身所带的负号) 各占一半.

\begin{example}
    求 $\displaystyle f(x)=\begin{vmatrix}
            2x & x & 1 & 2  \\
            1  & x & 1 & -1 \\
            3  & 2 & x & 1  \\
            1  & 1 & 1 & x
        \end{vmatrix}$ 中 $x^4$ 与 $x^3$ 的系数.
\end{example}
\begin{solution}
    只有行列式的对角线上的元素相乘才出现 $x^4$, 并且带正号, 于是 $x^4$ 的系数为 $2$;
    同理, 含 $x^3$ 的项只有一项, 即 $$x\cdot 1\cdot x\cdot x=x^3$$
    而且其列标所构成的排列为 $2134$, 于是 $\tau(2134)=1$, 即 $x^3$ 的系数为 $-1.$
\end{solution}

\begin{example}
    \label{n jie hang lie shi wei fen fa}
    证明 $n$ 阶行列式微分法:
    $$\frac{\dd  }{\dd  x} \begin{vmatrix}
            f_{11} & f_{12} & \cdots & f_{1n} \\
            \vdots & \vdots &        & \vdots \\
            f_{i1} & f_{i2} & \cdots & f_{in} \\
            \vdots & \vdots &        & \vdots \\
            f_{n1} & f_{n2} & \cdots & f_{nn}
        \end{vmatrix}=\sum_{i=1}^{n}
        \begin{vmatrix}
            f_{11}                       & f_{12}                      & \cdots & f_{1n}                       \\
            \vdots                       & \vdots                      &        & \vdots                       \\
            \dfrac{\dd  }{\dd x } f_{i1} & \dfrac{\dd  }{\dd x }f_{i2} & \cdots & \dfrac{\dd  }{\dd x } f_{in} \\
            \vdots                       & \vdots                      &        & \vdots                       \\
            f_{n1}                       & f_{n2}                      & \cdots & f_{nn}
        \end{vmatrix}.$$
\end{example}
\begin{proof}[{\songti \textbf{证}}]
    从行列式的定义出发予以证明, 
    \begin{flalign*}
         & \frac{\dd  }{\dd  x}
        \begin{vmatrix}
            f_{11} & f_{12} & \cdots & f_{1n} \\
            \vdots & \vdots &        & \vdots \\
            f_{i1} & f_{i2} & \cdots & f_{in} \\
            \vdots & \vdots &        & \vdots \\
            f_{n1} & f_{n2} & \cdots & f_{nn}
        \end{vmatrix}
        =\dfrac{\dd }{\dd x}\sum_{j_1~j_2~\cdots ~j_n}(-1)^{\tau (j_1~j_2~\cdots ~j_n)}\prod_{i=1}^{n}f_{i~j_i}                                                                                                                                                  \\
         & =\sum_{j_1~j_2~\cdots ~j_n}(-1)^{\tau (j_1~j_2~\cdots ~j_n)}\dfrac{\dd }{\dd x}\prod_{i=1}^{n}f_{i~j_i}=\sum_{j_1~j_2~\cdots ~j_n}(-1)^{\tau (j_1~j_2~\cdots ~j_n)}\sum_{i=1}^{n}f_{1~j_1}f_{2~j_2}\cdots\dfrac{\dd }{\dd x}f_{i~j_i}\cdots f_{n~j_n} \\
         & =\sum_{i=1}^{n}\sum_{j_1~j_2~\cdots ~j_n}(-1)^{\tau (j_1~j_2~\cdots ~j_n)}f_{1~j_1}f_{2~j_2}\cdots\dfrac{\dd }{\dd x}f_{i~j_i}\cdots f_{n~j_n}=
        \begin{vmatrix}
            f_{11}                       & f_{12}                      & \cdots & f_{1n}                       \\
            \vdots                       & \vdots                      &        & \vdots                       \\
            \dfrac{\dd  }{\dd x } f_{i1} & \dfrac{\dd  }{\dd x }f_{i2} & \cdots & \dfrac{\dd  }{\dd x } f_{in} \\
            \vdots                       & \vdots                      &        & \vdots                       \\
            f_{n1}                       & f_{n2}                      & \cdots & f_{nn}
        \end{vmatrix}.
    \end{flalign*}
\end{proof}

\subsection{行列式的基本性质}

\begin{theorem}[行列式的转置]
    行列式 $ D $ 与它的转置行列式 $ D^{\top} $ (将 $ D $ 行的项转为列的项, 如第 1 行转为第 1 列, $ \cdots $, 第 $ n $ 行转为第 $ n $ 列) 相等.
\end{theorem}
\begin{theorem}[变号定理]
    互换行列式的两行 (列), 行列式变号 (例如, 交换第 1 行和第 2 行, 记为 $ r_{1} \leftrightarrow r_{2} $; 交换第 1 列和第 2 列, 
    记为 $ c_{1} \leftrightarrow c_{2} $).
\end{theorem}
\begin{theorem}[行列式的数乘]
    行列式的某一行(列) 中所有的元素都乘以同一数 $ k $, 等于用数 $ k $ 乘此行列式.
\end{theorem}
\begin{inference}
    行列式中某一行 (列) 的所有元素的公因子可以提到整个行列式的外面.
\end{inference}
\begin{theorem}
    行列式中如果有两行 (列) 元素成比例, 则此行列式等于零.
\end{theorem}
\begin{theorem}[行列式拆分]
    若行列式的某一列 (行) 的所有元素都是两数之和, 例如第 $ i $ 列的元素都是两数之和, 即
    $$\begin{vmatrix}
            a_{11} & a_{12} & \cdots & a_{1i}+a_{1i}' & \cdots & a_{1n} \\
            a_{21} & a_{22} & \cdots & a_{2i}+a_{2i}' & \cdots & a_{2n} \\
            \vdots & \vdots &        & \vdots         &        & \vdots \\
            a_{n1} & a_{22} & \cdots & a_{ni}+a_{ni}' & \cdots & a_{2n} \\
        \end{vmatrix}$$
    则 $D$ 等于下列两个行列式之和
    $$D=\begin{vmatrix}
            a_{11} & a_{12} & \cdots & a_{1i} & \cdots & a_{1n} \\
            a_{21} & a_{22} & \cdots & a_{2i} & \cdots & a_{2n} \\
            \vdots & \vdots &        & \vdots &        & \vdots \\
            a_{n1} & a_{22} & \cdots & a_{2i} & \cdots & a_{nn} \\
        \end{vmatrix}
        =\begin{vmatrix}
            a_{11} & a_{12} & \cdots & a_{1i}' & \cdots & a_{1n} \\
            a_{21} & a_{22} & \cdots & a_{2i}' & \cdots & a_{2n} \\
            \vdots & \vdots &        & \vdots  &        & \vdots \\
            a_{n1} & a_{22} & \cdots & a_{ni}' & \cdots & a_{nn} \\
        \end{vmatrix}$$
\end{theorem}
\begin{theorem}[行列式变换]
    把行列式的某一行 (列) 的各元素乘以同一数然后加到另一行 (列) 对应的元素上, 行列式不变.
\end{theorem}

\subsection{几种特殊的行列式}

常见的行列式展开.
\setcounter{magicrownumbers}{0}
\begin{table}[H]
    \centering
    \begin{tabular}{l}
        二阶行列式                                                                                                                                                 \\
        (\rownumber) $\mqty|a_{11} & a_{12} \\
                              a_{21} & a_{22}|=a_{11}a_{22}-a_{12}a_{21}$                                                                                                     \\
        \midrule
        三阶行列式                                                                                                                                                 \\
        (\rownumber) $\begin{vmatrix}
                              a_{11} & a_{12} & a_{13} \\
                              a_{21} & a_{22} & a_{23} \\
                              a_{31} & a_{32} & a_{33}
                          \end{vmatrix}=a_{11} a_{22} a_{33}+a_{12} a_{23} a_{31}+a_{13} a_{21} a_{32}-a_{13} a_{22} a_{31}-a_{12} a_{21} a_{33}-a_{11} a_{23} a_{32}$ \\
        \midrule
        上三角形、下三角形、对角行列式                                                                                                                             \\
        (\rownumber) $\mqty|a_{11}         &        &        & *       \\
                                                   & a_{22} &        &         \\
                                                   &        & \ddots &         \\
                                    \boldsymbol{O} &        &        & a_{n n}|
            =\mqty|a_{11} &        &        & \boldsymbol{O} \\
                              & a_{22} &        &                \\
                              &        & \ddots &                \\
                       *      &        &        & a_{n n}|
        =\mqty|a_{11}         &        &        & \boldsymbol{O} \\
                                      & a_{22} &        &                \\
                                      &        & \ddots &                \\
                       \boldsymbol{O} &        &        & a_{n n}|=a_{11} a_{22} \cdots a_{n n}$                                                                                                   \\
        \midrule
        副对角线方向的行列式                                                                                                                                       \\
        (\rownumber) $\begin{vmatrix}
                              *      &                                        &           & a_{1n}         \\
                                     &                                        & a_{2,n-1} &                \\
                                     & \begin{sideways}$\ddots$\end{sideways} &           &                \\
                              a_{n1} &                                        &           & \boldsymbol{O}
                          \end{vmatrix}
            =\begin{vmatrix}
                 \boldsymbol{O} &                                        &           & a_{1n} \\
                                &                                        & a_{2,n-1} &        \\
                                & \begin{sideways}$\ddots$\end{sideways} &           &        \\
                 a_{n1}         &                                        &           & *
             \end{vmatrix}
        =\begin{vmatrix}
                 \boldsymbol{O} &                                        &           & a_{1n}         \\
                                &                                        & a_{2,n-1} &                \\
                                & \begin{sideways}$\ddots$\end{sideways} &           &                \\
                 a_{n1}         &                                        &           & \boldsymbol{O}
             \end{vmatrix}$                                                                      \\
        $=(-1)^{\frac{n(n-1)}{2}}\displaystyle\prod_{i=1}^{n}a_{i,n-i+1}$
    \end{tabular}
\end{table}