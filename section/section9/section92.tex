\section{Vandermonde 行列式与代数余子式}

\subsection{Vandermonde 行列式计算}

\begin{theorem}
    $n$ 阶 Vandermonde 行列式为
    $$D_n=
        \begin{vmatrix}
            1         & 1         & \cdots & 1         \\
            x_1       & x_2       & \cdots & x_n       \\
            x_1^2     & x_2^2     & \cdots & x_n^2     \\
            \vdots    & \vdots    &        & \vdots    \\
            x_1^{n-1} & x_2^{n-1} & \cdots & x_n^{n-1}
        \end{vmatrix}
        =\prod_{1\leqslant j<i\leqslant n}(x_i-x_j).$$
\end{theorem}
% \begin{proof}[{\songti \textbf{证法一}}]
%     将行列式第 $2$ 列至第 $n$ 列都减去第 $1$ 列,然后再将其按第 $1$ 行展开,
%     \begin{flalign*}
%         D_n & \xlongequal[j=2,\cdots,n]{c_j-c_1}
%         \begin{vmatrix}
%             1         & 0                   & \cdots & 0                       & 0                   \\
%             x_1       & x_2-x_1             & \cdots & x_{n-1}-x_1             & x_n-x_1             \\
%             x_1^2     & x_2^2-x_1^2         & \cdots & x_{n-1}^2-x_1^2         & x_n^2-x_1^2         \\
%             \vdots    & \vdots              &        & \vdots                  & \vdots              \\
%             x_1^{n-1} & x_2^{n-1}-x_1^{n-1} & \cdots & x_{n-1}^{n-1}-x_1^{n-1} & x_n^{n-1}-x_1^{n-1} \\
%         \end{vmatrix} \\
%             & =\begin{vmatrix}
%                    x_2-x_1             & \cdots & x_{n-1}-x_1             & x_n-x_1             \\
%                    x_2^2-x_1^2         & \cdots & x_{n-1}^2-x_1^2         & x_n^2-x_1^2         \\
%                    \vdots              &        & \vdots                  & \vdots              \\
%                    x_2^{n-1}-x_1^{n-1} & \cdots & x_{n-1}^{n-1}-x_1^{n-1} & x_n^{n-1}-x_1^{n-1} \\
%                \end{vmatrix}
%     \end{flalign*}
% \end{proof}

\begin{example}
    计算下列行列式:
    \setcounter{magicrownumbers}{0}
    \begin{table}[H]
        \centering
        \resizebox{.99\textwidth}{!}{
            \begin{tabular}{l || l}
                (\rownumber{}) $\displaystyle
                    D_{n+1}=
                    \begin{vmatrix}
                        a^n     & (a-1)^n     & \cdots & (a-n)^n     \\
                        a^{n-1} & (a-1)^{n-1} & \cdots & (a-n)^{n-1} \\
                        \vdots  & \vdots      &        & \vdots      \\
                        a       & a-1         & \cdots & a-n         \\
                        1       & 1           & \cdots & 1
                    \end{vmatrix}.$
                 & (\rownumber{}) $\displaystyle
                    D_{n-1}=
                    \begin{vmatrix}
                        2^n-2  & 2^{n-1}-2 & \cdots & 2^3-2  & 2      \\
                        3^n-3  & 3^{n-1}-3 & \cdots & 3^3-3  & 6      \\
                        \vdots & \vdots    &        & \vdots & \vdots \\
                        n^n-n  & n^{n-1}-n & \cdots & n^3-n  & n^2-n  \\
                    \end{vmatrix}.$
            \end{tabular}}
    \end{table}
\end{example}
\begin{solution}
    \begin{enumerate}[label=(\arabic{*})]
        \item 将 $D_{n+1}$ 的第 $n+1$ 行依次与第 $n,n-1,\cdots,1$ 行互换,再将新的行列式的第 $n+1$ 行依次与第 $n,n-1,\dots,2$ 行互换,
              如此下去,总共经过 $n+(n+1)+\cdots+2+1=\dfrac{n(n+1)}{2}$ 次行与行的互换,最后得 Vandermonde 行列式
              \begin{flalign*}
                  D_{n+1} & =(-1)^{\frac{n(n+1)}{2}}
                  \begin{vmatrix}
                      1       & 1           & \dots & 1           \\
                      a       & a-1         & ...   & a-n         \\
                      \vdots  & \vdots      &       & \vdots      \\
                      a^{n-1} & (a-1)^{n-1} & \dots & (a-n)^{n-1} \\
                      a^n     & (a-1)^n     & ...   & (a-n)^n
                  \end{vmatrix}
                  =(-1)^{\frac{n(n+1)}{2}}\prod_{0\leqslant j<i\leqslant n}[(a-i)-(a-j)] \\
                          & =\prod_{0\leqslant j<i\leqslant n}(i-j)=\prod_{k=1}^{n}k!.
              \end{flalign*}
        \item 利用升阶法,将 $D$ 添上一行一列,然后将第一行的 $i$ 倍加到后面各行,再将第 $j$ 列与后面的 $n-j$ 列逐次交换,$j=2,3,\cdots,n-1$,化为 Vandermonde 行列式,
              \begin{flalign*}
                  D & =
                  \begin{vNiceArray}{c:ccccc}
                    1      & 1      & 1         & \cdots & 1      & 1      \\ \hdottedline
                    0      & 2^n-2  & 2^{n-1}-2 & \cdots & 2^3-2  & 2      \\
                    0      & 3^n-3  & 3^{n-1}-3 & \cdots & 3^3-3  & 6      \\
                    \vdots & \vdots & \vdots    &        & \vdots & \vdots \\
                    0      & n^n-n  & n^{n-1}-n & \cdots & n^3    & n^2-n
                \end{vNiceArray}_{n}
                  \xlongequal[i=2,\cdots,n]{r_i+ir_1}
                  \begin{vmatrix}
                      1      & 1      & 1       & \cdots & 1      & 1      \\
                      2      & 2^n    & 2^{n-1} & \cdots & 2^3    & 2^2    \\
                      3      & 3^n    & 3^{n-1} & \cdots & 3^3    & 3^2    \\
                      \vdots & \vdots & \vdots  &        & \vdots & \vdots \\
                      n      & n^n    & n^{n-1} & \cdots & n^3    & n^2
                  \end{vmatrix}                            \\
                    & =n!\begin{vmatrix}
                             1 & 1       & 1       & \cdots & 1      & 1      \\
                             1 & 2^{n-1} & 2^{n-2} & \cdots & 2^2    & 2^1    \\
                             1 & 3^{n-1} & 3^{n-2} & \cdots & 3^2    & 3^1    \\
                             1 & \vdots  & \vdots  &        & \vdots & \vdots \\
                             1 & n^{n-1} & n^{n-2} & \cdots & n^2    & n^1    \\
                         \end{vmatrix}
                  =n! (-1)^{\frac{(n-2)(n-1)}{2}}\begin{vmatrix}
                                                     1 & 1      & 1      & \cdots & 1       & 1       \\
                                                     1 & 2      & 2^2    & \cdots & 2^{n-2} & 2^{n-1} \\
                                                     1 & 3      & 3^2    & \cdots & 3^{n-2} & 3^{n-1} \\
                                                     1 & \vdots & \vdots &        & \vdots  & \vdots  \\
                                                     1 & n      & n^2    & \cdots & n^{n-2} & n^{n-1} \\
                                                 \end{vmatrix} \\
                    & =(-1)^{\frac{n^2+n+2}{2}}\prod_{1\leqslant j<i\leqslant n+1}(i-j).
              \end{flalign*}
    \end{enumerate}
\end{solution}

\begin{example}
    计算下列行列式:
    \setcounter{magicrownumbers}{0}
    \begin{table}[H]
        \centering
        \begin{tabular}{l || l}
            (\rownumber{}) $\displaystyle
                D_n=\begin{vmatrix}
                        1      & a_1    & a_1^2  & \cdots & a_1^{n-2} & a_1^{n-1}+S/a_1 \\
                        1      & a_2    & a_2^2  & \cdots & a_2^{n-2} & a_2^{n-1}+S/a_2 \\
                        \vdots & \vdots & \vdots &        & \vdots    & \vdots          \\
                        1      & a_n    & a_n^2  & \cdots & a_n^{n-2} & a_n^{n-1}+S/a_n
                    \end{vmatrix}.$
             & (\rownumber{}) $\displaystyle
                D=\begin{vmatrix}
                      1      & x_1    & x_1^2  & \cdots & x_1^n  \\
                      1      & x_2    & x_2^2  & \cdots & x_2^n  \\
                      \vdots & \vdots & \vdots &        & \vdots \\
                      1      & x_n    & x_n^2  & \cdots & x_n^n  \\
                      0      & -2     & -2     & \cdots & -2
                  \end{vmatrix}.$
        \end{tabular}
    \end{table}
    其中 $\displaystyle S=\sum_{i=1}^{n}a_i$,且 $a_i\neq0.$
\end{example}
\begin{solution}
    \begin{enumerate}[label=(\arabic{*})]
        \item 先拆项,再利用 Vandermonde 行列式计算,
              \begin{flalign*}
                  D_n & =
                  \begin{vmatrix}
                      1      & a_1    & a_1^2  & \cdots & a_1^{n-2} & a_1^{n-1} \\
                      1      & a_2    & a_2^2  & \cdots & a_2^{n-2} & a_2^{n-1} \\
                      \vdots & \vdots & \vdots &        & \vdots    & \vdots    \\
                      1      & a_n    & a_n^2  & \cdots & a_n^{n-2} & a_n^{n-1}
                  \end{vmatrix}+
                  \begin{vmatrix}
                      1      & a_1    & a_1^2  & \cdots & a_1^{n-2} & S/a_1  \\
                      1      & a_2    & a_2^2  & \cdots & a_2^{n-2} & S/a_2  \\
                      \vdots & \vdots & \vdots &        & \vdots    & \vdots \\
                      1      & a_n    & a_n^2  & \cdots & a_n^{n-2} & S/a_n
                  \end{vmatrix}                                                               \\
                      & =\prod_{1\leqslant j<i\leqslant n}(a_i-a_j)
                  +\dfrac{S}{\displaystyle \prod_{i=1}^na_i}
                  \begin{vmatrix}
                      a_1    & a_1^2  & \cdots & a_1^{n-2} & a_1^{n-1} & 1      \\
                      a_2    & a_2^2  & \cdots & a_2^{n-2} & a_2^{n-1} & 1      \\
                      \vdots & \vdots &        & \vdots    & \vdots    & \vdots \\
                      a_n    & a_n^2  & \cdots & a_n^{n-2} & a_n^{n-1} & 1
                  \end{vmatrix}                                                            \\
                      & =\prod_{1\leqslant j<i\leqslant n}(a_i-a_j)
                  +(-1)^{n-1}\dfrac{S}{\displaystyle \prod_{i=1}^na_i}
                  \begin{vmatrix}
                      1      & a_1    & a_1^2  & \cdots & a_1^{n-2} & a_1^{n-1} \\
                      1      & a_2    & a_2^2  & \cdots & a_2^{n-2} & a_2^{n-1} \\
                      \vdots & \vdots & \vdots &        & \vdots    & \vdots    \\
                      1      & a_n    & a_n^2  & \cdots & a_n^{n-2} & a_n^{n-1}
                  \end{vmatrix}                                                            \\
                      & =\left[1+(-1)^{n-1}\dfrac{S}{\displaystyle \prod_{i=1}^na_i}\right]\prod_{1\leqslant j<i\leqslant n}(x_i-x_j).
              \end{flalign*}
        \item 先拆项,再利用 Vandermonde 行列式计算,
              \begin{flalign*}
                  D & =
                  \begin{vmatrix}
                      1      & x_1    & x_1^2  & \cdots & x_1^n  \\
                      1      & x_2    & x_2^2  & \cdots & x_2^n  \\
                      \vdots & \vdots & \vdots &        & \vdots \\
                      1      & x_n    & x_n^2  & \cdots & x_n^n  \\
                      -2     & -2     & -2     & \cdots & -2
                  \end{vmatrix}+
                  \begin{vmatrix}
                      0      & x_1    & x_1^2  & \cdots & x_1^n  \\
                      0      & x_2    & x_2^2  & \cdots & x_2^n  \\
                      \vdots & \vdots & \vdots &        & \vdots \\
                      0      & x_n    & x_n^2  & \cdots & x_n^n  \\
                      2      & -2     & -2     & \cdots & -2
                  \end{vmatrix}                                                                                                  \\
                    & =-2
                  \begin{vmatrix}
                      1      & x_1    & x_1^2  & \cdots & x_1^n  \\
                      1      & x_2    & x_2^2  & \cdots & x_2^n  \\
                      \vdots & \vdots & \vdots &        & \vdots \\
                      1      & x_n    & x_n^2  & \cdots & x_n^n  \\
                      1      & 1      & 1      & \cdots & 1
                  \end{vmatrix}
                  +(-1)^n2\prod_{i=1}^n
                  \begin{vmatrix}
                      1      & x_1     & x_1^2     & \cdots & x_1^{n-1}     \\
                      1      & x_2     & x_2^2     & \cdots & x_2^{n-1}     \\
                      \vdots & \vdots  & \vdots    &        & \vdots        \\
                      1      & x_{n-1} & x_{n-1}^2 & \cdots & x_{n-1}^{n-1} \\
                      1      & x_n     & x_n^2     & \cdots & x_n^{n-1}
                  \end{vmatrix}                                                                                       \\
                    & =-2\prod_{i=1}^{n}(1-x_i)\prod_{1\leqslant j<i\leqslant n}(x_i-x_j)+(-1)^n2\prod_{i=1}^{n}x_i\prod_{1\leqslant j<i\leqslant n}(x_i-x_j) \\
                    & =2\left[-\prod_{i=1}^{n}(1-x_i)+(-1)^n\prod_{i=1}^{n}x_i\right]\prod_{1\leqslant j<i\leqslant n}(x_i-x_j).
              \end{flalign*}
    \end{enumerate}
\end{solution}

\subsection{方幂指数跳跃}

\begin{example}[2006 山东大学]
    计算 $n$ 阶行列式\label{fmty}
    $$D=\begin{vmatrix}
            1         & 1         & \cdots & 1         \\
            x_1       & x_2       & \cdots & x_n       \\
            x_1^2     & x_2^2     & \cdots & x_n^2     \\
            \vdots    & \vdots    &        & \vdots    \\
            x_1^{n-2} & x_2^{n-2} & \cdots & x_n^{n-2} \\
            x_1^n     & x_2^n     & \cdots & x_n^n
        \end{vmatrix}.$$
\end{example}
\begin{solution}
    在 $D$ 中增加一行一列,凑成 $n+1$ 阶 Vandermonde 行列式:
    $$D_{n+1}=\begin{vmatrix}
            \begin{array}{cccc|c|}
                1         & 1         & \cdots & 1         & 1       \\
                x_1       & x_2       & \cdots & x_n       & y       \\
                x_1^2     & x_2^2     & \cdots & x_n^2     & y^2     \\
                \vdots    & \vdots    &        & \vdots    & \vdots  \\
                x_1^{n-2} & x_2^{n-2} & \cdots & x_n^{n-2} & y^{n-2} \\ \hline
                x_1^{n-1} & x_2^{n-1} & \cdots & x_n^{n-1} & y^{n-1} \\ \hline
                x_1^n     & x_2^n     & \cdots & x_n^n     & y^n
            \end{array}
        \end{vmatrix}=\prod_{k=1}^{n}(y-x_k)\prod_{1\leqslant j<i\leqslant n}(x_i-x_j)$$
        这是关于变量 $y$ 的恒等式,一方面,将上式左边 $D_{n+1}$ 按第 $D_{n+1}$ 列展开,得 $y$ 的 $n$ 次多项式:
        $$D_{n+1}=A_{1,n+1}+yA_{2,n+1}+\cdots+y^{n-1}A_{n,n+1}+y^nA_{n+1,n+1}$$
        其中 $A_{k,n+1}$ 是 $D_{n+1}$ 的 $(k,n+1)$ 元的代数余子式 $(1\leqslant k\leqslant n+1)$,且 $y^{n-1}$ 的系数为
        $$A_{n,n+1}=(-1)^{n+n+1}D=-D$$
        另一方面,对于 $D_{n+1}$ 的右边,因为 $$\prod_{k=1}^{n}(y-x_k)=y^n-\sum_{k=1}^{n}x_ky^{n-1}+\cdots+(-1)^n\prod_{k=1}^{n}x_k$$
        所以,$y^{n-1}$ 的系数为 $\displaystyle-\sum_{k=1}^{n}x_k\prod_{1\leqslant j<i\leqslant n}(x_i-x_j)$,因此,有
        $$D=\sum_{k=1}^{n}x_k\prod_{1\leqslant j<i\leqslant n}(x_i-x_j).$$
\end{solution}

\subsection{代数余子式问题}

\begin{definition}
    在 $ n $ 阶行列式 $ D=\left|a_{i j}\right| $ 中,去掉元素 $ a_{i j} $ 所在的第 $ i $ 行和第 $ j $ 列后,余下的 $ n-1 $ 阶行列式,称为 $ a_{i j} $ 的余子式,记为 $ M_{i j} .$
\end{definition}
\begin{definition}
    称 $ A_{i j}=(-1)^{i+j} M_{i j} $ 为 $ a_{i j} $ 的代数余子式.
\end{definition}
\begin{definition}
    在 $ n $ 阶行列式 $ D=\left|a_{i j}\right| $ 中,任意选定 $ k $ 行 $ k $ 列 $ (1 \leqslant k \leqslant n) $,
    位于这些行列交叉处的 $ k^{2} $ 个元素,按原来顺序构成一个 $ k $ 阶行列式,称为 $ D $ 的一个 $ k $ 阶子式.
\end{definition}
\begin{theorem}
    行列式等于它的任一行 (列) 的各元素与其对应的代数余子式乘积之和,即按第 $ i $ 行展开,$D=a_{i 1} A_{i 1}+a_{i 2} A_{i 2}+\cdots+a_{i n} A_{i n}(i=1,2, \cdots, n) $;
    按第 $ j $ 列展开,$D=a_{1 j} A_{1 j}+a_{2 j} A_{2 j}+\cdots+a_{n j} A_{n j}(j=1,2, \cdots, n) .$
\end{theorem}
\begin{theorem}
    行列式某一行 (列) 的元素与另一行 (列) 的对应元素的代数余子式乘积之和等于零,即
    $$a_{i 1} A_{j 1}+a_{i 2} A_{j 2}+\cdots+a_{i n} A_{j n}=0, i \neq j$$
    或
    $$a_{1 i} A_{1 j}+a_{2 i} A_{2 j}+\cdots+a_{n i} A_{n j}=0, i \neq j .$$
\end{theorem}

\begin{example}
    设四阶矩阵 $$\boldsymbol{A}=\begin{pmatrix}
            2  & 0 & 1 & 8  \\
            -2 & 1 & 4 & -7 \\
            3  & 0 & 5 & -9 \\
            a  & b & c & d
        \end{pmatrix}$$
    而 $A_{ij}$ 是 $\boldsymbol{A}$ 的 $(i,j)$ 元的代数余子式 $(i,j=1,2,3,4)$,试计算
    \begin{enumerate}[label=(\arabic{*})]
        \item $2A_{14}-2A_{24}+3A_{34}-3A_{44}$;
        \item $A_{41}+A_{42}+A_{43}+A_{44}.$
    \end{enumerate}
\end{example}
\begin{solution}
    \begin{enumerate}[label=(\arabic{*})]
        \item 由于 $a_{11}A_{14}+a_{21}A_{24}+a_{31}A_{34}+a_{41}A_{44}=0$,所以
              $$2A_{14}-2A_{24}+3A_{34}+a_{44}=0\Rightarrow 2A_{14}-2A_{24}+3A_{34}-3A_{44}=-(a+3)A_{44}$$
              且 $\displaystyle A_{44}=(-1)^{4+4}\begin{vmatrix}
                      2  & 0 & 1 \\
                      -2 & 1 & 4 \\
                      3  & 0 & 5
                  \end{vmatrix}=7$,所以 $2A_{14}-2A_{24}+3A_{34}-3A_{44}=-7(a+3)$.
        \item 构造一个新矩阵 $$\boldsymbol{B}=\begin{pmatrix}
                      2  & 0 & 1 & 8  \\
                      -2 & 1 & 4 & -7 \\
                      3  & 0 & 5 & -9 \\
                      1  & 1 & 1 & 1
                  \end{pmatrix}$$
              易知 $|\boldsymbol{B}|=-217$,所以 $A_{41}+A_{42}+A_{43}+A_{44}=-217.$
    \end{enumerate}
\end{solution}

\begin{example}[2010 云南大学]
    设四阶行列式
    $$D=\begin{vmatrix}
            3   & -5  & 2   & d   \\
            a   & b   & c   & d   \\
            a^2 & b^2 & c^2 & d^2 \\
            a^4 & b^4 & c^4 & d^4
        \end{vmatrix}$$
    计算 $A_{11}+A_{12}+A_{13}+A_{14}$,其中 $A_{ij}$ 是元素 $a_{ij}$ 的代数余子式.
\end{example}
\begin{solution}
    由于 $A_{ij}$ 与 $a_{ij}$ 的值无关,现构造一个新的行列式
    $$D_1=\begin{vmatrix}
            1   & 1   & 1   & 1   \\
            a   & b   & c   & d   \\
            a^2 & b^2 & c^2 & d^2 \\
            a^4 & b^4 & c^4 & d^4
        \end{vmatrix}$$
        由例 \ref{fmty} 知 $D_1=(a+b+c+d)(a-b)(a-c)(a-d)(b-c)(b-d)(c-d)$,
        将 $D_1$ 按第一行展开即得 
        $$A_{11}+A_{12}+A_{13}+A_{14}=(a+b+c+d)(a-b)(a-c)(a-d)(b-c)(b-d)(c-d).$$
\end{solution}

\begin{example}
    设 3 阶矩阵 $\vb*{A}=(a_{ij})_{3\times 3}$ 满足 $\vb*{A}^\top=k\vb*{A}^*~~(k>0)$,若 $a_{11}=a_{12}=a_{13}=c>0$,求 $c.$
\end{example}
\begin{solution}
    对 $\vb*{A}^\top=k\vb*{A}^*$ 两边取行列式,则 $$\det\qty(\vb*{A}^\top)=\det\qty(k\vb*{A}^*)\Rightarrow \det\vb*{A}=k^n(\det\vb*{A})^{n-1}=k^3(\det\vb*{A})^2\Rightarrow \det\vb*{A}=\dfrac{1}{k^3}$$
    又因为 $\vb*{A}^\top=k\vb*{A}^*$,所以 $a_{ij}=kA_{ij}$,其中 $A_{ij}$ 是 $a_{ij}$ 对应的代数余子式,那么 $$\det\vb*{A}=a_{11}A_{11}+a_{12}A_{12}+a_{13}A_{13}=\dfrac{1}{k}\sum_{i=1}^{3}a_{1i}^2=\dfrac{3c^2}{k}$$
    于是 $\dfrac{1}{k^3}=\dfrac{3c^2}{k}\Rightarrow c=\dfrac{\sqrt{3}}{3k}.$
\end{solution}
