\section{行列式按行 (列) 展开定理}

\subsection{行列式展开定理}

\begin{definition}[余子式]
    在 $ n $ 阶行列式 $ D=\left|a_{i j}\right| $ 中, 去掉元素 $ a_{i j} $ 所在的第 $ i $ 行和第 $ j $ 列后, 余下的 $ n-1 $ 阶行列式, 称为 $ a_{i j} $ 的\textit{余子式}, 记为 $ M_{i j} .$
\end{definition}
\begin{definition}[代数余子式]
    称 $ A_{i j}=(-1)^{i+j} M_{i j} $ 为 $ a_{i j} $ 的\textit{代数余子式}.
\end{definition}
\begin{definition}[$ k $ 阶子式]
    在 $ n $ 阶行列式 $ D=\left|a_{i j}\right| $ 中, 任意选定 $ k $ 行 $ k $ 列 $ (1 \leqslant k \leqslant n) $, 
    位于这些行列交叉处的 $ k^{2} $ 个元素, 按原来顺序构成一个 $ k $ 阶行列式, 称为 $ D $ 的一个 $ k $ \textit{阶子式}.
\end{definition}
\begin{theorem}[按行 (列) 展开]
    行列式等于它的任一行 (列) 的各元素与其对应的代数余子式乘积之和, 即按第 $ i $ 行展开, $D=a_{i 1} A_{i 1}+a_{i 2} A_{i 2}+\cdots+a_{i n} A_{i n}(i=1,2, \cdots, n) $;
    按第 $ j $ 列展开, $D=a_{1 j} A_{1 j}+a_{2 j} A_{2 j}+\cdots+a_{n j} A_{n j}(j=1,2, \cdots, n) .$

    行列式某一行 (列) 的元素与另一行 (列) 的对应元素的代数余子式乘积之和等于零, 即
    $$a_{i 1} A_{j 1}+a_{i 2} A_{j 2}+\cdots+a_{i n} A_{j n}=0, i \neq j$$
    或
    $$a_{1 i} A_{1 j}+a_{2 i} A_{2 j}+\cdots+a_{n i} A_{n j}=0, i \neq j .$$
\end{theorem}

\begin{example}
    设四阶矩阵 $$\boldsymbol{A}=\begin{pmatrix}
            2  & 0 & 1 & 8  \\
            -2 & 1 & 4 & -7 \\
            3  & 0 & 5 & -9 \\
            a  & b & c & d
        \end{pmatrix}$$
    而 $A_{ij}$ 是 $\boldsymbol{A}$ 的 $(i,j)$ 元的代数余子式 $(i,j=1,2,3,4)$, 试计算
    \begin{enumerate}[label=(\arabic{*})]
        \item $2A_{14}-2A_{24}+3A_{34}-3A_{44}$;
        \item $A_{41}+A_{42}+A_{43}+A_{44}.$
    \end{enumerate}
\end{example}
\begin{solution}
    \begin{enumerate}[label=(\arabic{*})]
        \item 由于 $a_{11}A_{14}+a_{21}A_{24}+a_{31}A_{34}+a_{41}A_{44}=0$, 所以
              $$2A_{14}-2A_{24}+3A_{34}+a_{44}=0\Rightarrow 2A_{14}-2A_{24}+3A_{34}-3A_{44}=-(a+3)A_{44}$$
              且 $\displaystyle A_{44}=(-1)^{4+4}\begin{vmatrix}
                      2  & 0 & 1 \\
                      -2 & 1 & 4 \\
                      3  & 0 & 5
                  \end{vmatrix}=7$, 所以 $2A_{14}-2A_{24}+3A_{34}-3A_{44}=-7(a+3)$.
        \item 构造一个新矩阵 $$\boldsymbol{B}=\begin{pmatrix}
                      2  & 0 & 1 & 8  \\
                      -2 & 1 & 4 & -7 \\
                      3  & 0 & 5 & -9 \\
                      1  & 1 & 1 & 1
                  \end{pmatrix}$$
              易知 $|\boldsymbol{B}|=-217$, 所以 $A_{41}+A_{42}+A_{43}+A_{44}=-217.$
    \end{enumerate}
\end{solution}

\begin{example}[2010 云南大学]
    设四阶行列式
    $$D=\begin{vmatrix}
            3   & -5  & 2   & d   \\
            a   & b   & c   & d   \\
            a^2 & b^2 & c^2 & d^2 \\
            a^4 & b^4 & c^4 & d^4
        \end{vmatrix}$$
    计算 $A_{11}+A_{12}+A_{13}+A_{14}$, 其中 $A_{ij}$ 是元素 $a_{ij}$ 的代数余子式.
\end{example}
\begin{solution}
    由于 $A_{ij}$ 与 $a_{ij}$ 的值无关, 现构造一个新的行列式
    $$D_1=\begin{vmatrix}
            1   & 1   & 1   & 1   \\
            a   & b   & c   & d   \\
            a^2 & b^2 & c^2 & d^2 \\
            a^4 & b^4 & c^4 & d^4
        \end{vmatrix}$$
        由例 \ref{fmty} 知 $D_1=(a+b+c+d)(a-b)(a-c)(a-d)(b-c)(b-d)(c-d)$, 
        将 $D_1$ 按第一行展开即得 
        $$A_{11}+A_{12}+A_{13}+A_{14}=(a+b+c+d)(a-b)(a-c)(a-d)(b-c)(b-d)(c-d).$$
\end{solution}

\begin{example}
    设 3 阶矩阵 $\vb*{A}=(a_{ij})_{3\times 3}$ 满足 $\vb*{A}^\top=k\vb*{A}^*~~(k>0)$, 若 $a_{11}=a_{12}=a_{13}=c>0$, 求 $c.$
\end{example}
\begin{solution}
    对 $\vb*{A}^\top=k\vb*{A}^*$ 两边取行列式, 则 $$\det\qty(\vb*{A}^\top)=\det\qty(k\vb*{A}^*)\Rightarrow \det\vb*{A}=k^n(\det\vb*{A})^{n-1}=k^3(\det\vb*{A})^2\Rightarrow \det\vb*{A}=\dfrac{1}{k^3}$$
    又因为 $\vb*{A}^\top=k\vb*{A}^*$, 所以 $a_{ij}=kA_{ij}$, 其中 $A_{ij}$ 是 $a_{ij}$ 对应的代数余子式, 那么 $$\det\vb*{A}=a_{11}A_{11}+a_{12}A_{12}+a_{13}A_{13}=\dfrac{1}{k}\sum_{i=1}^{3}a_{1i}^2=\dfrac{3c^2}{k}$$
    于是 $\dfrac{1}{k^3}=\dfrac{3c^2}{k}\Rightarrow c=\dfrac{\sqrt{3}}{3k}.$
\end{solution}

\subsection{Laplace 展开定理}

\begin{theorem}[Laplace 展开定理]
    取定行指标 $i_1,i_2,\cdots,i_p,1\leqslant i_1<i_2<\cdots<i_p\leqslant n$, 遍取行列式 $\det\vb*{A}$ 中第 $i_1,i_2,\cdots,i_p$ 行上的 $p$ 阶子式, 
    并分别乘以相应的代数余子式, 其和即为 $\det\vb*{A}$, 具体地说, 有
    $$\det\vb*{A} =\sum_{1\leqslant j_1<j_2<\cdots<j_p\leqslant n}\vb*{A}
        \mqty(i_1i_2\cdots i_p\\j_1j_2\cdots j_p)
        \qty[(-1)^{\sum\limits_{k=1}^pi_k+\sum\limits_{k=1}^pj_k}\vb*{A}
            \mqty(i_{p+1}i_{p+2}\cdots i_n\\
            j_{p+1}j_{p+2}\cdots j_n)]$$
    其中 $i_1i_2\cdots i_pi_{p+1}\cdots i_n$ 和 $j_1j_2\cdots j_pj_{p+1}\cdots j_n$ 都是 $1,2,\cdots,n$ 的排列, 
    并且 $1\leqslant i_{p+1}<i_{p+2}<\cdots\leqslant i_n,1\leqslant j_{p+1}<j_{p+2}<\cdots\leqslant j_n$.
\end{theorem}

\begin{example}
    利用 Laplace 展开定理计算下列行列式:
    \setcounter{magicrownumbers}{0}
    \begin{table}[H]
        \centering
        \begin{tabular}{l || l}
            (\rownumber{}) $D=\begin{vmatrix}
                                      -4 & 1  & 2 & -2 & 1  \\
                                      0  & 3  & 0 & 1  & -5 \\
                                      2  & -3 & 1 & -3 & 1  \\
                                      -1 & -1 & 3 & -1 & 0  \\
                                      0  & 4  & 0 & 2  & 5
                                  \end{vmatrix}.$
             & (\rownumber{}) $\displaystyle
                D=\begin{vmatrix}
                      1     & 1     & 0     & 0     & 0     & 1     \\
                      x_1   & x_2   & 0     & 0     & 0     & x_3   \\
                      a_1   & b_1   & 1     & 1     & 1     & c_1   \\
                      a_2   & b_2   & x_1   & x_2   & x_3   & c_2   \\
                      a_3   & b_3   & x_1^2 & x_2^2 & x_3^2 & c_3   \\
                      x_1^2 & x_2^2 & 0     & 0     & 0     & x_3^2
                  \end{vmatrix}.$
        \end{tabular}
    \end{table}
\end{example}
\begin{solution}
    \begin{enumerate}[label=(\arabic{*})]
        \item 将行列式 $D$ 按第 1 列、第 3 列作 Laplace 展开, 得到
              \begin{flalign*}
                  D & =\sum_{1\leqslant i_1<i_2\leqslant 5}\vb*{A} \mqty(i_1 & i_2 \\1&3)\qty[(-1)^{i_1+i_2+1+3}\vb*{A}\mqty(i_3&i_4&i_5\\2&4&5)]\\
                    & =(-1)^{1+3+1+3}\mqty|-4                                & 2   \\2&1|\mqty|3&1&-5\\-1&-1&0\\4&2&5|+(-1)^{1+4+1+3}\mqty|-4&2\\-1&3|\mqty|3&1&-5\\-3&-3&1\\4&2&5|\\
                    & ~  +(-1)^{3+4+1+3}\mqty|2                              & 1   \\-1&3|\mqty|1&-2&1\\3&1&-5\\4&2&5|\\
                    & =-8\times(-20)-(-10)\times(-62)-7\times87=-1069.
              \end{flalign*}
        \item 将行列式 $D$ 按第 3 列、第 4 列和第 5 列作 Laplace 展开, 得到
              \begin{flalign*}
                  D & =\sum_{1\leqslant i_1<i_2<i_3\leqslant 6}\vb*{A}\mqty(i_1 & i_2 & i_3 \\3&4&5)\qty[(-1)^{i_1+i_2+i_3+3+4+5}\vb*{A}\mqty(i_4&i_5&i_6\\1&2&6)]\\
                    & =(-1)^{3+4+5+3+4+5}\mqty|1                                & 1   & 1   \\x_1&x_2&x_3\\x_1^2&x_2^2&x_3^2|\cdot\mqty|1&1&1\\x_1&x_2&x_3\\x_1^2&x_2^2&x_3^2|=\prod_{1\leqslant j<i\leqslant 3}(x_i-x_j)^2.
              \end{flalign*}
    \end{enumerate}
\end{solution}