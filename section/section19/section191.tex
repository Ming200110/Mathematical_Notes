\section{Chebyshev 不等式}

方差可以描述随机变量的离散程度, 方差越大, 说明随机变量取值越分散, 偏离其均值 $ E(X) $ 越远.
具体来讲, 设 $ \varepsilon $ 为任意正数, 事件 $ |X-E(X)| \geqslant \varepsilon $ 发生的概率 $ P\{|X-E(X)| \geqslant \varepsilon\} $ 应该与方差 $ D(X) $ 关系密切, $ D(X) $ 越大, $ P\{|X-E(X)| \geqslant \varepsilon\} $ 应该越大, 其关系就是下面著名的 Chebyshev 不等式.

\begin{theorem}[Chebyshev 不等式]
    设随机变量 $X$ 的数学期望 $E(X)$ 和方差 $D(X)$ 均存在, 则对于任意 $\varepsilon>0$, 不等式
    $$P\qty{|X-E(X)|\geqslant \varepsilon}\leqslant \dfrac{D(X)}{\varepsilon^2}\text{ 或 }P\qty{|X-\mu|<\varepsilon}\geqslant 1-\dfrac{D(X)}{\varepsilon^2}.$$
\end{theorem}

\begin{example}
    设电站供电网有 $10000$ 盏灯, 夜晚每一盏灯开灯的概率都是 $0.7$, 假定所有电灯开或关是彼此独立的,
    试用切比雪夫不等式估计夜晚同时开着的电灯数目在 $6800$ 与 $7200$ 之间的概率.
\end{example}
\begin{solution}
    设 $ X $ 表示夜晚同时开着的电灯数目, 它服从参数 $ n=10000, p=0.7 $ 的二项分布. 于是有
    $$\begin{array}{l}
            E(X)=n p=10000 \times 0.7=7000, D(X)=n p q=10000 \times 0.7 \times 0.3=2100, \\
            P\{6800<X<7200\}=P\{|X-7000|<200\} \geqslant 1-\frac{2100}{200^{2}} \approx 0.95 .
        \end{array}$$
    计算结果表明, 虽然有 $10000$ 盏灯, 但是电站只要有供应 $7000$ 盏灯的电力就能够以相当大的概率保证电量够用.
\end{solution}