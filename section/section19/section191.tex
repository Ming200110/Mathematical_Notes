\section{Chebyshev 不等式}

方差可以描述随机变量的离散程度, 方差越大, 说明随机变量取值越分散, 偏离其均值 $ E(X) $ 越远.
具体来讲, 设 $ \varepsilon $ 为任意正数, 事件 $ |X-E(X)| \geqslant \varepsilon $ 发生的概率 $ P\{|X-E(X)| \geqslant \varepsilon\} $ 应该与方差 $ D(X) $ 关系密切, $ D(X) $ 越大, $ P\{|X-E(X)| \geqslant \varepsilon\} $ 应该越大, 其关系就是下面著名的 Chebyshev 不等式.

\begin{theorem}[Chebyshev 不等式]
    设随机变量 $X$ 的数学期望 $E(X)$ 和方差 $D(X)$ 均存在, 则对于任意 $\varepsilon>0$, 不等式
    $$P\qty{|X-E(X)|\geqslant \varepsilon}\leqslant \dfrac{D(X)}{\varepsilon^2}\text{ 或 }P\qty{|X-E(X)|<\varepsilon}\geqslant 1-\dfrac{D(X)}{\varepsilon^2}.$$
\end{theorem}

\begin{example}
    设电站供电网有 $10000$ 盏灯, 夜晚每一盏灯开灯的概率都是 $0.7$, 假定所有电灯开或关是彼此独立的,
    试用切比雪夫不等式估计夜晚同时开着的电灯数目在 $6800$ 与 $7200$ 之间的概率.
\end{example}
\begin{solution}
    设 $ X $ 表示夜晚同时开着的电灯数目, 它服从参数 $ n=10000, p=0.7 $ 的二项分布. 于是有
    $$\begin{array}{l}
            E(X)=n p=10000 \times 0.7=7000, D(X)=n p q=10000 \times 0.7 \times 0.3=2100, \\
            P\{6800<X<7200\}=P\{|X-7000|<200\} \geqslant 1-\dfrac{2100}{200^{2}} \approx 0.95 .
        \end{array}$$
    计算结果表明, 虽然有 $10000$ 盏灯, 但是电站只要有供应 $7000$ 盏灯的电力就能够以相当大的概率保证电量够用.
\end{solution}

\begin{example}
    设随机变量 $X$ 服从指数分布 $E(1)$, 用 Chebyshev 不等式得到估计 $P\qty{X\geqslant 3}\leqslant a$, 则 $a$ 等于
    \begin{tasks}(4)
        \task $\dfrac{1}{2}$.
        \task $\dfrac{1}{4}$.
        \task $\dfrac{1}{8}$.
        \task $\e ^{-3}$.
    \end{tasks}
\end{example}
\begin{solution}
    因为 $X\sim E(1)$, 所以 $EX=1, DX=1$, 那么 
    $$
    P\qty{x\geqslant 3}=P\qty{X-1\geqslant 2}=P\qty{|X-1|\geqslant 2}-P\qty{X-1\leqslant -2}=P\qty{|X-EX|\geqslant 2}\leqslant \dfrac{DX}{2^2}=\dfrac{1}{4}
    $$
    故答案选 B.
\end{solution}

\begin{example}
    从编号为 1 到 9 的九张卡片中有放回地任抽取 5 张, 试用 Chebyshev 不等式估计所取号码之和在 15 和 35 之间的概率.
\end{example}
\begin{solution}
    $EX=\dfrac{1}{9}(1+2+ \cdots +9)=5,~EX^2=\dfrac{1}{9}\qty(1^2+2^2+ \cdots +9^2)=\dfrac{95}{3}$, 那么 
    $$
    E\qty(\displaystyle \sum_{k=1}^{5} X_k)=25,~DX=EX^2-E^2X=\dfrac{95}{3}-\dfrac{75}{3}=\dfrac{20}{3},~D\qty(\displaystyle \sum_{k=1}^{5} X_k)=\dfrac{100}{3}
    $$
    则 $$
    P\qty{15\leqslant \displaystyle \sum_{k=1}^{5} X_k\leqslant 35}=P\qty{\qty|\displaystyle \sum_{k=1}^{5} X_k-25|\leqslant 10}=1-\dfrac{D\qty(\displaystyle \sum_{k=1}^{5} X_k)}{10^2}=\dfrac{2}{3}.
    $$
\end{solution}