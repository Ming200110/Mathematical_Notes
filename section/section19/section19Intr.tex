\begin{flushright}
    \begin{tabular}{r|||}
        \textit{"对于任意一个正数, 不论它多大, 总有一个足够大的整数, 使得它的倒数小于这个正数. "}\\
        ——\textit{切比雪夫}
    \end{tabular}
\end{flushright}


大数定律和中心极限定理是概率论和数理统计中非常重要的两个定理, 它们是统计学中的基础. 

1. 大数定律: 大数定律是指在独立同分布的随机变量序列中, 随着样本容量的增大, 样本均值会越来越接近于总体均值. 换句话说, 样本均值的极限等于总体均值. 大数定律是统计学的基础, 它保证了样本的可靠性和稳定性, 使得我们可以通过样本来推断总体的特征. 

2. 中心极限定理: 中心极限定理是指在独立同分布的随机变量序列中, 当样本容量趋近于无穷大时, 样本均值的分布趋近于正态分布. 也就是说, 当样本容量足够大时, 样本均值的分布呈现出正态分布的特征. 中心极限定理是统计学中非常重要的定理, 它保证了样本均值的可靠性和稳定性, 使得我们可以通过样本均值来推断总体的特征, 并进行统计推断. 

大数定律和中心极限定理是统计学中的基本定理, 它们为我们提供了理论支持和工具, 使得我们可以通过样本来推断总体的特征, 并进行统计推断. 