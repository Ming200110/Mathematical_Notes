\section{正态总体的抽样分布}

设总体 $ X $ (不管服从什么分布, 只要均值和方差存在) 的均值为 $ \mu $, 方差为 $ \sigma^{2}, X_{1}, X_{2}, \cdots ,  X_{n} $ 为来自 $ X $ 的一个样本, 样本均值 $\displaystyle \bar{X}=\frac{1}{n} \sum_{i=1}^{n} X_{i} $, 样本方差 $\displaystyle S^{2}=\frac{1}{n-1} \sum_{i=1}^{n}\left(X_{i}-\bar{X}\right)^{2} $, 则有 $\displaystyle E(\bar{X})=\mu, D(\bar{X})=\frac{\sigma^{2}}{n}, E\left(S^{2}\right)=\sigma^{2} $.

鉴于正态总体在数理统计中的重要性, 我们将不加证明地给出有关来自于正态总体样本均值及样本方差的统计量分布的结论. 这些结论将在总体参数的区间估计和假设检验问题中用到.

\subsection{单正态总体的抽样分布}

\begin{theorem}[单正态总体的抽样分布]
    设 $ X_{1}, X_{2}, \cdots, X_{n} $ 是来自正态总体 $ X \sim N\left(\mu, \sigma^{2}\right) $ 的简单随机样本, 样本均值为 $ \bar{X} $, 样本方差为 $ S^{2}$, 则
    \begin{enumerate}[label=(\arabic{*})]
        \item $\bar{X} \sim N\left(\mu, \dfrac{\sigma^{2}}{n}\right), \dfrac{\bar{X}-\mu}{\sigma / \sqrt{n}} \sim N(0,1) $;
        \item $\bar{X} $ 与 $ S^{2} $ 相互独立, 且
              $\displaystyle \dfrac{(n-1) S^{2}}{\sigma^{2}}=\dfrac{1}{\sigma^{2}} \sum_{i=1}^{n}\left(X_{i}-\bar{X}\right)^{2} \sim \chi^{2}(n-1)$;
        \item $\displaystyle \dfrac{\bar{X}-\mu}{S / \sqrt{n}} \sim t(n-1) $;
        \item $\displaystyle \dfrac{1}{\sigma^{2}} \sum_{i=1}^{n}\left(X_{i}-\mu\right)^{2} \sim \chi^{2}(n) .$
    \end{enumerate}
\end{theorem}

\begin{example}
    设相互独立的总体 $X\sim N\qty(\mu,\sigma^2),Y\sim N\qty(\mu,\sigma^2)$, $(X_1, X_2, \cdots ,X_{10})$ 与 $(Y_1, Y_2, \cdots ,Y_{10})$ 分别来自总体 $X,Y$ 的简单随机样本,
    $$
    \bar{X}=\dfrac{1}{10}\sum_{i=1}^{10} X_i, \bar{Y}=\dfrac{1}{10}\sum_{i=1}^{10} Y_i, S_1^2=\dfrac{1}{9}\sum_{i=1}^{10} \qty(X_i-\bar{X})^2,S_2^2=\dfrac{1}{9}\sum_{i=1}^{10} \qty(Y_i-\bar{Y})^2
    $$
    令 $U=\dfrac{a\qty(\bar{X}-\bar{Y})}{\sqrt{S_1^2+S_2^2}}$, 则 $a$ 的值及 $U$ 服从的分别分别为 
    \begin{tasks}(4)
        \task $\sqrt{10},t(18)$
        \task $\sqrt{10},t(9)$
        \task $\sqrt{5},t(18)$
        \task $\sqrt{5},t(9)$
    \end{tasks}
\end{example}
\begin{solution}
    因为 $\bar{X}\sim N\qty(\mu,\dfrac{\sigma^2}{10}),\bar{Y}\sim N\qty(\mu,\dfrac{\sigma^2}{10})$, 所以 $$
    \bar{X}-\bar{Y}\sim N\qty(0,\dfrac{\sigma^2}{5})\Rightarrow \dfrac{\bar{X}-\bar{Y}}{\sigma /\sqrt{5}}\sim N(0,1)
    $$
    又因为 $\dfrac{9S_1^2}{\sigma^2}\sim \chi^2(9),\dfrac{9S_2^2}{\sigma^2}\sim \chi^2(9)$, 所以 $$
    \dfrac{9\qty(S_1^2+S_2^2)}{\sigma^2}\sim \chi^2(18)
    $$
    因此 $$
    \dfrac{\dfrac{\bar{X}-\bar{Y}}{\sigma /\sqrt{ 5}}}{\sqrt{\dfrac{9\qty(S_1^2+S_2^2)}{\sigma^2}/18} }=\dfrac{\sqrt{5}\qty(\bar{X}-\bar{Y})}{\sqrt{2}\sqrt{S_1^2+S_2^2}}=\dfrac{\sqrt{10}\qty(\bar{X}-\bar{Y})}{\sqrt{S_1^2+S_2^2}}\sim t(18)
    $$
    因此选 A.
\end{solution}

\subsection{双正态总体的抽样分布}

\begin{theorem}[双正态总体的抽样分布]
    设 $ X_{1}, X_{2}, \cdots, X_{n_{1}} $ 与 $ Y_{1}, Y_{2}, \cdots, Y_{n_{2}} $ 分别为来自于总体 $ X \sim N\left(\mu_{1}, \sigma_{1}^{2}\right) $ 和 $ Y \sim N\left(\mu_{2}, \sigma_{2}^{2}\right) $ 的样本, 且这两个样本相互独立. 设 $\displaystyle \bar{X}=\dfrac{1}{n_{1}} \sum_{i=1}^{n_{1}} X_{i} $, $\displaystyle \bar{Y}=\dfrac{1}{n_{2}} \sum_{i=1}^{n_{2}} Y_{i} $ 分别是两个样本的样本均值, $$\displaystyle S_{1}^{2}=\dfrac{1}{n_{1}-1} \sum_{i=1}^{n_{1}}\left(X_{i}-\bar{X}\right)^{2}, S_{2}^{2}=\dfrac{1}{n_{2}-1} \sum_{i=1}^{n_{2}}\left(Y_{i}-\bar{Y}\right)^{2}$$ 分别是两个样本的样本方差, 则有:
    \begin{enumerate}[label=(\arabic{*})]
        \item $\displaystyle \dfrac{\bar{X}-\bar{Y}-\left(\mu_{1}-\mu_{2}\right)}{\sqrt{\dfrac{\sigma_{1}^{2}}{n_{1}}+\dfrac{\sigma_{2}^{2}}{n_{2}}}} \sim N(0,1) ;$
        \item $\displaystyle \dfrac{S_{1}^{2} / S_{2}^{2}}{\sigma_{1}^{2} / \sigma_{2}^{2}} \sim F\left(n_{1}-1, n_{2}-1\right) ;$
        \item $\displaystyle \dfrac{\bar{X}-\bar{Y}-\left(\mu_{1}-\mu_{2}\right)}{S_{\mathrm{w}} \sqrt{\dfrac{1}{n_{1}}+\dfrac{1}{n_{2}}}} \sim t\left(n_{1}+n_{2}-2\right) $, 其中 $\displaystyle S_{\mathrm{w}}^{2}=\dfrac{\left(n_{1}-1\right) S_{1}^{2}+\left(n_{2}-1\right) S_{2}^{2}}{n_{1}+n_{2}-2} .$
    \end{enumerate}
\end{theorem}