\section{导数的综合应用}

\subsection{极值问题}

\subsubsection{极值、最值}

求极值的方法步骤一般如下:
\begin{enumerate}[label=(\arabic{*})]
    \item 求可疑点, 可疑点包括:
          \begin{enumerate}[label=(\roman{*})]
              \item 稳定点 (亦称驻点或逗留点, 皆值一阶导数等于零的点);
              \item 导数不存在的点;
              \item 区间端点.
          \end{enumerate}
    \item 对可疑点进行判断, 基本方法是:
          \begin{enumerate}[label=(\roman{*})]
              \item 直接利用定义判断;
              \item 利用实际背景来判断;
              \item 查看一阶导数的符号, 当 $x$ 从左向右穿越可疑点 $x_0$ 时, \\
                    若 $f'(x)$ 由“正”变“负”, 则 $f(x_0)$ 为严格极大值点;\\
                    若 $f'(x)$ 由“负”变“正”, 则 $f(x_0)$ 为严格极小值点;\\
                    若 $f'(x)$ 不变号, 则 $f(x_0)$ 不是极值.
              \item 若 $f'(x_0)=0$, $f''(x_0)\begin{cases}
                            >0 ,&\text{则 } f(x_0) \text{ 为严格极小值,} \\
                            <0 ,&\text{则 } f(x_0) \text{ 为严格极大值;}
                        \end{cases}$
              \item $f^{(k)}(x_0)=0~ (k=1,2,\cdots,n-1),~f^{(n)}(x_0)\neq 0$, \\
                    若 $n$ 为偶数, 则 $f(x_0)$ 为极值: $\begin{cases}
                            f^{(n)}(x_0)>0 ,&\text{严格极小值,} \\
                            f^{(n)}(x_0)<0 ,&\text{严格极大值;}
                        \end{cases}$\\
                    若 $n$ 为奇数, 则 $f(x_0)$ 不是极值.
          \end{enumerate}
\end{enumerate}

\begin{example}
    求函数 $f(x)=\qty|x\qty(x^2-1)|$ 的极值以及在 $[-2,2]$ 上的最值.
\end{example}
\begin{solution}
    $f(x)=\mathrm{sgn}\qty(x\qty(x^2-1))\cdot x\qty(x^2-1)$, 那么 $f'(x)=\mathrm{sgn}\qty(x\qty(x^2-1))\cdot\qty(3x^2-1)~ (x\neq0,\pm1)$, 令 $f'(x)=0$, 得
    $x=\pm\dfrac{\sqrt{3}}{3}$
    $$f''\qty(\dfrac{\sqrt{3}}{3})=\mathrm{sgn}\qty(x\qty(x^2-1))\biggl |_{x=\frac{\sqrt{3}}{3}}\cdot 6x\biggl |_{x=\frac{\sqrt{3}}{3}}=-2\sqrt{3}<0$$
    故 $f(x)$ 在 $x=\dfrac{\sqrt{3}}{3}$ 处取极大值, 又 $f(x)$ 是偶函数, 所以在 $x=-\dfrac{\sqrt{3}}{3}$ 处亦取极大值, 
    当 $x=0,\pm1$ 时, $f(0)=f(1)=f(-1)=0\leqslant f(x)(\forall x)$, 故为最小值, 也是极小值;
    $f(2)=f(-2)=6\geqslant f(x)(\forall x\in[-2,2])$, 故 $f(x)$ 在 $[-2,2]$ 上最大值为 6, 最小值为 0.
\end{solution}

\begin{example}
    设 $f(x)$ 在 $\qty(-\dfrac{\pi}{2},\dfrac{\pi}{2})$ 可导, $f(0)=1,~f(x)>0$, $$\displaystyle\lim_{h\to0}\qty[\dfrac{f\qty(x+h\cdot \cos^2x)}{f(x)}]^{\frac{1}{h}}=\e^{x\cos^2x+\tan x}$$
    求 $f(x)$ 的表达式及极值.
\end{example}
\begin{solution}
    $\displaystyle\lim_{h\to0}\qty[\dfrac{f\qty(x+h\cdot \cos^2x)}{f(x)}]^{\frac{1}{h}}=\lim_{h\to0}\dfrac{1}{h}\ln\dfrac{f\qty(x+h\cdot\cos^2x)}{f(x)}=x\cos^2x+\tan x$, 
    等式左边化为 $$\lim_{h\to0}\dfrac{1}{h}\dfrac{f\qty(x+h\cdot\cos^2x)-f(x)}{f(x)}=\lim_{h\to0}\dfrac{\cos^2x}{f(x)}\dfrac{f\qty(x+h\cdot\cos^2x)-f(x)}{h\cos^2x}=\dfrac{\cos^2x}{f(x)}\cdot f'(x)$$
    即 $$\dfrac{f'(x)}{f(x)}=x+\tan x\cdot\sec^2x\Rightarrow \qty[\ln f(x)]'=x+\tan x\cdot\sec2x\Rightarrow \ln f(x)=\dfrac{1}{2}x^2+\dfrac{1}{2}\tan^2x+C$$
    又 $f(0)=1$, 解得 $C=0$, 于是 
    $f(x)=\e^{\frac{1}{2}x^2+\frac{1}{2}\tan^2x}$, 则 
    $$f'(x)=\e^{\frac{1}{2}x^2+\frac{1}{2}\tan^2x}\cdot\qty(x+\tan x\cdot\sec^2x)$$
    因为 $\e^{x}$ 恒大于 0, 因此只需讨论 $x+\tan x\cdot\sec^2x$ 的符号即可, 令 $g(x)=x+\tan x\cdot\sec^2x$, 对 $g(x)$ 求导得 $$g'(x)=1+\sec^4x+4\tan^2x\sec^2x>1$$
    所以 $g(x)\nearrow$, 而 $g(0)=0$, 故当 $x\in\qty(-\dfrac{\pi}{2},0)$ 时, $g(x)<0\Rightarrow f'(x)<0\Rightarrow f(x)\searrow$; 当 $x\in\qty(0,\dfrac{\pi}{2})$ 时, $f(x)\nearrow$, 
    所以当 $x=0$ 时, $f(x)$ 取得极小值, 无极大值, 且 $f(0)=1.$
\end{solution}

\subsubsection{极值点偏移}

\begin{example}
    已知函数 $f(x)=x\e^{-x},x\in\mathbb{R}$, 如果有 $f(x_1)=f(x_2)$ 且 $x_1\neq x_2$, 求证 $x_1+x_2>2.$
\end{example}
\begin{proof}[{\songti \textbf{证法一}}]
    由 $f(x_1)=f(x_2)$ 得 $\dfrac{x_2}{x_1}=\dfrac{\e^{x_2}}{\e^{x_1}}\Rightarrow \ln\dfrac{x_2}{x_1}=x_2-x_1$, 不妨设 $0<x_1<1<x_2$, 作换元 $t=\dfrac{x_2}{x_1}$, 那么 $t>1$, 且 
    $$x_1=\dfrac{\ln t}{t-1},x_2=\dfrac{t\ln t}{t-1}$$
    于是要证 $\forall t>1,\dfrac{t\ln t+\ln t}{t-1}>2$, 构造函数 $F(t)=\ln t-\dfrac{2(t-1)}{t+1}$, 那么 $F'(t)=\dfrac{1}{t}-\dfrac{4}{(t+1)^2}=\dfrac{(t-1)^2}{t(t+1)^2}>0$, 所以 $F(t)$ 在 $(1,+\infty)$ 上单调递增, 所以 $F(t)>F(1)=0$, 即得待证式成立.
\end{proof}
\begin{proof}[{\songti \textbf{证法二}}]
    由题可得 $\ln x_2-\ln x_1=x_2-x_1$, 即 $\dfrac{x_2-x_1}{\ln x_2-\ln x_1}=1$, 要证 $x_1+x_2>2$, 即证 $x_1+x_2>2\dfrac{x_2-x_1}{\ln x_2-\ln x_1}$, 
    不妨设 $0<x_1<1<x_2$, 作换元 $t=\dfrac{x_2}{x_1}$, 那么原不等式可化为 $\ln t>\dfrac{2(t-1)}{t+1}$, 由证法一可得该不等式成立, 所以原命题成立.
\end{proof}

\begin{theorem}[对数平均不等式 (ALG)]
    对任意 $0<a<b$, 满足 $\sqrt{ab}<\dfrac{b-a}{\ln b-\ln a}<\dfrac{a+b}{2}.$
\end{theorem}

\subsection{导数不等式证明中的应用}

\begin{example}
    设 $\qty{a_n}$ 为常数列, 且 $\displaystyle\qty|\sum_{k=1}^{n}a_k\sin(kx)|\leqslant |\sin x|,~\qty|\sum_{k=1}^{n}a_{n-k+1}\sin(kx)|\leqslant |\sin x|$, 证明: $$\displaystyle \qty|\sum_{k=1}^{n}a_k|\leqslant \dfrac{2}{n+1}.$$
\end{example}
\begin{proof}[{\songti \textbf{证}}]
    分别令 $\displaystyle f(x)=\sum_{k=1}^{n}a_k\sin(kx),~g(x)=\sum_{k=1}^{n}a_{n-k+1}\sin(kx)$, 则 $f,g$ 在定义域内连续, 且 $f(0)=g(0)=0$, 于是
    $$f'(x)=\sum_{k=1}^{n}ka_k\cos kx,~g'(x)=\sum_{k=1}^{n}ka_{n-k+1}\cos kx$$
    由 Lagrange 中值定理知, $\exists\xi_1,\xi_2\in(0,x)$, 使得
    $$f(x)-f(0)=f'(\xi_1)x\Rightarrow f(x)=f'(\xi_1)x,~g(x)-g(0)=g'(\xi_2)x\Rightarrow g(x)=g'(\xi_2)x$$
    因此 $f'(\xi_1)+g'(\xi_2)=\dfrac{f(x)}{x}+\dfrac{g(x)}{x}$, 一方面
    $$\qty|f'(\xi_1)+g'(\xi_2)|=\qty|\dfrac{f(x)}{x}+\dfrac{g(x)}{x}|\leqslant \qty|\dfrac{f(x)}{x}|+\qty|\dfrac{g(x)}{x}|\leqslant 2\qty|\dfrac{\sin x}{x}|\to 2~~(x\to0)$$
    另一方面当 $x\to0$ 时, $\xi_1,\xi_2\to0$, 则
    $$\qty|f'(0)+g'(0)|\to\qty|\sum_{k=1}^{n}ka_k+\sum_{k=1}^{n}ka_{n-k+1}|=\qty|\sum_{k=1}^{n}ka_k+\sum_{k=1}^{n}(n-k+1)a_k|=\qty|\sum_{k=1}^{n}\qty[k+(n-k+1)]a_k|=(n+1)\qty|\sum_{k=1}^{n}a_k|$$
    所以, 当 $x\to0$ 时, 有 $\displaystyle (n+1)\qty|\sum_{k=1}^{n}a_k|\leqslant 2\Rightarrow \qty|\sum_{k=1}^{n}a_k|\leqslant \dfrac{2}{n+1}.$
\end{proof}

