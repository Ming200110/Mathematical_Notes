\section{导数与微分}

导数和微分都是描述函数变化率的概念, 但是它们有一些不同之处.

导数是函数在某一点的变化率, 表示函数在该点的斜率.
导数可以用极限的概念来定义, 即函数在某一点的导数等于该点的函数值的极限与该点的自变量取值的极限的比值.
导数可以用符号表示为 $f'(x)$, 表示函数 $f$ 在点 $x$ 处的导数.

微分是函数在某一点的线性近似, 表示函数在该点的局部变化率.
微分可以用导数来计算, 即函数在某一点的微分等于函数在该点的导数与自变量的微小增量的乘积.
微分可以用符号表示为 $\dd f(x)$, 表示函数 $f$ 在点 $x$ 处的微分.

\subsection{导数的定义与可微性质}

\subsubsection{导数定义的运用}

\begin{definition}[导数]
    设函数 $y=f(x)$ 在点 $x_0$ 的某邻域内有定义, 当自变量 $x$ 在点 $x_0$ 处取得增量 $\Delta x~(\Delta x\neq 0)$ 时, 相应地, 函数 $y$ 取得增量 $\Delta y=f(x_0+\Delta x)-f(x_0)$, 如果极限
    \label{theDefinitionOfTheDerivationFunction}
    $$\lim_{\Delta x\to0}\dfrac{\Delta y}{\Delta x}=\lim_{\Delta x\to0}\dfrac{f(x_0+\Delta x)-f(x_0)}{\Delta x}$$
    存在, 则称函数 $y=f(x)$ 在点 $x_0$ 处\textit{可导}, 并称整个极限值为\textit{函数} $y=f(x)$ \textit{在点} $x_0$ \textit{处的导数}, 记作 $f'(x_0),~y'(x_0),~\eval{\dfrac{\dd y}{\dd x}}_{x=x_0}$,
    如果记 $x=x_0+\Delta x$, 则导数又可表示为 $f'(x_0)=\displaystyle\lim_{x\to x_0}\dfrac{f(x)-f(x_0)}{x-x_0}.$
    \index{导数}
\end{definition}

\begin{theorem}[导数与极限式]
    若 $f(x)$ 在 $x=x_0$ 处连续, 且 $\displaystyle \lim_{x \to x_0}\dfrac{f(x)}{x-x_0}=a$, 则 $f(x_0)=0$, 且 $f'(x_0)=a.$
\end{theorem}

\begin{example}
    设 $f(x)$ 为连续可导函数, 且 $\displaystyle f(x)=x+x\int_{0}^{1} f(t) \dd t+x^2\cdot\lim_{x \to 0}\dfrac{f(x)}{x}$, 求 $f(x)$.
\end{example}
\begin{solution}
    易知 $f(0)=0, f'(0)=\displaystyle \lim_{x \to 0}\dfrac{f(x)}{x}$, 设 $\displaystyle \int_{0}^{1} f(t) \dd t=A$, 那么 
    \begin{equation*}
        f(x)=x+Ax+f'(0)x^2
        \tag{*}
    \end{equation*}
    对 (*) 式两边积分, 得 $$\displaystyle A=\int_{0}^{1} f(x) \dd x=\int_{0}^{1} \qty(x+Ax+f'(0)x^2) \dd x \Rightarrow 3+2f'(0)=3A$$
    对 (*) 式两边求导, 并令 $x=0$, 得 $$f'(0)=1+A$$
    于是解得 $f'(0)=6, \displaystyle \int_{0}^{1} f(x) \dd x=5$, 故 $f(x)=6x+6x^2.$
\end{solution}

\begin{example}[2004 数一]
    设函数 $f(x)$ 连续, 且 $f'(0)>0$, 则存在 $\delta>0$, 使得 (\quad).
    \begin{tasks}(2)
        \task $f(x)$ 在 $(0,\delta)$ 内单调递增
        \task $f(x)$ 在 $(-\delta,0)$ 内单调递增
        \task 对任意 $x\in(0,\delta)$ 有 $f(x)>f(0)$
        \task 对任意 $x\in(-\delta,0)$ 有 $f(x)>f(0)$
    \end{tasks}
\end{example}
\begin{solution}
    由于 $f'(0)=\displaystyle\lim_{x\to0}\dfrac{f(x)-f(0)}{x}>0$, 所以由函数极限的局部保号性知, 存在 $\delta>0$, 使得当 $0<|x|<\delta$ 时, $\dfrac{f(x)-f(0)}{x}>0$,
    则当 $x\in(-\delta,0)$ 时, $f(x)-f(0)<0$, 即有 $f(x)<f(0)$; 当 $x\in(0,\delta)$ 时, $f(x)-f(0)>0$, 即有 $f(x)>f(0)$, 因此选 C.
\end{solution}
\begin{inference}
    若 $f'(x_0)>0$, 则存在 $\delta>0$, 当 $x\in(x_0-\delta,x_0)$ 时, $f(x)<f(x_0)$; 当 $x\in(x_0,x_0+\delta)$ 时, $f(x)>f(x_0)$, 同理 $f'(x_0)<0$ 也有类似结论.
\end{inference}

\begin{example}
    下列 $f$ 在 $x_0$ 的某个邻域内均有定义, 哪项极限式能作为 $f$ 在 $x_0$ 处的导数定义 (\quad).
    \begin{tasks}(2)
        \task $\displaystyle \lim_{\Delta x\to0}\dfrac{f(x_0+\Delta x)-f(x_0-\Delta x)}{2\Delta x}$
        \task $\displaystyle \lim_{n\to\infty}n\qty[f\qty(x_0+\dfrac{1}{n})-f(x_0)]$
        \task $\displaystyle \lim_{x\to0}\dfrac{f\qty(x_0+x^2)-f(x_0)}{x^2}$
        \task $\displaystyle \lim_{x\to0}\dfrac{f(x_0)-f(x_0-x)}{x}$
    \end{tasks}
\end{example}
\begin{solution}
    按照定义 \ref{theDefinitionOfTheDerivationFunction}, 不但要求 $f$ 在 $x_0$ 的某个邻域有定义, 而且 $f$ 在 $x_0$ 处的导数 $f'(x_0)$ 是否存在以及 $f'(x_0)$ 的值的大小都与 $f$ 在 $x_0$ 处的值 $f(x_0)$ 有关, 由此可知 A 不能作为 $f$ 在 $x_0$ 处导数的定义,
    实际上, 从 A 的分子结构易见, 该极限存在与否同 $f$ 在 $x_0$ 处的值无关, 即使 $f$ 在点 $x_0$ 处没有定义, 或者有定义但不连续, A 极限也可能存在, 例如, $$f(x)=\begin{cases}
            \dfrac{1}{x^2}, & x\neq0 \\[6pt]
            0,              & x=0
        \end{cases}$$在点 $x=0$ 处不连续 ($x=0$ 为第二类间断点), 因而在 $x=0$ 处必不可导, 但极限却存在 $$\lim_{\Delta x\to0}\dfrac{f(0+\Delta x)-f(0-\Delta x)}{2\Delta x }=\dfrac{1}{2}\lim_{\Delta x\to0}\dfrac{\dfrac{1}{\qty(\Delta x)^2}-\dfrac{1}{(\Delta x)^2}}{\Delta x}=0$$
    因此, 用 A 极限来定义函数 $f$ 在点 $x_0$ 处的导数是不恰当的, 实际上, 对于任何偶函数 $f$, 在点 $x_0=0$ 处 A 极限总存在, 且 $$\lim_{\Delta x\to0}\dfrac{f(0+\Delta x)-f(0-\Delta x)}{2\Delta x}=0$$
    但根据定义 \ref{theDefinitionOfTheDerivationFunction} 却不一定存在, 也就是说, 函数 $f$ 在点 $x_0=0$ 处不一定可导;\\
    B 极限也不能作为 $f$ 在点 $x_0$ 处的导数定义, 因为
    $$\exists\lim_{n\to\infty}n\qty[f\qty(x_0+\dfrac{1}{n})-f(x_0)]\not\Rightarrow \exists\lim_{x\to h}\dfrac{f(x_0+h)-f(x_0)}{h}$$
    例如, 函数 $f(x)=\begin{cases}
            1, & x\text{ 为无理数} \\
            0, & x\text{ 为有理数}
        \end{cases}$ 处处有定义, 处处不连续, 从而处处不可导, 但是, $$\forall x\in\mathbb{R},~\lim_{n\to\infty}n\qty[f\qty(x_0+\dfrac{1}{n})-f(x_0)]=\lim_{n\to\infty}\dfrac{f\qty(x_0+\dfrac{1}{n})-f(x_0)}{\dfrac{1}{n}}=\lim_{n\to\infty}\dfrac{0}{\dfrac{1}{n}}=0$$
    C 极限也不能作为 $f$ 在 $x_0$ 处导数的定义, 实际上, 极限 $$\exists\lim_{x\to0}\dfrac{f\qty(x_0+x^2)-f(x_0)}{x^2}\Leftrightarrow \exists f_+'(x_0)$$
    由此不能推出定义 \ref{theDefinitionOfTheDerivationFunction} 存在, 例如函数 $f(x)=\begin{cases}
            x,  & x\geqslant 0 \\
            -1, & x<0
        \end{cases}$ 虽然极限 $$\lim_{x\to0}\dfrac{f\qty(0+x^2)-f(0)}{x^2}=\lim_{x\to0}\dfrac{x^2-0}{x^2}=1$$
    但是由于 $f(x)$ 在 $x=0$ 处不连续, 所以它在点 $x=0$ 处不可导;\\
    D 极限与导数定义等价, 因为 $$\lim_{x\to0}\dfrac{f(x_0)-f(x_0-x)}{x}=\lim_{x\to0}\dfrac{f(x_0-x)-f(x_0)}{-x} \xlongequal{h=-x}\lim_{h\to0}\dfrac{f(x_0+h)-f(x_0)}{h}=f'(x_0).$$
    综上所述, 选 D.
\end{solution}

\begin{example}[2011 数二]
    已知 $f(x)$ 在 $x=0$ 处可导, 且 $f(0)=0$, 则 $\displaystyle\lim_{x\to0}\dfrac{x^2f(x)-2f\qty(x^3)}{x^3}$ 等于 (\quad).
    \begin{tasks}(4)
        \task $-2f'(0)$
        \task $-f'(0)$
        \task $f'(0)$
        \task $0$
    \end{tasks}
\end{example}
\begin{solution}
    \textbf{法一: }凑导数定义,
    $$\text{原式}=\lim_{x\to0}\dfrac{x^2f(x)-x^2f(0)+2f(0)-2f\qty(x^3)}{x^3}=\lim_{x\to0}\qty(\dfrac{f(x)-f(0)}{x}-2\dfrac{f\qty(x^3)-f(0)}{x^3})=f'(0)-2f'(0)=-f'(0)$$
    \textbf{法二: }由于 $f(x)$ 在 $x=0$ 处可导, 则
    $$f(x)=f(0)+f'(0)x+o(x)=f'(0)x+o(x),~f\qty(x^3)=f'(0)x^3+o\qty(x^3)$$
    于是 $$\text{原式}=\lim_{x\to0}\dfrac{x^2(f'(0)x+o(x))-2\qty(f'(0)x^3+o\qty(x))}{x^3}=f'(0)-2f'(0)=-f'(0).$$
\end{solution}

\begin{example}
    证明: $\left(\dfrac{ax+b}{cx+d}\right)'=\dfrac{1}{(cx+d)^2}
        \begin{vmatrix}
            a & b \\
            c & d
        \end{vmatrix}.$
\end{example}
\begin{proof}[{\songti \textbf{证}}]
    $\left(\dfrac{ax+b}{cx+d}\right)'=\dfrac{a(cx+d)-c(ax+b)}{(cx+d)^2}=\dfrac{ad-bc}{(cx+d)^2}=\dfrac{1}{(cx+d)^2}
        \begin{vmatrix}
            a & b \\
            c & d
        \end{vmatrix}~  (cx+d\neq0).$
\end{proof}

% \begin{theorem}
%     $n$ 阶行列式求导公式
% \end{theorem}

\begin{example}
    求下列一阶导数
    \setcounter{magicrownumbers}{0}
    \begin{table}[H]
        \centering
        \begin{tabular}{l | l}
            (\rownumber{}) $\displaystyle F(x)=\begin{vmatrix}x-1 & 1 &2 \\-3 & x &3 \\-2 & -3 &x+1\end{vmatrix}.$ & (\rownumber{}) $\displaystyle F(x)=\begin{vmatrix}x &x^2 &x^3 \\1 & 2x &3x^2 \\0 & 2 &6x\end{vmatrix}.$ \\
        \end{tabular}
    \end{table}
\end{example}
\begin{solution}
    \begin{enumerate}[label=(\arabic{*})]
        \item 利用例 \ref{n jie hang lie shi wei fen fa} 的结论, 有
              \begin{flalign*}
                  F'(x) & =
                  \begin{vmatrix}
                      1  & 0  & 0   \\
                      -3 & x  & 3   \\
                      -2 & -3 & x+1
                  \end{vmatrix}+
                  \begin{vmatrix}
                      x-1 & 1  & 2   \\
                      0   & 1  & 0   \\
                      -2  & -3 & x+1
                  \end{vmatrix}+
                  \begin{vmatrix}
                      x-1 & 1 & 2 \\
                      -3  & x & 3 \\
                      0   & 0 & 1
                  \end{vmatrix}                                   \\
                        & =(x^2+x+9)+(x^2-1+4)+(x^2-x+3)=3(x^2+5).
              \end{flalign*}
        \item 同上, 有 $\displaystyle F'(x)=
                  \begin{vmatrix}
                      1 & 2x & 3x^2 \\
                      1 & 2x & 3x^2 \\
                      0 & 2  & 6x
                  \end{vmatrix}+
                  \begin{vmatrix}
                      x & x^2 & x^3 \\
                      0 & 2   & 6x  \\
                      0 & 2   & 6x
                  \end{vmatrix}+
                  \begin{vmatrix}
                      x & x^2 & x^3  \\
                      1 & 2x  & 3x^2 \\
                      0 & 0   & 6
                  \end{vmatrix}=
                  0+0+6(2x^2-x^2)=6x^2.$
    \end{enumerate}
\end{solution}

\begin{example}
    设 $f(x)$ 在 $x=0$ 处可导, $f(0)\neq0,~f'(0)\neq 0$, 且 $$af(h)+bf(2h)-f(0)=o(h)~  h\to0$$
    求 $a,b.$
\end{example}
\begin{solution}
    两边同时除以 $h$, 有
    $$o(1)=\dfrac{af(h)+bf(2h)-f(0)}{h}=a\cdot\dfrac{f(h)-f(0)}{h}+2b\cdot\dfrac{f(2h)-f(0)}{2h}+\dfrac{(a+b-1)f(0)}{h}$$
    令 $h\to0$ 得方程组 $\begin{cases}
            a+2b=0 \\a+b-1=0
        \end{cases}\Rightarrow a=2,b=-1.$
\end{solution}

\begin{example}
    设 $f(0)=0,f'(0)=A$, 求极限 $\displaystyle\lim_{n\to\infty}\sum_{k=1}^{n}f\left(\frac{k}{n^2}\right).$
\end{example}
\begin{solution}
    $\displaystyle\lim_{x\to0}\left|\frac{f(x)-f(0)}{x}-f'(0)\right|=0\Rightarrow\frac{f(x)-f(0)}{x}-f'(0)=o(1)\Rightarrow f(x)=x\cdot o(1)+xf'(0)+f(0)$, 故
    $$f\left(\frac{k}{n^2}\right)=\frac{k}{n^2}\cdot o(1)+\frac{k}{n^2} f'(0)+f(0)=\frac{k}{n^2}(A+o(1))~  (n\to\infty)$$
    因此 $$\sum_{k=1}^{n}f\left(\frac{k}{n^2}\right)=\sum_{k=1}^{n}\frac{k}{n^2}(A+o(1))=\frac{n+1}{2n}(A+o(1))~  (n\to\infty)$$
    取极限有 $\displaystyle\lim_{n\to\infty}\sum_{k=1}^{n}f\left(\frac{k}{n^2}\right)=\frac{A}{2}.$
\end{solution}
\begin{example}
    设 $f(x)$ 在 $x=x_0$ 处可微, $\alpha_n<x_0<\beta_n~  (n=1,2,\cdots),~\lim\limits_{n\to\infty}\alpha_n=\lim\limits_{n\to\infty}\beta_n=x_0$,
    证明: $$\lim\limits_{n\to\infty}\dfrac{f(\beta_n)-f(\alpha_n)}{\beta_n-\alpha_n}=f'(x_0).$$
\end{example}
\begin{proof}[{\songti \textbf{证}}]
    首先, 容易得到
    \begin{flalign*}
        \dfrac{f\left( \beta n\right) -f\left( \alpha _{n}\right) }{\beta _{n}-\alpha _{n}} & =\dfrac{f\left( \beta _{n}\right) -f\left( x_{0}\right) +f\left( x_{0}\right) -f\left( \alpha _{n}\right) }{\beta _{n}-\alpha _{n}}                                                                                                              \\
                                                                                            & =\dfrac{\beta _{n}-x_{0}}{\beta n-\alpha _{n}}\dfrac{f\left( \beta _{n}\right) -f\left( x_{0}\right) }{\beta n-x_{0}}-\dfrac{\alpha _{n}-x_{0}}{\beta n-\alpha _{n}}\dfrac{f\left( \alpha _{n}\right) -f\left( x_{0}\right) }{\alpha _{n}-x_{0}}
    \end{flalign*}
    若记 $\lambda _{n}=\dfrac{\beta _{n}-x_{0}}{\beta _{n}-\alpha _{n}}$, 那么 $\dfrac{x_{0}-\alpha _{n}}{\beta _{n}-\alpha _{n}}=1-\lambda _{n}$, 且 $0<\lambda_n<1,~0<1-\lambda_n<1$, 故上式可改写为
    $$\dfrac{f\left( \beta _{n}\right) -f\left( \alpha _{n}\right) }{\beta _{n}-\alpha _{n}}=\lambda _{n}\dfrac{f\left( \beta _{n}\right) -f\left( x_{0}\right) }{\beta _{n}-x_{0}}+\left( 1-\lambda _{n}\right) \dfrac{f\left( \alpha _{n}\right) -f\left( x_{0}\right) }{\alpha _{n}-x_{0}}$$
    即 $f'\left( x_{0}\right) =\lambda _{n}f'\left( x_{0}\right) +\left( 1-\lambda _{n}\right) f'\left( x_{0}\right) $, 故有 $\forall\varepsilon>0,\exists N>0$, 当 $n>N$ 时,
    \begin{flalign*}
        \left| \dfrac{f\left( \beta _{n}\right) -f\left( \alpha _{n}\right) }{\beta _{n}-\alpha _{n}}-f'\left( x_{0}\right) \right| & \leqslant \lambda _{n}\left| \dfrac{f\left( \beta _{n}\right) -f\left( x_{0}\right) }{\beta _{n}-x_{0}}-f'\left( x_{0}\right) \right| +\left( 1-\lambda _{n}\right) \left| \dfrac{f\left( \alpha _{n}\right) -f\left( x_{0}\right) }{\alpha _{n}-x_{0}}-f'\left( x_{0}\right) \right| \\
                                                                                                                                    & <\lambda _{n}\varepsilon +\left( 1-\lambda _{n}\right) \varepsilon =\varepsilon
    \end{flalign*}
    原极限获证.
\end{proof}
\begin{example}
    设 $f(x)$ 在 $x=0$ 处连续, 且 $\lim\limits_{x\to0}\dfrac{f(2x)-f(x)}{x}=A$, 求证:
    $f'(0) \text{ 存在, 并且 } f'(0)=A.$
\end{example}
\begin{proof}[{\songti \textbf{证}}]
    因已知 $\lim\limits_{x\to0}\dfrac{f(2x)-f(x)}{x}=A$, 即 $\forall \varepsilon>0,\exists\delta>0,\text{当 }|x|<\delta$ 时, 有
    $$A-\dfrac{\varepsilon }{2} <\dfrac{f(2x) -f(x) }{x} <A+\dfrac{\varepsilon }{2}$$
    特别地, 取 $x_n=\dfrac{x}{2^k}~  (k\in\mathbb{N})$, 上式亦成立, 则有
    $$\dfrac{1}{2^{k}}\left( A-\dfrac{\varepsilon }{2}\right)  <\dfrac{f\left( \dfrac{x}{2^{k-1}}\right) -f\left( \dfrac{x}{2^{k}}\right) }{x} <\dfrac{1}{2^{k}}\left( A+\dfrac{\varepsilon }{2}\right) $$
    将此 $n$ 式相加, 得
    $$\sum ^{n}_{k=1}\left[ f\left( \dfrac{x}{2^{k-1}}\right) -f\left( \dfrac{x}{2^{k}}\right) \right] =f(x) -f\left( \dfrac{x}{2^{n}}\right) =f(x) -f\left( x_{n}\right) $$
    因为 $\displaystyle\sum ^{n}_{k=1}\dfrac{1}{2^{k}}=1-\dfrac{1}{2^{k}}$, 于是
    $$\left( 1-\dfrac{1}{2^{n}}\right) \left( A-\dfrac{\varepsilon }{2}\right)  <\dfrac{f(x) -f\left( x_{n}\right) }{x} <\left( 1-\dfrac{1}{2^{n}}\right) \left( A+\dfrac{\varepsilon }{2}\right) $$
    再令 $n\to\infty$, 取极限, 有 $x_n=\dfrac{x}{2^n}\to0$, 而 $f$ 在 $x=0$ 处连续,  $\lim\limits_{n\to\infty}f(x_n)=f(0)$, 故
    $$A-\dfrac{\varepsilon }{2}\leqslant \dfrac{f(x) -f(0) }{x}\leqslant A+\dfrac{\varepsilon }{2}$$
    即 $\left| \dfrac{f(x) -f(0) }{x}-A\right| \leqslant \dfrac{\varepsilon }{2} <\varepsilon $, $f'(0)$ 存在且 $f'(0)=A.$
\end{proof}

\subsubsection{两种符号的区别}

\begin{definition}[左右导数]
    若极限 $\displaystyle\lim_{\Delta x\to 0^-}\dfrac{\Delta y}{\Delta x}=\lim_{\Delta x\to0^-}\dfrac{f(x_0+\Delta x)-f(x_0)}{\Delta x}$ 存在, 则该极限值称为 $f(x)$ \textit{在点} $x_0$ \textit{处的左导数}, 记作 $f'_(x_0)$.
    同理可得\textit{右导数}的定义.

    $f_-'(x_0)~(f_+'(x_0))$ 表示函数在点 $x_0$ 处的左 (右) 导数, 而 $\displaystyle\lim_{x\to x_0^-}~(\lim_{x\to x_0^+})$ 表示导函数在 $x_0$ 处的左 (右) 极限,
    一般情况下 $f_{\pm}'(x_0)\neq \displaystyle\lim_{x\to x_0^{\pm}}f'(x).$
    \index{左右导数}
\end{definition}

\begin{theorem}[函数的可导性与连续性]
    若函数 $y=f(x)$ 在点 $x_0$ 可导, 则 $y=f(x)$ 在点 $x_0$ 必连续, 但连续不一定可导.

    函数在某一点可导不能保证它在该点的某一邻域内可导, 例如: $f(x)=\begin{cases}
            x^2, & x\text{ 为无理数} \\
            0,   & x\text{ 为有理数}
        \end{cases}$

    函数在某一点可导不能保证其导函数在该点连续, 例如: $f(x)=\begin{cases}
            x^2\sin\dfrac{1}{x}, & x\neq0 \\
            0,                   & x=0
        \end{cases}$

    两个函数至少有一个不可导, 那么复合后不一定不可导.
    \index{函数的可导性与连续性}
\end{theorem}

\begin{example}
    设函数 $f(x)=\begin{cases}
            \dfrac{x}{1+\e ^{\frac{1}{x}}}, & x<0 \\0,&x=0\\ \dfrac{2x}{1+\e^x},&x>0
        \end{cases}$ 求函数在点 $x=0$ 处的导数.
\end{example}
\begin{solution}
    $f(x)$ 是分段函数, 按定义分别求 $f(x)$ 在 $x=0$ 处的左、右导数,
    $$f'_(0)=\lim_{x\to0^-}\dfrac{\dfrac{x}{1+\e ^{\frac{1}{x}}}-0}{x}=\lim_{x\to0 ^-}\dfrac{1}{1+\e ^{\frac{1}{x}}}=1,~f'_+(0)=\lim_{x\to0^+}\dfrac{\dfrac{2x}{1+\e^x}-0}{x}=\lim_{x\to0^+}\dfrac{2}{1+\e^x}=1$$
    因为左右导数相等, 所以 $f'(0)=1$.
\end{solution}

\begin{example}[2018 数一]
    下列函数中, 在 $x=0$ 处不可导的是 (\quad).
    \begin{tasks}(4)
        \task $f(x)=|x|\sin |x|$
        \task $f(x)=|x|\sin \sqrt{|x|}$
        \task $f(x)=\cos |x|$
        \task $f(x)=\cos \sqrt{|x|}$
    \end{tasks}
\end{example}
\begin{solution}
    对选项 D, 有导数定义知
    $$f'_-(0)=\lim_{x\to0^-}\dfrac{\cos\sqrt{|x|}-1}{x}=\lim_{x\to0^-}\dfrac{-\dfrac{1}{2}|x|}{x}=\dfrac{1}{2},~f'_+(0)=\lim_{x\to0^+}\dfrac{\cos\sqrt{|x|}-1}{x}=\lim_{x\to0^+}\dfrac{-\dfrac{1}{2}|x|}{x}=-\dfrac{1}{2}$$
    则 $f(x)=\cos \sqrt{|x|}$ 在 $x=0$ 处不可导, 故选 D.
\end{solution}



\subsection{反函数、用参数方程确定的函数、隐函数的导数}

\begin{theorem}[反函数的导数]
    导数 $f'(x)\neq0$ 的可微函数 $y=f(x)~  (a<x<b)$ 具有单值连续的反函数 $x=f^{-1}(y)$,
    此反函数可微, 那么 $$x'_y=\dfrac{1}{y_x'}$$ 反函数的导数等于原函数的导数的倒数.
    \index{反函数的导数}
\end{theorem}

\begin{example}
    设 $f(x)$ 为单调可微函数, $g(x)$ 与 $f(x)$ 互为反函数, 且 $f(2)=4,f'(2)=\sqrt{5},f'(4)=6$, 则 $g'(4)$ 等于 (\quad).
    \begin{tasks}(4)
        \task $\dfrac{1}{4}$
        \task $\dfrac{1}{\sqrt{5}}$
        \task $\dfrac{1}{6}$
        \task $4$
    \end{tasks}
\end{example}
\begin{solution}
    $f(2)=4\Leftrightarrow g(4)=2$, 故 $g'(4)=\dfrac{1}{f'(2)}=\dfrac{1}{\sqrt{5}}$, 选 B.
\end{solution}

\begin{theorem}[反函数的二阶导数]
    反函数二阶导数 $\displaystyle \dv[2]{y}{x}=-\dv[2]{x}{y}\cdot\qty(\dv{y}{x})^{3}$.
\end{theorem}
\begin{proof}[{\songti \textbf{证}}]
    因为 $\displaystyle \dv{y}{x}=\qty(\dv{x}{y})^{-1}$, 所以 
    $$
    \dv[2]{y}{x}=\dv{x}\qty(\dv{x}{y})^{-1}=\dv{y}\qty(\dv{x}{y})^{-1}\cdot\dv{y}{x}=-\qty(\dv{x}{y})^{-2}\dv[2]{y}{x}\cdot\dv{y}{x}=-\dv[2]{x}{y}\cdot\qty(\dv{y}{x})^{3}.
    $$
\end{proof}

\begin{theorem}[用参数方程确定的函数的导数]\index{用参数方程确定的函数的导数}
    若方程组
    $\begin{cases}
            x=\varphi(t) \\
            y=\psi (t)
        \end{cases}(\alpha<t<\beta)$, 其中 $\varphi(t)$ 和 $\psi(t)$ 为可微函数, 且 $\varphi'(t)\neq0$,
    在某区域内确定 $y$ 为 $x$ 的单值连续函数: $y=\psi\left(\varphi^{-1}(x)\right)$, 则此函数的导数为
    $$y'_x=\dfrac{y'_t}{x'_t}.$$
\end{theorem}

\begin{example}[2017 数二]
    设函数 $y=y(x)$ 由参数方程 $\begin{cases}
            x=t+\e^t \\y=\sin t
        \end{cases}$ 确定, 求 $\displaystyle\eval{\dv[2]{y}{x}}_{x=0}$
\end{example}
\begin{solution}
    $\displaystyle\dv{y}{x}=\dv{y}{t}\cdot\qty(\dv{x}{t})^{-1}=\dfrac{\cos t}{1+\e^t}$, 那么
    $$\dv[2]{y}{x}=\dv{x}\qty(\dv{y}{x})=\dv{t}\qty(\dv{y}{x})\cdot\qty(\dv{x}{t})^{-1}=\dfrac{-\sin t\qty(1+\e^t)-\e^t\cos t}{\qty(1+\e^t)^3}$$
    因此 $\displaystyle\eval{\dv[2]{y}{x}}_{x=0}=-\dfrac{1}{8}.$
\end{solution}

\begin{theorem}[隐函数的导数]
    \index{隐函数的导数}若可微函数 $y=y(x)$ 满足方程 $F(x,y)=0$, 则此隐函数的导数为
    $$\dfrac{\dd }{\dd x}[F(x,y)]=0.$$
\end{theorem}

\begin{example}
    设 $u=xy+y^2$, 其中 $y=y(x)$ 是由 $x^2+y^2=5$ 确定的函数, 求 $\displaystyle\dv{u}{x}\biggl |_{\substack{x=2\\y=-1}}.$
\end{example}
\begin{solution}
    等式两边同时对 $x$ 求导, $\begin{cases}
            \displaystyle \dv{u}{x}=y+\dv{y}{x}+2y\dv{y}{x} \\[6pt]
            \displaystyle 2x+2y\dv{x}{y} =0
        \end{cases}$, 并将 $x=2,y=-1$ 代入, 得 $\displaystyle\eval{\dv{u}{x}}_{\substack{x=2\\y=-1}}=-1.$
\end{solution}

\begin{example}
    设 $y=y(x)$ 由方程 $x^3+y^3+xy=1$ 确定, 求 $\displaystyle\lim_{x\to0}\dfrac{3y+x-3}{x^2}.$
\end{example}
\begin{solution}
    当 $x=0$ 时, $y(0)=1$,
    方程 $x^3+y^3+xy=1$ 两边同时对 $x$ 求导, 得 $$3x^2+3y^2\cdot y'+y+xy'=0\Rightarrow y'(0)=-\dfrac{1}{3}$$
    继续对 $x$ 求导, 得 $$6x+6y\cdot\qty(y')^2+3y^2\cdot y''+2y'+xy''=0\Rightarrow y''(0)=0$$
    将 $y(x)$ 在 $x=0$ 处 Taylor 展开, 得
    $$y(x)=y(0)+y'(0)x+\dfrac{1}{2}y''(x)x^2+o\qty(x^2)$$
    于是待求极限式等于 $\displaystyle\lim_{x\to0}\dfrac{3-x+o\qty(x^2)+x-3}{x^2}=0.$
\end{solution}

\begin{example}[2007 数二]
    已知函数 $f(u)$ 具有二阶导数, 且 $f'(0)=1$, 函数 $y=y(x)$ 由方程 $y-x\e ^{y-1}=1$ 所确定, 设 $z=f(\ln y-\sin x)$, 求 $\displaystyle \eval{\dv{z}{x}}_{x=0},~\eval{\dv[2]{z}{x}}_{x=0}.$
\end{example}
\begin{solution}
    当 $x=0$ 时, 由 $y-x\e ^{y-1}=1$ 知 $y=1$, 并且对方程 $y-x\e ^{y-1}=1$ 两端的 $x$ 求导, 得 $y'-\qty(\e ^{y-1}+x\e ^{y-1}\cdot y')=0$, 将 $x=0,y=1$, 代入该式求得 $y'(0)=1$, 那么
    $\displaystyle \eval{\dv{z}{x}}_{x=0}=\eval{f'(\ln y-\sin x)\qty(\dfrac{y'}{y}-\cos x)}_{x=0}=0$, 且 $$y''-\qty[2\e ^{y-1}\cdot y'+x\qty(\e ^{y-1}\cdot y'^2+\e ^{y-1}\cdot y'')]=0$$
    解得 $y''(0)=0$, 那么
    $$\eval{\dv[2]{z}{x}}_{x=0}=\eval{f''(\ln y-\sin x)\qty(\dfrac{y'}{y}-\cos x)^2+f'(\ln y-\sin x)\qty(\dfrac{y''\cdot y-y'^2}{y^2}+\sin x)}_{x=0}=f'(0)(2-1)=f'(0)=1.$$
\end{solution}

\begin{example}
    设函数 $y=y(x)$ 由方程组 $\begin{cases}
            x=3t^2+2t+3 \\ \e^y\sin t-y+1=0
        \end{cases}$ 所确定, 试求 $\displaystyle\eval{\dv[2]{y}{x}}_{t=0}.$
\end{example}
\begin{solution}
    对方程组每个方程两边分别取微分, 得 $$\begin{cases}
            \dd x=6t\dd t+2\dd t \\ \e^y\sin t\dd y+\e^y\cos t\dd t-\dd y=0
        \end{cases}$$ 则 $$\displaystyle\dv{x}{t}=6t+2,~\dv{y}{t}=\dfrac{\e^y\cos t}{1-\e^y\sin t}$$ 那么
    \begin{flalign*}
        \dv{y}{x}    & =\dv{y}{t}\cdot\qty(\dv{x}{t})^{-1}=\dfrac{\e^y\cos t}{\qty(1-\e^y\sin t)(6t+2)}=\dfrac{\e^y\cos t}{\qty(2-y)(6t+2)}                                                     \\
        \dv[2]{y}{x} & =\dv{t}\qty[\dfrac{\e^y\cos t}{(2-y)(6t+2)}]\qty(\dv{t}{x})=\dfrac{\qty(\e^yy'_t\cos t-\e^y\sin t)(2-y)(6t+2)-\qty[\qty(-y'_t)(6t+2)+6(2-y)]\e^y\cos t}{(2-y)^2(6t+2)^3}
    \end{flalign*}
    由 $\displaystyle\eval{\dv{y}{t}}_{t=0},~\eval{y}_{t=0}=1$, 代入上式, 得 $\displaystyle\eval{\dv[2]{y}{x}}_{t=0}=\dfrac{\e(2\e-3)}{4}.$
\end{solution}

\begin{example}
    求下列 $y'_x$ (参数均为正实数, 且 $r=\sqrt{x^2+y^2},~\varphi=\arctan\dfrac{y}{x}$).
    \setcounter{magicrownumbers}{0}
    \begin{table}[H]
        \centering
        \begin{tabular}{l | l | l}
            (\rownumber{}) $x=\sin^2t~  y=\cos^2t.$      & (\rownumber{}) $x=a\cos t~  y=b\sin t.$                           & (\rownumber{})  $x=a\cos^3t~  y=a\sin^3t.$               \\
            (\rownumber{}) $\sqrt{x}+\sqrt{y}=\sqrt{a}.$ & (\rownumber{}) $x^{\frac{2}{3}}+y^{\frac{2}{3}}=a^{\frac{2}{3}}.$ & (\rownumber{})  $\arctan\dfrac{y}{x}=\ln\sqrt{x^2+y^2}.$ \\
            (\rownumber{}) $r=a\varphi.$                 & (\rownumber{}) $r=a(1+\cos \varphi).$                             & (\rownumber{})  $r=a\e ^{m\varphi}.$
        \end{tabular}
    \end{table}
\end{example}
\begin{solution}
    \begin{enumerate}[label=(\arabic{*})]
        \item $\dfrac{\dd y}{\dd x}=\dfrac{\dfrac{\dd y}{\dd t}}{\dfrac{\dd x}{\dd t}}=\dfrac{-2\cos t\sin t}{2\sin t\cos t}=-1~  ( 0 <x < 1). $
        \item $\dfrac{\dd y}{\dd x}=\dfrac{b\cos t}{-a\sin t}=-\cot x~  ( 0 <\left| t\right|  < \pi ) $
        \item $\dfrac{\dd y}{\dd x}=\dfrac{3a\sin ^{2}t\cos t}{-3a\cos ^{2}t\sin t}=-\tan t~  ( t\neq  \dfrac{2k+1}{2}\pi k\text{ 为正整数}) .$
        \item 两边对 $x$ 求导, 得 $\dfrac{1}{2\sqrt{x}}+\dfrac{1}{2\sqrt{y}}.y_{x}'=0\Rightarrow y_{x}'=-\sqrt{\dfrac{y}{x}}~  ( x,y > 0) .$
        \item 两边对 $x$ 求导, 得 $\dfrac{2}{3}x^{-\frac{1}{3}}+\dfrac{2}{3}y^{-\frac{1}{3}}y_{x}'=0\Rightarrow y_{x}'=-\sqrt[3] {\dfrac{y}{x}}~  ( x\neq  0) .$
        \item 两边对 $x$ 求导, 得 $\dfrac{1}{1+\dfrac{y^{2}}{x^{2}}}\cdot \dfrac{xy_{x}'-y}{x^{2}}=\dfrac{x+yyx'}{x^{2}+y^{2}}\Rightarrow y_{x}'=\dfrac{x+y}{x-y}~  ( x\neq  y,0) .$
        \item $x=r\cos\varphi,~y=r\sin\varphi,~r=r(\varphi)$, 所以
              \begin{equation}
                  \dfrac{\dd y}{\dd x}=\dfrac{\dfrac{\dd y}{\dd \varphi }}{\dfrac{\dd x}{\dd \varphi }}=\dfrac{\dfrac{\dd r}{\dd \varphi }\sin \varphi +r\cos \varphi }{\dfrac{\dd r}{\dd \varphi }\cos \varphi -r\sin \varphi }\tag{*}
              \end{equation}
              所以 $\dfrac{\dd y}{\dd x}=\dfrac{\arcsin \varphi +a\varphi \cos \varphi }{\arccos \varphi -a\varphi \sin \varphi }=\tan \left( \varphi +\arctan \varphi \right) .$
        \item $\dfrac{\dd r}{\dd \varphi}=-a\sin\varphi$, 代入 (*) 式, 且 $(\varphi\neq0,\pm\dfrac{2\pi}{3})$,
              $$\dfrac{\dd y}{\dd x}=\dfrac{-a\sin ^{2}\varphi +a\left( 1+\cos \varphi \right) \cos \varphi }{-a\sin \varphi \cos \varphi -a\left( 1+\cos \varphi \right) \sin \varphi }=-\dfrac{\cos 2\varphi +\cos \varphi }{\sin 2\varphi +\sin \varphi }=-\dfrac{\cos \dfrac{3\varphi }{2}\cos \dfrac{\varphi }{2}}{2\sin \dfrac{3\varphi }{2}\cos \dfrac{\varphi }{2}}=-\cot\dfrac{3\varphi}{2}.$$
        \item $\dfrac{\dd r}{\dd \varphi }=ma\e ^{m\varphi }$, 代入 (*) 式得
              $$\dfrac{\dd y}{\dd x}=\dfrac{ma\e ^{m\varphi }\sin \varphi +a\e ^{m\varphi }\cos \varphi }{ma\e ^{m\varphi }\cos \varphi -a\e ^{m\varphi }\sin \varphi }=\dfrac{m\sin \varphi +\cos \varphi }{m\cos \varphi -\sin \varphi }=\tan \left( \varphi +\arctan \dfrac{1}{m}\right) .$$
    \end{enumerate}
\end{solution}

\subsection{高阶导数与 Leibniz 公式}

\subsubsection{先拆项再求导}

%有些式子不易直接求高阶导数, 当拆项以后, 变成易于求高阶导数的一些基本形式之和, 便立即可以直接求导.

常见函数的高阶导数基本形式主要有:

\setcounter{magicrownumbers}{0}
\begin{table}[H]
    \centering
    \caption{初等函数的高阶导数}
    \begin{tabular}{l l}
        (\rownumber{}) $\displaystyle (x^k)^{(n)}=k(k-1)\cdots(k-n+1)x^{k-n}~  (n\leqslant k).$    & (\rownumber{}) $\displaystyle \left(\frac{1}{ax+b}\right)^{(n)}=(-1)^n\frac{n!a^n}{(ax+b)^{n+1}}$ \\
        \midrule
        (\rownumber{}) $\displaystyle(\e ^x)^{(n)}=\e ^x.$                                         & (\rownumber{}) $\displaystyle (a^x)^{(n)}=a^x\ln^n a.$                                            \\
        (\rownumber{}) $\displaystyle(\ln x)^{(n)}=(-1)^{n-1}(n-1)!x^{-n}.$                        & (\rownumber{}) $\displaystyle(\log_ax)^{(n)}=(-1)^{n-1}\frac{(n-1)!}{x^n\ln a}$                   \\
        \midrule
        (\rownumber{}) $\displaystyle[\sin (kx+b)]^{(n)}=k^n\sin\left(kx+b+\frac{n\pi}{2}\right).$ & (\rownumber{}) $\displaystyle[\cos (kx+b)]^{(n)}=k^n\cos\left(kx+b+\frac{n\pi}{2}\right).$
    \end{tabular}
\end{table}
并特别注意 $[f(ax+b)]^{(n)}=a^nf^{(n)}(ax+b)$, 因子 $a^n$ 不要漏掉.

\begin{example}
    求下列 $y^{(n)}$, $n$ 足够的大.
    \setcounter{magicrownumbers}{0}
    \begin{table}[H]
        \centering
        \begin{tabular}{l | l | l | l}
            (\rownumber{}) $\displaystyle y=\frac{1}{x(1-x)}.$      & (\rownumber{}) $\displaystyle y=\frac{x^4}{x-1}.$ & (\rownumber{}) $\displaystyle y=\frac{x^2+x+1}{x^2-5x+6}.$ & (\rownumber{}) $y=\dfrac{1}{x^2-3x+2}$                    \\
            (\rownumber{}) $\displaystyle y=\sin^2x.$               & (\rownumber{}) $y=\cos^3x.$                       & (\rownumber{}) $y=\cos ax\cos bx.$                         & (\rownumber{}) $y=\sin ax\cos bx.$                        \\
            (\rownumber{}) $\displaystyle y=\frac{1}{\sqrt{1-2x}}.$ & (\rownumber{}) $y=\dfrac{1}{\sqrt{3x+2}}$         & (\rownumber{}) $\displaystyle y=\frac{x}{\sqrt{1-x}}.$     & (\rownumber{}) $\displaystyle y=\frac{x}{\sqrt[3]{1+x}}.$
        \end{tabular}
    \end{table}
\end{example}
\begin{solution}
    \begin{enumerate}[label=(\arabic{*})]
        \item 因为 $\displaystyle y=\frac{1}{x}+\frac{1}{1-x}$, 所以 $\displaystyle y^{(n)}=(-1)^n\frac{n!}{x^{n+1}}+\frac{n!}{(1-x)^{n+1}},~x\not=0,1.$
        \item 因为 $\displaystyle y=(x^2+1)(x+1)+\frac{1}{x-1}$, 所以 $\displaystyle y^{(n)}=(-1)^n\frac{n!}{(x-1)^{n+1}},~n\geqslant 4.$
        \item 因为 $\displaystyle y=1-\frac{7}{x-2}+\frac{13}{x-3}$, 所以 $\displaystyle y^{(n)}=7\cdot\frac{n!}{(2-x)^{n+1}}+(-1)^n\cdot13\cdot\frac{n!}{(x-3)^{n+1}},~x\not=2,3.$
        \item 因为 $y=\dfrac{1}{x-2}-\dfrac{1}{x-1}$, 所以 $y^{(n) }=(-1) ^{n}n!\left[ \dfrac{1}{\left( x-2\right) ^{n+1}}-\dfrac{1}{\left( x-1\right) ^{n+1}}\right],~ x\neq  1,2. $
        \item 因为 $\displaystyle y=\frac{1-\cos2x}{2}$, 所以 $\displaystyle y^{(n)}=-2^{n-1}\cos\left(2x+\frac{n\pi}{2}\right).$
        \item 由三倍角公式 $\cos 3\alpha=4\cos^3\alpha-3\cos\alpha$, 所以 $\displaystyle y^{(n)}=\frac{3^n}{4}\cos\left(3x+\frac{n\pi}{2}\right)+\frac{3}{4}\cos\left(x+\frac{n\pi}{2}\right).$
        \item 因为 $y=\dfrac{1}{2}\cos (a-b)  x+\dfrac{1}{2}\cos \left( a+b\right) x$, 所以
              $$y^{(n)} =\dfrac{(a-b)  ^{n}}{2}\cos \left[ (a-b)  x+\dfrac{n\pi }{2}\right] +\dfrac{1}{2}(a-b)  ^{n}\cos \left[ \left( a+b\right) x+\dfrac{n\pi }{2}\right] .$$
        \item 由积化和差公式得 $\displaystyle y=\frac{1}{2}\sin(a+b)x+\frac{1}{2}\sin(a-b)x$, 所以 $$ y^{(n)}=\frac{(a+b)^n}{2}\sin\left[(a+b)x+\frac{n\pi}{2}\right]+\frac{(a-b)^n}{2}\sin\left[(a-b)x+\frac{n\pi}{2}\right].$$
        \item $\displaystyle y^{(n)}=\left(-\frac{1}{2}\right)\left(-\frac{3}{2}\right)\cdots\left(-\frac{2n-1}{2}\right)\cdot(-2)^n\cdot(1-2x)^{-\frac{2n+1}{2}}=\frac{(2n-1)!!}{(1-2x)^{n+\frac{1}{2}}}.$
        \item $y^{(n)} =\left( -\dfrac{1}{2}\right) \left( -\dfrac{3}{2}\right) \ldots \left( -\dfrac{2n-1}{2}\right) \cdot 3^{n}\cdot \left( 3x+2\right) ^{-\frac{2n+1}{2}}=\left( -\dfrac{3}{2}\right) ^{n}\cdot\dfrac{\left( 2n-1\right) !!}{\left( 3x+2\right) ^{\frac{2}{n+1}}}.$
        \item 因为 $\displaystyle y=\frac{1}{\sqrt{1-x}}-\sqrt{1-x}$, 所以 $\displaystyle y^{(n)}=\frac{(2n-1)!!}{2^n}(1-x)^{-\frac{2n+1}{2}}+\frac{(2n-3)!!}{2^n}(1-x)^{-\frac{2n-1}{2}}.$
        \item 因为 $\displaystyle y=\frac{(x+1)-1}{\sqrt[3]{1+x}}=(1+x)^{\frac{2}{3}}-(1+x)^{-\frac{1}{3}}$, 所以当 $n\geqslant 2,x\not=-1$ 时,
              \begin{flalign*}
                  y^{(n)} & =\frac{2}{3}\left(-\frac{1}{3}\right)\left(-\frac{4}{3}\right)\cdots\left(-\frac{3n-5}{3}\right)(1+x)^{-\frac{3n-2}{3}}+\frac{1}{3}\left(-\frac{4}{3}\right)\cdots\left(-\frac{3n-2}{3}\right)(1+x)^{-\frac{3n+1}{3}} \\
                          & =\frac{(-1)^{n+1}\cdot1\cdot4\cdots(3n-5)}{3^n(1+x)^{n+\frac{1}{3}}}[2(1+x)+(3n-2)]=\frac{(-1)^{n+1}1\cdot4\cdot7\cdots(3n-5)(3n+2x)}{3^n(1+x)^{n+\frac{1}{3}}}.
              \end{flalign*}
    \end{enumerate}
\end{solution}

\subsubsection{Leibniz 公式}

把要求导的函数写成两项相乘, 然后直接应用 Leibniz 公式:
$$(u\cdot v)^{(n)}=\sum_{k=0}^{n}\mathrm{C}_n^ku^{(k)}v^{(n-k)}.$$

% \begin{inference}
%     (广义 Leibniz 公式) 设 $u_i(x),~i=1,2,\cdots,m$ 是 $x$ 的 $n$ 次可微函数, 则有
%     $$\left(\prod_{i=1}^{m}u_i\right)^{(n)}=\sum_{r_1+r_2+\cdots+r_m=n}n!\prod_{i=1}^{m}\frac{u_i^{(r_i)}}{r_i!}.$$
% \end{inference}
\begin{example}
    求下列 $y^{(n)}$.
    \setcounter{magicrownumbers}{0}
    \begin{table}[H]
        \centering
        \begin{tabular}{l | l | l | l}
            (\rownumber{}) $y=x\cos ax.$           & (\rownumber{}) $y=x^2\sin ax.$       & (\rownumber{}) $y=\sin^2ax\cos bx.$ & (\rownumber{}) $y=\dfrac{\sin(2x+1)}{4x-3}.$ \\
            (\rownumber{}) $y=(x^2+2x+2)\e ^{-x}.$ & (\rownumber{}) $y=\dfrac{\e ^x}{x}.$ & (\rownumber{}) $y=x^2\e ^{ax}.$     & (\rownumber{}) $y=x\ln x.$
        \end{tabular}
    \end{table}
\end{example}
\begin{solution}
    \begin{enumerate}[label=(\arabic{*})]
        \item $y^{(n)}=x(\cos ax)^{(n)}+n(\cos ax)^{(n-1)}=a^nx\cos\left(ax+\dfrac{n\pi}{2}\right)+na^{n-1}\sin\left(ax+\dfrac{n\pi}{2}\right).$
        \item $y^{(n)}=a^nx^2\sin\left(ax+\dfrac{n\pi}{2}\right)+2na^{n-1}x\sin\left(ax+\dfrac{n-1}{2}\pi\right)+n(n-1)a^{n-1}\sin\left(ax+\dfrac{n-2}{2}\pi\right).$
        \item 将 $\sin^2ax$ 降幂, 再用积化和差公式, 得 $y=\dfrac{1}{2}\cos bx-\dfrac{1}{4}\cos(2a+b)x-\dfrac{1}{4}\cos(2a-b)x$, 于是
              $$y^{(n)}=\frac{1}{2}b^2cos\left(bx+\frac{n\pi}{2}\right)-\frac{1}{4}(2a+b)^n\cos\left[(2a+b)x+\frac{n\pi}{2}\right]-\frac{1}{4}(2a-b)^n\cos\left[(2a-b)x+\frac{n\pi}{2}\right].$$
        \item $y^{(n)}=\displaystyle \sum_{k=0}^{n}\C_n^k(-1)^{n-k}\dfrac{(n-k)!\cdot 2^{3n-2k}}{(4x-3)^{n-k+1}}\sin\qty(2x+1+\dfrac{n\pi}{2})$.
        \item $y^{(n)}=(-1)^n\e ^{-x}\left[x^2-2(n-1)x+(n-1)(n-2)\right].$
        \item $\displaystyle y^{(n) }=\sum ^{n}_{k=0}\mathrm{C}_{n}^{k}\e ^{x}\left( \dfrac{1}{x}\right) ^{\left( k\right) }=\e ^{x}\left[ \dfrac{1}{x}+\sum ^{n}_{k=1}(-1) ^{k}\dfrac{n\left( n-1\right) \ldots \left( n-k+1\right) }{x^{k+1}}\right] .$
        \item $\displaystyle y^{(n)} =a^{n}x^{2}\e ^{ax}+2na^{n-1}x\e ^{ax}+n\left( n-1\right) a^{n-2}\e ^{ax}.$
        \item $\displaystyle y^{(n) }=\begin{cases}
                      (-1)^{n-2}(n-2)!x^{1-n}, & n\geqslant 2 \\ \ln x+1,&n=1.
                  \end{cases} $
    \end{enumerate}
\end{solution}

\begin{example}
    设函数 $y=\qty(x^2-11x+30)^{2022}\cdot\cos\dfrac{\pi x}{18}$, 求 $y^{(2022)}(6).$
\end{example}
\begin{solution}
    $y=\qty(x^2-11x+30)^{2022}\cdot\cos\dfrac{\pi x}{18}=(x-6)^{2022}(x-5)^{2022}\cos\dfrac{\pi x}{18}$, 由 Leibniz 计算公式, 可知对 $x=6$ 处, 仅仅有一项不为 $0$, 即
    \begin{flalign*}
        y^{(2022)}(6)=\mathrm{C}_{2022}^{2022}\cdot2022!\qty[(x-5)^{2022}\cos\dfrac{\pi x}{18}]_{x=6}=\dfrac{2022!}{2}.
    \end{flalign*}
\end{solution}

% \begin{example}
%     试证 $\displaystyle (a^2+b^2)^{\frac{n}{2}}\e ^{ax}\sin(bx+n\varphi)=\sum_{i=0}^{n}\mathrm{C}_n^ia^{n-i}b^i\e ^{ax}\sin\left(bx+\frac{\pi}{2}i\right)$, 其中 $\displaystyle \varphi=\arctan \frac{b}{a}.$
% \end{example}
% \begin{proof}[{\songti \textbf{证}}]
%     令 $f(x)=\e ^{ax}\sin bx$, 则
%     $$f'(x)=\e ^{ax}(a\sin bx+b\cos bx)=\e ^{ax}(a^2+b^2)^{1/2}\sin(bx+\varphi)~  \left(\varphi=\arctan\frac{b}{a}\right)$$
%     反复求导, 得 $f^{(n)}(x)=(a^2+b^2)^{n/2}\e ^{ax}\sin(bx+n\varphi)$, 又有 Leibniz 公式, 
%     $$f^{(n)}=\sum_{i=0}^{n}\mathrm{C}_n^ia^{n-i}b^i\e ^{ax}\sin\left(bx+\frac{\pi}{2}i\right)$$
%     即得所求的等式.
% \end{proof}

\subsubsection{用数学归纳法求高阶导数}

当高阶导数不能一次求出时, 可先求出前几阶导数, 归纳总结, 找出规律, 然后用数学归纳法加以证明.

\begin{example}
    证明: $\displaystyle \left(x^{n-1}\e ^{\frac{1}{x}}\right)^{(n)}=\frac{(-1)^n}{x^{n+1}}\e ^{\frac{1}{x}}.$
\end{example}
\begin{proof}[{\songti \textbf{证}}]
    当 $n=1$ 时, 由于 $\left(\e ^{\frac{1}{x}}\right)'=-\dfrac{1}{x^2}\e ^{\frac{1}{x}}$, 故等式成立, \\
    设当 $n=k$ 时等式成立, 即有 $\left(x^{k-1}\e ^{\frac{1}{x}}\right)^{(k)}=\dfrac{(-1)^k}{x^{k+1}}\e ^{\frac{1}{x}}$, 要证等式对 $n=k+1$ 时也成立,
    \begin{flalign*}
        \left(x^k\e ^{\frac{1}{x}}\right)^{(k+1)} & =\left[\left(x\cdot x^{k-1}\e ^{\frac{1}{x}}\right)^{(k)}\right]'=\left[x\left(x^{k-1}\e ^{\frac{1}{x}}\right)^{(k)}+k\left(x^{k-1}\e ^{\frac{1}{x}}\right)^{(k-1)}\right]'                  \\
                                                  & =x\left(x^{k-1} \e ^{\frac{1}{x}}\right)^{(k+1)}+\left(x^{k-1} \e ^{\frac{1}{x}}\right)^{(k)}+k\left(x^{k-1} \e ^{\frac{1}{x}}\right)^{(k)}                                                  \\
                                                  & =x\left[\frac{(-1)^{k}}{x^{k+1}} \e ^{\frac{1}{x}}\right]^{\prime}+(k+1) \frac{(-1)^{k}}{x^{k+1}} \e ^{\frac{1}{x}}                                                                          \\
                                                  & =\frac{(-1)^{k+1}(k+1)}{x^{k+1}} \e ^{\frac{1}{x}}+\frac{(-1)^{k+1}}{x^{k+2}} \e ^{\frac{1}{x}}+\frac{(-1)^{k}(k+1)}{x^{k+1}} \e ^{\frac{1}{x}}=\frac{(-1)^{k+1}}{x^{k+2}} \e ^{\frac{1}{x}}
    \end{flalign*}
    于是, 由数学归纳法得知, $\displaystyle \left(x^{n-1} \e ^{\frac{1}{x}}\right)^{(n)}=\frac{(-1)^{n}}{x^{n+1}} \e ^{\frac{1}{x}}$ 对于一切正整数 $n$ 均成立.
\end{proof}

\subsubsection{用递推公式求导}

\begin{example}
    设 $f(x)=x^n\cdot\ln x$, 求 $f^{(n)}(x).$
\end{example}
\begin{solution}
    令 $f_n(x)=x^n\cdot\ln x$, 则 \newline
    \begin{minipage}{0.72\linewidth}
        $$f'_n(x)=nx^{n-1}\cdot\ln x+x^{n-1}=n\cdot f_{n-1}(x)+x^{n-1}$$
        对上式两边同时除以 $n!$, 则有 $\dfrac{f'_n(x)}{n!}=\dfrac{f_{n-1}(x)}{(n-1)!}+\dfrac{x^{n-1}}{n!}$, 再同时求 $(n-1)$ 阶导, 即
        $$\dfrac{f_n^{(n)}(x)}{n!}=\dfrac{f_{n-1}^{(n-1)}(x)}{(n-1)!}+\dfrac{1}{n}:=u_n-u_{n-1}=\dfrac{1}{n}$$
        于是 $u_n=\ln x+\displaystyle\sum_{k=1}^{n}\dfrac{1}{k}$, 所以 $f_n^{(n)}(x)=n!\qty(\ln x+1+\dfrac{1}{2}+\cdots+\dfrac{1}{n}).$
    \end{minipage}\hfill
    \begin{minipage}{0.27\linewidth}
        $\begin{NiceArray}{c@{\;}c@{\;}c@{\;}c@{\;}c}[create-medium-nodes]
                u_1           & - & u_0           & =              & \dfrac{1}{1}                            \\[6pt]
                u_2           & - & u_1           & =              & \dfrac{1}{2}                            \\[6pt]
                u_3           & - & u_2           & =              & \dfrac{1}{3}                            \\[6pt]
                u_4           & - & u_3           & =              & \dfrac{1}{4}                            \\[6pt]
                \phantom{u_5} &   & \phantom{u_4} & \smash{\vdots} & \phantom{\dfrac{1}{5}}                  \\[6pt]
                u_n           & - & u_{n-1}       & =              & \dfrac{1}{n}                            \\[6pt]
                \hline
                u_n           & - & u_0           & =              & \displaystyle\sum_{k=1}^{n}\dfrac{1}{k} \\
                \CodeAfter
                \tikz[very thick, red, opacity=0.4,name suffix = -medium]
                \draw (1-1.north west) -- (2-3.south east)
                (2-1.north west) -- (3-3.south east)
                (3-1.north west) -- (4-3.south east)
                (4-1.north west) -- (5-3.south east)
                (5-1.north west) -- (6-3.south east) ;
            \end{NiceArray}$
    \end{minipage}
\end{solution}

\begin{example}[2023 四川大学]
    设 $y=\arcsin x$, 求 $\dfrac{y^{(7)}(0)}{7!}.$
\end{example}
\begin{solution}
    直接求导, 有 $$y'=\dfrac{1}{\sqrt{1-x^2}},~y''=\dfrac{x}{1-x^2}y'$$
    则 $\qty(1-x^2)y''=xy'$, 对其两边求 $n$ 阶导数, 利用 Leibniz 公式得
    $$\qty(1-x^2)y^{(n+2)}+n(-2x)y^{(n+1)}+\dfrac{n(n-1)}{2}(-2)y^{(n)}=xy^{(n+1)}+ny^{(n)}$$
    整理得 $$\qty(1-x^2)y^{(n+2)}-(2n+1)xy^{(n+1)}-n^2y^{(n)}=0$$
    令 $x=0$, 可得递推公式可知: $y^{(n+2)}(0)=n^2y^{(n)}(0)$, 又因为 $y'(0)=y''(0)=0$, 则
    $$y^{(n)}(0)=\begin{cases}
            0, & x=2k \\\qty[(2k-1)!!]^2,&n=2k+1
        \end{cases},k\in\mathbb{N}_+ $$
    所以 $\dfrac{y^{(7)}(0)}{7!}=\dfrac{(5!!)^2}{7!}=\dfrac{5}{112}.$
\end{solution}

\begin{example}
    设 $f(x)=(\arcsin x)^2$, 求 $f^{(n)}(0).$
\end{example}
\begin{solution}
    由 $f'(x)=2\dfrac{\arcsin x}{\sqrt{1-x^2}}$, 得
    $$(1-x^2)\qty[f'(x)]^2=4f(x)$$ 再次求导, 整理得 $$-xf'(x)+(1-x^2)f''(x)=2$$
    由 Leibniz 公式得
    $$-xf^{(n+1)}(x)-nf^{(n)}(x)+(1-x^2)f^{(n+2)}(x)-2nxf^{(n+1)}(x)-n(n-1)f^{(n)}(x)=0$$
    令 $x=0$, 得 $$f'(0) =0,f''(0) =2,f^{(n+2) }(0) =n^{2}f^{(n) }(0) $$
    从而
    \begin{flalign*}
        f^{(2k+1)}(0)            & =0,~k=0,1,\cdots,                                                                                             \\
        f^{\left( 2k\right) }(0) & =\left( 2k-2\right) ^{2}\left( 2k-4\right) ^{2}\cdots 2^{2}\cdot2=2\prod ^{k-1}_{i=1}\left( 2k-2i\right) ^{2} \\
                                 & =2^{2(k-1) }\cdot 2\prod ^{k-1}_{i=1}(k-i) ^{2}=2^{2k-1}\left[ \prod ^{k-1}_{i=1}(k-i) \right] ^{2}           \\
                                 & =2^{2k-1}\left[ (k-1) !\right] ^{2},~k=1,2,\cdots.
    \end{flalign*}
\end{solution}

\subsubsection{用 Taylor 展开式求导}

\begin{theorem}[Taylor 展开式]
    $f(x)$ 按 $(x-a)$ 的幂展开的幂级数必是 $f(x)$ 的 Taylor 展开式:
    $$f(x) =\sum ^{\infty }_{n=0}\dfrac{f^{(n) }(a) }{n!}(x-a) ^{n}$$
    因此, 若一旦得到展开式 $\displaystyle f(x)=\sum_{n=0}^{\infty}a_n(x-a)^n$, 则 $f^{(n)}(a)=a_nn!~  (n=0,1,2,\cdots).$
    \index{Taylor 展开式}
\end{theorem}

\begin{example}
    设 $f(x)=(x-1)^n\cdot x^{2n}\cdot\sin\dfrac{\pi x}{2}$, 求 $f^{(n)}(1).$
\end{example}
\begin{solution}
    当 $x\to1$ 时, $\displaystyle\lim_{x\to1}\dfrac{f(x)}{(x-1)^n}=\lim_{x\to1}\qty(x^{2n}\cdot\sin\dfrac{\pi x}{2})=1$, 将 $f(x)$ 在 $x=1$ 处 Taylor 展开, 有
    $$\lim_{x\to1}\dfrac{f(1)+f'(1)(x-1)+\cdots+\dfrac{f^{(n)}(1)}{n!}(x-1)^n+o\qty((x-1)^{n})}{(x-1)^n}=1$$
    即 $$f(1)=f'(1)=\cdots=f^{(n-1)}(1)=0,~\dfrac{f^{(n)}(1)}{n!}=1$$
    故 $f^{(n)}(1)=n!.$
\end{solution}

\begin{example}
    设 $f(x)=\arctan\dfrac{1+x}{1-x}$, $n$ 为正整数, 求 $f^{(2n+1)}(0)$.
\end{example}
\begin{solution}
    因为 $f'(x)=\dfrac{1}{1+\qty(\dfrac{1+x}{1-x})^2}\cdot\qty(\dfrac{1+x}{1-x})'=\dfrac{1}{1+x^2}=\displaystyle \sum_{n=0}^{\infty} (-1)^{n}x^{2n}$, 所以 
    $$
    f(x)=\sum_{n=0}^{\infty} (-1)^{n}\int_{0}^{x} t^{2n} \dd t=\sum_{n=0}^{\infty} (-1)^{n}\dfrac{x^{2n+1}}{2n+1}
    $$
    由 Taylor 展开的唯一性, $f^{(2n+1)}(0)=(2n+1)!\cdot\dfrac{(-1)^{n}}{2n+1}=(-1)^{n}(2n)!.$
\end{solution}

\begin{example}
    设 $y=\arctan x$, 求 $y^{(n)}(0)$.
    \label{arctan x y(n)(0)}
\end{example}
\begin{solution}
    \textbf{法一: }因为
    $$\arctan x=\int _{0}^{x}\dfrac{\dd t}{1+t^{2}}=\int _{0}^{x}\left[ 1-t^{2}+t^{4}-\cdots +(-1) ^{n}t^{2n}\right] \dd t=x-\dfrac{x^{3}}{3}+\dfrac{x^{5}}{5}-\cdots ~  ( -1 <x < 1) $$
    所以由幂级数展开式的唯一性, 得 $$y^{(2n)  }(0) =0,~y^{(2n+1) }(0) =(2n+1) !\dfrac{(-1) ^{n}}{2n+1}=(-1) ^{n}(2n)  !$$
    \textbf{法二: }由 $y=\arctan x$, 则 $y'=\dfrac{1}{1+x^2}$, 即 $y'(1+x^2)=1$, 等式两端对 $x$ 求 $n+1$ 阶导数, 由 Leibniz 公式得
    $$\left( 1+x^{2}\right) y^{(n+2) }+(n+1) y^{(n+1) }\left( 1+x^{2}\right) '+\dfrac{n(n+1) }{2!}y^{(n) }\left( 1+x^{2}\right) ''=0$$
    于是得 $\left( 1+x^{2}\right) y^{(n+2) }+2(n+1) xy^{(n+1) }+n(n+1) y^{(n) }=0$, 令 $x=0$, 得
    $$y^{(n+2) }(0) =-n(n+1) y^{(n) }(0) $$
    又 $y'(0)=1,y''(0)=0$, 故 $y^{(2m)}(0)=0,y^{(2m+1)}(0)=(-1)^m(2m).$\\
    \textbf{法三: }转化为 $x=\tan y$, 于是
    $$y'=\dfrac{1}{1+x^{2}}=\dfrac{1}{1+\tan ^{2}y}=\cos ^{2}y=\cos y\cdot \sin \left( y+\dfrac{\pi }{2}\right) $$
    两边关于 $x$ 求导, 得
    \begin{flalign*}
        y'' & =-\sin y\cdot y'\cdot \sin \left( y+\dfrac{\pi }{2}\right) +\cos y\cdot \cos \left( x+\dfrac{\pi }{2}\right) \cdot y'
        =y'\cdot \left[ -\sin y\cdot \sin \left( y+\dfrac{\pi }{2}\right) +\cos y\cdot \cos \left( y+\dfrac{\pi }{2}\right) \right]                                        \\
            & =y'\cos \left( 2y+\dfrac{\pi }{2}\right) =\cos ^{2}y\cdot \sin \left( 2y+2\cdot \dfrac{\pi }{2}\right) =\cos ^{2}y\cdot \sin \left( y+\dfrac{\pi }{2}\right)
    \end{flalign*}
    由数学归纳法可证得 $$y^{(n) }=\left( n-1\right) !\cdot \cos ^{n}y\cdot \sin n\left( y+\dfrac{\pi }{2}\right) $$
    当 $x=0$ 时 $y=0$, 于是
    $$y^{(n) }(0) =\left( n-1\right) !\cdot \sin \dfrac{n\pi }{2}=\begin{cases}0                                       , & n=2m   \\
             (-1) ^{m}\left( 2m\right) ! ,             & n=2m+1\end{cases}\text{其中 }m=0,1,\cdots.$$
    \textbf{法四: }由 $y'=\dfrac{1}{1+x^{2}}=\dfrac{1}{2\mathrm{i}}\left( \dfrac{1}{x-\mathrm{i}}-\dfrac{1}{x+\mathrm{i}}\right) ,x\pm \mathrm{i}=re^{\pm \mathrm{i}\theta }$, 其中 $r=\sqrt{1+x^2},\theta=\arccot x$, 所以
    \begin{flalign*}
        y^{(n+1) } & =\left( \dfrac{1}{1+x^{2}}\right) ^{(n) }=\dfrac{1}{2\mathrm{i}}\left[ \left( \dfrac{1}{x-\mathrm{i}}\right) ^{(n) }-\left( \dfrac{1}{x+\mathrm{i}}\right) ^{(n) }\right] =\dfrac{1}{2\mathrm{i}}\left[ \dfrac{(-1) ^{n}n!}{\left( x-\mathrm{i}\right) ^{n+1}}-\dfrac{(-1) ^{n}n!}{\left( x+\mathrm{i}\right) ^{n+1}}\right] \\
                   & =\dfrac{(-1) ^{n}n!}{2\mathrm{i}\left( 1+x^{2}\right) ^{n+1}}\cdot \left[ \left( x+\mathrm{i}\right) ^{n+1}-\left( x-\mathrm{i}\right) ^{n+1}\right] =\dfrac{(-1) ^{n}n!}{2\mathrm{i}\left( 1+x^{2}\right) ^{n+1}}\cdot \left[ r^{n+1}\e ^{\mathrm{i}(n+1) \theta }-r^{n+1}\e ^{-\mathrm{i}(n+1) \theta }\right]             \\
                   & =\dfrac{(-1) ^{n}n!}{\left( 1+x^{2}\right) ^{n+1}}r^{n+1}\sin (n+1) \theta =\dfrac{(-1) ^{n}n!}{\left( 1+x^{2}\right) ^{\frac{n+1}{2}}}\sin \left[ (n+1) \cdot \arccot x\right]
    \end{flalign*}
    于是 $y'(0) =1,~y(n+1) (0) =(-1) ^{n}n!\sin \dfrac{(n+1) }{2}\pi ,~n=1,2,\ldots $, 故当 $n=2m$ 时,
    $$y^{\left( 2m+1\right) }(0) =(-1) ^{2m}\left( 2m\right) !\sin \dfrac{2m+1}{2}\pi =(-1) ^{m}\left( 2m\right) !$$
    当 $n=2m-1$ 时, $y^{(2m)}(0)=0.$
\end{solution}

\begin{example}
    设 $f(x)=x^{2021}\e ^{2020x}\sin x$, 求 $f^{(2023)}(0)$.
\end{example}
\begin{solution}
    \textbf{法一: }令 $h(x)=x^{2021},~g(x)=\e ^{2020x}\sin x$, 根据 Leibniz 求导公式,
    \begin{flalign*}
        f^{(2023)}(x) & =\sum_{k=0}^{2023}\mathrm{C}_{2023}^kh^{(2023-k)}(x)g^{(k)}(x)=\sum_{k=0}^{2023}\mathrm{C}_{2023}^k\qty(x^{2021})^{(2023-k)}\qty(\e ^{2020x}\sin x)^{(k)} \\
                      & =\sum_{k=0}^{2023}\mathrm{C}_{2023}^k\qty(x^{2021})^{(2023-k)}\qty[\sum_{m=0}^{k}\mathrm{C}_k^m\qty(\e ^{2020x})^{(m-k)}\sin^{(m)}x]                      \\
                      & =\sum_{k=0}^{2023}\mathrm{C}_{2023}^k\dfrac{2021!}{(k-2)!}x^{k-2}\qty[\sum_{m=0}^{k}\mathrm{C}_k^m (2020)^{k-m}\e ^{2020x}\sin\qty(x+\dfrac{m\pi}{2})]
    \end{flalign*}
    当 $x=0$ 时,
    \begin{flalign*}
        f^{(2023)}(0)=\mathrm{C}_{2023}^2\dfrac{2021!}{(2-2)!}\qty[\sum_{m=0}^{2}\mathrm{C}_2^m(2020)^{2-m}\sin\qty(\dfrac{m\pi}{2})]=\dfrac{2023!}{2!\cdot2021!}2021!\cdot2!\cdot2020=2020\cdot2023!.
    \end{flalign*}
    \textbf{法二: }
    利用 Taylor 公式的唯一性,
    \begin{flalign*}
        f(x)  =x^{2021}\e ^{2020x}\sin x=x^{2021}\qty(1+2020x+1010x^2+o(x^2))\qty(x-\dfrac{x^3}{6}+o(x^3))
        =x^{2022}+2020x^{2023}+o(x^{2023})
    \end{flalign*}
    故 $f^{(2023)}(0)=2020\cdot2023!.$
\end{solution}

\begin{example}[首届长三角高校高等数学竞赛]
    设 $ \displaystyle f(x)=\prod_{k=1}^{2022} \tan kx $, 求 $\displaystyle f^{(2024)}(0).$
\end{example}
\begin{solution}
    因为 $ \displaystyle \tan x=x+\dfrac{1}{3}x^3+o\qty(x^3) $, 所以
    $$\displaystyle f(x)=\prod_{k=1}^{2022} \tan kx=\prod_{k=1}^{2022} \qty[kx+\dfrac{1}{3}k^3x^3+o\qty(x^3)]=2022! x^{2022}+\dfrac{1}{3}2022!x^{2024}\sum_{k=1}^{2022}k^2+o\qty(x^{2024})$$
    所以 $f^{(2024)}(0)=\dfrac{1}{3}\cdot 2022!\cdot 2024! \cdot\dfrac{2022\cdot 2023\cdot 4045}{6}.$
\end{solution}