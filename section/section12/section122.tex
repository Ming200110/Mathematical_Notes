\section{非齐次线性方程组}

\subsection{方程组与行列式}

\begin{example}
    已知线性方程组 $\left\{\begin{matrix}
            ax_1 & + &      &   & x_3  & = & 1 \\
            x_1  & + & ax_2 & + & x_3  & = & 0 \\
            x_1  & + & 2x_2 & + & ax_3 & = & 0 \\
            ax_1 & + & bx_2 &   &      & = & 2
        \end{matrix}\right.$ 有解, 其中 $a,b$ 为常数, 若 $\mqty|a&0&1\\1&a&1\\1&2&a|=4$, 求 $\mqty|1&a&1\\1&2&a\\a&b&0|.$
\end{example}
\begin{solution}
    设 $\vb*{x}$ 的系数矩阵为 $\vb*{A}$, $(1,0,0,2)^\top=\vb*{b}$, 因为方程组有解, 则 $\rank(\vb*{A},\vb*{b})=\rank(\vb*{A})$, 由于 $\mqty|a&0&1\\1&a&1\\1&2&a|=4\neq0$, 所以 $\rank\vb*{A}\geqslant 3$,
    又因为 $\vb*{A}_{4\times3}$, 所以 $\rank\vb*{A}\leqslant 3$, 于是 $$\rank\vb*{A}=3\Rightarrow \rank(\vb*{A},\vb*{b})=3\Rightarrow\det(\vb*{A},\vb*{b})=0$$
    对行列式 $\det(\vb*{A},\vb*{b})$ 按最后一列展开, 得
    $$\det(\vb*{A},\vb*{b})=-\mqty|1&a&1\\1&2&a\\a&b&0|+2\mqty|a&0&1\\1&a&1\\1&2&a|\Rightarrow\mqty|1&a&1\\1&2&a\\a&b&0|=8. $$
\end{solution}

\subsection{Cramer 法则的应用}

\begin{theorem}[Cramer 法则]\index{Cramer 法则}
    如果数域 $ K $ 上的含有 $ n $ 个末知量 $ n $ 个方程的线性方程组
    $$\begin{cases}
            a_{11} x_{1}+a_{12} x_{2}+\cdots+a_{1 n} x_{n}=b_{1}                  \\
            a_{21} x_{1}+a_{22} x_{2}+\cdots+a_{2 n} x_{n}=b_{2}                  \\
            \cdots \cdots \cdots \cdots \cdots \cdots \cdots \cdots \cdots \cdots \\
            a_{n 1} x_{1}+a_{n 2} x_{2}+\cdots+a_{n n} x_{n}=b_{n}
        \end{cases}$$
    的系数行列式 $ D \neq 0$, 那么此方程组有唯一解 $ \left(\dfrac{D_{1}}{D}, \dfrac{D_{2}}{D}, \cdots, \dfrac{D_{n}}{D}\right)$,
    其中 $ D_{j} $ 是把 $ D $ 中第 $ j $ 列的元素换成方程组的常数项 $ b_{1}, b_{2}, \cdots, b_{n} $ 而成的行列式 $ (j=1,2, \cdots, n)$, 即
    $$\mqty|a_{11}  & \cdots & a_{1, j-1} & b_{1}  & a_{1, j+1} & \cdots & a_{1 n} \\
        a_{21}  & \cdots & a_{2, j-1} & b_{2}  & a_{2, j+1} & \cdots & a_{2 n} \\
        \vdots  &        & \vdots     & \vdots & \vdots     &        & \vdots  \\
        a_{n 1} & \cdots & a_{n, j-1} & b_{n}  & a_{n, j+1} & \cdots & a_{n n}|$$
    特别地, 当 $ b_{i}=0(i=1,2, \cdots, n) $ 时, 如果系数行列式 $ D \neq 0 $, 那么方程组只有零解;
    反之, 如果方程组有非零解,那么必有 $ D=0 .$
\end{theorem}

\begin{example}[2005 华中科技大学]
    解线性方程组 $\begin{cases}
            x_1+ax_2+a^2x_3=a^3 \\
            x_1+bx_2+b^2x_3=b^3 \\
            x_1+cx_2+c^2x_3=c^3 \\
        \end{cases}$
    其中 $a,b,c$ 是互不相等的常数.
\end{example}
\begin{solution}
    利用 Cramer 法则求解, 因为系数行列式 $$D=\mqty|1&a&a^2\\1&b&b^2\\1&c&c^2|=(c-b)(c-a)(b-a)\neq0$$
    所以方程组有唯一解, 又因为
    $$D_1=\mqty|a^3&a&a^2\\b^3&b&b^2\\b^3&b&b^2|=abcD,~D_2=\mqty|1&a^3&a^2\\1&b^3&b^2\\1&c^3&c^2|=-(ab+ac+bc)D,~D_3=\mqty|1&a&a^3\\1&b&b^3\\1&c&c^3|=(a+b+c)D$$
    因此 $x_1=\dfrac{D_1}{D}=abc,~x_2=\dfrac{D_2}{D}=-ab-ac-bc,~x_3=\dfrac{D_3}{D}=a+b+c.$
\end{solution}

\subsection{线性相关与线性无关}

\subsubsection{线性无关}

\begin{example}[2009 数一]
    设 $\vb*{A}=\mqty(1&-1&-1\\-1&1&1\\0&-4&-2),~\vb*{\xi}_1=\mqty(-1\\1\\-2)$,
    \begin{enumerate}[label=(\arabic{*})]
        \item 求满足 $\vb*{A\xi}_2=\vb*{\xi}_1,~\vb*{A}^2\vb*{\xi}_3=\vb*{\xi}_1$ 的所有向量 $\vb*{\xi}_2,~\vb*{\xi}_3$;
        \item 对 (1) 中的任意向量 $\vb*{\xi}_2,~\vb*{\xi}_3$, 证明: $\vb*{\xi}_1,\vb*{\xi}_2,\vb*{\xi}_3$ 线性无关.
    \end{enumerate}
\end{example}
\begin{solution}
    \begin{enumerate}[label=(\arabic{*})]
        \item 对矩阵 $(\vb*{A},\vb*{\xi}_1)$ 施行初等行变换, 得
              $$(\vb*{A},\vb*{\xi}_1)=\begin{pNiceArray}{ccc:c}
                      1  & -1 & -1 & -1 \\
                      -1 & 1  & 1  & 1  \\
                      0  & -4 & -2 & -2
                  \end{pNiceArray}\to\begin{pNiceArray}{ccc:c}
                      1 & 0 & -\dfrac{1}{2} & -\dfrac{1}{2} \\[6pt]
                      0 & 1 & \dfrac{1}{2}  & \dfrac{1}{2}  \\[6pt]
                      0 & 0 & 0             & 0
                  \end{pNiceArray}$$
              由此可解得 $\vb*{\xi}_2=\mqty(-\dfrac{1}{2}+\dfrac{k}{2},\dfrac{1}{2}-\dfrac{k}{2},k)^\top$, 其中 $k$ 为任意常数,
              再对矩阵 $(\vb*{A}^2,\vb*{\xi}_1)$ 施行初等行变换, 得
              $$(\vb*{A}^2,\vb*{\xi}_1)=\begin{pNiceArray}{ccc:c}
                      2  & 2  & 0 & -1 \\
                      -2 & -2 & 0 & 1  \\
                      4  & 4  & 0 & -2
                  \end{pNiceArray}\to\begin{pNiceArray}{ccc:c}
                      1 & 1 & 0 & -\dfrac{1}{2} \\[6pt]
                      0 & 0 & 0 & 0             \\
                      0 & 0 & 0 & 0
                  \end{pNiceArray}$$
              故可解得 $\vb*{\xi}_3=\mqty(-\dfrac{1}{2}-a,b)^\top$, 其中 $a,b$ 为任意常数.
        \item 由 (1) 的结果, 有
              \begin{flalign*}
                  \qty|\vb*{\xi}_1,\vb*{\xi}_2,\vb*{\xi}_3|=\mqty|-1 & \dfrac{k-1}{2} & -\dfrac{1}{2}-a \\[6pt]1&\dfrac{1-k}{2}&a\\[6pt]-2&k&b|\xlongequal{r_1+r_2}\mqty|0&0&-\dfrac{1}{2}\\[6pt]1&\dfrac{1-k}{2}&a\\[6pt]-2&k&b|=-\dfrac{1}{2}\neq0
              \end{flalign*}
              所以 $\vb*{\xi}_1,\vb*{\xi}_2,\vb*{\xi}_3$ 线性无关.
    \end{enumerate}
\end{solution}

\subsection{非齐次线性方程组解的讨论}

\begin{theorem}[非齐次线性方程组有解的条件]
    $n$ 元非齐次线性方程组 $\vb*{Ax}=\vb*{b}$ 解的情况:\index{非齐次线性方程组有解的条件}
    \begin{enumerate}[label=(\arabic{*})]
        \item 方程组 $\vb*{Ax}=\vb*{b}$ 有唯一解 $\Leftrightarrow \rank\vb*{A}=\rank\bar{\vb*{A}}=n$;
        \item 方程组 $\vb*{Ax}=\vb*{b}$ 有无穷多解 $\Leftrightarrow \rank\vb*{A}=\rank\bar{\vb*{A}}<n$;
        \item 方程组 $\vb*{Ax}=\vb*{b}$ 无解 $\Leftrightarrow \rank\vb*{A}\neq\rank\bar{\vb*{A}}$.
    \end{enumerate}
\end{theorem}

\subsubsection{方程组无解的充要条件}

\begin{example}[2000 数一]
    已知方程组 $\begin{pmatrix} 1 & 2 & 1 \\ 2 & 3 & a+2 \\ 1 & a & -2 \\\end{pmatrix}\mqty( x_1 \\ x_2  \\ x_n )=\mqty( 1 \\ 3  \\ 0 )$ 无解, 求 $a$.
\end{example}
\begin{solution}
    方程组无解的充要条件是 $\rank\vb*{A}\neq\rank\vb*{\bar{A}}$, 故应对增广矩阵作初等行变换, 由 
    $$
    \begin{pNiceArray}{ccc:c}
        1 & 2 & 1 & 1 \\ 2 & 3 & a+2 & 3 \\ 1 & a & -2 & 0
    \end{pNiceArray}\xrightarrow[r_3-r_1]{r_2-2r_1}\begin{pNiceArray}{ccc:c}
        1 & 2 & 1 & 1 \\ 0 & -1 & a & 1 \\ 0 & a-2 & -3 & -1
    \end{pNiceArray}\xrightarrow{r_3+(a-2)\times r_2}\begin{pNiceArray}{ccc:c}
        1 & 2 & 1 & 1 \\ 0 & -1 & a & 1 \\ 0 & 0 & a^2-2a-3 & a-3
    \end{pNiceArray}
    $$
    可知, 若 $a=-1$, 则 $\vb*{\bar{A}}\to\begin{pNiceArray}{ccc:c}
        1 & 2 & 1 & 1 \\ 0 & -1 & -1 & 1 \\ 0 & 0 & 0 & -4
    \end{pNiceArray}$, 于是 $\rank\vb*{A}=2, \rank\vb*{\bar{A}}=3$, 从而方程组无解, 因此 $a=-1.$
\end{solution}

% \subsubsection{无穷多解}

% \begin{example}
%     设方程组 $\begin{pmatrix} a & 1 & 1 \\ 1 & a & 1 \\ 1 & 1 & a \\\end{pmatrix}\mqty( x_1 \\ x_2 \\ x_n )=\mqty(1\\1\\-2)$ 有无穷多解, 求 $a$.
% \end{example}
% \begin{solution}
%     对增广矩阵 $\vb*{\bar{A}}$ 作初等行变换化为行最简单形:
    
% \end{solution}

\begin{example}[2016 南京大学]
    讨论当 $a,b$ 为何值时, 线性方程组
    $$\left\{\begin{array}{rrrrrrrrrrr}
            x_{1}   & + & x_{2}   & + & x_{3}   & + & x_{4}   & + & x_{5}   & = & 1 \\
            3 x_{1} & + & 2 x_{2} & + & x_{3}   & + & x_{4}   & - & 3x_{5}  & = & a \\
                    &   & x_{2}   & + & 2 x_{3} & + & 2 x_{4} & + & 6 x_{5} & = & 3 \\
            5 x_{1} & + & 4 x_{2} & + & 3 x_{3} & + & 3 x_{4} & - & x_{5}   & = & b
        \end{array}\right.$$
    有解? 无解? 并求出有解时的一般解.
\end{example}
\begin{solution}
    对方程组的增广矩阵施行初等行变换, 化为简化的行阶梯形矩阵:
    $$\begin{pNiceArray}{c:c}
            \vb*{A} & \vb*{B}
        \end{pNiceArray}\xrightarrow[r_1-r_3]{\substack{r_2-3r_1+r_3\\r_4-5r_1+r_3}}
        \begin{pNiceArray}{ccccc:c}
            1 & 0 & -1 & -1 & -5 & -2  \\
            0 & 1 & 2  & 2  & 6  & 3   \\
            0 & 0 & 0  & 0  & 0  & a   \\
            0 & 0 & 0  & 0  & 0  & b-2
        \end{pNiceArray}$$
    \begin{enumerate}[label=(\arabic{*})]
        \item 当 $a\neq0$ 或 $b\neq2$ 时, 方程组无解;
        \item 当 $a=0$ 且 $b=2$ 时, 方程组有解, 此时, 方程组的一般解为
              $$\vb*{x}=\mqty(-2\\3\\0\\0\\0)+k_1\mqty(1\\-2\\1\\0\\0)+k_2\mqty(1\\-2\\0\\1\\0)+k_3\mqty(5\\-6\\0\\0\\1)$$
              其中 $k_1,k_2,k_3$ 为任意常数.
    \end{enumerate}
\end{solution}

\subsubsection{方程组解的结构与通解}

\begin{example}
    已知 $\vb*{\xi}_1=(-9,1,2,11)^\top, \vb*{\xi}_2=(1,-5,13,0)^\top, \vb*{\xi}_3=(-7,-9,24,11)^\top$ 是方程组
    $$
    \begin{cases}
        a_1x_1+7x_2+a_3x_3+x_4=d_1\\ 3x_1+b_2x_2+2x_3+2x_4=d_2\\9x_1+4x_2+x_3+7x_4=2
    \end{cases}
    $$的解, 求方程组的通解.
\end{example}
\begin{solution}
    只要知道 $\rank\vb*{A}$, 算出 $n-\rank\vb*{A}$ 就知道解的结构, 因为矩阵 $\vb*{A}=\begin{pmatrix} a_1 & 7 & a_3 & 1\\ 3 & b_2 & 2 & 2\\ 9 & 4 & 1 & 7\\\end{pmatrix}$ 中有二阶子式 $\mqty|2&2\\1&7|\neq0$, 所以 $\rank\vb*{A}\geqslant 2$, 又 
    $$
    \vb*{xi}_1-\vb*{xi}_2=(-10,6,-11,11)^\top, \quad \vb*{xi}_1-\vb*{xi}_3=(-2,10,-22,0)^\top
    $$
    是齐次方程组 $\vb*{Ax}=\vb*{0}$ 的两个线性无关的解, 而有 $$4-\rank\vb*{A}\geqslant 2\Rightarrow \rank\vb*{A}\leqslant 2$$
    从而得 $\rank\vb*{A}=2$, 所以方程组的通解为 $\mqty(-9\\1\\2\\11)+k_1\mqty(-10\\6\\-11\\11)+k_2\mqty(-1\\5\\-11\\0),\forall k_1,k_2\in \mathbb{R}.$
\end{solution}

\begin{example}[2014 数一]
    设 $\vb*{A}=\mqty(1&-2&3&-4\\0&1&-1&1\\1&2&0&-3)$, $\vb*{E}$ 为 $3$ 阶矩阵,
    \begin{enumerate}[label=(\arabic{*})]
        \item 求方程组 $\vb*{Ax}=\vb*{0}$ 的一个基础解系;
        \item 求满足 $\vb*{AB}=\vb*{E}$ 的所有矩阵 $\vb*{B}$.
    \end{enumerate}
\end{example}
\begin{solution}
    \begin{enumerate}[label=(\arabic{*})]
        \item 对矩阵 $\vb*{A}$ 作初等行变换化为行最简单形, 有
        $$
        \vb*{A}=\mqty(1&-2&3&-4\\0&1&-1&1\\1&2&0&-3)\xrightarrow{r_3-r_1-4r_2}\mqty(1&-2&3&-4\\0&1&-1&1\\0&0&1&-3)\xrightarrow[r_1+2r_2-3r_3]{r_2+r_3}\mqty(1&0&0&1\\0&1&0&-2\\0&0&1&-3)
        $$
        因 $n-\rank\vb*{A}=4-3=1$, 令 $x_4=1$, 则有 $x_3=3, x_2=2, x_1=-1$, 因此基础解系为 $\vb*{\eta}=(-1,2,3,1)^\top.$
        \item $\vb*{AB}=\vb*{E}$ 中 $\vb*{B}$ 的列向量其实是三个非齐次线性方程组
        $$
        \vb*{Ax}=\mqty(1\\0\\0),\quad\vb*{Ax}=\mqty(0\\1\\0),\quad\vb*{Ax}=\mqty(0\\0\\1)
        $$
        的解, 由于这三个方程组的系数矩阵是相同的, 所以令 $\vb*{\bar{A}}=(\vb*{A},\vb*{E})$ 作初等行变换, 
        \begin{flalign*}
            \vb*{\bar{A}}=\begin{pNiceArray}{cccc:ccc}1&-2&3&-4&1&0&0\\0&1&-1&1&0&1&0\\1&2&0&-3&0&0&1\end{pNiceArray}\xrightarrow[\substack{r_2+r_3\\ r_1+2r_2-3r_3}]{r_3-r_1-4r_2}
            \begin{pNiceArray}{cccc:ccc}1&0&0&1&2&6&-1\\0&1&0&-2&-1&-3&1\\0&0&1&-3&-1&-4&1\end{pNiceArray}
        \end{flalign*}
        由此解得三个方程组的通解 
        $$
        (-2,-1,-1,0)^\top+k_1\vb*{\eta},\quad
        (6,-3,-4,0)^\top+k_2\vb*{\eta},\quad
        (-1,1,1,0)^\top+k_3\vb*{\eta}
        $$
        因此 $\vb*{B}=\mqty(2-k_1&6+2k_2&-1-k_3\\-1+2k_1&-3+2k_2&1+2k_3\\-3+3k_1&-4+3k_2&1+3k_2\\k_1&k_2&k_3),\forall k_1, k_2, k_3\in\mathbb{R}.$
    \end{enumerate}
\end{solution}

\subsection{无解方程组的最小误差解——最小二乘解}

\begin{theorem}[最小二乘解]
    方程组 $\vb*{Ax}=\vb*{b}$ 的最小二乘解为 $\vb*{x}=\qty(\vb*{A}^\top\vb*{A})^{-1}\vb*{A}^\top\vb*{b}$.
    \index{最小二乘解}
\end{theorem}