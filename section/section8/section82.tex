\section{高阶微分方程}

高阶微分方程是指涉及一个未知函数及其多阶导数的微分方程. 通常形式为:
$$F\qty(x,y,y',y'',\cdots,y^{(n)})=0$$
其中 $y$ 是未知函数, $y^{(i)}~~(i=0,1,\cdots,n)$ 是 $y$ 的 $i$ 阶导数,
$F$ 是关于 $x,y,y',y'',\cdots,y^{(n)}$ 的函数, 解高阶微分方程需要使用微积分知识、积分技巧以及一些特定的方法, 如常数变易法、特征方程法等.

\subsection{常系数齐次线性微分方程}

\subsubsection{二阶常系数线性齐次微分方程的通解}

\begin{theorem}[二阶常系数线性齐次微分方程的通解结构]
    设\index{二阶常系数线性齐次微分方程的通解结构}
    \begin{equation*}
        y^{\prime \prime}+p y^{\prime}+q y=0
        \tag{1}
    \end{equation*}
    \begin{enumerate}[label=(\arabic{*})]
        \item 特征方程有两个相异实根 $ r_{1} \neq r_{2}$, 则方程 (1) 的通解为 $ Y=C_{1} \e^{r_{1} x}+C_{2} \e^{r_{2} x} $;
        \item 特征方程有两个相等实根 $ r_{1}=r_{2}=r $, 则方程 (1) 的通解为 $ Y=\left(C_{1}+C_{2} x\right) \e^{r x} $;
        \item 特征方程有一对共轭复根 $ r_{1,2}=\alpha \pm \mathrm{i} \beta $, 则方程 (1) 的通解为
              $$Y=\e^{\alpha x}\left(C_{1} \cos \beta x+\right.   \left.C_{2} \sin \beta x\right) $$
              且虚根对应的特征方程为 $$
                  \alpha \pm \mathrm{i} \beta \Leftrightarrow (r-\alpha)^2+\beta^2=0
              $$
    \end{enumerate}
\end{theorem}

\begin{example}
    求解下列微分方程的通解.
    \setcounter{magicrownumbers}{0}
    \begin{table}[H]
        \centering
        \begin{tabular}{l | l | l}
            (\rownumber{}) $ y''-12y'+35y=0 $             . & (\rownumber{}) $ y''+2y'+3y=0 $  . & (\rownumber{}) $y'''-6y''+3y'+10y=0$ . \\
            (\rownumber{}) $ y^{(4)}-2y'''+2y''-2y'+y=0 $ . & (\rownumber{}) $y'''-2y''+5y'=0$ . & (\rownumber{}) $ y'''-2y''+y'-2y=0 $.
        \end{tabular}
    \end{table}
\end{example}
\begin{solution}
    \begin{enumerate}[label=(\arabic{*})]
        \item 特征方程 $r^2-12r+35=0\Rightarrow r_1=5, r_2=7$, 因此微分方程的通解为 $y=C_1\e ^{5x}+C_2\e ^{7x}.$
        \item 特征方程 $r^2+2r+3=0\Rightarrow r_{1, 2}=-1\pm \sqrt{2}\mathrm{i}$, 因此微分方程的通解为 $y=\e ^{-x}\qty(C_1\cos\sqrt{2}x+C_2\sin\sqrt{2}x)$.
        \item 特征方程 $r^3-6r^2+3r+10=0\Rightarrow r_1=-1, r_2=2, r_3=5$, 因此微分方程的通解为 $y=C_1\e ^{-x}C_2 \e ^{2x}+C_3 \e ^{5x}$.
        \item 特征方程 $r^4-2r^3+2r^2-2r+1=0\Rightarrow (r-1)^2\qty(r^2+1)=0$ 得二重实根 $1$, 单重共轭复根 $\pm \i$, 因此微分方程的通解为
              $$
                  y=(C_1+C_2x)\e ^{x}+C_3\cos x+C_4\sin x
              $$
        \item 特征方程 $r^3-2r^2+5r=0\Rightarrow r_1=0, r_{2, 3}=1\pm 2\i$, 因此微分方程的通解为 $y=C_1+\e ^{x}(C_2\cos 2x+C_3\sin 2x).$
        \item 特征方程 $r^3-2r^2+r-2=0\Rightarrow r_1=2, r_2=-1, r_3=1$, 因此微分方程的通解为 $y=C_1\e ^{2x}+C_2\e ^{-x}+C_3 \e ^{x}.$
    \end{enumerate}
\end{solution}

\begin{example}[第十三届数学竞赛决赛]
    求区间 $[0,1]$ 上的连续函数 $f(x)$, 使之满足 $$f(x)=1+(1-x)\int_{0}^{x}yf(y)\dd y+x\int_{x}^{1}(1-y)f(y)\dd y.$$
\end{example}
\begin{solution}
    分别令 $x=0,1$, 代入关系式, 得 $f(0)=f(1)=1$, 并对关系式求导, 有
    \begin{flalign*}
        f'(x)=-\int_{0}^{x}yf(y)\dd y+\int_{x}^{1}(1-y)f(y)\dd y
    \end{flalign*}
    再次求导有
    \begin{flalign*}
        f''(x)+f(x)=0
    \end{flalign*}
    该常系数齐次线性微分方程的特征方程为 $r^2+1=0$, 特征根为 $r=\pm i$, 故通解为 $$f(x)=C_1\cos x+C_2\sin x$$
    代入初值 $f(0)=f(1)=1$, 得 $$C_1=1,~C_2=\dfrac{1-\cos 1}{\sin 1}=\tan\dfrac{1}{2}$$
    故所求函数为 $$f(x)=\cos x+\tan\dfrac{1}{2}\sin x.$$
\end{solution}

\subsubsection{$n$ 阶常系数线性齐次微分方程}

\begin{definition}[特征方程与特征根]
    \index{特征方程与特征根}
    设 $ n $ 阶常系数线性齐次微分方程是
    \begin{equation*}
        y^{(n)}+p_{1} y^{(n-1)}+p_{2} y^{(n-2)}+\cdots+p_{n-1} y^{\prime}+p_{n} y=0
        \tag{2}
    \end{equation*}
    其中 $ p_{1}, p_{2}, \cdots, p_{n} $ 是常数, 代数方程
    $$r^{n}+p_{1} r^{n-1}+p_{2} r^{n-2}+\cdots+p_{n-1} r+p_{n}=0$$
    称为\textit{微分方程} (2) \textit{的特征方程}, 特征方程的根叫作\textit{微分方程} (2) \textit{的特征根}.
\end{definition}

\begin{theorem}[$n$ 阶常系数线性齐次微分方程通解结构]
    \index{高阶常系数线性齐次微分方程通解结构}
    对于 $n$ 阶常系数线性齐次微分方程,
    \begin{enumerate}[label=(\arabic{*})]
        \item 如果 $ r_{1} $ 是特征方程的单根, 则
              $$y=C \e^{r_{1} x}$$
              是微分方程 (2) 的解;
        \item 如果特征方程有一对共轭复根 $ r_{1}=\alpha+i \beta, r_{2}=\alpha-i \beta$, 则
              $$y=\e^{\alpha x}\left(C_{1} \cos \beta x+C_{2} \sin \beta x\right)$$
              是微分方程 (2) 的解;
        \item 如果 $ r_{1} $ 是特征方程的 $ k $ 重根, 则
              $$y=\left(C_{1}+C_{2} x+\cdots+C_{k} x^{k-1}\right) \e^{r_{1} x}$$
              是微分方程 (2) 的解;
        \item 如果 $ r_{1}=\alpha+i \beta, r_{2}=\alpha-i \beta $ 都是特征方程的 $ k $ 重根, 则
              $$y=\e^{\alpha x}\left[\left(C_{1}+C_{2} x+\cdots+C_{k} x^{k-1}\right) \cos \beta x+\left(D_{1}+D_{2} x+\cdots+D_{k} x^{k-1}\right) \sin \beta x\right]$$
              是微分方程 (2) 的解.
    \end{enumerate}
\end{theorem}

\subsection{常系数非齐次线性微分方程}

\subsubsection{二阶常系数非齐次线性微分方程的通解结构}

\begin{theorem}[二阶常系数非齐次线性微分方程解的性质]
    若 $y_1, y_2, y_3$ 为二阶非齐次微分方程的解, 令 $$
        y=k_1y_1 +k_2y_2 +k_3y_3
    $$
    若 $k_1+ k_2+ k_3=1$, 则 $y$ 为该非齐次微分方程的解; 若 $k_1+ k_2+ k_3=0$, 则 $y$ 为该方程对应齐次方程的解.
\end{theorem}

\begin{example}
    设线性无关的函数 $y_1, y_2, y_3$ 都是二阶非齐次线性方程 $y''+p(x)y'+q(x)y=f(x)$ 的解, $c_1,c_2$ 是任意常数,
    则该非齐次方程的通解是
    \begin{tasks}(2)
        \task $c_1y_1+c_2y_2+y_3.$
        \task $c_1y_1+c_2y_2-(c_1+c_2)y_3.$
        \task $c_1y_1+c_2y_2-(1-c_1-c_2)y_3.$
        \task $c_1y_1+c_2y_2+(1-c_1-c_2)y_3.$
    \end{tasks}
\end{example}
\begin{solution}
    对于 D 选项, 由 $c_1+c_2+(1-c_1-c_2)=1$, 则 $y=c_1y_1+c_2y_2+(1-c_1-c_2)y_3$ 是二阶非齐次线性方程的解, 再用解的结构进一步分析:
    $$
        y=c_1y_1+c_2y_2+(1-c_1-c_2)y_3=c_1(y_1-y_3)+c_2(y_2-y_3)+y_3
    $$
    其中 $$
        (y_1-y_3,y_2-y_3)=(y_1, y_2, y_3)\mqty(1&0\\ 0&1\\ -1&-1)
    $$
    令 $\vb*{A}=\mqty(1&0\\ 0&1\\ -1&-1)$, 则 $\rank\vb*{A}=2$, 故 $\rank((y_1-y_3,y_2-y_3))=2$, 从而 $y_1-y_3$ 与 $y_2-y_3$ 二者无关, 那么 $y=c_1(y_1-y_3)+c_2(y_2-y_3)$ 是对应齐次方程的通解.
\end{solution}

\begin{theorem}
    当方程 $ y^{\prime \prime}+p y^{\prime}+q y=f(x) $ 的自由项
    $$f(x)=\mathrm{e}^{\alpha x}\left[P_{m}(x) \cos \beta x+P_{n}(x) \cos \beta x\right] $$ 时
    非齐次特解要设为
    $$y^{*}=\mathrm{e}^{\alpha x}\left[Q_{l}^{(1)}(x) \sin \beta x+Q_{l}^{(2)}(x) \cos \beta x\right] x^{k}$$
    其中 $\begin{cases}
        \mathrm{e}^{\alpha x} \text { 照抄,  } \\ l=\max \{m, n\}, Q_{l}^{(1)}, Q_{l}^{(2)} \text { 分别为 } x \text { 的 } l \text { 次多项式,  } \\ k=\begin{cases}
            0, \alpha \pm \beta i \text { 不是特征值,  } \\ 1, \alpha \pm \beta i \text { 是特征值. }
        \end{cases}
    \end{cases}$
\end{theorem}

\begin{example}
    如果二阶常系数非齐次线性微分方程 $y''+ay'+by=\e ^{-x}\cos x$ 有一个特解 $$y^*=\e ^{-x}(x\cos x+x\sin x)$$ 则
    \begin{tasks}(4)
        \task $a=-1,b=1.$
        \task $a=1,b=-1.$
        \task $a=2,b=-1.$
        \task $a=2,b=2.$
    \end{tasks}
\end{example}
\begin{solution}
    由特解 $y^*=\e ^{-x}(x\cos x+x\sin x)$, 得特征根 $-1\pm i$, 对应的特征方程为 $(r+1)^2+1=0\Rightarrow r^2+2r+2=0$, 对应的齐次微分方程为 $y''+2y'+2y=0$, 故 $a=b=2$, 选 D.
\end{solution}

若 $ y_{1}, y_{2} $ 为二阶线性齐次微分方程 $ y^{\prime \prime}+P(x) y^{\prime}+Q(x) y=0 $ 的两个线性无关的解, 则其通解为 $ y=C_{1} y_{1}+C_{2} y_{2} $, 即 $ y-C_{1} y_{1}-C_{2} y_{2}=0 $, 继而有
$
    \begin{cases}
        y-C_{1} y_{1}-C_{2} y_{2}=0                            \\
        y^{\prime}-C_{1} y_{1}^{\prime}-C_{2} y_{2}^{\prime}=0 \\
        y^{\prime \prime}-C_{1} y_{1}^{\prime \prime}-C_{2} y_{2}^{\prime \prime}=0
    \end{cases}
$
即 $$\begin{pmatrix}  y & y_{1} & y_{2} \\ y^{\prime} & y_{1}^{\prime} & y_{2}^{\prime} \\ y^{\prime \prime} & y_{1}^{\prime \prime} & y_{2}^{\prime \prime}\\\end{pmatrix} \begin{pmatrix}1 \\ -C_{1} \\ -C_{2}\end{pmatrix}=\begin{pmatrix}0 \\ 0 \\ 0\end{pmatrix}$$
该方程组有非零解, 故其系数行列式
$$
    \begin{vmatrix}y                 & y_{1}                 & y_{2}                 \\
               y^{\prime}        & y_{1}^{\prime}        & y_{2}^{\prime}        \\
               y^{\prime \prime} & y_{1}^{\prime \prime} & y_{2}^{\prime \prime}\end{vmatrix}=0
$$
据此即可解出待求的二阶线性齐次方程. 最后得到的齐次方程等号右边是 0, 此时将 0 换成 $ f(x) $, 再找一个二阶非齐次方程的特解代进去, 将 $ f(x) $ 解出来, 就可以得到二阶非齐次方程了.

\begin{example}
    求以 $y=x^2-\e^x$ 和 $y=x^2$ 为特解的一阶非齐次线性微分方程.
\end{example}
\begin{solution}
    令 $y_1=x^2-\qty(x^2-\e^x)=\e^x$, 那么 $$\begin{vmatrix} y & y_1 \\ y' & y'_1 \\\end{vmatrix}=\begin{vmatrix} y & \e^x \\ y' & \e^x \\\end{vmatrix}=\e^x\begin{vmatrix} y & 1 \\ y' & 1 \\\end{vmatrix}=0$$
    即 $y-y'=0$, 又 $y=x^2$, 所以 $f(x)=x^2-2x$, 故一节非齐次线性微分方程为 $y'-y=2x-x^2.$
\end{solution}

\begin{example}
    已知 $y_1=x,~y_2=x+\sin x,~y_3=\e ^x+x$ 为某二阶非齐次线性微分方程的三个特解, 求该微分方程, 并求其通解.
\end{example}
\begin{solution}
    $y_2-y_1=\sin x, y_3-y_1=\e^x$, 那么 $$\begin{vmatrix} y & \sin x & \e^x \\ y' & \cos x & \e^x \\ y'' & -\sin x & \e^x \\\end{vmatrix}=0\Rightarrow (\sin x-\cos x)y''-2\sin xy'+(\sin x+\cos x)y=f(x)$$
    又 $y_1=x$, 所以 $f(x)=-2\sin x+x(\sin x+\cos x)$, 于是通解为
    $$
        (\cos x-\sin x)y''+2\sin xy'-(\sin x+\cos x)y=2\sin x-x(\sin x+\cos x).
    $$
\end{solution}

% \begin{theorem}
%     一般地, 设线性无关的函数 $y_1,~y_2,~y_3$ 都是以下二阶非齐次线性微分方程的解, 则
%     $$y''(x)+p(x)y'+q(x)y=f(x)$$
%     \begin{enumerate}[label=(\arabic{*})]
%         \item $y_1-y_2,~y_1-y_3,~y_2-y_3$ 都是 $y''+p(x)y'+q(x)y=0$ 的解,
%               且 $y_1-y_3$ 与 $y_2-y_3$ 线性无关;
%         \item 非齐次线性微分方程的通解为 $$y=C_1(y_1-y_3)+C_2(y_2-y_3)+y_3=C_1y_1+C_2y_2+(1-C_1-C_2)y_3.$$
%     \end{enumerate}
% \end{theorem}

% \begin{example}
%     设二阶线性微分方程 $y''+p(x)y'+q(x)y=f(x)$ 有三个特解 $y_1=\e ^x,~y_2=\e ^x+\e ^{\frac{x}{2}},~y_3=\e ^x+\e ^{-x}$, 
%     试求该微分方程, 并求其通解.
% \end{example}

\begin{example}[1997 数二]
    设某个二阶常系数非齐次线性微分方程有三个特解:
    $$y_1=x\e ^x+\e ^{2x},~y_2=x\e ^x+\e ^{-x},~y_3=x\e ^x+\e ^{2x}-\e ^{-x}$$
    求该微分方程, 并给出其通解.
\end{example}
\begin{solution}
    $y_1-y_3=\e ^{-x}, y_1-y_2+(y_1-y_3)=\e ^{2x}$, 那么 $$
        \begin{vmatrix} y & \e ^{-x} & \e ^{2x} \\ y' & -\e ^{-x} & 2\e ^{2x} \\ y'' & \e ^{-x} & 4\e ^{2x} \\\end{vmatrix}=0\Rightarrow y''-y'-2y=f(x)
    $$
    又 $y_1=x\e ^{x}+\e ^{2x}$, 所以 $f(x)=\e ^{x}-2x\e ^{x}$, 那么通解为 $$
        y''-y'-2y=\e ^{x}-2x\e ^{x}.
    $$
\end{solution}

\begin{example}[1993 数三]
    设二阶常系数非齐次线性微分方程 $ y^{\prime \prime}+a y^{\prime}+b y=k \mathrm{e}^{x} $ 的一个特解为 $$ y=\mathrm{e}^{2 x}+(1+x) \mathrm{e}^{x} $$ 试确定 $ a, b, k $, 并求该方程的通解.
\end{example}
\begin{solution}
    \textbf{法一: }
    将特解 $ y=\mathrm{e}^{2 x}+(1+x) \mathrm{e}^{x} $ 代人原方程, 得
    $$(4+2 a+b) \mathrm{e}^{2 x}+(3+2 a+b) \mathrm{e}^{x}+(1+a+b) x \mathrm{e}^{x}=k \mathrm{e}^{x} $$
    比较两边同类项的系数, 得 $ 2 a+b=-4,2 a+b-k=-3, a+b=-1 $, 解得 $ a=-3 , b=2, k=-1 $ 故原方程为
    $$y^{\prime \prime}-3 y^{\prime}+2 y=-\mathrm{e}^{x} $$
    又对应的齐次线性方程的特征方程为 $ r^{2}-3 r+2=0 $, 解得特征根为 $ r_{1}=1, r_{2}=2 $, 于是可设原方程的特解形式为 $ y^{*}=c x \mathrm{e}^{x} $, 把它代人原方程中得 $ c=1 $, 故由通解结构知原方程的通解为 $$ y=C_{1} \mathrm{e}^{x}+C_{2} \mathrm{e}^{2 x}+x \mathrm{e}^{x}$$ 其中 $ C_{1}, C_{2} $ 为任意常数.\\
    \textbf{法二: }
    由已知及通解的结构与特解形式, 得 $ \mathrm{e}^{2 x} $ 与 $ \mathrm{e}^{x} $ 应是对应齐次线性方程的特解, $ x \mathrm{e}^{x} $ 是非齐次线性方程的一个特解. 于是特征方程为 $$ r^{2}+a r+b=(r-1)(r-2)=r^{2}-3 r+2$$
    即得 $ a=-3, b=2$, 又将 $ y^{*}=x \mathrm{e}^{x} $ 代人原方程 $ y^{\prime \prime}-3 y^{\prime}+2 y=k \mathrm{e}^{x} $ 中得 $ k=-1$, 故原方程为 $ y^{\prime \prime}-3 y^{\prime}+2 y=-\mathrm{e}^{x} $, 其通解为 $ y=C_{1} \mathrm{e}^{x}+C_{2} \mathrm{e}^{2 x}+x \mathrm{e}^{x} $, 其中 $ C_{1}, C_{2} $ 为任意常数.\\
    \textbf{法三: }由已知及通解的结构与特解形式, 得 $ \mathrm{e}^{2 x}, \mathrm{e}^{x} $ 是对应齐次线性方程的特解, $ x \mathrm{e}^{x} $ 是非齐次线性方程的一个特解, 分别代人对应的方程, 得
    $$ \begin{cases}
            4+2 a+b=0, \\ 1+a+b=0, \\ (2+x)+a(1+x)+b x=k
        \end{cases}$$
    解得 $ a=-3, b=2, k=-1 $, 故原方程为 $ y^{\prime \prime}-3 y^{\prime}+   2 y=-\mathrm{e}^{x} $, 其通解为 $ y=C_{1} \mathrm{e}^{x}+C_{2} \mathrm{e}^{2 x}+x \mathrm{e}^{x} $ 其中 $ C_{1}, C_{2} $ 为任意常数.
\end{solution}

\begin{example}
    设 $y=\e^x(C_1\sin x+C_2\cos x)$ ($C_1,C_2$ 为任意常数) 为某二阶常系数齐次线性微分方程的通解, 求该微分方程.
\end{example}
\begin{solution}
    \textbf{法一: }由二阶常系数齐次线性微分方程的通解公式, 可知特征方程的根应为 $ \lambda_{1,2}=1 \pm i $, 从而特征方程为 $ \left(\lambda-\lambda_{1}\right)\left(\lambda-\lambda_{2}\right)=(\lambda-1-i) \cdot(\lambda-1+i)=0 $, 即 $ \lambda^{2}-2 \lambda+2=0 $. 故所求微分方程为 $ y^{\prime \prime}-2 y^{\prime}+2 y=0 $.\\
    \textbf{法二: }设所求微分方程为 $ y^{\prime \prime}+p y^{\prime}+q y=0 $, 其中 $ p, q $ 为常数. 将已知解 $ y=\mathrm{e}^{x}\left(C_{1} \sin x+C_{2} \cos x\right) $ 代人上述方程, 得
    $$\mathrm{e}^{x} \sin x\left[p\left(C_{1}-C_{2}\right)+q C_{1}-2 C_{2}\right]+\mathrm{e}^{x} \cos x\left[p\left(C_{1}+C_{2}\right)+q C_{2}+2 C_{1}\right]=0$$
    因为 $ \mathrm{e}^{x} \sin x $ 与 $ \mathrm{e}^{x} \cos x $ 线性无关, 所以 $$ p\left(C_{1}-C_{2}\right)+q C_{1}-2 C_{2}=0, p\left(C_{1}+\right.   \left.C_{2}\right)+q C_{2}+2 C_{1}=0 $$
    解得 $ p=-2, q=2 $, 故所求微分方程为 $ y^{\prime \prime}-2 y^{\prime}+2 y=0 $.\\
    \textbf{法三: }设所求微分方程为 $ y^{\prime \prime}+p y^{\prime}+q y=0 $, 其中 $ p, q $ 为常数.
    因为 $ y=\mathrm{e}^{x}\left(C_{1} \sin x+C_{2} \cos x\right) $ 为上述方程的通解, 所以 $ y_{1}=\mathrm{e}^{x} \sin x $ 和 $ y_{2}=   \mathrm{e}^{x} \cos x $ 为 $ y^{\prime \prime}+p y^{\prime}+q y=0 $ 的特解, 代人方程, 得
    $$\begin{cases}
            p(\cos x+\sin x)+q \sin x+2 \cos x=0 \\
            p(\cos x-\sin x)+q \cos x-2 \sin x=0
        \end{cases}$$
    改写为矩阵形式, 有 $ \mqty(\cos x+\sin x & \sin x \\ \cos x-\sin x & \cos x)\mqty(p\\ q)=\mqty(-2\cos x\\ 2\sin x) $, 而
    $$\mqty(\cos x+\sin x & \sin x \\
        \cos x-\sin x & \cos x)^{-1}=\mqty(\cos x        & -\sin x       \\
        \sin x-\cos x & \cos x+\sin x)$$
    且 $\mqty|\cos x+\sin x & \sin x \\
        \cos x-\sin x & \cos x|=1$, 所以
    $$\mqty(p\\ q)=\mqty(\cos x        & -\sin x       \\
        \sin x-\cos x & \cos x+\sin x)\mqty(-2\cos x\\ 2\sin x)=\mqty(-2\\ 2)$$
    故所求微分方程为 $ y^{\prime \prime}-2 y^{\prime}+2 y=0 .$\\
    \textbf{法四: }由通解 $ y=\mathrm{e}^{x}\left(C_{1} \sin x+C_{2} \cos x\right) $, 求出 $ y^{\prime}, y^{\prime \prime} $:
    $$y^{\prime}=\mathrm{e}^{x}\left[\left(C_{1}-C_{2}\right) \sin x+\left(C_{1}+C_{2}\right) \cos x\right], \quad y^{\prime \prime}=\mathrm{e}^{x}\left(-2 C_{2} \sin x+2 C_{1} \cos x\right)$$
    消去 $ C_{1} $ 与 $ C_{2} $, 得 $ y^{\prime \prime}-2 y^{\prime}+2 y=0 $, 此即为所求微分方程.
\end{solution}

\subsubsection{微分算子法}

微分算子法是在求解非齐次线性微分方程 $$y^{(n)}+a_1y^{(n-1)}+\cdots+a_{n-1}y'+a_ny=f(x)$$
的一个特解时常采用的一种方法. 具体地说, 在求解常系数微分方程上述方程时, 可以先求解其特征方程,
得到相应齐次方程的通解, 再求出非齐次方程的一个特解即可.

\begin{definition}[微分算子]
    为了求非齐次方程的一个特解, 引入\textit{微分算子}:
    $$\mathrm{D}=\frac{\mathrm{d}}{\mathrm{d}x},~\mathrm{D}^2=\frac{\mathrm{d^2}}{\mathrm{d}x^2},~\cdots,~\mathrm{D}^n=\frac{\mathrm{d}^n}{\mathrm{d}x^n}$$
    以及\textit{算子多项式} $$L(\mathrm{D})=\mathrm{D}^n+a_1\mathrm{D}^{n-1}+\cdots+a_{n-1}\mathrm{D}+a_n$$
    这样一来, 微分方程可以写成 $$L(\mathrm{D})y=f(x)$$
    显然, 如果记 $\displaystyle\frac{1}{\mathrm{D}}f(x)$ 为此方程的任一解, 则
    $$L(\mathrm{D})\left(\frac{1}{L(\mathrm{D})}f(x)\right)=f(x)$$
    因此, 在形式上可以把 $\displaystyle\frac{1}{\mathrm{D}}f(x)$ 看成是 $L(\mathrm{D})$ 的逆算子.
\end{definition}

\begin{theorem}[算子逆运算]
    \index{算子逆运算}
    $\displaystyle\frac{1}{\mathrm{D}}f(x)=\int f(x)\mathrm{d}x.$
\end{theorem}
\begin{theorem}[算子特性]
    \index{算子特性}
    对于任意常数 $a$, $\displaystyle\frac{1}{\mathrm{D}-a}f(x)=\e ^{ax}\int\e ^{-ax}f(x)\mathrm{d}x.$
\end{theorem}
\begin{theorem}[算子分配律]
    \index{算子分配律}
    对于任意常数 $\alpha$ 和 $\beta$, $\displaystyle\frac{1}{L(\mathrm{D})}(\alpha u(x)+\beta v(x))=\alpha\frac{1}{L(\mathrm{D})}u(x)+\beta\frac{1}{L(\mathrm{D})}v(x).$
\end{theorem}
\begin{theorem}[算子结合律]
    \index{算子结合律}
    $\displaystyle\frac{1}{L_1(\mathrm{D})L_2(\mathrm{D})}f(x)=\frac{1}{L_1(\mathrm{D})}\left(\frac{1}{L_2(\mathrm{D})}f(x)\right)=\frac{1}{L_2(\mathrm{D})}\left(\frac{1}{L_1(\mathrm{D})}f(x)\right).$
\end{theorem}

\setcounter{magicrownumbers}{0}
\begin{table}[H]
    \centering
    \caption{微分算子常用结论}
    \begin{tabular}{l l}
        (\rownumber) $L(\mathrm{D})=\mathrm{D}^m$ 时, $\dfrac{1}{L(\mathrm{D})}f^{(m)}(x)=f(x)$.                                                                                                                            \\
        \midrule
        $L(\mathrm{D})=\mathrm{D}-k$ 时                                                                                                                                                                                     \\
        (\rownumber) $\dfrac{1}{L(\mathrm{D})}\e^{bx}=\dfrac{e^{bx}}{b-k},~b\neq k$.     & (\rownumber) $\dfrac{1}{L(\mathrm{D})}x^{\alpha}=\qty(-\displaystyle\sum_{n=0}^{\alpha}\dfrac{\mathrm{D}^n}{k^{n+1}})x^{\alpha}$ \\
        \midrule
        (\rownumber) $\dfrac{1}{L(\mathrm{D})}\e^{kx}=\dfrac{1}{L(k)}\e^{kx},~L(k)\neq0$ & (\rownumber) $\dfrac{1}{L(\mathrm{D})}u(x)\e^{kx}=\e^{kx}\dfrac{1}{L(\mathrm{D}+k)}u(x)$                                         \\
        \multicolumn{2}{l}{(\rownumber) $L(\mathrm{D})=(\mathrm{D}-b)(\mathrm{D}-a)$ 时, $\dfrac{1}{L(\mathrm{D})}f(x)=\dfrac{1}{\mathrm{D}-b}\dfrac{1}{\mathrm{D}-a}f(x)$, 当 $ab\neq0$ 时, $\dfrac{1}{\mathrm{D}-b}\dfrac{1}{\mathrm{D}-a}=\dfrac{1}{ab}$.}
    \end{tabular}
\end{table}

\begin{example}
    用微分算子法计算下列微分方程的特解.
    \setcounter{magicrownumbers}{0}
    \begin{table}[H]
        \centering
        \begin{tabular}{l | l}
            (\rownumber{}) $y''+3y'+2y=5\e ^{3x}$      . & (\rownumber{}) $y''+3y'+2y=\e ^{-x}$.     \\
            (\rownumber{}) $y''+3y=\sin 2x$            . & (\rownumber{}) $y''+3y'-2y=\sin 2x$     . \\
            (\rownumber{}) $y''+3y'-2y=\e ^{x}\sin 2x$ . & (\rownumber{}) $y''+y=x\cos 2x$.          \\
            (\rownumber{}) $y''-3y'+2y=2x^2+x+1$.
        \end{tabular}
    \end{table}
\end{example}
\begin{solution}
    \begin{enumerate}[label=(\arabic{*})]
        \item $y^*=\dfrac{1}{\mathrm{D}^2+3\mathrm{D}+2}5\e ^{3x}=\dfrac{1}{3^2+3\cdot 3+2}5\e ^{3x}=\dfrac{1}{4}\e ^{3x}$.
        \item $y^*=\dfrac{1}{\mathrm{D}^2+3\mathrm{D}+2}\e ^{-x}=x\dfrac{1}{2\mathrm{D}+3}\e ^{-x}=x\e ^{-x}$.
        \item $y^*=\dfrac{1}{\mathrm{D}^2+3}\sin 2x=-\sin 2x$.
        \item $y^*=\dfrac{1}{\mathrm{D}^2+3\mathrm{D}-2}\sin 2x=\dfrac{1}{-(2)^2+3\mathrm{D}-2}\sin 2x=\dfrac{1}{3}\dfrac{\mathrm{D}+2}{\mathrm{D}^2-4}\sin 2x=-\dfrac{1}{12}(\sin2x+\cos 2x)$.
        \item $y^*=\dfrac{1}{\mathrm{D}^2+3\mathrm{D}-2}\e ^{x}\sin 2x=\e ^{x}\dfrac{1}{(\mathrm{D}+1)^2+3(\mathrm{D}+1)-2}\sin 2x=\e ^{x}\dfrac{5\mathrm{D}+2}{25\mathrm{D}^2-4}\sin 2x= -\dfrac{\e ^{x}}{52}(5\cos2x+\sin2x)$.
        \item $y^*=\dfrac{1}{\mathrm{D}^2+1}x\cos2x=\qty(x-\dfrac{2\mathrm{D}}{\mathrm{D}^2+1})\dfrac{1}{\mathrm{D}^2+1}\cos2x=-\dfrac{1}{3}\qty(x-\dfrac{2\mathrm{D}}{\mathrm{D}^2+1})\cos 2x=\dfrac{4}{9}\sin 2x-\dfrac{1}{3}x\cos 2x.$
        \item $y^*=\dfrac{1}{\mathrm{D}^2-3\mathrm{D}+2}\qty(2x^2+x+1)=\qty(\dfrac{1}{2}+\dfrac{3}{4}\mathrm{D}+\dfrac{7}{8}\mathrm{D}^2)\qty(2x^2+x+1)=x^2+\dfrac{7}{2}x+\dfrac{19}{4}$.
    \end{enumerate}
\end{solution}

\begin{inference}[微分算子法快速计算]
    对于下列的情况有,
    \begin{enumerate}[label=(\arabic{*})]
        \item $f(x)=\e ^{kx}$ 型: 所有 $\mathrm{D}$ 的值都为 $k$ 的值; 当发现分母为 0 时, 需要对分母求导, 求 $l$ 次导, 前面添加 $x^l$ (类似于洛必达法则);
        \item $\sin\alpha x$ 或 $\cos \alpha x$ 型: 将 $\mathrm{D}^2$ 换成 $-\alpha^2$, 当分母没有 $\mathrm{D}$ 的平方项时, 用平方差公式创造出平方项;
        \item $\e ^{kx}y(x)$ 型: 采用“漂移法”, 将 $\e ^{kx}$ 移至左侧;
        \item $x \sin \alpha x$ 或 $x \cos \alpha x$ 型: 有公式 $y^*=\qty(x-\dfrac{2\mathrm{D}}{\mathrm{D}^2+1})\dfrac{1}{L(\mathrm{D})}\sin\alpha x (\text{ 或} \cos\alpha x).$
        \item 多项式型: 采用“长除法”.
    \end{enumerate}
\end{inference}

\begin{example}
    若 $A,B$ 为非零常数, $k$ 为常数, 则微分方程 $y''+k^2y=\cos x$ 的特解可能具有的形式是
    \begin{tasks}(4)
        \task $A\sin x+B\cos x$
        \task $Ax\cos x$
        \task $Ax \sin x$
        \task $Ax\sin x+Bx\cos x$
    \end{tasks}
\end{example}
\begin{solution}
    该微分方程的特解为 $y^*=\dfrac{\cos x}{\mathrm{D}^2+k^2}$, 若 $k^2=1$ 时, $$
        y^*=\dfrac{\cos x}{\mathrm{D}^2+k^2}=x\dfrac{\cos x}{2 \mathrm{D}}=\dfrac{x}{2}\sin x
    $$
    若 $k^2\neq 1$ 时,
    $$
        y^*=\dfrac{\cos x}{\mathrm{D}^2+k^2}=\dfrac{\cos x}{k^2-1}
    $$
    因为 $A,B$ 都是非零常数, 所以只有 $C$ 符合.
\end{solution}

\begin{example}[2010 数学 (一)]
    \label{y3y2y2xex}求微分方程 $y''-3y'+2y=2x\e ^x$ 的通解.
\end{example}
\begin{solution}
    记 $\displaystyle \mathrm{D}=\frac{\mathrm{d}}{\mathrm{d}x},~\mathrm{I}y=y$, 则原方程可改写为 $\left(\mathrm{D}^2-3\mathrm{D}+2\mathrm{I}\right)y=2x\e ^x$, 即 $(\mathrm{D}-\mathrm{I})(\mathrm{D}-2\mathrm{I})y=2x\e ^x$,
    令 $(\mathrm{D}-2\mathrm{I})y=z(x)$, 则有 $(\mathrm{D}-\mathrm{I})z(x)=2x\e ^x$, 即 $\displaystyle \frac{\mathrm{d}z(x)}{\mathrm{d}x}-z(x)=2x\e ^x$, 解得
    $$z(x)=\e ^{\int\mathrm{d}x}\left[\int 2x\e ^x\cdot \e ^{-\int \mathrm{d}x}+C_1\right]=\e ^x\left(x^2+C_1\right)$$
    再由 $(\mathrm{D}-2\mathrm{I})y=z(x)$ 化为 $\displaystyle \frac{\mathrm{d}y}{\mathrm{d}x}-2y=\e ^x\left(x^2+C_1\right)$, 故原方程的通解为
    \begin{flalign*}
        y & =\e ^{\int \mathrm{d}x}\left[\int \e ^x\left(x^2+C_1\right)\cdot\e ^{-\int 2\mathrm{d}x}+C_2\right]=\e ^{2x}\left[\int \e ^{-x}\left(x^2+C_1\right)\mathrm{d}x+C_2\right] \\
          & =\e ^{2x}\left[(-x^2\e ^{-x}-2x\e ^{-x}-2\e ^{-x})-C_1\e ^{-x}+C_2\right]=\left(-x^2-2x-2-C_1\right)\e ^x+C_2\e ^{2x}
    \end{flalign*}
    其中 $C_1,~C_2$ 为任意常数.
\end{solution}

\begin{example}
    求微分方程 $y''+y'-2y=\dfrac{\e ^x}{1+\e ^x}.$
\end{example}
\begin{solution}
    记 $\displaystyle \mathrm{D}=\dv{x}$, 则原方程可改写为 $\qty(\mathrm{D}^2+\mathrm{D}-2)y=\dfrac{\e ^x}{1+\e ^x}$, 即 $(\mathrm{D}+2)(\mathrm{D}-1)y=\dfrac{\e ^x}{1+\e ^x}$,
    令 $z(x)=(\mathrm{D}-1)y$, 则有 $(\mathrm{D}+2)z(x)=\dfrac{\e ^x}{1+\e ^x}$, 即 $\displaystyle\dv{z(x)}{x}+2z=\dfrac{\e ^x}{1+\e ^x}$, 解得
    $$z(x)=\e ^{-2\int\dd x}\qty[\int\dfrac{\e ^x}{1+\e ^x}\cdot \e ^{2\int\dd x}\dd x+C_1]=\dfrac{1}{2}-\dfrac{1}{\e ^x}+\dfrac{\ln\qty(\e ^x+1)}{\e ^{2x}}$$
    再由 $(\mathrm{D}-1)y=z(x)$ 化为 $\displaystyle \dv{y}{x}-y=\dfrac{1}{2}-\dfrac{1}{\e ^x}+\dfrac{\ln\qty(\e ^x+1)}{\e ^{2x}}$, 故原方程的通解为
    \begin{flalign*}
        y & =\e ^{\int\dd x}\qty{\int\qty[\dfrac{1}{2}-\dfrac{1}{\e ^x}+\dfrac{\ln\qty(\e ^x+1)}{\e ^{2x}}]\cdot\e ^{-\int\dd x}\dd x+C_2}=\e ^x \qty{\int \qty[\dfrac{1}{2\e ^x}-\dfrac{1}{\e ^{2x}}+\dfrac{\ln\qty(\e ^x+1)}{\e ^{3x}}]\dd x+C_2} \\
          & =-\dfrac{\qty(\e ^{3x}+1)\ln\qty(\e ^x+1)}{3\e ^{2x}}+\dfrac{x\e ^x}{3}+C_2\e ^x+\dfrac{1}{3\e ^x}+\dfrac{C_1}{\e ^{2x}}-\dfrac{1}{6}
    \end{flalign*}
    其中 $\displaystyle\int \dfrac{\ln\qty(\e ^x+1)}{\e ^{3x}}\dd x\xlongequal[]{\e ^x=t}\int\dfrac{\ln(t+1)}{t^4}\dd t=-\dfrac{1}{3}\int \ln(t+1)\dd \dfrac{1}{t^3}=-\dfrac{1}{3}\qty[\dfrac{\ln(t+1)}{t^3}-\int\dfrac{\dd t}{t^3(t+1)}]$, 且
    \begin{flalign*}
        \int\dfrac{\dd t}{t^3(t+1)}\xlongequal[]{u=\frac{t}{t+1}}\int\qty(\dfrac{1}{u}-\dfrac{2}{u^2}+\dfrac{1}{u^3})\dd u=\ln|u|+\dfrac{4u-1}{2u^2}+C
    \end{flalign*}
    即得 $\displaystyle\int\dfrac{\ln\qty(\e ^x+1)}{\e ^{3x}}\dd x=-\dfrac{\ln\qty(\e ^x+1)}{3}+\dfrac{-2\ln\qty(\e ^x+1)+2\e ^{2x}-\e ^x}{6\e ^{3x}}+\dfrac{x}{3}+C.$
\end{solution}

\subsubsection{Laplace 变换法}

\begin{definition}[Laplace 变换]
    假设函数 $f(x)$ 在区间 $[0,+\infty)$ 上分段连续, 如果参变量积分
    $$F(s)=\int_{0}^{+\infty}\e ^{-st}f(t)\mathrm{d}t$$
    收敛, 则称函数 $F(s)$ 为 $f(x)$ 的 \textit{Laplace 变换}, 并记为 $$F(s)=\mathcal{L}\{f(t)\}$$
    称函数 $f(t)$ 为 $F(s)$ 的 \textit{Laplace 逆变换}, 并记为 $$f(t)=\mathcal{L}^{-1}\{F(s)\}.$$
\end{definition}

常见的 Laplace 变换 (以下 $s>0,\alpha,\lambda,~n>-1$).
\setcounter{magicrownumbers}{0}
\begin{table}[H]
    \centering
    \caption{常见的 Laplace 变换}
    \begin{tabular}{l l}
        (\rownumber) $\displaystyle \mathcal{L}\{1\}=\frac{1}{s}$                                                                       & (\rownumber) $\displaystyle \mathcal{L}\{t^n\}=\frac{n!}{s^{n+1}}$                                                                  \\
        (\rownumber) $\displaystyle \mathcal{L}\{\e ^{\alpha t}\}=\frac{1}{s-\alpha}$                                                   & (\rownumber) $\displaystyle \mathcal{L}\{t^n\e ^{\alpha t}\}=\frac{n!}{(s-\alpha)^{n+1}}$                                           \\
        \midrule
        (\rownumber) $\displaystyle \mathcal{L}\{\sin \omega t\}=\frac{\omega}{s^2+\omega}$                                             & (\rownumber) $\displaystyle \mathcal{L}\{\cos \omega t\}=\frac{s}{s^2+\omega}$                                                      \\
        (\rownumber) $\displaystyle \mathcal{L}\{t\sin \omega t\}=\frac{2s\omega}{(s^2+\omega^2)^2}$                                    & (\rownumber) $\displaystyle \mathcal{L}\{t\cos \omega t\}=\frac{s^2-\omega^2}{(s^2+\omega^2)^2}$                                    \\
        \midrule
        (\rownumber) $\displaystyle \mathcal{L}\{\e ^{\lambda t}\sin \omega t\}=\frac{\omega}{(s-\lambda)^2+\omega^2}$                  & (\rownumber) $\displaystyle \mathcal{L}\{\e ^{\lambda t}\cos \omega t\}=\frac{s-\lambda}{(s-\lambda)^2+\omega^2}$                   \\
        (\rownumber) $\displaystyle \mathcal{L}\{t\e ^{\lambda t}\sin \omega t\}=\frac{2\omega(s-\lambda)}{[(s-\lambda)^2+\omega^2]^2}$ & (\rownumber) $\displaystyle \mathcal{L}\{t\e ^{\lambda t}\cos \omega t\}=\frac{(s-\lambda)^2-\omega^2}{[(s-\lambda)^2+\omega^2]^2}$
    \end{tabular}
\end{table}

\begin{example}[2010 数学 (一)]\scriptsize\linespread{0.8}
    (用 Laplace 变换) 求解例题 \ref{y3y2y2xex}.
\end{example}
\begin{solution}\scriptsize\linespread{0.8}
    对方程两边进行 Laplace 变换, 得 $\displaystyle s^2Y(s)-3sY(s)+2Y(s)=\frac{2}{(s-1)^2}$, 解得 $$Y(s)=2\left[\frac{1}{s-2}-\frac{1}{s-1}-\frac{1}{(s-1)^2}-\frac{1}{(s-1)^3}\right]$$
    所以一个特解为 $\displaystyle y^*=2\left(\e ^{2x}-\e ^x-x\e ^x-\frac{x^2}{2}\e ^x\right)$, 又对应齐次线性微分方程 $y''-3y'+2y=0$ 的通解为 $Y=C_1\e ^x+C_2\e ^{2x}$,
    故原方程的通解为 $$y=Y+y^*=C_1\e ^x+C_2\e ^{2x}-\left(x^2+2x+2\right)\e ^x.$$
\end{solution}

\subsection{可降阶的高阶微分方程}

\subsubsection{$y^{(n)}=f(x)$ 型}

方程特点是右端为自变量 $x$ 的函数, 且不含有函数 $y$ 及其导数 $y',y'',\cdots,y^{(n-1)}$, 将方程两边对 $x$ 积分 $n$ 次, 即得其通解
$$y=\int\dd x\cdots\int f(x)\dd x+\sum_{i=1}^{n}\dfrac{\C_i}{(n-i)!}x^{n-i}.$$

\begin{theorem}
    对于二阶常系数非齐次线性微分方程 $y''+py'+qy=f(x)$, 若由特征方程 $r^2+pr+q=0$ 求出特征根为 $\lambda_1,~\lambda_2$, 且有
    $$\lambda_1+\lambda_2=-p,~\lambda_1\cdot\lambda_2=q$$
    那么方程总可以化为 $$\qty(y'-\lambda_1y)'-\lambda_2\qty(y'-\lambda_1y)=f(x).$$
\end{theorem}

\begin{example}
    试用降阶法求解例题 \ref{y3y2y2xex}.
\end{example}
\begin{solution}
    将原方程改写为 $\qty(y'-2y)'-(y'-2y)=2x\e ^x$, 设 $u=y'-2y$, 则上述方程化为 $u'-u=2x\e ^x$, 解得
    $$u=\e ^{\int\dd x}\qty[\int 2x\e ^x\e ^{-\int\dd x}\dd x+C_1]=\e ^x\qty(\int 2x\dd x+C_1)=x^2\e ^x+C_1\e ^x$$
    再求一阶微分方程 $y'-2y=x^2\e ^x+C_1\e ^x$ 得原方程的通解为
    $$y=\e ^{2x}\qty[\int\qty(x^2\e ^x+C_1\e ^x)\e ^{-2x}\dd x+C_2]=-\qty(x^2+2x+1)\e ^x-C_1\e ^x+C_2\e ^{2x}.$$
\end{solution}

\subsubsection{$y''=f\qty(x,y')$ 型}

方程的特点是右端不显含 $y$, 令 $y'=p(x), y''=\dfrac{\dd p}{\dd x}=p'$, 代入原方程即可化为一阶方程 $p'=f(x,p)$, 若其解为 $p=\varphi(x,C_1)$, 则原方程的通解为 $$y=\int\varphi(x,C_1)\dd x+C_2.$$

\subsubsection{$y''=f\qty(y,y')$ 型}

方程的特点是右端不显含自变量 $x$, 令 $y'=p(x)$, 并利用复合函数的求导法则, 有
$$y''=\dv{p}{x}=\dv{p}{y}\cdot\dv{y}{x}=p\cdot\dv{p}{y}$$
代入原方程即可化为一阶方程 $$p\cdot \dv{p}{y}=f(y,p)$$
若其解为 $p=\varphi(y,C_1)$, 即 $\dfrac{\dd y}{\dd x}=\varphi(y,C_1)$, 则原方程的通解为
$$\int \dfrac{\dd y}{\varphi(y,C_1)}=x+C_2.$$

\subsection{Euler 微分方程}

\subsubsection{齐次 Euler 微分方程}

\begin{definition}[齐次 Euler 微分方程]
    形如 $$x^ny^{(n)}+a_1x^{n-1}y^{(n-1)}+\cdots+a_{n-1}xy'+a_ny=0.$$
    的微分方程称为\textit{齐次 Euler 微分方程}.
\end{definition}

\begin{theorem}
    若采用记号 $\mathrm{D}$ 表示对 $t~(t=\ln x)$ 求导的运算 $\displaystyle\dv{t}$, 那么有
    $$x^ky_{x}^{(k)}=\mathrm{D}(\mathrm{D}-1)\cdots(\mathrm{D}-k+1)y_{t}=\mathrm{D}^{\underline{k}}y_{t}~(k\in \mathbb{N}).$$
\end{theorem}

\begin{example}
    \label{x2y2xy2y0}求微分方程 $x^2y''-2xy'+2y=0$ 的通解.
\end{example}
\begin{solution}
    令 $x=\e ^t,~\mathrm{D}=\displaystyle\dv{t}$, 那么原方程化为 $(\mathrm{D}-1)(\mathrm{D}-2)y=0$, 易得 $y=C_1x^2+C_2x.$
\end{solution}

\begin{example}
    求 Euler 方程 $\displaystyle x^2\dv[2]{y}{x}+4x\dv{y}{x}+2y=0~(x>0)$ 的通解.
\end{example}
\begin{solution}
    令 $x=\e ^t,~\mathrm{D}=\displaystyle\dv{t}$, 那么原方程化为 $(\mathrm{D}+1)(\mathrm{D}+2)y=0$, 易得 $y=\dfrac{C_1}{x}+\dfrac{C_2}{x^2}.$
\end{solution}

\begin{example}
    求方程 $\displaystyle x^3\dv[3]{y}{x}+x\dv{y}{x}-y=0$ 的通解.
\end{example}
\begin{solution}
    令 $x=\e ^t,~\mathrm{D}=\displaystyle\dv{t}$, 那么原方程化为 $(\mathrm{D}-1)^3y=0$, 令 $z_1=(\mathrm{D}-1)^2y$, 则有
    $$
        (\mathrm{D}-1)z=0\Rightarrow z_1=(\mathrm{D}-1)^2y=C_1\e ^{t}
    $$
    再令 $z_2=(\mathrm{D}-1)y$, 有 $$(\mathrm{D}-1)z_{2}=C_1\e ^{t}\Rightarrow z_2=(\mathrm{D}-1)y=C_1t\e ^{t}+C_2\e ^{t}$$
    进而得到 $$y=\e ^{\int \dd t}\qty[\int\qty(C_1t \e ^{t}+C_2\e ^{t})\cdot \e ^{-\int \dd  t}\dd  t+C_3]=\qty(C_1t^2+C_2t+C_3)\e ^{t}$$ 回代 $t=\ln x$, 得 $y=x\qty(C_1\ln^2x+C_2\ln x+C_3).$
\end{solution}

\subsubsection{非齐次 Euler 微分方程}

\begin{definition}[非齐次 Euler 微分方程]
    形如 $$x^ny^{(n)}+a_1x^{n-1}y^{(n-1)}+\cdots+a_{n-1}xy'+a_ny=f(x).$$
    的微分方程称为\textit{非齐次 Euler 微分方程}.
\end{definition}

\begin{example}
    求微分方程 $\displaystyle xy'+2y=4\ln x$ 的通解.
\end{example}
\begin{solution}
    $\displaystyle \text{原式} \xLongrightarrow[]{x=\e ^t}\frac{\mathrm{d}y}{\mathrm{d}t}+2y=4t\Rightarrow y=\e ^{-2\int\mathrm{d}t}\left[\int 4t\cdot\e ^{2\int\mathrm{d}t}\mathrm{d}t+C\right]=(2t-1)+C\e ^{-2t}=2\ln x-1+\frac{C}{x^2}.$
\end{solution}

\begin{example}
    求微分方程 $4x^4y'''-4x^3y''+4x^2y'=1$ 的通解.
\end{example}
\begin{solution}
    将原方程改写为 $x^3y'''-x^2y''+xy'=\dfrac{1}{4x}$, 并令 $x=\e ^t,~\mathrm{D}=\displaystyle\dv{x}$, 于是原式化为
    $$\mathrm{D}(\mathrm{D}-1)(\mathrm{D}-2)y-\mathrm{D}(\mathrm{D}-1)y+\mathrm{D}y=\mathrm{D}(\mathrm{D}-2)^2y=\dfrac{1}{4\e ^t}$$
    令 $(\mathrm{D}-2)^2y=z$, 那么 $z=\dfrac{1}{\mathrm{D}}\dfrac{1}{4\e ^t}=\displaystyle\int\dfrac{\dd t}{4\e ^t}=-\dfrac{1}{4}\e ^{-t}+C_1$, 再令 $w=(\mathrm{D}-2)y$, 于是,
    $(\mathrm{D}-2)w=w'-2w=-\dfrac{1}{4}\e ^{-t}+C_1$, 由一阶线性微分方程通解公式得
    $$w=\e ^{2\int\dd t}\qty[\int\qty(-\dfrac{1}{4}\e ^{-t}+C_1)\e ^{-2\int\dd t}\dd t+C_2]=\e ^{2t}\qty[\dfrac{1}{12}\e ^{-3t}-\dfrac{C_1}{2}\e ^{-2t}+C_2]=\dfrac{1}{12}\e ^{-t}+\dfrac{C_1}{2}+C_2\e ^{2t}$$
    又 $y'-2y=w$, 再由一阶线性微分方程通解公式得
    $$y=C_2x^2\ln x+C_1x^2-\dfrac{1}{36x}+C_3.$$
\end{solution}

\begin{example}
    求微分方程 $(1+x)^2y''-(1+x)y'+y=\dfrac{1}{1+x}$ 满足初值条件 $y(0)=y'(0)=0$ 的特解.
\end{example}
\begin{solution}
    令 $1+x=t$, 再令 $t=\e ^u,~\mathrm{D}=\displaystyle\dv{t}$, 于是原式化为
    $$\mathrm{D}(\mathrm{D}-1)y-\mathrm{D}y+y=\e ^{-u}\Rightarrow (\mathrm{D}-1)^2y=\e ^{-u}$$
    令 $z=(\mathrm{D}-1)y$, 于是上式化为 $z'-z=\e ^{-u}$, 由一阶线性微分方程通解公式得
    $$z=\e ^{\int\dd u}\qty[\int\e ^{-u}\cdot\e ^{-\int\dd u}\dd u+C_1]=\e ^u\qty[-\dfrac{1}{2}\e ^{-2u}+C_1]=-\dfrac{1}{2}\e ^{-u}+C_1\e ^u$$
    由已知得 $z(0)=\eval{-\dfrac{1}{2(1+x)}+C_1(1+x)}_{x=0}=0\Rightarrow C_1=\dfrac{1}{2}$, 又 $y'-y=-\dfrac{1}{2}\e ^{-u}+\dfrac{1}{2}\e ^u$, 再求一阶微分方程通解公式
    $$y=\e ^{\int\dd u}\qty[\int\qty(-\dfrac{1}{2}\e ^{-u}+\dfrac{1}{2}\e ^u)\e ^{-\int\dd u}\dd u+C_2]=\e ^u\qty[\dfrac{1}{4}\e ^{-2u}+\dfrac{1}{2}u+C_2]=\dfrac{1}{4}\e ^{-u}+\dfrac{1}{2}u\e ^u+C_2$$
    回代 $\e ^u=1+x$, 并 $y(0)=0$, 解得 $C_2=-\dfrac{1}{4}$, 于是该方程的特解为 $$y=\dfrac{x+1}{2}\ln(x+1)+\dfrac{1}{4}(x+1)-\dfrac{x+1}{4}.$$
\end{solution}

\begin{example}
    \label{x2y2xy2y2x3}求微分方程 $x^2y''-2xy'+2y=2x^3$ 的通解.
\end{example}
\begin{solution}
    \textbf{法一: }令 $\displaystyle x=\e ^t,~\mathrm{D}=\dv{t}$, 则 $xy'=\mathrm{D}y,~x^2y''=\mathrm{D}(\mathrm{D}-1)y$,
    原方程化为 $$\qty(\mathrm{D}^2-3\mathrm{D}+2)y=2\e ^{3t}=(\mathrm{D}-1)(\mathrm{D}-2)y=2\e ^{3t}$$
    令 $(\mathrm{D}-2)y=z$, 那么有 $\displaystyle(\mathrm{D}-1)z=\dv{z}{t}-z=2\e ^{3t}$, 为一阶线性微分方程, 则有通解
    $$z=\e ^{\int\dd t}\qty[\int 2\e ^{3t}\cdot\e ^{-\int\dd t}+C_1]=\e ^t\qty[\int 2\e ^{2t}\dd t+C_1]=\e ^{3t}+C_1\e ^t$$
    又 $z=(\mathrm{D}-2)y=\displaystyle\dv{y}{t}-2y=\e ^{3t}+C_1\e ^t$, 故又有通解公式
    $$y=\e ^{2\int\dd t}\qty[\int\qty(\e ^{3t}+C_1\e ^t)\cdot\e ^{-2\int\dd t}\dd t+C_2]=\e ^{2t}\qty[\e ^t-C_1\e ^{-t}+C_2]=\e ^{3t}-C_1\e ^t+C_2\e ^{2t}$$
    回代 $\e ^t=x$, 解得该方程的通解为 $y=x^3+C_2x^2-C_1x.$\\
    \textbf{法二: }设 $y=x^k$, 代入对应齐次线性微分方程 $$x^2y''-2xy'+2y=0$$
    得到 $k$ 满足的方程, 即特征方程为 $k^2-3k+2=0$, 解得 $k_1=1,~k_2=2$, 于是齐次 Euler 微分方程的通解为 $Y=C_1x+C_2x^2$,
    又因为 $f(x)=2x^3$ 于是 $f\qty(\e ^t)=2\e ^{3t}$, 而 3 不是特征根, 故可设其特解为 $y^*=a\e ^{3t}$, 即原方程的特解为 $y^*=ax^3$, 代入原方程解得 $a=1$, 所以原方程的通解为 $y=C_1x+C_2x^2+x^3.$
\end{solution}

\begin{example}
    求微分方程 $x^2y''-2xy'+2y+x-2x^3=0$ 的通解.
\end{example}
\begin{solution}
    \textbf{法一: }令 $x=\e ^t,~\mathrm{D}=\displaystyle\dv{t}$, 那么原方程化为
    $$(\mathrm{D}-1)(\mathrm{D}-2)y=2\e ^{3t}-\e ^t$$
    令 $(\mathrm{D}-2)y=z$, 那么有 $(\mathrm{D}-1)z=\displaystyle\dv{z}{t}-z=2\e ^{3t}-\e ^t$, 为一阶线性微分方程, 则有通解
    $$z=\e ^{\int\dd t}\qty[\int\qty(2\e ^{3t}-\e ^t)\e ^{-\int\dd t}\dd t+C_1]=\e ^t\qty[\e ^{2t}-t+C_1]=\e ^{3t}-t\e ^{t}+C_1\e ^t$$
    又 $z=(\mathrm{D}-2)y=\displaystyle\dv{y}{t}-2y=\e ^{3t}-t\e ^{t}+C_1\e ^t$, 故又有通解公式
    \begin{flalign*}
        y & =\e ^{2\int\dd t}\qty[\int\qty(\e ^{3t}-t\e ^t+C_1\e ^t)\e ^{-2\int\dd t}\dd t+C_2]=\e ^{2t}\qty[\e ^t+(t+1)\e ^{-t}-C_1\e ^{-t}+C_2] \\
          & =\e ^{3t}+(t+1)\e ^t-C_1\e ^t+C_2\e ^{2t}=x^3+C_2x^2+x\ln x+C_1'x
    \end{flalign*}
    其中 $C_1'=1-C_1.$\\
    \textbf{法二: }令 $y=x^k$, 代入对应齐次线性微分方程, 即例题 \ref{x2y2xy2y0}, 得通解为 $Y=C_1x^2+C_2x$,
    又 $f(x)=2x^3-x$ 即 $f\qty(\e ^t)=2\e ^{3t}-\e ^t$, 而 3 不是特征根, 但 1 是方程的一重特征根,
    故可设其特解为 $y^*=2A\e ^{3t}+Bt\e ^t$,
    即设原方程的特解为 $y^*=2Ax^3+Bx\ln x$, 代入原方程解得, $A=\dfrac{1}{2},~B=1$, 所以原方程的通解为
    $y=x^3+C_1x^2+C_1x+x\ln x.$
\end{solution}

\subsection{二阶变系数线性微分方程}

\subsubsection{二阶变系数齐次线性微分方程}

\begin{lemma}[Liouville 引理]
    \label{Liouville}设函数 $y_1(x)$ 为二阶变系数齐次线性微分方程 $$y''+P(x)y'+Q(x)y=0$$
    的一个非零解, 则该方程的与 $y_1(x)$ 线性无关的另一个特解为
    $$y_2(x)=y_1(x)\cdot\int\dfrac{\exp\qty(-\displaystyle\int P(x)\dd x)}{y_1^2(x)}\dd x.$$
\end{lemma}

\begin{example}[2016 数二]
    已知函数 $y_1(x)=\e ^x,~y_2(x)=u(x)\e ^x$ 是二阶微分方程 $$(2x-1)y''-(2x+1)y'+2y=0$$ 的两个解,
    若 $u(-1)=\e ,~u(0)=-1$, 求 $u(x)$, 并写出该微分方程的通解.
\end{example}
\begin{solution}
    由引理 $\ref{Liouville}$ 知, $u(x)=\displaystyle\int\dfrac{\exp\qty(\displaystyle\int\dfrac{2x+1}{2x-1}\dd x)}{\e ^{2x}}\dd x=C_1\int(2x-1)\e ^{-x}\dd x=-C_1\e ^{-x}(1+2x)+C_2$,
    由 $u(-1)=\e ,~u(0)=-1$, 解得 $C_1=1,~C_2=0$, 则 $u(x)=-\e ^{-x}(1+2x)$, 那么 $y_1$ 与 $y_2$ 线性无关, 则微分方程的通解为
    $$y=k_1\e ^x-k_2\e ^{-x}(1+2x)~  k_{1,2}\in\mathbb{R}.$$
\end{solution}

\begin{theorem}
    设有二阶变系数齐次线性微分方程 $p_1(x)y''+p_2(x)y'+p_3(x)y=0$ 那么 \label{erjiebianxis}
    \begin{enumerate}[label=(\arabic{*})]
        \item 当关于 $r$ 的一元二次方程 $r^2p_1(x)+rp_2(x)+p_3(x)=0$ 的两个根
              $$r_{1,2}=\dfrac{-p_2(x)\pm\sqrt{p_2^2(x)-4p_1(x)p_3(x)}}{2p_1(x)}$$
              中至少有一个 $r$ 是非零常数时, $y^*=\e ^{rx}$ 是该方程的一个特解;
              特别地, 当 $$p_1(x)\pm p_2(x)+p_3(x)=0$$
              时, $y^*=\e ^{\pm x}$ 是原方程的一个特解;
        \item 当 $\dfrac{p_2(x)+xp_3(x)}{p_3(x)}=a$ (其中 $a$ 为常数) 时,
              $y^*=x-a$ 是该方程的一个特解; 特别地, 当 $$p_2(x)+xp_3(x)=0$$ 时,
              $y^*=x$ 是该方程的一个特解;
        \item 当 $\alpha(\alpha-1)p_1(x)+\alpha xp_2(x)+x^2p_3(x)=0$ ($\alpha$ 为非零常数) 时,
              $y^*=x^\alpha$ 是该方程的一个特解;
        \item 当 $\dfrac{p_3(x)-p_1(x)}{p_2(x)}=-\cot x$, 时 $y^*=\sin x$ 是该方程的一个特解;
        \item 当 $\dfrac{p_3(x)-p_1(x)}{p_2(x)}=\tan x$, 时 $y^*=\cos x$ 是该方程的一个特解.
    \end{enumerate}
\end{theorem}

\begin{example}
    求微分方程 $(2x-1)y''-(2x+1)y'+2y=0$ 满足初值条件 $y(0)=0,~y'(0)=1$ 的特解.
\end{example}
\begin{solution}
    记 $p_1(x)=2x-1,~p_2(x)=-(2x+1),~p_3(x)=2$, 则有 $p_1(x)+p_2(x)+p_3(x)=0$, 且
    $$\dfrac{p_2(x)+xp_3(x)}{p_3(x)}=-\dfrac{1}{2}$$
    于是由定理 \ref{erjiebianxis} (1)(2) 可知 $y_1(x)=\e ^x,~y_2(x)=x+\dfrac{1}{2}$ 为原方程的两个特解,
    且它们线性无关, 于是方程的通解为 $y=C_1\e ^x+C_2\qty(x+\dfrac{1}{2})$, 又有初值条件 $y(0)=0,~y'(0)=1$ 代入解得
    $C_1=-1,~C_2=2$, 于是该方程的特解为 $y=2x+1-\e ^x.$
\end{solution}

\begin{example}
    求微分方程 $xy''+(x-2)y'-(2x+4)y=0$ 的通解.
\end{example}
\begin{solution}
    记 $p_1(x)=x,~p_2(x)=x-2,~p_3(x)=-(2x+4)$, 则由 $r^2p_1(x)+rp_2(x)+p_3(x)=0$ 解得
    \begin{flalign*}
        r_{1,2} & =\dfrac{-p_2(x)\pm\sqrt{p_2^2(x)-4p_1(x)p_3(x)}}{2p_1(x)}=\dfrac{2-x\pm\sqrt{(x-2)^2+4x(2x+4)}}{2x} \\
                & =\dfrac{2-x\pm\qty|3x+2|}{2x}=1+\dfrac{2}{x}\text{ 或 }-2
    \end{flalign*}
    于是由定理 \ref{erjiebianxis} (1) 知 $y_1(x)=\e ^{-2x}$ 是该方程的一个特解,
    再由引理 \ref{Liouville} 得与 $y_1(x)$ 线性无关的另一个特解为
    \begin{flalign*}
        y_2(x) & =y_1(x)\cdot\int\dfrac{\exp\qty(-\displaystyle\int P(x)\dd x)}{y_1^2(x)}\dd x=\e ^{-2x}\int\dfrac{\exp\qty(-\displaystyle\int\dfrac{x-2}{x}\dd x)}{\e ^{-4x}}\dd x \\
               & =\e ^{-2x}\int x^2\e ^{3x}\dd x=\e ^x\qty(\dfrac{1}{3}x^2-\dfrac{2}{9}x+\dfrac{2}{27})
    \end{flalign*}
    故由通解结构知原方程的通解为 $y=C_1\e ^{-2x}+C_2\e ^x\qty(\dfrac{1}{3}x^2-\dfrac{2}{9}x+\dfrac{2}{27}).$
\end{solution}

\begin{example}
    求微分方程的通解 $\qty(x^2\ln x)y''-xy'+y=0$ 的通解.
\end{example}
\begin{solution}
    记 $p_1(x)=x^2\ln x,~p_2(x)=-x,~p_3(x)=1$, 则 $p_2(x)+xp_3(x)=0$,
    于是由定理 \ref{erjiebianxis} (2) 知 $y_1(x)=x$ 是该方程的一个特解, 再由引理 \ref{Liouville} 得与 $y_1(x)$ 线性无关的另一个特解为
    \begin{flalign*}
        y_2(x) & =y_1(x)\cdot\int\dfrac{\exp\qty(-\displaystyle\int P(x)\dd x)}{y_1^2(x)}\dd x=x\int\dfrac{\exp\qty(\displaystyle\int\dfrac{x}{x^2\ln x}\dd x)}{x^2}\dd x \\
               & =x\int\dfrac{\ln x}{x^2}\dd x=-x\int\ln x\dd \qty(\dfrac{1}{x})=-(\ln x+1)
    \end{flalign*}
    故由通解结构知原方程的通解为 $y=C_1x+C_2(\ln x+1).$
\end{solution}

\begin{example}
    求微分方程 $y''\cos x-2y'\sin x+3y\cos x=0$ 的通解.
\end{example}
\begin{solution}
    记 $p_1(x)=\cos x,~p_2(x)=-2\sin x,~p_3(x)=3\cos x$, 则 $\dfrac{p_3(x)-p_1(x)}{p_2(x)}=-\cot x$,
    于是由定理 \ref{erjiebianxis} (4) 知, $y_1(x)=\sin x$ 是该方程的一个特解, 再由引理 \ref{Liouville} 得与 $y_1(x)$ 线性无关的另一个特解为
    \begin{flalign*}
        y_2(x) & =y_1(x)\cdot\int\dfrac{\exp\qty(-\displaystyle\int P(x)\dd x)}{y_1^2(x)}\dd x=\sin x\int\dfrac{\exp\qty(\displaystyle\int\dfrac{2\sin x}{\cos x}\dd x)}{\sin^2x}\dd x \\
               & =\sin x\int\dfrac{\dd x}{\sin^2x\cos^2x}=\sin x(\tan x-\cot x)
    \end{flalign*}
    故由通解结构知原方程的通解为 $y=C_1\sin x+C_2\sin x(\tan x-\cot x).$
\end{solution}

\subsubsection{二阶变系数非齐次线性微分方程}

\begin{example}
    试用定理 \ref{Liouville}, 求解例题 \ref{x2y2xy2y2x3} 的通解.
\end{example}
\begin{solution}
    记 $p_1(x)=x^2,~p_2(x)=-2x,~p_3(x)=2$, 则 $p_2(x)+xp_3(x)=0$ 且
    $$\alpha(\alpha-1)p_1(x)+\alpha xp_2(x)+x^2p_3(x)=\qty(\alpha^2-3\alpha+2)x^2$$
    令 $\qty(\alpha^2-3\alpha+2)=0$ 解得 $\alpha_1=1,~\alpha_2=2$, 于是 $y_1(x)=x,~y_2(x)=x^2$
    是对应齐次方程 $x^2y''-2xy'+2y=0$ 的两个线性无关的特解, 从而由通解结构知齐次方程的通解为
    $$Y=C_1x+C_2x^2$$
    下面利用常系数变易法求原方程的解, 将原方程改写为 $$y''-\dfrac{2}{x}y'+\dfrac{2}{x^2}y=2x$$
    令 $y=C_1(x)x+C_2(x)x^2$ 为原方程的解, 代入原方程, 且由
    $$\left\{\begin{matrix}
            C_2''(x) &   &          & = & 0 \\
            C_1''(x) & + & 2C_2'(x) & = & 2
        \end{matrix}\right.$$
    解得 $\begin{cases}
            C_1(x)=C_1-x^2 \\
            C_2(x)=C_2+2x
        \end{cases}$, 故所求通解为 $$y=\qty(C_1-x^2)x+(C_2+2x)x^2=C_1x+C_2x^2+x^3.$$
\end{solution}

\begin{example}
    已知 $y_1=\e ^x$ 是微分方程 $y''+\dfrac{x}{1-x}y'-\dfrac{1}{1-x}y=x-1$ 对应的齐次线性微分方程的一个特解,
    求所给二阶变系数非齐次线性微分方程的通解.
\end{example}
\begin{solution}
    记 $p_1(x)=1,~p_2(x)=\dfrac{x}{1-x},~p_3(x)=-\dfrac{1}{1-x}$, 则 $p_2(x)+xp_3(x)=0$,
    于是由定理 \ref{Liouville} (2) 知 $y_2(x)=x$ 是该方程的一个特解, 则该方程对应的齐次线性微分方程的通解为
    $$Y=C_1\e ^x+C_2x$$
    利用常系数变易法求原方程的同解为 $$y=-1-x-x^2+C_1\e ^x-C_2x.$$
\end{solution}
