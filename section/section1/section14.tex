\section{极限典型问题}

这一节讨论与极限相关联的三类典型问题,即极限的存在性问题、极限的局部逆问题和无穷小量及其阶的比较.

\subsection{极限的存在性问题}

讨论极限的存在性,是高等数学中既十分典型又经常遇到的问题,在研究非初等函数的连续性与可导性,
往往归结为这类问题.

\begin{example}
    设函数 $$f(x)=\begin{cases}
        \dfrac{\displaystyle \int_{0}^{x^2}\sqrt{1+\tan^2t}\dd t}{2x^2}&,x<0\\[6pt]
        \dfrac{\sqrt{1+x^2}-1}{\ln\qty(1+x^2)}&,x>0
    \end{cases}$$
    求极限 $\displaystyle\lim_{x\to0}f(x).$
\end{example}
\begin{solution}
    利用 L'Hospital 法则易求得在 $x=0$ 处的左右极限值,
    $$\lim_{x\to0^-}f(x)=\lim_{x\to0^-}\dfrac{\displaystyle\int_{0}^{x^2}\sqrt{1+\tan^2t}\dd t}{2x^2} \xlongequal{L'}\lim_{x\to0^-}\dfrac{2x\cdot\sqrt{1+\tan^2x^2}}{4x}=\dfrac{1}{2}$$
    并且 $$\lim_{xto0^+}f(x)=\lim_{x\to0^+}\dfrac{\sqrt{1+x^2}-1}{\ln\qty(1+x^2)}=\dfrac{1}{2}$$
    由 $f(0^-)=f(0^+)=\dfrac{1}{2}$,故 $\displaystyle \lim_{x\to0}f(x)=\dfrac{1}{2}.$
\end{solution}

\subsection{极限的局部逆问题}

如果已知函数的极限存在,但是在函数的表达式中含有一个 (或多个) 待定的参数,要求确定待定参数的值,
这就是所谓的函数极限的局部逆问题.

% \begin{example}
%     试确定常数 $a,b,c$ 的值,使 $\displaystyle\lim_{x\to0}\dfrac{ax-\sin x}{\displaystyle\int_{b}^{x}\dfrac{\ln\qty(1+t^3)}{t}\dd t}=c\neq0.$
% \end{example}

\begin{example}[2018 数二]
    若 $\displaystyle\lim_{x\to0}\qty(\e^x+ax^2+bx)^{x^{-2}}=1$,则
    \begin{tasks}(4)
        \task $a=\dfrac{1}{2},~b=-1$
        \task $a=-\dfrac{1}{2},~b=-1$
        \task $a=\dfrac{1}{2},~b=1$
        \task $a=-\dfrac{1}{2},~b=1$
    \end{tasks}
\end{example}
\begin{solution}
    由题设条件 $\displaystyle\lim_{x\to0}\qty(\e^x+ax^2+bx)^{x^{-2}}=\exp\lim_{x\to0}\dfrac{1}{x^2}\ln\qty(\e^x+ax^2+bx)=1$,
    于是有 $\displaystyle\lim_{x\to0}\dfrac{\ln\qty(\e^x+ax^2+bx)}{x^2}=0$,即
    \begin{flalign*}
        \lim_{x\to0}\dfrac{\ln\qty(\e^x+ax^2+bx)}{x^2} & =\lim_{x\to0}\dfrac{\e^x+ax^2+bx+1}{x^2}=\lim_{x\to0}\dfrac{1+x+\dfrac{1}{2}x^2+ax^2+bx+1+o\qty(x^2)}{x^2} \\
                                                       & =\lim_{x\to0}\qty[\qty(\dfrac{1}{2}+a)+\dfrac{b+1}{x}+\dfrac{2}{x^2}]=0\Rightarrow\begin{cases}
                                                                                                                                               \dfrac{1}{2}+a=0 \\[6pt]1+b=0
                                                                                                                                           \end{cases}
    \end{flalign*}
    故解得选 B.
\end{solution}

\begin{example}[2018 数一]
    若 $\displaystyle\lim_{x\to0}\qty(\dfrac{1-\tan x}{1+\tan x})^{\frac{1}{\sin kx}}=\e$,求 $k.$
\end{example}
\begin{solution}
    $\displaystyle \exp\lim_{x\to0}\dfrac{1}{\sin kx}\ln\qty(\dfrac{1-\tan x}{1+\tan x})=\exp\lim_{x\to0}\dfrac{\ln(1-\tan x)-\ln(1+\tan x)}{\sin kx}=\exp\lim_{x\to0}\dfrac{-2\tan x}{\xi_x \cdot \sin kx}=\e$,
    于是 $$\lim_{x\to0}\dfrac{-2\tan x}{\xi_x\cdot\sin kx}=\lim_{x\to0}\dfrac{-2x}{\xi_xk}=1\Rightarrow k=-2$$ 
    其中 $\xi_x\to1~ (x\to0).$
\end{solution}

\begin{example}
    试确定常数 $A,B,C$,使下式当 $x\to0 $ 时成立:
    $$\dfrac{\e^{\sin x}}{\sin x}=\dfrac{1+Bx+Cx^2}{x+Ax^2}+o\qty(x^2).$$
\end{example}
\begin{solution}
    将所给等式两边同时乘以 $(1+Ax)\sin x$,并注意到 $x\to0$ 时,$(1+Ax)\sin x\cdot o\qty(x^2)=o\qty(x^3)$,得
    $$(1+Ax)\e^{\sin x}=\dfrac{\sin x}{x}\qty(1+Bx+Cx^2)+o\qty(x^3)$$
    将 $\e^{\sin x},\dfrac{\sin x}{x}$ 分别展开到 $x$ 的三阶,于是有
    $$\e^{\sin x}=1+x+\dfrac{1}{2}x^2+o\qty(x^2),~\dfrac{\sin x}{x}=1-\dfrac{1}{6}x^2+o\qty(x^2)$$
    代入上式,得
    \begin{flalign*}
        \qty(1+x+\dfrac{1}{2}x^2)(1+Ax)=\qty(1-\dfrac{1}{6}x^2)\qty(1+Bx+Cx^2)+o\qty(x^3) \\
        (A+1-B)x+\qty(A-C+\dfrac{2}{3})x^2+\qty(\dfrac{A}{2}+\dfrac{B}{6})x^3=o\qty(x^3)
    \end{flalign*}
    欲使上式成立,必须有 $\begin{cases}
            A+1-B=0 \\[6pt]A-C+\dfrac{1}{3}=0\\[6pt]\dfrac{A}{2}+\dfrac{B}{6}=0
        \end{cases}$,联立解得 $A=-\dfrac{1}{4},~B=\dfrac{3}{4},~C=\dfrac{5}{12}.$
\end{solution}

\subsection{无穷小量及其阶的比较}

有关无穷小量的概念可参考定义 \ref{infinitesimalDefinitions}.

\begin{example}[2019 数一]
    当 $x\to0$ 时,若 $x-\tan x$ 与 $x^{k}$ 是同阶无穷小,则 $k$ 等于 
    \begin{tasks}(4)
        \task 1
        \task 2
        \task 3
        \task 4
    \end{tasks}
\end{example}
\begin{solution}
    由于当 $x\to0$ 时,$x-\tan x\sim-\dfrac{1}{3}x^3$,则 $\displaystyle\lim_{x\to0}\dfrac{x-\tan x}{x^3}=-\dfrac{1}{3}$,所以 $k=3$,选 C.
\end{solution}

\begin{example}
    设 $x\to0$ 时,$\e ^{x\cos x^2}-\e ^{x}$ 与 $x^n$ 是同阶无穷小,则 $n $ 为
    \begin{tasks}(4)
        \task 5
        \task 4
        \task 3
        \task 2
    \end{tasks}
\end{example}
\begin{solution}
    因为 $\e ^{x\cos x^2}-\e ^{x}=\e^x\qty[\e ^{x\qty(\cos x^2-1)}-1]\sim x\qty(\cos x^2-1)\sim -\dfrac{1}{2}x^5$,所以 $n=5$,故选 A
\end{solution}

\begin{example}[2001 数二]
    设当 $x\to0 $ 时,$(1-\cos x)\ln\qty(1+x^2)$ 是比 $x\sin x^n$ 高阶的无穷小,而 $x\sin x^n$ 是比 $\e ^{x^2}-1$ 高阶的无穷小,则正整数 $n$ 
    \begin{tasks}(4)
        \task 1
        \task 2
        \task 3
        \task 4
    \end{tasks}
\end{example}
\begin{solution}
    因为 $(1-\cos x)\ln\qty(1+x^2)\sim\dfrac{1}{2}x^4,~x\sin x^n\sim x^{n+1},~\e ^{x^2}-1\sim x^2$,由题意知 $4>n+1>2$,解得 $n=2$,故选 B.
\end{solution}