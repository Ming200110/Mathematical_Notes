\begin{flushright}
    \begin{tabular}{r|}
        \textit{“迟序之数,非出神怪,有形可检,有数可推。”}\\
        ——\textit{祖冲之}
    \end{tabular}
\end{flushright}

函数是微积分讨论的主要对象,它以极限理论为基础,在研究函数时我们总是通过函数值 $f(x)$
的变化来看函数关系的性质,所以应该用运动变化的观点来掌握函数.
极限与函数的连续性理论是微积分的基础,如何用已知的、可求的来逼近未知的、要求的,
用有限来逼近无限,在无限变化的过程中考察变量的变化趋势,从有限过渡到无限,这是本章需掌握的基本思想.