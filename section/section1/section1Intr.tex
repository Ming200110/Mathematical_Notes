\begin{flushright}
    \begin{tabular}{r|}
        \textit{“迟序之数, 非出神怪, 有形可检, 有数可推. ”}\\
        ——\textit{祖冲之}
    \end{tabular}
\end{flushright}

函数、数列和极限是数学中重要的概念, 它们在分析、微积分、数学分析等领域都有广泛的应用. 下面简要介绍这几个概念: 

1. 函数: 函数是一种映射关系, 它将一个集合中的每个元素映射到另一个集合中的唯一元素. 在数学中, 我们通常用 $f(x)$ 或 $y = f(x)$ 表示函数, 其中 $x$ 是自变量, $y$ 是因变量. 函数可以是线性的、多项式的、三角函数、指数函数、对数函数等各种形式. 函数的性质包括定义域、值域、奇偶性、周期性等. 

2. 数列: 数列是按照一定规律排列的一组数的序列. 数列可以是有限的, 也可以是无限的. 常见的数列有等差数列、等比数列、斐波那契数列等. 数列的极限是指当数列的项趋向于某个值时, 这个值称为数列的极限. 数列的极限可以是有限的, 也可以是无穷的. 

3. 极限: 极限是数学中一个重要的概念, 用来描述函数或数列在某个点或无穷远处的“接近程度”. 对于函数 $f(x)$, 当 $x$ 趋近于某个值 $a$ 时, 如果 $f(x)$ 的取值趋近于某个确定的值 $L$, 则称 $L$ 是函数 $f(x)$ 在 $x$ 趋近于 $a$ 时的极限, 记作 $\displaystyle\lim_{x \to a} f(x) = L$. 类似地, 对于数列 $\{a_n\}$, 当 $n$ 趋向于无穷大时, 如果数列的项趋近于某个确定的值 $L$, 则称 $L$ 是数列 $\{a_n\}$ 的极限, 记作 $\displaystyle\lim_{n \to \infty} a_n = L$. 

函数、数列和极限是数学中基础而重要的概念, 它们在理论研究和实际问题求解中都有着重要的作用.  
