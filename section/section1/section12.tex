\section{极限的概念、性质及存在准则}

极限是微积分中非常重要的概念, 它描述了函数在某一点或无穷远处的趋势或取值. 极限的性质包括唯一性、局部性、保号性、保序性和四则运算法则等. 极限的存在准则有夹逼准则、单调有界准则等. 

\subsection{数列、函数极限的定义}

\subsubsection{数列的极限定义}

\begin{definition}[数列极限 A]
    设 $\qty{a_n}$ 是一数列, 如果存在常数 $a$, 当 $n$ 无限增大时, $a_n$ 无限接近 (或趋近) 于 $a$, 则称数列 $\qty{a_n}$ \textit{收敛}, $a$ 称为\textit{数列} $\qty{a_n}$ \textit{的极限}, 
    或称数列 $\qty{a_n}$ 收敛于 $a$, 记作 $\displaystyle\lim_{n\to\infty}a_n=a$, 或 $a_n\to\infty,~n\to\infty$.
    当 $n\to\infty$ 时, 若不存在这样的常数 $a$, 则称数列 $\qty{a_n}$ \textit{发散} 或 \textit{不收敛}, 也可以说极限 $\displaystyle\lim_{n\to\infty}a_n$ 不存在.
    \index{收敛}
    \index{数列的极限}
    \index{发散}
    \index{不收敛}
\end{definition}

\begin{definition}[数列极限 B]
    设 $\qty{a_n}$ 为一数列, $a$ 为一个常数, 若对任意给定的 $\varepsilon>0$, 都存在一个正整数 $N$, 使得当 $n>N$ 时, 有 $|a_n-a|<\varepsilon$, 则称 $a$ 为数列 $\qty{a_n}$ 的极限, 记作 $\displaystyle\lim_{n\to\infty}a_n=a.$
\end{definition}

\begin{example}[2003 数一]
    设 $\qty{a_n},\qty{b_n},\qty{c_n}$ 均为非负整数, 且 $\displaystyle\lim_{n\to\infty}a_n=0,~\lim_{n\to\infty}b_n=1,~\lim_{n\to\infty}c_n=+\infty$, 则必有
    \begin{tasks}(2)
        \task $a_n<b_n$ 对任意 $n$ 成立
        \task $b_n<c_n$ 对任意 $n$ 成立
        \task 极限 $\displaystyle\lim_{n\to\infty}a_nc_n$ 不存在
        \task 极限 $\displaystyle\lim_{n\to\infty}b_nc_n$ 不存在
    \end{tasks}
\end{example}
\begin{solution}
    取 $a_n=\dfrac{2}{n},~b_n=1,~c_n=\dfrac{n}{2}$, 则可排除选项 A、B、C, 因此选 D.
\end{solution}

\begin{example}[2014 数三]
    设 $\displaystyle\lim_{n\to\infty}a_n=0\neq 0$, 则当 $n$ 充分大时有
    \begin{tasks}(4)
        \task $|a_n|>\dfrac{|a|}{2}$
        \task $|a_n|<\dfrac{|a|}{2}$
        \task $a_n>a-\dfrac{1}{n}$
        \task $a_n>a+\dfrac{1}{n}$
    \end{tasks}
\end{example}
\begin{solution}
    因为 $\displaystyle\lim_{n\to\infty}a_n=a\neq 0$, 所以 $\forall\varepsilon>0$, 都存在一个正整数 $N$, 使得当 $n>N$ 时, 有 $|a_n-a|<\varepsilon$, 
    即 $a-\varepsilon<a_n<a+\varepsilon$, 则 $|a|-\varepsilon<|a_n|\leqslant |a|+\varepsilon$, 取 $\varepsilon=\dfrac{|a|}{2}$, 得 $|a_n|>\dfrac{|a|}{2}$, 选 A.
\end{solution}

\subsubsection{函数的极限定义}

\begin{definition}[函数的极限]
    设函数 $f(x)$ 在点 $x_0$ 的邻域内 (点 $x_0$ 可除外) 有定义, $A$ 为一个常数, 若对任意给定的 $\varepsilon>0$, 都存在一个正数 $\delta$, 
    当 $0<|x-x_0|<\delta$ 时, 有 $|f(x)-A|<\varepsilon$, 则称 $A$ 为\textit{函数} $f(x)$ 当 $x\to x_0$ \textit{时的极限}, 记作 $\displaystyle\lim_{x\to x_0}f(x)=A.$
    \index{函数的极限}
\end{definition}

\begin{definition}[左右极限]
    若对于满足 $0<x_0-x<\delta~~(0<x-x_0<\delta)$ 的一切 $x$ 所对应的 $f(x)$ 都不满足 $|f(x)-A|<\varepsilon$, 则称 $A$ 为函数 $f(x)$ 当 $x$ 自 $x_0$ 左 (右) 侧趋于 $x_0$ 时的极限, 即\textit{左 (右) 极限}, 
    分别记作 $$\lim_{x\to x_0^-}f(x)=f(x_0-0)=A~~(\lim_{x\to x_0^+}f(x)=f(x_0+0)=A)$$
    类似地, 可以给出当 $x\to\infty,~x\to+\infty,~x\to-\infty$ 时, $f(x)$ 的极限为 $A$ 的定义.
    \index{左极限}
    \index{右极限}
\end{definition}

\begin{example}
    设 $f(x)$ 在 $[a,+\infty)$ 连续, 则 "$\exists x_n\in[a,+\infty)$ 有 $\displaystyle\lim_{n\to\infty}x_n=+\infty$ 且 $\displaystyle\lim_{n\to\infty}f(x_n)=\infty$" 是 $f(x)$ 在 $[a,+\infty)$ 无界的
    \begin{tasks}(2)
        \task 充分非必要条件
        \task 必要非充分条件
        \task 充分必要条件
        \task 既非充分也非必要条件
    \end{tasks}
\end{example}
\begin{solution}
    题目中的两个条件分别为
    \begin{enumerate}[label=(\roman{*})]
        \item $\exists x_n\in [a,+\infty)$ 有 $\displaystyle\lim_{n\to\infty}x_n=+\infty$ 且 $\displaystyle\lim_{n\to\infty}f(x_n)=\infty$ ;
        \item $f(x)$ 在 $[a,+\infty)$ 无界, 
    \end{enumerate}
    \textbf{讨论充分性, 即 (i)$\to$(ii)}\\
    因为 $\displaystyle\lim_{n\to\infty}f(x_n)=\infty$ 可知对于 $\forall M>0,\exists N\in [a,+\infty)$ 使得 $|f(x_N)|>M$, 故 $f(x)$ 在 $[a,+\infty)$ 无界;\\
    \textbf{讨论必要性, 即 (ii)$\to$(i)}\\
    因为 $f(x)$ 在 $[a,+\infty)$ 无界, 则对 $\forall M_1>0,\exists X_1\in [a,+\infty)$, 使得 $|f(X_1)|>M_1$, 且 $\exists X_2\in [X_1,+\infty)$, 使得 $|f(X_2)|>|f(X_1)|$, 同理可取 $x_1$, 
    $\exists x_2\in [x_1,+\infty)$, 使得 $|f(x_2)|>|f(x_1)|$; $\exists x_3\in [x_2,+\infty)$, 使得 $|f(x_3)|>|f(x_2)|$, 由此递推, 
    得存在严格 $\nearrow$ 的数列 $\qty{x_n}$, 有 $\displaystyle\lim_{n\to\infty}x_n=\infty$, 且存在单调的 $|f(x_n)|$, 有 $\displaystyle\lim_{n\to\infty}|f(x_n)|=+\infty$, 即 $\displaystyle\lim_{n\to\infty}f(x_n)=\infty.$
\end{solution}

\subsection{数列与其子列极限之间的关系}

\begin{theorem}[子列极限定理]
    \label{theRelationshipBetweenASequenceAndItsSubColumnLimits}
    $\displaystyle\lim_{n\to\infty}x_n=a\Leftrightarrow \lim_{n\to\infty}x_{2n}=\lim_{n\to\infty}x_{2n-1}\Leftrightarrow\lim_{n\to\infty}x_{3n}=\lim_{n\to\infty}x_{3n+1}=\lim_{n\to\infty}x_{3n+2}=a.$
    \index{子列极限定理}
\end{theorem}

\begin{example}[2015 数三]
    设 $\qty{x_n}$ 是数列, 下列命题中不正确的是 
    \begin{tasks}(1)
        \task 若 $\displaystyle\lim_{n\to\infty}x_n=a$, 则 $\displaystyle\lim_{n\to\infty}x_{2n}=\lim_{n\to\infty}x_{2n+1}=a$
        \task 若 $\displaystyle\lim_{n\to\infty}x_{2n}=\lim_{n\to\infty}x_{2n+1}=a$, 则 $\displaystyle\lim_{n\to\infty}x_n=a$
        \task 若 $\displaystyle\lim_{n\to\infty}x_n=a$, 则 $\displaystyle\lim_{n\to\infty}x_{3n}=\lim_{n\to\infty}x_{3n+1}=a$
        \task 若 $\displaystyle\lim_{n\to\infty}x_{3n}=\lim_{n\to\infty}x_{3n+1}=a$, 则 $\displaystyle\lim_{n\to\infty}x_n=a$
    \end{tasks}
\end{example}
\begin{solution}
    如 $x_{3n}=1+\dfrac{1}{3n},~x_{3n+1}=1+\dfrac{1}{3n+1},~x_{3n+2}=2+\dfrac{1}{3n+2}$, 则 $\displaystyle\lim_{n\to\infty}x_{3n}=1,~\lim_{n\to\infty}x_{3n+1}=1$, 但 $\displaystyle\lim_{n\to\infty}x_{3n+2}=2$, 故 $\displaystyle\lim_{n\to\infty}\neq 1$, 故选 D.
\end{solution}

\begin{example}
    设 $x_n=(-1)^{n}\cdot\dfrac{n+1}{n}$, 证明: 数列 $\qty{x_n}$ 发散.
\end{example}
\begin{proof}[{\songti \textbf{证}}]
    考察子列
    $$x_{2n}=\dfrac{2n+1}{2n}=1+\dfrac{1}{2n}\to 1~~(n\to\infty),~x_{2n+1}=-\dfrac{2n+2}{2n+1}=-1-\dfrac{1}{2n+1}\to-1~~(n\to\infty)$$
    由定理 \ref{theRelationshipBetweenASequenceAndItsSubColumnLimits} 可知 $\displaystyle\lim_{n\to\infty}x_n$ 不存在, 即得证数列 $\qty{x_n}$ 发散.
\end{proof}

\subsection{数列、函数极限的性质}

\subsubsection{数列极限的性质}

\begin{theorem}[数列极限的唯一性]
    收敛数列的极限是唯一的, 即若数列 $\qty{a_n}$ 收敛, 且 $\displaystyle\lim_{n\to\infty}a_n=a$ 和 $\displaystyle\lim_{n\to\infty}a_n=b$, 则 $a=b$.
    \index{数列极限的唯一性}
\end{theorem}

\begin{theorem}[数列极限的有界性]
    设数列 $\qty{a_n}$ 收敛, 则数列 $\qty{a_n}$ 有界, 即存在常数 $M>0$, 使得 $|a_n|<M~(\forall n\in N)$.
    \index{数列极限的有界性}
\end{theorem}

\begin{theorem}[数列极限的保号性]
    设数列 $\qty{a_n}$ 收敛, 其极限为 $a$
    \begin{enumerate}[label=(\arabic{*})]
        \item 若有正整数 $N$, 使得当 $n>N$, 有 $a_n>0$ (或 $<0$), 则 $a\geqslant 0$ (或 $\leqslant 0$).
        \item 若 $a>0$ (或 $<0$), 则有正整数 $N$, 使得当 $n>N$, 时, 有 $a_n>0$ (或 $<0$).
    \end{enumerate}
    \index{数列极限的保号性}
\end{theorem}

\begin{example}[2017 数二]
    设数列 $\qty{x_n}$ 收敛, 则 
    \begin{tasks}(2)
        \task 当 $\displaystyle\lim_{n\to\infty}\sin x_n=0$ 时, $\displaystyle\lim_{n\to\infty}x_n=0$
        \task 当 $\displaystyle\lim_{n\to\infty}\qty(x_n+\sqrt{|x_n|})=0$ 时, $\displaystyle\lim_{n\to\infty}x_n=0$
        \task 当 $\displaystyle\lim_{n\to\infty}\qty(x_n+x_n^2)=0$ 时, $\displaystyle\lim_{n\to\infty}x_n=0$
        \task 当 $\displaystyle\lim_{n\to\infty}(x_n+\sin x_n)=0$ 时, $\displaystyle\lim_{n\to\infty}x_n=0$
    \end{tasks}
\end{example}
\begin{solution}
    因为数列 $\qty{x_n}$ 收敛, 故令 $\displaystyle\lim_{n\to\infty}x_n=a$, 则有 A 知 $\sin a=0\not\Rightarrow a=0$, 同理 B、C 不正确, 而由 D 可知 $\sin a=-a\Rightarrow a=0$, 故选 D.
\end{solution}

\subsubsection{函数极限的性质}

\begin{theorem}[函数极限的唯一性]
    若 $\displaystyle\lim_{x\to x_0}f(x)=A$, 则 $A$ 必唯一.
    \index{函数极限的唯一性}
\end{theorem}

\begin{theorem}[函数极限的有界性]
    若 $\displaystyle\lim_{x\to x_0}f(x)=A$, 则 $f(x)$ 在点 $x_0$ 的某一去心邻域内有界.
    \index{函数极限的有界性}
\end{theorem}

\begin{theorem}[函数极限的保号性]
    设 $f(x)$ 在 $x_0$ 的某去心邻域内均有 $f(x)\geqslant 0$ (或 $f(x)\leqslant 0$), 且 $\displaystyle\lim_{x\to x_0}f(x)=A$, 则 $a\geqslant 0$ (或 $A\leqslant 0$).
    \index{函数极限的保号性}
\end{theorem}

\begin{theorem}[函数极限的充要条件]
    \begin{enumerate}[label=(\arabic{*})]
        \item $\displaystyle\lim _{x \rightarrow x_{0}} f(x)=A \Leftrightarrow \lim _{x \rightarrow x_{0}^{-}} f(x)=\lim _{x \rightarrow x_{0}} f(x)=A $.
        \item $\displaystyle\lim _{x \rightarrow \infty} f(x)=A \Leftrightarrow \lim _{x \rightarrow-\infty} f(x)=\lim _{x \rightarrow+\infty} f(x)=A .$
    \end{enumerate}
    \index{函数极限的充要条件}
\end{theorem}

\subsubsection{证明函数 $ f(x) $ 的极限不存在的方法}

\begin{enumerate}[label=(\arabic{*})]
    \item 若 $ f\left(x_{0}-0\right) \neq f\left(x_{0}+0\right)$, 则 $\displaystyle \lim _{x \rightarrow x_{0}} $ 不存在. 当 $ x \rightarrow \infty $ 时, 对含有 $ a^{x}(a>0, a \neq 1) $ 或 $ \arctan x $ 或 $ \arccot x $ 的函数极限, 一定要对 $ x \rightarrow+\infty $ 与 $ x \rightarrow-\infty $ 分别求极限, 若两者的极限值相等,则 $ x \rightarrow \infty $ 时极限存在, 否则不存在.
    \item 若存在数列 $ \left\{x_{n}\right\}: x_{n} \rightarrow x_{0}, x_{n} \neq x_{0} $, 使得 $\displaystyle \lim _{n \rightarrow \infty} f\left(x_{n}\right) $ 不存在; 或有两个数列 $ \left\{x_{n}\right\} $ 与 $ \left\{y_{n}\right\} $, 满足 $ x_{n} \rightarrow x_{0}\left(x_{n} \neq x_{0}\right), y_{n} \rightarrow y_{0}\left(y_{n} \neq y_{0}\right) $ 使得 $ \displaystyle\lim _{n \rightarrow \infty} f\left(x_{n}\right) \neq \lim _{n \rightarrow \infty} f\left(y_{n}\right) $, 则 $ \displaystyle\lim _{x \rightarrow x_{0}} f(x) $ 不存在.
    \item 利用结论: 设 $\displaystyle \lim _{x \rightarrow x_{0}} f(x)=A, \lim _{x \rightarrow x_{0}} g(x) $ 不存在, 则 $\displaystyle \lim _{x \rightarrow x_{0}}[f(x)+g(x)] $ 不存在; 若又有 $ A \neq 0 $, 则 $ \displaystyle\lim _{x \rightarrow x_{0}} f(x) g(x) $ 不存在.
\end{enumerate}

\subsection{数列、函数极限的存在准则}

\subsubsection{数列极限的存在准则}

\begin{theorem}[数列的夹逼准则]
    \label{pinchGuidelines}
    若 $\exists N$, 使得当 $n>N$ 时有 $y_n\leqslant x_n\leqslant z_n$, 且 $\displaystyle\lim_{n\to\infty}y_n=\lim_{n\to\infty}z_n=a$, 则 $\displaystyle\lim_{n\to\infty}x_n=a.$
    \index{数列的夹逼准则}
\end{theorem}

\begin{theorem}[数列的单调有界准则]
    若数列 $\qty{x_n}$ 单调上升有上界 (或单调下降有下界), 即 $x_{n+1}\leqslant x_n~(\text{或 }x_{n+1}\geqslant x_n)~(n=1,2,\cdots)$, 并存在一个数 $M~ (m)$ 使得对一切 $n$ 有 $x_n\leqslant M~(\text{或 }x_n\geqslant m)$, 则 $\qty{x_n}$ 收敛, 即存在一个数 $a$, 使得 $\displaystyle\lim_{n\to\infty}x_n=a$, 且有 $x_n\leqslant a~(\text{或 }x_n\geqslant a)~(n=1,2,\cdots)$.
    \index{数列的单调有界准则}
\end{theorem}

\begin{example}[2008 数一]
    设函数 $f(x)$ 在 $(-\infty,+\infty)$ 内单调有界, $\qty{x_n}$ 为数列, 下列命题正确的是 
    \begin{tasks}(2)
        \task 若 $\qty{x_n}$ 收敛, 则 $\qty{f(x_n)}$ 收敛
        \task 若 $\qty{x_n}$ 单调, 则 $\qty{f(x_n)}$ 收敛
        \task 若 $\qty{f(x_n)}$ 收敛, 则 $\qty{x_n}$ 收敛
        \task 若 $\qty{f(x_n)}$ 单调, 则 $\qty{x_n}$ 收敛
    \end{tasks}
\end{example}
\begin{solution}
    若 $\qty{x_n}$ 单调, $f(x)$ 单调有界, 则数列 $\qty{f(x_n)}$ 单调有界, 因此数列 $\qty{f(x_n)}$ 收敛, 故选 B.
\end{solution}

\subsubsection{函数极限的存在准则}

\begin{theorem}[函数的夹逼准则]
    若 $\exists\delta>0$, 使得当 $0<|x-x_0|<\delta$ 时有 $h(x)\leqslant f(x)\leqslant g(x)$, 且 $\displaystyle \lim_{x \to x_0}h(x)=\lim_{x \to x_0}g(x)=A$, 则 $\displaystyle \lim_{x \to x_0}f(x)=A.$\index{函数的夹逼准则}
\end{theorem}

\begin{example}[2000 数三]
    设对任意 $x$, 总有 $\varphi(x)\leqslant f(x)\leqslant g(x)$, 且 $\displaystyle\lim_{x\to\infty}[g(x)-\varphi(x)]=0$, 则 $\displaystyle\lim_{x\to\infty}f(x)$
    \begin{tasks}(4)
        \task 存在且等于零
        \task 存在但不一定为零
        \task 一定不存在
        \task 不一定存在
    \end{tasks}
\end{example}
\begin{solution}
    本题中所给条件比夹逼准则的条件弱, 事实上, 若 $\displaystyle\lim_{x\to\infty}g(x)=\lim_{x\to\infty}\varphi(x)=A$ (有限), 则必有 $\displaystyle\lim_{x\to\infty}[g(x)-\varphi(x)]=0$, 反之则不然, 因为当 $\displaystyle\lim_{x\to\infty}[g(x)-\varphi(x)]=0$ 时, 
    极限 $\displaystyle\lim_{x\to\infty}g(x)$ 和 $\displaystyle\lim_{x\to\infty}\varphi(x)$ 可以都不存在, 如 $g(x)=\varphi(x)=x$.
    \begin{enumerate}[label=(\arabic{*})]
        \item 若取 $\varphi(x)=x-\dfrac{1}{x^2},~f(x)=x,~g(x)=x+\dfrac{1}{x^2}$, 显然有 $\varphi(x)\leqslant f(x)\leqslant g(x)$, 且 $\displaystyle\lim_{x\to\infty}[g(x)-\varphi(x)]=0$, 但 $\displaystyle\lim_{x\to\infty}f(x)$ 不存在, 则排除选项 A、B.
        \item 若取 $\varphi(x)=f(x)=g(x)=1$, 满足题设条件, 但 $\displaystyle\lim_{x\to\infty}f(x)=1$ 存在, 则排除选项 C, 故选 D.
    \end{enumerate}
\end{solution}