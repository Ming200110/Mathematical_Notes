\section{协方差与相关系数}

\subsection{随机变量的独立性与相关性}

\begin{definition}[随机变量的相关性定义]
    若随机变量 $ X $ 和 $ Y $ 的相关系数 $ \rho_{X Y}=0 $ 则称 $ X $ 和 $ Y $ 不相关, 否则称 $ X $ 和 $ Y $ 相关.
\end{definition}

\begin{theorem}[独立性与相关性的判定]
    随机变量 $X$ 和 $Y$ 不相关, 即 $\rho_{XY}=0\Leftrightarrow \cov(X,Y)=0\Leftrightarrow E(XY)=E(X)E(Y)\Leftrightarrow D(X\pm Y)=D(X)\pm D(Y)\Leftarrow X\text{ 与 }Y\text{ 相互独立}$.
    \newline
    特别地, 当 $(X,Y)\sim N(\mu_1,\mu_2;\sigma_1^2,\sigma_2^2;\rho)$ 时, $X$ 和 $Y$ 不相关 $\Leftrightarrow X$ 和 $Y$ 相互独立;\\
    当随机变量 $X$ 和 $Y$ 都服从 (0,1) 分布, 则 $X$ 和 $Y$ 不相关 $\Leftrightarrow X$ 和 $Y$ 相互独立.
\end{theorem}

\subsection{协方差}

\begin{theorem}[协方差的性质]
    \begin{enumerate}[label=(\arabic{*})]
        \item $\cov(X,Y)=\cov(Y,X)$, $\cov(X,X)=D(X)$;
        \item $\cov(aX,bY)=ab\cov(X,Y)$, $a,b$ 是常数, $\cov(X,a)=0$;
        \item $\cov(X_1+X_2,Y)=\cov(X_1,Y)+\cov(X_2,Y)$;
        \item 若 $X,Y$ 相互独立, 则 $\cov(X,Y)=0$, 反之不对.
        \item $\cov(X,Y)=E(XY)-E(X)E(Y)$.
    \end{enumerate}
\end{theorem}

\begin{example}[2002 数三]
    设随机变量 $X$ 和 $Y$ 的联合概率分布为表 \ref{-101007018015}, 求协方差 $\cov\qty(X^2,Y^2).$
\end{example}
\begin{solution}
    \begin{minipage}{0.3\linewidth}
        \begin{table}[H]
            \centering
            \caption*{联合概率分布表}
            \label{-101007018015}
            \begin{tabular}{c | c c c}
                $X^{\displaystyle\setminus Y}$ & -1   & 0    & 1    \\
                \midrule
                0                              & 0.07 & 0.18 & 0.15 \\
                1                              & 0.08 & 0.32 & 0.20
            \end{tabular}
        \end{table}
    \end{minipage}\hfill
    \begin{minipage}{0.66\linewidth}
        $X,~Y$ 的分布律分别为 $\begin{array}{c|cc}
                X & 0   & 1   \\\hline
                P & 0.4 & 0.6
            \end{array},~\begin{array}{c|ccc}
                Y & -1   & 0   & 1    \\\hline
                P & 0.15 & 0.5 & 0.35
            \end{array}$, 那么
        $$E\qty(X^2)=0.6,~E\qty(Y^2)=0.5,~E\qty(X^2Y^2)=\sum_i\sum_j x_i^2y_j^2 p_{ij}=0.08+0.20=0.28$$
        所以 $\cov\qty(X^2,Y^2)=E\qty(X^2Y^2)-E\qty(X^2)E\qty(Y^2)=0.28-0.3=-0.02.$
    \end{minipage}
\end{solution}

\begin{example}
    设随机变量 $(X,Y)$ 的概率分布为表 \ref{012140140130} 求\newline
    \begin{enumerate*}[label=(\arabic{*})]
        \item $P\qty{X=2Y}$;
        \item $\cov(X-Y,Y)$.
    \end{enumerate*}
\end{example}
\begin{solution}
    \begin{minipage}{0.3\linewidth}
        \begin{table}[H]
            \centering
            \caption*{概率分布}
            \label{012140140130}
            \begin{tabular}{c | c c c}
                $X^{\displaystyle\setminus Y}$ & 0               & 1              & 2               \\
                \midrule
                0                              & $\dfrac{1}{4}$  & 0              & $\dfrac{1}{4}$  \\[6pt]
                1                              & 0               & $\dfrac{1}{3}$ & 0               \\[6pt]
                2                              & $\dfrac{1}{12}$ & 0              & $\dfrac{1}{12}$
            \end{tabular}
        \end{table}
    \end{minipage}\hfill
    \begin{minipage}{0.69\linewidth}
        \begin{enumerate}[label=(\arabic{*})]
            \item $P\qty{X=2Y}=P\qty{X=0,Y=0}+P\qty{X=2,Y=1}=\dfrac{1}{4}.$
            \item $X,~Y$ 的分布律分别为 $\begin{array}{c|ccc}
                          X & 0           & 1           & 2           \\\hline
                          P & \frac{1}{2} & \frac{1}{3} & \frac{1}{6}
                      \end{array},~\begin{array}{c|ccc}
                          Y & 0           & 1           & 2           \\\hline
                          P & \frac{1}{3} & \frac{1}{3} & \frac{1}{3}
                      \end{array}$, 那么
                  \begin{flalign*}
                      \cov(X-Y,Y) & =\cov(X,Y)-\cov(Y,Y)=E(XY)-E(X)E(Y)-D(Y) \\
                                  & =E(XY)-E(X)E(Y)-E\qty(Y^2)+E^2(Y)
                  \end{flalign*}
                  其中 $\displaystyle E(XY)=\sum_i\sum_j x_iy_jp_{ij}=1\cdot\dfrac{1}{3}+2\cdot\dfrac{2}{12}=\dfrac{2}{3}$,
                  $E(X)=\dfrac{1}{3}+\dfrac{2}{6}=\dfrac{2}{3},~E(Y)=\dfrac{1}{3}+\dfrac{2}{3}=1$,
                  $E\qty(Y^2)=\dfrac{1}{3}+\dfrac{4}{3}=\dfrac{5}{3}$, 代入算得 $\cov(X-Y,Y)=-\dfrac{2}{3}.$
        \end{enumerate}
    \end{minipage}
\end{solution}

\begin{theorem}[协方差与方差]
    $D(X\pm Y)=D(X)+D(Y)\pm 2\cov(X,Y).$
\end{theorem}

\begin{example}[2011 数一]
    设随机变量 $X$ 与 $Y$ 的概率分布分别为 $\begin{array}{c|cc}
            X & 0           & 1           \\\hline
            P & \frac{1}{3} & \frac{2}{3}
        \end{array},~\begin{array}{c|ccc}
            Y & -1          & 0           & 1           \\\hline
            P & \frac{1}{3} & \frac{1}{3} & \frac{1}{3}
        \end{array}$, 且 $P\qty{X^2=Y^2}=1.$\newline
    \begin{enumerate*}[label=(\arabic{*})]
        \item 二维随机变量 $(X,Y)$ 的概率分布;
        \item $Z=XY$ 的概率分布;
        \item $X$ 和 $Y$ 的相关系数 $\rho_{XY}$.
    \end{enumerate*}
\end{example}
\begin{solution}
    \begin{minipage}{0.28\linewidth}
        $\begin{matrix}
                \begin{array}{c|cccc}
                    X^{\displaystyle\setminus Y} & -1           & 0            & 1                           \\ \hline
                    0                            & a_{11}       & a_{12}       & a_{13}       & \dfrac{1}{3} \\[6pt]
                    1                            & a_{21}       & a_{22}       & a_{23}       & \dfrac{2}{3} \\[6pt]
                                                 & \dfrac{1}{3} & \dfrac{1}{3} & \dfrac{1}{3}
                \end{array} \\
                \Downarrow                                                                                        \\
                \begin{array}{c|cccc}
                    X^{\displaystyle\setminus Y} & -1           & 0            & 1                        \\ \hline
                    0                            & 0            & \dfrac{1}{3} & 0            & ~~~~~~~~~ \\[6pt]
                    1                            & \dfrac{1}{3} & 0            & \dfrac{1}{3}
                \end{array}
            \end{matrix}$
    \end{minipage}\hfill
    \begin{minipage}{0.68\linewidth}
        \begin{enumerate}[label=(\arabic{*})]
            \item 因为 $P\qty{X^2=Y^2}=1$, 所以 $P\qty{X^2\neq Y^2}=1-1=0$, 于是有
                  $$\begin{cases}
                          a_{11}=P\qty{X=0,Y=-1}=0\Rightarrow a_{21}=\dfrac{1}{3} \\[6pt]
                          a_{22}=P\qty{X=1,Y=0}=0\Rightarrow a_{12}=\dfrac{1}{3}\Rightarrow a_{13}=0\Rightarrow a_{23}=\dfrac{1}{3}
                      \end{cases}$$
            \item $Z=XY$ 的可能取值为 $-1,0,1$, 且 $P\qty{Z=1}=P\qty{X=1,Y=1}=\dfrac{1}{3}$, $$P\qty{Z=0}=P\qty{X=0,Y=*}+P\qty{X=*,Y=0}=\dfrac{1}{3}$$
                  同理 $P\qty{Z=-1}=\dfrac{1}{3}$, 于是 $Z$ 的概率分布为 $\begin{array}{c|ccc}
                          Z & -1          & 0           & 1           \\\hline
                          P & \frac{1}{3} & \frac{1}{3} & \frac{1}{3}
                      \end{array}.$
            \item 易求得 $\cov(X,Y)=0$, 于是 $\rho_{XY}=0.\begin{array}{c|ccc}
                          Y & -1          & 0           & 1           \\\hline
                          P & \frac{1}{3} & \frac{1}{3} & \frac{1}{3}
                      \end{array}$
        \end{enumerate}
    \end{minipage}
\end{solution}

\subsection{相关系数}

\begin{definition}[相关系数]
    对于随机变量 $X,Y$, $D(X)>0,D(Y)>0$, 则称 $\dfrac{\cov(X,Y)}{\sqrt{D(X)}\sqrt{D(Y)}}$ 为随机变量 $X$ 和 $Y$ 的\textit{相关系数}, 记作 $\rho_{XY}$.
\end{definition}

\begin{theorem}[相关系数的性质]
    \begin{enumerate}[label=(\arabic{*})]
        \item $|\rho_{XY}|\leqslant 1$;
        \item $|\rho_{XY}|=1\Leftrightarrow $ 存在 $a,b$ 使得 $P{Y=aX+b}=1$, 其中 $a>0$ 时, $\rho_{XY}=1$ (正线性相关); $a<0$ 时, $\rho_{XY}=-1$ (负线性相关).
    \end{enumerate}
\end{theorem}

\begin{example}
    设随机变量 $X$ 和 $Y$ 的联合概率分布为表 \ref{-101007018}, 求 $X$ 和 $Y$ 的相关系数 $\rho.$
\end{example}
\begin{solution}
    \begin{minipage}{0.3\linewidth}
        \begin{table}[H]
            \centering
            \caption*{联合概率分布表}
            \label{-101007018}
            \begin{tabular}{c | c c c}
                $X^{\displaystyle\setminus Y}$ & -1   & 0    & 1    \\
                \midrule
                0                              & 0.07 & 0.18 & 0.15 \\
                1                              & 0.08 & 0.32 & 0.20
            \end{tabular}
        \end{table}
    \end{minipage}\hfill
    \begin{minipage}{0.66\linewidth}
        $X,~Y$ 的分布律分别为 $\begin{array}{c|cc}
                X & 0   & 1   \\\hline
                P & 0.4 & 0.6
            \end{array},~\begin{array}{c|ccc}
                Y & -1   & 0   & 1    \\\hline
                P & 0.15 & 0.5 & 0.35
            \end{array}$, 那么
        $$E(X)=0.6,~E(Y)=0.2,~E(XY)=\sum_i\sum_j x_iy_j p_{ij}=-0.08+0.20=0.12$$
        所以 $\cov(X,Y)=E(XY)-E(X)E(Y)=0$, 所以 $\rho_{XY}=\dfrac{\cov(X,Y)}{\sqrt{D(X)}\sqrt{D(Y)}}=0.$
    \end{minipage}
\end{solution}

\begin{example}[1994 数一]
    已知随机变量 $X$ 和 $Y$ 分布服从正态分布 $N\qty(1,3^2)$ 和 $N\qty(0,4^2)$, 且 $X$ 与 $Y$ 的相关系数 $\rho_{XY}=-\dfrac{1}{2}$,
    设 $Z=\dfrac{X}{3}+\dfrac{Y}{2}$.\newline
    \begin{enumerate*}[label=(\arabic{*})]
        \item 求 $Z$ 的数学期望 $E(X)$ 和方差 $D(Z)$;
        \item 求 $X$ 和 $Z$ 的相关系数 $\rho_{XZ}.$
    \end{enumerate*}
\end{example}
\begin{solution}
    \begin{enumerate}[label=(\arabic{*})]
        \item $E(Z)=\dfrac{1}{3}E(X)+\dfrac{1}{2}E(Y)=\dfrac{1}{3}+0=\dfrac{1}{3}$,
              $$D(Z)=\dfrac{1}{9}D(X)+\dfrac{1}{4}D(X)+2\cdot\dfrac{1}{3}\cdot\dfrac{1}{2}\cov(X,Y)=\dfrac{1}{9}D(X)+\dfrac{1}{4}D(X)+2\cdot\dfrac{1}{3}\cdot\dfrac{1}{2}\rho_{XY}\sqrt{D(X)}\sqrt{D(Y)}=3.$$
        \item 由协方差的性质,
              \begin{flalign*}
                  \cov(X,Z)=\cov\qty(X,\dfrac{1}{3}X+\dfrac{1}{2}Y)=\dfrac{1}{3}\cov(X,X)+\dfrac{1}{2}\cov(X,Y)=\dfrac{1}{3}D(X)+\dfrac{1}{2}\cdot\rho_{XY}\sqrt{D(X)}\sqrt{D(Y)}=0
              \end{flalign*}
              所以 $\rho_{XZ}=0.$
    \end{enumerate}
\end{solution}

\begin{example}[2000 数一]
    设二维随机变量 $(X,Y)$ 服从二维正态分布, 则随机变量 $\xi=X+Y$ 与 $\eta=X-Y$ 不相关的充分必要条件为
    \begin{tasks}(2)
        \task $E(X)=E(Y)$
        \task $E\qty(X^2)-E^2(X)=E\qty(Y^2)-E^2(Y)$
        \task $E\qty(X^2)=E\qty(Y^2)$
        \task $E\qty(X^2)+E^2(X)=E\qty(Y^2)+E^2(Y)$
    \end{tasks}
\end{example}
\begin{solution}
    因为不相关, 所以 $\rho_{\xi\eta}=0\Rightarrow \cov(\xi,\eta)=0$, 于是
    \begin{flalign*}
        \cov(\xi,\eta)=E(\xi\eta)-E(\xi)E(\eta)=E\qty(X^2-Y^2)-E(X+Y)E(X-Y)=E\qty(X^2-Y^2)-E^2(X)+E^2(Y)=0
    \end{flalign*}
    即 $E\qty(X^2)-E^2(X)=E\qty(Y^2)-E^2(Y)$, 故选 B.
\end{solution}

\begin{theorem}
    当 $Y=aX+b$ 时, $\rho_{XY}=\sgn(a).$
\end{theorem}

\begin{example}
    设随机变量 $X\sim N(0,1),~Y\sim N(1,4)$, 且相关系数 $\rho_{XY}=1$, 则
    \begin{tasks}(2)
        \task $P\qty{Y=-2X -1}=1$
        \task $P\qty{Y=2X -1}=1$
        \task $P\qty{Y=-2X+1}=1$
        \task $P\qty{Y=2X+1}=1$
    \end{tasks}
\end{example}
\begin{solution}
    用排除法, 设 $Y=aX+b$, 由 $\rho_{XY}=1\Rightarrow a>0$, 排除 A、C,
    由 $E(X)=0,~E(Y)=1$, 且 $$E(Y)=E(aX+b)=aE(X)+b\Rightarrow b=1$$
    排除 B, 故选 D.
\end{solution}

\begin{example}
    设二维随机变量 $(X,Y)$ 的概率密度为 $$f(x,y)=\begin{cases}
            \dfrac{2}{\pi}\qty(1-x^2-y^2), & x^2+y^2\leqslant 1 \\
            0,                             & \text{其他}
        \end{cases}$$
    \begin{enumerate}[label=(\arabic{*})]
        \item $X$ 与 $Y$ 是否相互独立, 说明理由;
        \item 求 $\rho_{XY}$;
        \item 求 $Z=\sqrt{X^2+Y^2}$ 的概率密度.
    \end{enumerate}
\end{example}
\begin{solution}
    \begin{enumerate}[label=(\arabic{*})]
        \item 注意到 $X$ 与 $Y$ 轮换对此, 那么
              \begin{flalign*}
                  f_{X}(x) & =\int_{-\infty}^{+\infty} f(x, y) \dd y=\frac{2}{\pi} \int_{-\sqrt{1-x^{2}}}^{\sqrt{1-x^{2}}}\left(1-x^
                  {2}-y^{2}\right) \dd y =\frac{4}{\pi} \int_{0}^{\sqrt{1-x^{2}}}\left[\left(1-x^{2}\right)-y^{2}\right] \dd y                                                                               \\
                           & =\frac{4}{\pi}\left[\left(1-x^{2}\right)^{\frac{3}{2}}-\frac{1}{3}\left(1-x^{2}\right)^{\frac{3}{2}}\right]  =\frac{8}{3 \pi}\left(1-x^{2}\right)^{\frac{3}{2}},|x| \leqslant 1
              \end{flalign*}
              故 $$f_{X}(x)=\begin{cases}
                      \displaystyle \frac{8}{3 \pi}\left(1-x^{2}\right)^{\frac{3}{2}}, & |x| \leqslant 1 \\
                      0,                                                               & |x|>1
                  \end{cases}$$
              同理可求得
              $$f_{Y}(y)=\begin{cases}
                      \displaystyle \frac{8}{3 \pi}\left(1-y^{2}\right)^{\frac{3}{2}}, & |y| \leqslant 1 \\
                      0,                                                               & |y|>1
                  \end{cases}$$
              由于 $ f_{X}(x) \cdot f_{Y}(y) \neq f(x, y) $, 所以 $ X $ 与 $ Y $ 不相互独立.
        \item 由 $\displaystyle EX=\int_{-\infty}^{+\infty} x f_{X} \dd x=\int_{-1}^{1} x \cdot \frac{8}{3 \pi}\left(1-x^{2}\right)^{\frac{3}{2}} \dd x=0$, 且被积函数关于 $ x, y $ 均为奇函数, $$\displaystyle E_{X Y}  =\int_{-\infty}^{+\infty} \int_{-\infty}^{+\infty} x y f(x, y) \dd x \dd y
                  =\iint\limits_{x^{2}+y^{2} \leqslant 1}\frac{2}{\pi} x y\left(1-x^{2}-y^{2}\right) \dd x \dd y=0$$
              所以 $ \cov(X, Y)=E X Y-E X E Y=0 $, 从而可得 $ \rho_{X Y}=0 $.
        \item 当 $ z<0 $ 时, $ Z=\sqrt{X^{2}+Y^{2}} $ 的分布函数为 $ F_{Z}(z)=0 $.
              当 $ 0 \leqslant z<1 $ 时,
              $$
                  F_{z}(z) =P\{Z \leqslant z\}=P\left\{\sqrt{X^{2}+Y^{2}} \leqslant z\right\} \\
                  =\iint\limits_{x^{2}+y^{2} \leqslant z^{2}} f(x, y) \dd x \dd y \\
                  =\int_{0}^{2 \pi} \dd \theta \int_{0}^{z} \frac{2}{\pi}\left(1-r^{2}\right) r \dd r=2 z^{2}-z^{4}
              $$
              当 $ z \geqslant 1 $ 时, $F_{Z}(z)=1 $, 故 $ Z $ 的概率密度为
              $f_{Z}(z)=F_{Z}^{\prime}(z)=\begin{cases}
                      4 z-4 z^{3}, & 0<z<1,          \\
                      0,           & \text { 其他. }
                  \end{cases}$
    \end{enumerate}
\end{solution}