\section{矩阵的合同}

\begin{example}
    对于二次型 $f=\vb*{x}^\top\vb*{Ax}$ 经非奇异线性变换 $\vb*{x}=\vb*{Cy}$ 化为 $f=\vb*{y}^\top\vb*{By}$, 则 $\vb*{A}$ 与 $\vb*{B}$ 必定是
    \begin{tasks}(4)
        \task 相等
        \task 相似
        \task 合同
        \task 具有相同的特征值
    \end{tasks}
\end{example}
\begin{solution}
    令 $\vb*{x}=\vb*{Cy}$, 则 
    $$f=\vb*{x}^\top\vb*{Ax}=(\vb*{Cy})^\top\vb*{A}(\vb*{Cy})=\qty(\vb*{y}^\top\vb*{C}^\top)\vb*{A}(\vb*{Cy})=\vb*{y}^\top\qty(\vb*{C}^\top\vb*{AC})\vb*{y}=\vb*{y}^\top\vb*{By}$$
    因此 $\vb*{B}=\vb*{C}^\top\vb*{AC}$, 那么 $\vb*{A}$ 与 $\vb*{B}$ 合同.
\end{solution}

\begin{example}
    设矩阵 $\vb*{A}=\mqty(-3&0&0\\0&-1&2\\0&2&2)\text{, }\vb*{B}=\mqty(0&3&0\\3&0&0\\0&0&k)$, 若 $\vb*{A}$ 与 $\vb*{B}$ 合同但不相似, 则常数 $k$ 的取值范围是多少.
\end{example}
\begin{solution}
    若 $\vb*{A}$ 与 $\vb*{B}$ 合同但不相似, 说明 $\vb*{A}$ 与 $\vb*{B}$ 的正负特征值个数分别相同, 但是 $\vb*{A}$ 与 $\vb*{B}$ 的特征值不全相同, 则
    \begin{flalign*}
        |\lambda\vb*{E}-\vb*{A}| & =\mqty|\lambda+3 & 0  & 0 \\0&\lambda+1&-2\\0&-2&\lambda-2|=(\lambda+3)\qty(\lambda^2-\lambda-6)\\
        |\lambda\vb*{E}-\vb*{B}| & =\mqty|\lambda   & -3 & 0 \\-3&\lambda&0\\0&0&\lambda-k|=(\lambda-k)\qty(\lambda^2-9)
    \end{flalign*}
    那么 $\vb*{A}$ 的特征值为 $\lambda_1=-3\text{, }\lambda_2=-2\text{, }\lambda_3=3$, $\vb*{B}$ 的特征值为 $\lambda_1=-3\text{, }\lambda_2=k\text{, }\lambda_3=3$, 所以 $k<0$ 且 $k\neq-2.$
\end{solution}



% \subsubsection{合同变换法的改进——行初等变换法}

% 对矩阵 $\vb*{A}$ 作一系列初等行变换化为上三角形矩阵, 相同的列变换就将它化成了对角矩阵 $\vb*{\varLambda}$ (上三角形矩阵的对角元就是 $\vb*{\varLambda}$ 的对角元), 所用行变换同时将单位矩阵化成了 $\vb*{C}^\top$, 即

% $$(\vb*{A},\vb*{I})\xrightarrow{\text{行初等变换}}\mqty(\mqty(d_1 & & *\\&\ddots&\\& & d_n),\vb*{C}^\top)$$
% 则 $\vb*{C}^\top\vb*{AC}=\mqty(d_1 & & \\&\ddots&\\& & d_n).$

\begin{example}
    用行初等变换法将矩阵 $\vb*{A}=\mqty(1&1&0\\1&2&-1\\0&-1&-4)$ 化为标准形.
\end{example}
\begin{solution}
    $(\vb*{A},\vb*{I})=\begin{pNiceArray}{ccc:ccc}
            1&1&0&1&0&0\\
            1&2&-1&0&1&0\\
            0&-1&-4&0&0&1
        \end{pNiceArray}\xrightarrow[]{r_2-r_1}\begin{pNiceArray}{ccc:ccc}
            1&1&0&1&0&0\\
            0&1&-1&-1&1&0\\
            0&-1&-4&0&0&1
        \end{pNiceArray}\xrightarrow[]{r_3+r_2}\begin{pNiceArray}{ccc:ccc}
            1 &  1 &  0  & 1  & 0 &  0\\
            0 &  1 &  -1 &  -1& 1 &  0\\
            0 &  0 &  -5 &  -1& 1 &1
        \end{pNiceArray}$
    由此可得 $\vb*{\varLambda}=\mqty(\dmat{1,1,-5}),~\vb*{C}=\mqty(1&-1&-1\\0&1&1\\0&0&1)$, 且 $\vb*{C}^\top\vb*{AC}=\vb*{\varLambda}.$
\end{solution}
