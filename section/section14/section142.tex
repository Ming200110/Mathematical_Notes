\section{正定性与矩阵的合同}

正定矩阵是正实数概念的延伸——它定义了何为“正”的矩阵, 一种方法是全部元素为正数, 但这显然不是一个好的定义, 
因为它完全不涉及矩阵的具体结构。因此尝试借助正实数的概念推广至矩阵的正定性。

一个正实数 $\alpha$ 满足: 对任意一个矢量\footnote{当强调方向时, 采用矢量的说法} $\vb*{v}$, 矢量 $\alpha\vb*{v}$ 会和 $\vb*{v}$ 在同一方向上\footnote{仅描述新矢量在原矢量的分量方向上的投影是正的情況, 不要求两个矢量之间的夹角为零角即完全重合.}, 即内积为正 $\vb*{v}^\top\alpha\vb*{v}>0.$

借助这个准则, 一个正的矩阵 $\vb*{A}$ 需要满足: 对任意一个矢量 $\vb*{v}$, $\vb*{A}v$ 和 $\vb*{v}$ 在同一方向上, 即内积为正: $\vb*{v}^\top\vb*{A}\vb*{v}>0.$

\subsection{正定性}

\begin{theorem}[Hurwitz 定理]
    对称矩阵 $\vb*{A}$ 为正定的充分必要条件: $\vb*{A}$ 的各阶主子式 (顺序主子式) 都为正, 即
    $$a_{11}>0,~\begin{vmatrix}
            a_{11} & a_{12} \\
            a_{21} & a_{22}
        \end{vmatrix}>0,~\cdots,~\begin{vmatrix}
            a_{11} & \cdots & a_{1n}  \\
            \vdots &        & \vdots\ \\
            a_{n1} & \cdots & a_{nn}
        \end{vmatrix}>0.$$
    对称矩阵 $\vb*{A}$  为负定的充分必要条件: 奇数阶主子式为负, 偶数主子式为正, 即
    $$(-1)^r\begin{vmatrix}
            a_{11} & \cdots & a_{1r} \\
            \vdots &        & \vdots \\
            a_{r1} & \cdots & a_{rr}
        \end{vmatrix}>0~  (r=1,2,\cdots ,n).$$
    \label{Hurwitztheorem}
\end{theorem}

\begin{theorem}[实对称方阵的等价命题]
    设 $\vb*{S}=(s_{ij})$ 是 $n$ 阶实对称方阵, 则下述命题等价:
    \begin{enumerate}[label=(\arabic{*})]
        \item 方阵 $\vb*{S}$ 是正定的;
        \item 方阵 $\vb*{S}$ 的每一个特征值均为正的;
        \item 存在正定对称方阵 $\vb*{S}_1$, 使得 $\vb*{S}=\vb*{S}_1^2$;
        \item 存在可逆方阵 $\vb*{P}$, 使得 $\vb*{S}=\vb*{P}^\top\vb*{P}$;
        \item 方阵 $\vb*{S}$ 的每个主子式均为正的;
        \item 方阵 $\vb*{S}$ 的顺序主子式均为正的.
    \end{enumerate}
\end{theorem}

\begin{example}
    判断二次型 $f\qty(x_1,x_2,x_3)=5x_1^2+6x_2^2+4x_3^2-4x_1x_2-4x_2x_3$ 是否正定.
\end{example}
\begin{solution}
    因为 $\begin{NiceArray}{c:ccc}
        & x_1 & x_2 & x_3 \\ \hdottedline
            x_1 & 5   & -2  & 0   \\
            x_2 & -2  & 6   & -2  \\
            x_3 & 0   & -2  & 4
    \end{NiceArray}$, 所以 $f$ 的矩阵 $\vb*{A}=\begin{pmatrix}
            5  & -2 & 0  \\
            -2 & 6  & -2 \\
            0  & -2 & 4
        \end{pmatrix}$, 故 $\vb*{A}$ 的顺序主子式分别为
    $$|5|=5>0,~\begin{vmatrix}
            5  & -2 \\
            -2 & 6
        \end{vmatrix}=26>0,~\begin{vmatrix}
            5  & -2 & 0  \\
            -2 & 6  & -2 \\
            0  & -2 & 4
        \end{vmatrix}=84>0$$
    所以这个二次型是正定的.
\end{solution}

\begin{example}
    $t$ 满足什么条件时, 下列二次型是正定的:
    \begin{enumerate}[label=(\arabic{*})]
        \item $f=x_1^2+x_2^2+5x_3^2+2tx_1x_2-2x_1x_3+4x_2x_3$;
        \item $f=x_1^2+4x_2^2+x_3^2+2tx_1x_2+10x_1x_3+6x_2x_3$.
    \end{enumerate}
\end{example}
\begin{solution}
    \begin{enumerate}[label=(\arabic{*})]
        \item 因为 $\begin{NiceArray}{c:ccc}
            & x_1 & x_2 & x_3 \\ \hdottedline
                          x_1 & 1   & t   & -1  \\
                          x_2 & t   & 1   & 2   \\
                          x_3 & -1  & 2   & 5
        \end{NiceArray}$, 所以 $f$ 的矩阵 $\vb*{A}_1=\begin{pmatrix}
                      1  & t & -1 \\
                      t  & 1 & 2  \\
                      -1 & 2 & 5
                  \end{pmatrix}$, 故 $\vb*{A}$ 的顺序主子式分别为
              $$|1|=1>0,~\begin{vmatrix}
                      1 & t \\
                      t & 1
                  \end{vmatrix}=1-t^2>0,~\begin{vmatrix}
                      1  & t & -1 \\
                      t  & 1 & 2  \\
                      -1 & 2 & 5
                  \end{vmatrix}=-t(5t+4)>0$$
              所以当 $-\dfrac{4}{5}<t<0$ 时, $f(x_1,x_2,x_3)$ 是正定的.
        \item $f$ 的矩阵 $\vb*{A}_2=\begin{pmatrix}
                      1 & t & 5 \\
                      t & 4 & 3 \\
                      5 & 3 & 1
                  \end{pmatrix}$, 那么 $\vb*{A}$ 的顺序主子式分别为
              $$|1|=1>0,~\begin{vmatrix}
                      1 & t \\
                      t & 4
                  \end{vmatrix}=4-t^2>0,~\begin{vmatrix}
                      1 & t & 5 \\
                      t & 4 & 3 \\
                      5 & 3 & 1
                  \end{vmatrix}=-t^2+30t-105>0$$
              因为 $4-t^2$ 与 $-t^2+30t-105$ 不能同时大于零, 所以无论 $t$ 取什么值, 这个二次型都不能是正定的.
    \end{enumerate}
\end{solution}

\begin{example}
    判断二次型 $f=-5x^2-6y^2-4z^2+4xy+4xz$ 的正定性.
\end{example}
\begin{solution}
    因为 $\begin{NiceArray}{c:ccc}
        & x  & y  & z  \\ \hdottedline
            x & -5 & 2  & 2  \\
            y & 2  & -6 & 0  \\
            z & 2  & 0  & -4
    \end{NiceArray}$, 所以 $f$ 的矩阵 $\vb*{A}=\begin{pmatrix}
            -5 & 2  & 2  \\
            2  & -6 & 0  \\
            2  & 0  & -4
        \end{pmatrix}$, 故 $\vb*{A}$ 的顺序主子式分别为
    $$|-5|=-5<0,~\begin{vmatrix}
            -5 & 2  \\
            2  & -6
        \end{vmatrix}=26>0,~\begin{vmatrix}
            -5 & 2  & 2  \\
            2  & -6 & 0  \\
            2  & 0  & -4
        \end{vmatrix}=-80<0$$
    即 $f$ 是负定的.
\end{solution}

\begin{example}
    判断二次型 $\displaystyle f=\sum_{i=1}^{n}x_i^2+\sum_{i=1}^{n-1}x_ix_{i+1}$ 的正定性.
\end{example}
\begin{solution}
    $f$ 的矩阵为 $\vb*{A}=\mqty(
        1&\dfrac{1}{2}&0&\cdots& 0&0\\
        \dfrac{1}{2}& 1& \dfrac{1}{2} &\cdots &0&0 \\
        0&\dfrac{1}{2}&1&\cdots&0&0\\
        \vdots& \vdots&\vdots&\ddots&\vdots&\vdots \\
        0&0&0&\cdots & 1&\dfrac{1}{2} \\
        0&0&0&\cdots&\dfrac{1}{2}&1
        )$, 任取 $\vb*{A}$ 的一个 $k$ 阶顺序主子式, 有
    \begin{flalign*}
        |\vb*{A}_k|&=\mqty|
        1            & \dfrac{1}{2} & 0            & \cdots & 0              & 0                                                                           \\
        \dfrac{1}{2} & 1            & \dfrac{1}{2} & \cdots & 0              & 0                                                                           \\
        0            & \dfrac{1}{2} & 1            & \cdots & 0              & 0                                                                           \\
        \vdots       & \vdots       & \vdots       & \ddots & \vdots         & \vdots                                                                      \\
        0            & 0            & 0            & \cdots & 1              & \dfrac{1}{2}                                                                \\
        0            & 0            & 0            & \cdots & \dfrac{1}{2}   & 1
        |=\dfrac{1}{2^k} \mqty|
        2            & 1            & 0            & \cdots & 0              & 0                                                                           \\
        1            & 2            & 1            & \cdots & 0              & 0                                                                           \\
        0            & 1            & 2            & \cdots & 0              & 0                                                                           \\
        \vdots       & \vdots       & \vdots       & \ddots & \vdots         & \vdots                                                                      \\
        0            & 0            & 0            & \cdots & 2              & 1                                                                           \\
        0            & 0            & 0            & \cdots & 1              & 2|\xlongequal[i=1,\cdots,k-1]{r_{i+1}-\frac{i}{i+1}r_i}\dfrac{1}{2^k}\mqty|
        2            & 1            & 0            & \cdots & 0              & 0                                                                           \\
        0            & \dfrac{3}{2} & 1            & \cdots & 0              & 0                                                                           \\
        0            & 0            & \dfrac{4}{3} & \cdots & 0              & 0                                                                           \\
        \vdots       & \vdots       & \vdots       &        & \vdots         & \vdots                                                                      \\
        0            & 0            & 0            & \cdots & \dfrac{k}{k-1} & 1                                                                           \\
        0            & 0            & 0            & \cdots & 0              & \dfrac{k+1}{k}|\\ 
        &= \dfrac{k+1}{k}>0,~k=1,2,\cdots, n
    \end{flalign*}
    所以 $\vb*{A}$ 正定, 从而二次型正定.
\end{solution}

\begin{theorem}[特征值与正定性]
    若一矩阵 $\vb*{A}$ 是正定的, 则其所有特征值 $\lambda_i>0~ (i=1,2,\cdots,n).$
\end{theorem}
\begin{example}
    设 $\vb*{A}$ 是 3 阶正定矩阵, 证明 $|\vb*{A}+2\vb*{E}|>8.$
\end{example}
\begin{proof}[{\songti \textbf{证}}]
    由 $\vb*{A}$ 是正定的, 则 $\lambda_{1,2,3}>0$, 那么 $|\vb*{A}+2\vb*{E}|=(\lambda_1+2)(\lambda_2+2)(\lambda_3+2)\geqslant 8.$
\end{proof}

\begin{example}[2000 清华大学]
    设 $n$ 阶实方阵 $(n\geqslant2)$ $$\vb*{A}=\begin{pmatrix}
            b+8    & 3      & 3      & \cdots & 3      \\
            3      & b      & 1      & \cdots & 1      \\
            3      & 1      & b      & \ddots & \vdots \\
            \vdots & \vdots & \ddots & \ddots & 1      \\
            3      & 1      & \cdots & 1      & b
        \end{pmatrix}$$ 试求 $b$ 的取值范围, 使 $\vb*{A}$ 为正定矩阵.
\end{example}
\begin{solution}
    设 $k$ 维列向量 $\vb*{\alpha}=(3,1,\cdots,1)^{\top}$, 则 $\vb*{A}$ 的 $k$ 阶顺序主子式为
    \begin{flalign*}
        |\vb*{A}_k|&=\begin{vmatrix}
            \begin{pmatrix}
                b-1 &     &        &     \\
                    & b-1 &        &     \\
                    &     & \ddots &     \\
                    &     &        & b-1
            \end{pmatrix}+\begin{pmatrix}
                              3      \\
                              1      \\
                              \vdots \\
                              1
                          \end{pmatrix}(3,1,\cdots,1)
        \end{vmatrix}=|(b-1)\vb*{E}_k+\vb*{\alpha\alpha}^{\top}|\\
        &=(b-1)^{k-1}|(b-1)\vb*{E}_1+\vb*{\alpha}^{\top}\vb*{\alpha}|=(b-1)^{k-1}(b+k+7)
    \end{flalign*}
    因为 $\vb*{A}$ 的正定的充分必要条件是 $$|\vb*{A}_k|>0\Leftrightarrow b>1\text{ 且 } b>-(k+7),~k=1,2,\cdots,n$$
    所以, 当 $b>1$ 时 $\vb*{A}$ 是正定矩阵.
\end{solution}

\begin{example}[2019 数一]
    已知 $\vb*{A}=\mqty(a&1&-1\\1&a&-1\\-1&-1&a)$, 
    \begin{enumerate}[label=(\arabic{*})]
        \item 求正交矩阵 $\vb*{P}$, 使得 $\vb*{P}^\top\vb*{AP}$ 为对角矩阵;
        \item 求正定矩阵 $\vb*{C}$, 使得 $\vb*{C}^2=(a+3)\vb*{E}-\vb*{A}.$
    \end{enumerate}
\end{example}
\begin{solution}
    \begin{enumerate}[label=(\arabic{*})]
        \item \textbf{法一: }$|\lambda\vb*{E}-\vb*{A}|=\mqty|\lambda-a&-1&1\\-1&\lambda-a&1\\1&1&\lambda-a|\xlongequal[r_2+r_3]{r_1+r_3}\mqty|\lambda-a+1&0&\lambda-a+1\\0&\lambda-a+1&\lambda-a+1\\1&1&\lambda-a|=(\lambda-a+1)^2(\lambda-a-2)=0$, 
        因此特征值为 $\lambda_1=a+2$, $\lambda_{2,3}=a-1$ (二重), 则 $\lambda_1$ 对应的特征向量为 $\vb*{\xi}_1=(-1,-1,1)^\top$, $\lambda_{2,3}$ 对应的特征向量分别为 $\vb*{\xi}_2=(-1,1,0)^\top,~\vb*{\xi}_3=(1,0,1)^\top$, 下作特征向量的正交单位化, 
        $$\vb*{\alpha}_1=\vb*{\xi}_1,~\vb*{\alpha}_2=\vb*{\xi}_2-\dfrac{\qty[\vb*{\alpha}_1,\vb*{\xi}_2]}{\qty[\vb*{\alpha}_1,\vb*{\alpha}_1]}\vb*{\alpha}_1=\vb*{\xi}_2,~\vb*{\alpha}_3=\vb*{\xi}_3-\dfrac{\qty[\vb*{\alpha}_1,\vb*{\xi}_3]}{\qty[\vb*{\alpha}_1,\vb*{\alpha}_1]}\vb*{\alpha}_1-\dfrac{\qty[\vb*{\alpha}_2,\vb*{\xi}_3]}{\qty[\vb*{\alpha}_2,\vb*{\alpha}_2]}\vb*{\alpha}_2=(1,1,2)^\top$$
        单位化 $$\vb*{e}_1=\dfrac{1}{||\vb*{\alpha}_1||}\vb*{\alpha}_1=\dfrac{1}{\sqrt{3}}\mqty(-1\\-1\\1),~\vb*{e}_2=\dfrac{1}{||\vb*{\alpha}_2||}\vb*{\alpha}_2=\dfrac{1}{\sqrt{2}}\mqty(-1\\1\\0),~\vb*{e}_3=\dfrac{1}{||\vb*{\alpha}_3||}\vb*{\alpha}_3=\dfrac{1}{\sqrt{6}}\mqty(1\\1\\2)$$
        得正交矩阵 $\vb*{P}=\mqty(-\dfrac{1}{\sqrt{3}}&-\dfrac{1}{\sqrt{2}}&\dfrac{1}{\sqrt{6}}\\[6pt]-\dfrac{1}{\sqrt{3}}&\dfrac{1}{\sqrt{2}}&\dfrac{1}{\sqrt{6}}\\[6pt]\dfrac{1}{\sqrt{3}}&0&\dfrac{2}{\sqrt{6}})$, 使得 $\vb*{P}^\top\vb*{AP}=\mqty(a+2\\&a-1\\&&a-1).$\\
        \textbf{法二: }因为 $\vb*{A}=\mqty(a&1&-1\\1&a&-1\\-1&-1&a)$, 于是 
        $$f(\lambda)=\lambda^3-\tr\vb*{A}\lambda^2+\sum_{1\leqslant j_1<j_2\leqslant 3}A\mqty|a_{j_1j_1}&a_{j_1j_2}\\a_{j_2j_1}&a_{j_2j_2}|-\det\vb*{A}=\lambda^3-3a\lambda^2+3\qty(a^2-1)\lambda-(a-1)^2(a+2)=0$$
        即得特征值 $\lambda_1=a+2,~\lambda_{2,3}=a-1$ (二重), 当 $\lambda=\lambda_1$ 时, $\lambda_1\vb*{E}-\vb*{A}=\mqty(2&-1&1\\-1&2&1\\1&1&2)\xrightarrow{r_3-r_1-r_2}\mqty(2&-1&1\\-1&2&1\\0&0&0)$, 那么 $r_1$ 可由 $r_2$ 与 $(3,-3,0)$ 线性表示, 则
        $$(2,-1,1)\times(3,-3,0)=(3,3,-3)\Rightarrow (1,1,-1)\Rightarrow \vb*{\xi}_1=(1,1,-1)^\top$$
        当 $\lambda=\lambda_{2,3}$ 时, $\lambda_{2,3}\vb*{E}-\vb*{A}=\mqty(-1&-1&1\\-1&-1&1\\1&1&-1)\xrightarrow[r_1\leftrightarrow r_3]{\substack{r_1+r_3\\r_2+r_3}}\mqty(1&1&-1\\0&0&0\\0&0&0)$, 那么 $$r_1\cdot (1,1,2)=0,~r_1\times(1,1,2)=(-3,3,0)\Rightarrow(-1,1,0)$$
        则取 $\vb*{\xi}_2=(-1,1,0)^\top,~\vb*{\xi}_3=(1,1,2)^\top$, 因此 $\vb*{\xi}_{1,2,3}$ 已经两两相互正交, 则正交矩阵 $\vb*{P}=\mqty(-\dfrac{1}{\sqrt{3}}&-\dfrac{1}{\sqrt{2}}&\dfrac{1}{\sqrt{6}}\\[6pt]-\dfrac{1}{\sqrt{3}}&\dfrac{1}{\sqrt{2}}&\dfrac{1}{\sqrt{6}}\\[6pt]\dfrac{1}{\sqrt{3}}&0&\dfrac{2}{\sqrt{6}})$, 使得 $\vb*{P}^\top\vb*{AP}=\mqty(a+2\\&a-1\\&&a-1).$
        \item 因为 $\vb*{P}^\top\vb*{AP}=\mqty(a+2\\&a-1\\&&a-1)$, 所以 $\vb*{P}^\top\qty[(a+3)\vb*{E}-\vb*{A}]\vb*{P}=\mqty(1\\&4\\&&4)$, 即 
        $$(a+3)\vb*{E}-\vb*{A}=\vb*{P}\mqty(1\\&4\\&&4)\vb*{P}^\top=\vb*{P}\mqty(1\\&2\\&&2)\vb*{P}^\top\cdot\vb*{P}\mqty(1\\&2\\&&2)\vb*{P}^\top$$
        $\vb*{C}=\vb*{P}\mqty(1\\&2\\&&2)\vb*{P}^\top\Rightarrow\mqty(-\dfrac{1}{\sqrt{3}}&-\dfrac{1}{\sqrt{2}}&\dfrac{1}{\sqrt{6}}\\[6pt]-\dfrac{1}{\sqrt{3}}&\dfrac{1}{\sqrt{2}}&\dfrac{1}{\sqrt{6}}\\[6pt]\dfrac{1}{\sqrt{3}}&0&\dfrac{2}{\sqrt{6}})\mqty(1\\&2\\&&2)\mqty(-\dfrac{1}{\sqrt{3}}&-\dfrac{1}{\sqrt{2}}&\dfrac{1}{\sqrt{6}}\\[6pt]-\dfrac{1}{\sqrt{3}}&\dfrac{1}{\sqrt{2}}&\dfrac{1}{\sqrt{6}}\\[6pt]\dfrac{1}{\sqrt{3}}&0&\dfrac{2}{\sqrt{6}})^\top=\dfrac{1}{3}\mqty(5&-1&1\\-1&5&1\\1&1&5).$
    \end{enumerate}
\end{solution}

\subsection{半正定性}

\begin{theorem}[对称方阵的等价命题]
    设 $\vb*{S}$ 是 $n$ 阶对称方阵, 则下述命题等价:
    \begin{enumerate}[label=(\arabic{*})]
        \item 方阵 $\vb*{S}$ 是半正定的;
        \item 方阵 $\vb*{S}$ 的所有特征值均为非负的;
        \item 存在 $n$ 阶半正定对称方阵 $\vb*{S}_1,~\rank\vb*{S}_1=\rank\vb*{S}$, 使得 $\vb*{S}=\vb*{S}_1^2$;
        \item 存在 $n$ 阶方阵 $\vb*{P},~\rank\vb*{P}=\rank\vb*{S}$, 使得 $\vb*{S}=\vb*{P}^\top\vb*{P}$;
        \item 方阵 $\vb*{S}$ 的所有主子式均为非负的;
        \item 方阵 $\vb*{S}$ 的所有 $k$ 阶主子式之和均为非负的, $k=1,2,\cdots,n.$
    \end{enumerate}
\end{theorem}

\begin{theorem}
    若 $\vb*{A}$  是 $n$ 阶正定矩阵, $\vb*{B}$ 是 $n$ 阶非零半正定矩阵, 则 $\vb*{A}+\vb*{B}$ 是正定矩阵, 且 $$\det(\vb*{A}+\vb*{B})>\det\vb*{A}+\det\vb*{B}.$$
\end{theorem}

\begin{example}[2005 武汉大学]
    已知实二次型 $f=x_1^2+x_2^2+x_3^2+9x_4^2+2a(x_1x_2+x_2x_3+x_3x_1)$, 问当 $a$ 取何值时, $f$ 是正定的、半正定的以及不定的二次型?
\end{example}
\begin{solution}
    二次型 $f$ 的矩阵为 $\vb*{A}=\begin{pmatrix}
            1 & a & a & 0 \\
            a & 1 & a & 0 \\
            a & a & 1 & 0 \\
            0 & 0 & 0 & 9
        \end{pmatrix}$, 得 $\vb*{A}$ 的顺序主子式分别为
    $$\varDelta_1=1,~\varDelta_2=(1-a)(1+a),~\varDelta_3=(1-a)^2(1+2a),~\varDelta_4=9\varDelta_3$$
    由此可见, \begin{enumerate}[label=(\arabic{*})]
        \item 当 $-\dfrac{1}{2}<a<1$ 时, $f$ 为正定二次型;
        \item 当 $a=-\dfrac{1}{2}$ 或 $a=1$, $f$ 为半正定二次型;
        \item 当 $a<-\dfrac{1}{2}$ 或 $a>1$ 时, $f$ 为不定二次型.
    \end{enumerate}
\end{solution}

\begin{example}[2014 南京大学]
    设 $\vb*{A},~\vb*{B}$ 均为 $n$ 阶实对称矩阵, $\vb*{B}$ 正定, 且 $\vb*{A}-\vb*{B}$ 半正定, 证明:
    \begin{enumerate}[label=(\arabic{*})]
        \item 若 $\lambda$ 是 $|\vb*{A}-\lambda\vb*{B}|=0$ 的根, 则 $\lambda\geqslant 1$;
        \item $\det\vb*{A}\geqslant \det\vb*{B}.$
    \end{enumerate}
\end{example}
\begin{proof}[{\songti \textbf{证}}]
    \begin{enumerate}[label=(\arabic{*})]
        \item 因为 $\vb*{A},~\vb*{B}$ 均为 $n$ 阶实对称矩阵, 所以存在可逆实矩阵 $\vb*{P}$, 使得 $\vb*{P}^\top\vb*{BP}=\vb*{E}$, $\vb*{P}^\top\vb*{AP}$ 仍为实对称矩阵, 故又存在正交矩阵 $\vb*{Q}$ 使得, 
              $$\vb*{Q}^\top\qty(\vb*{P}^\top\vb*{AP})\vb*{Q}=\diag(\lambda_1,\lambda_2,\cdots,\lambda_n)$$
              其中 $\lambda_k,~(k=1,2,\cdots,n)$ 为 $\vb*{P}^\top\vb*{AP}$ 的特征值, 又因为
              $$\qty|\lambda\vb*{E}-\vb*{P}^\top\vb*{AP}|=\qty|\lambda \vb*{P}^\top\vb*{BP}-\vb*{P}^\top\vb*{AP}|=\qty|\vb*{P}^2|~\qty|\lambda\vb*{B}-\vb*{A}|=|\vb*{B}|~\qty|\vb*{P}^2|~\qty|\lambda\vb*{E}-\vb*{AB}^{-1}|=0$$
              因为 $\vb*{B}$ 是正定的, $\det\vb*{P}^2>0$, 所以 $|\vb*{B}|~\qty|\vb*{P}^2|>0$, 所以 $\lambda_1,\cdots,\lambda_n$ 是 $\vb*{AB}^{-1}$ 的全部特征值, 也是 $|\vb*{A}-\lambda\vb*{B}|$ 的所有根, 
              又因为 $\vb*{P}^\top(\vb*{A}-\vb*{B})\vb*{P}$ 是实对称矩阵, 其特征值为 $\lambda_1-1,\lambda_2-1,\cdots,\lambda_n-1$, 
              所以 $\vb*{A}-\vb*{B}$ 半正定当且仅当 $\vb*{P}^\top(\vb*{A}-\vb*{B})\vb*{P}$ 半正定, 这又等价为 $\lambda_k-1\geqslant 0,~(k=1,2,\cdots,n)$, 即 $|\vb*{A}-\lambda\vb*{B}|=0$ 的任一根 $\lambda\geqslant 1.$
        \item 由 (1) 可知 $\det\qty(\vb*{AB}^{-1})=\displaystyle\prod_{i=1}^{r}\lambda_k\geqslant 1$, 所以 $\det\vb*{A}\geqslant \det\vb*{B}.$
    \end{enumerate}
\end{proof}

\begin{example}[1994 华中师范大学]
    设 $\vb*{A}_{m\times n},~\vb*{B}_{s\times n}$ 均为行满秩实矩阵, $\vb*{Q}=\vb*{AB}^\top\qty(\vb*{BB}^\top)^{-1}\vb*{BA}^\top$, 证明:
    \begin{enumerate}[label=(\arabic{*})]
        \item $\vb*{AA}^\top-\vb*{Q}$ 是半正定的;
        \item $0\leqslant \det\vb*{Q}\leqslant \det\qty(\vb*{AA}^\top).$
    \end{enumerate}
\end{example}
\begin{proof}[{\songti \textbf{证}}]
    \begin{enumerate}[label=(\arabic{*})]
        \item 由题意知: $\rank\vb*{A}=m,~\rank\vb*{B}=s$, 且 $\rank\qty(\vb*{AA}^\top)=m,~\rank\qty(\vb*{BB}^\top)=s$, 
              所以 $\vb*{AA}^\top,~\vb*{BB}^\top$ 均为可逆矩阵, 并且分块矩阵 $\mqty(\vb*{AA}^\top&\vb*{AB}^\top\\\vb*{BA}^\top&\vb*{BB}^\top)=\mqty(\vb*{A}\\\vb*{B})\mqty(\vb*{A}\\\vb*{B})^\top$ 是半正定的, 且与 $\mqty(\vb*{AA}^\top-\vb*{Q}&\vb*{O}\\\vb*{O}&\vb*{BB}^\top)$ 合同, 
              因此 $\vb*{AA}^\top-\vb*{Q}$ 是半正定的.
        \item 易知 $\vb*{AA}^\top$ 是正定的, $\vb*{Q}$ 是半正定的, 所以存在可逆实矩阵 $P$, 使得 $$\vb*{P}^\top\qty(\vb*{AA}^\top)\vb*{P}=\vb*{E},~\vb*{P}^\top\vb*{QP}=\diag(\lambda_1,\lambda_2,\cdots,\lambda_n),~(\lambda_k\geqslant 0)$$
              又因为 $\vb*{AA}^\top-\vb*{Q}$ 是半正定的, 所以 $1-\lambda_k\geqslant 0$ 即 $0\leqslant \lambda_k\leqslant 1$, 其中 $k=1,2,\cdots,n$, 
              $$\det\qty(\vb*{P}^\top\vb*{QP})=\qty|\vb*{P}^2|~\qty|\vb*{Q}|=\prod_{i=1}^{r}\lambda_i\leqslant 1=\qty|\vb*{P}^2|\qty|\vb*{AA}^\top|\Rightarrow 0\leqslant \det\vb*{Q}\leqslant \det\qty(\vb*{AA}^\top).$$
    \end{enumerate}
\end{proof}

\begin{inference}[实对称半正定矩阵的单调性]
    设 $\vb*{A},~\vb*{B},~\vb*{A}-\vb*{B}$ 均为 $n$ 阶实对称半正定矩阵, 即 $\vb*{A}\geqslant \vb*{B}\geqslant 0$, 则 $\det\vb*{A}\geqslant \det\vb*{B}\geqslant 0.$
\end{inference}

\subsection{矩阵的合同}

\begin{example}
    对于二次型 $f=\vb*{x}^\top\vb*{Ax}$ 经非奇异线性变换 $\vb*{x}=\vb*{Cy}$ 化为 $f=\vb*{y}^\top\vb*{By}$, 则 $\vb*{A}$ 与 $\vb*{B}$ 必定是
    \begin{tasks}(4)
        \task 相等
        \task 相似
        \task 合同
        \task 具有相同的特征值
    \end{tasks}
\end{example}
\begin{solution}
    令 $\vb*{x}=\vb*{Cy}$, 则 
    $$f=\vb*{x}^\top\vb*{Ax}=(\vb*{Cy})^\top\vb*{A}(\vb*{Cy})=\qty(\vb*{y}^\top\vb*{C}^\top)\vb*{A}(\vb*{Cy})=\vb*{y}^\top\qty(\vb*{C}^\top\vb*{AC})\vb*{y}=\vb*{y}^\top\vb*{By}$$
    因此 $\vb*{B}=\vb*{C}^\top\vb*{AC}$, 那么 $\vb*{A}$ 与 $\vb*{B}$ 合同.
\end{solution}

\begin{example}
    设矩阵 $\vb*{A}=\mqty(-3&0&0\\0&-1&2\\0&2&2)\text{, }\vb*{B}=\mqty(0&3&0\\3&0&0\\0&0&k)$, 若 $\vb*{A}$ 与 $\vb*{B}$ 合同但不相似, 则常数 $k$ 的取值范围是多少.
\end{example}
\begin{solution}
    若 $\vb*{A}$ 与 $\vb*{B}$ 合同但不相似, 说明 $\vb*{A}$ 与 $\vb*{B}$ 的正负特征值个数分别相同, 但是 $\vb*{A}$ 与 $\vb*{B}$ 的特征值不全相同, 则
    \begin{flalign*}
        |\lambda\vb*{E}-\vb*{A}| & =\mqty|\lambda+3 & 0  & 0 \\0&\lambda+1&-2\\0&-2&\lambda-2|=(\lambda+3)\qty(\lambda^2-\lambda-6)\\
        |\lambda\vb*{E}-\vb*{B}| & =\mqty|\lambda   & -3 & 0 \\-3&\lambda&0\\0&0&\lambda-k|=(\lambda-k)\qty(\lambda^2-9)
    \end{flalign*}
    那么 $\vb*{A}$ 的特征值为 $\lambda_1=-3\text{, }\lambda_2=-2\text{, }\lambda_3=3$, $\vb*{B}$ 的特征值为 $\lambda_1=-3\text{, }\lambda_2=k\text{, }\lambda_3=3$, 所以 $k<0$ 且 $k\neq-2.$
\end{solution}



% \subsubsection{合同变换法的改进——行初等变换法}

% 对矩阵 $\vb*{A}$ 作一系列初等行变换化为上三角形矩阵, 相同的列变换就将它化成了对角矩阵 $\vb*{\varLambda}$ (上三角形矩阵的对角元就是 $\vb*{\varLambda}$ 的对角元), 所用行变换同时将单位矩阵化成了 $\vb*{C}^\top$, 即

% $$(\vb*{A},\vb*{I})\xrightarrow{\text{行初等变换}}\mqty(\mqty(d_1 & & *\\&\ddots&\\& & d_n),\vb*{C}^\top)$$
% 则 $\vb*{C}^\top\vb*{AC}=\mqty(d_1 & & \\&\ddots&\\& & d_n).$

\begin{example}
    用行初等变换法将矩阵 $\vb*{A}=\mqty(1&1&0\\1&2&-1\\0&-1&-4)$ 化为标准形.
\end{example}
\begin{solution}
    $(\vb*{A},\vb*{I})=\begin{pNiceArray}{ccc:ccc}
            1&1&0&1&0&0\\
            1&2&-1&0&1&0\\
            0&-1&-4&0&0&1
        \end{pNiceArray}\xrightarrow[]{r_2-r_1}\begin{pNiceArray}{ccc:ccc}
            1&1&0&1&0&0\\
            0&1&-1&-1&1&0\\
            0&-1&-4&0&0&1
        \end{pNiceArray}\xrightarrow[]{r_3+r_2}\begin{pNiceArray}{ccc:ccc}
            1 &  1 &  0  & 1  & 0 &  0\\
            0 &  1 &  -1 &  -1& 1 &  0\\
            0 &  0 &  -5 &  -1& 1 &1
        \end{pNiceArray}$
    由此可得 $\vb*{\varLambda}=\mqty(\dmat{1,1,-5}),~\vb*{C}=\mqty(1&-1&-1\\0&1&1\\0&0&1)$, 且 $\vb*{C}^\top\vb*{AC}=\vb*{\varLambda}.$
\end{solution}

\begin{example}
    设 3 阶矩阵 $\vb*{A}=\begin{pmatrix} 1 & 1 & 0 \\ 1 & 0 & 1 \\ 0 & 1 & -1 \\\end{pmatrix}, \vb*{B}=\begin{pmatrix} 1 & 3 & 1 \\ 3 & a & 1 \\ 1 & 1 & 0 \\\end{pmatrix}$,
    \begin{enumerate}[label=(\arabic{*})]
        \item 若矩阵 $\vb*{A}$ 与 $\vb*{B}$ 合同, 求 $a$ 与可逆矩阵 $\vb*{P}$, 使得 $\vb*{P}^\top\vb*{AP}=\vb*{B}$;
        \item 矩阵 $\vb*{A}$ 与 $\vb*{B}$ 是否相似.
    \end{enumerate}
\end{example}
\begin{solution}
    \begin{enumerate}[label=(\arabic{*})]
        \item 因为 $\vb*{A}$ 与 $\vb*{B}$ 合同, 所以 $\rank\vb*{A}=\rank\vb*{B}$, 那么
        $$\det\vb*{A}=0=\det\vb*{B}=5-a\Rightarrow a=5$$
        实对称矩阵 $\vb*{A}$ 对应二次型为 $f(x_1,x_2,x_3)=x_1^2-x_3^2+2x_1x_2+2x_2x_3=(x_1+x_2)^2-(x_2-x_3)^2$
        作可逆线性变化 $\begin{cases}
            z_1=x_1+x_2\\ z_2=x_2-x_3\\ z_3=x_3
        \end{cases}$ 即 $$\mqty( z_1 \\ z_2 \\ z_3 )=\begin{pmatrix} 1 & 1 & 0 \\ 0 & 1 & -1 \\ 0 & 0 & 1 \\\end{pmatrix}\mqty( x_1 \\ x_2 \\ x_n )=\vb*{P}_1\vb*{X}$$
        实对称矩阵 $\vb*{B}$ 对应二次型为 $f(y_1, y_2, y_3)=y_1^2+5y_2^2+6y_1y_2+2y_1y_3+2y_2y_3=(y_1+3y_2+y_3)^2-(y_2-y_3)^2$ 作可逆线性变化 $\begin{cases}
            z_1=y_1+3y_2+y_3\\ z_2=y_2-y_3\\ z_3=y_3
        \end{cases}$
        即 
        $$
        \mqty( z_1 \\ z_2 \\ z_n )=\begin{pmatrix} 1 & 3 & 1 \\ 0 & 2 & 1 \\ 0 & 0 & 1 \\\end{pmatrix}\mqty( y_1 \\ y_2 \\ y_n )=\vb*{P}_2\vb*{X}
        $$
        则 $$\vb*{X}=\vb*{P}_1^{-1}\vb*{P}_2\vb*{Y}=\begin{pmatrix} 1 & 1 & 0 \\ 0 & 1 & -1 \\ 0 & 0 & 1 \\\end{pmatrix}^{-1}\begin{pmatrix} 1 & 3 & 1 \\ 0 & 2 & 1 \\ 0 & 0 & 1 \\\end{pmatrix}\mqty( y_1 \\ y_2 \\ y_n )=\begin{pmatrix} 1 & 1 & -1 \\ 0 & 2 & 2 \\ 0 & 0 & 1 \\\end{pmatrix}\mqty( y_1 \\ y_2 \\ y_n )$$
        记 $\vb*{P}=\begin{pmatrix} 1 & 1 & -1 \\ 0 & 2 & 2 \\ 0 & 0 & 1 \\\end{pmatrix}$, 于是在可逆线性变换 $\vb*{X}=\vb*{PY}$ 下, 二次型 $f(\vb*{X})$ 化为二次型 $g(\vb*{Y})$, 从而有 $\vb*{P}^\top\vb*{AP}=\vb*{B}.$
        \item 矩阵 $\vb*{A}$ 与 $\vb*{B}$ 不相似, 理由如下, 因为若矩阵 $\vb*{A} \sim \vb*{B}$ 则 $|\vb*{A}|=|\vb*{B}|,\tr\vb*{A}=\tr\vb*{B}$, 由于 
        $$
        |\vb*{A}|=0,|\vb*{B}|=5-a,\tr\vb*{A}=0,\tr\vb*{B}=1+a
        $$
        方程组 $\begin{cases}
            5-a=0\\ 1+a=0
        \end{cases}$ 无解, 所以对任意实数 $a$, 矩阵 $\vb*{A}$ 与 $\vb*{B}$ 均不相似.
    \end{enumerate}
\end{solution}