\section{幂级数}

幂级数可以在某个区间内收敛,也可以在某些点处发散. 对于幂级数,我们通常关心以下几个问题: 

\begin{enumerate}
    \item 收敛半径: 幂级数的收敛半径 $R$ 是一个非负实数,表示幂级数在哪些区间内收敛. 收敛半径的计算可以使用根试验、比值试验、比较试验等方法.
    \item 收敛区间: 根据收敛半径 $R$,我们可以确定幂级数的收敛区间. 在收敛区间内,幂级数可以表示为某个函数的展开形式.
    \item 函数表示: 在收敛区间内,幂级数可以表示为一个函数的展开形式. 这种展开形式有助于我们研究函数的性质和行为.
    \item 求和: 在幂级数的收敛区间内,我们可以通过幂级数的求和来计算函数在某一点的值,或者对函数进行近似.
\end{enumerate}

\subsection{下降阶乘幂}

\begin{definition}[下降阶乘幂]
    $x^{\underline{m} }=x(x-1)\cdots(x-m+1)~~(m\in\mathbb{N}_+).$
\end{definition}

\begin{definition}[上升阶乘幂]
    $x^{\overline{m}}=x(x+1)\cdots(x+m-1)~~(m\in\mathbb{N}_+).$
\end{definition}

\begin{definition}
    $x^{\underline{0}}=x^{\overline{0}}=1.$
\end{definition}

\begin{theorem}[阶乘与阶乘幂]
    $n!=n^{\underline{n}}=1^{\overline{n}}.$
    \index{阶乘与阶乘幂}
\end{theorem}

\begin{theorem}
    $\varDelta \qty(x^{\underline{m}})=mx^{\underline{m-1}}.$
\end{theorem}
\begin{proof}[{\songti \textbf{证}}]
    $\varDelta\qty(x^{\underline{m}})=(x+1)^{\underline{m}}-x^{\underline{m}}=(x+1)x\cdots(x-m+2)-x\cdots(x-m+2)(x-m+1)=mx(x-1)\cdots(x-m+2)=mx^{\underline{m-1}}.$
\end{proof}

\begin{definition}[不定和]
    $\displaystyle g(x)=\varDelta f(x)\iff \sum g(x)\delta x=f(x)+C.$
\end{definition}

\begin{definition}[定和]
    $\displaystyle \sum\nolimits_{a}^{b} g(x)\delta x=\eval{f(x)}_{a}^{b}=f(b)-f(a).$
\end{definition}

\begin{enumerate}[label=(\arabic{*})]
    \item $\displaystyle \sum\nolimits_a^a g(x)\delta x=f(a)-f(a)=0$;
    \item $\displaystyle \sum\nolimits_a^{a+1} g(x)\delta x=f(a+1)-f(a)=g(a)$;
    \item $\displaystyle \sum\nolimits_{a}^{b+1}g(x)\delta x-\sum\nolimits_{a}^{b}g(x)\delta x=f(b+1)-f(a)-f(b)+f(a)=g(b)$.
\end{enumerate}

\begin{theorem}
    $\displaystyle \sum\nolimits_{a}^{b}g(x)\delta x=\sum_{k=a}^{b-1}g(k)=\sum_{a\leqslant k<b}g(k)~~(b\geqslant a).$
\end{theorem}

\begin{theorem}
    $\displaystyle \sum\nolimits_{a}^{b}g(x)\delta x+\sum\nolimits_b^c g(x)\delta x=\sum\nolimits_a^cg(x)\delta x.$
\end{theorem}

\begin{theorem}
    $\displaystyle \sum_{0\leqslant k<n}k^{\underline{m}}=\eval{\dfrac{k^{\underline{m+1}}}{m+1}}_{0}^{n}=\dfrac{n^{\underline{m+1}}}{m+1}~~(m,n\in\mathbb{N}_+).$
\end{theorem}

\begin{definition}
    $x^{\underline{-m}}=\dfrac{1}{(x+1)(x+2)\cdots(x+m)}~~(m>0).$
\end{definition}

\begin{theorem}
    $x^{\underline{m+n}}=x^{\underline{m}}(x-m)^{\underline{n}} ~~(m,n\in\mathbb{Z}).$
\end{theorem}

\begin{theorem}
    $\displaystyle\sum\nolimits_a^b x^{\underline{m}}\delta x=\begin{cases}
            \eval{\dfrac{x^{\underline{m+1}}}{m+1}}_{a}^{b}, & m\neq -1 \\
            \eval{H_x}_{a}^{b},                              & m=-1
        \end{cases}$,
    其中 $\displaystyle H_x=\sum_{k=1}^{x}\dfrac{1}{k}$ 称为调和级数.
\end{theorem}

\begin{definition}
    $\mathrm{E}f(x)=f(x+1).$
\end{definition}

\begin{theorem}[分部和公式]
    $\displaystyle\sum u\varDelta v=uv-\sum\mathrm{E}v\varDelta u.$
    \index{分部和公式}
\end{theorem}

\setcounter{magicrownumbers}{0}
\begin{table}[H]
    \centering
    \begin{tabular}{c c | c c}
        $f=\sum g$                         & $\varDelta f=g$                     & $f=\sum g$           & $\varDelta f=g$                       \\
        \midrule
        $x^{\underline{0}}=1$              & 0                                   & $2^{x}$              & $2^{x}$                               \\
        $x^{\underline{1}}=x$              & 1                                   & $c^{x}$              & $(c-1)c^{x}$                          \\
        $x^{\underline{2}}=x(x-1)$         & $2x$                                & $\dfrac{c^{x}}{c-1}$ & $c^x$                                 \\
        $x^{\underline{m}}$                & $mx^{\underline{m-1}}$              & $cf$                 & $c\varDelta f$                        \\
        $\dfrac{x^{\underline{m+1}}}{m+1}$ & $x^{\underline{m}}$                 & $f+g$                & $\varDelta f+\varDelta g$             \\
        $H_x$                              & $x^{\underline{-1}}=\dfrac{1}{x+1}$ & $fg$                 & $f\varDelta g+\mathrm{E}v\varDelta f$
    \end{tabular}
\end{table}

\begin{example}[第五届数学竞赛初赛]
    判断级数 $\displaystyle\sum_{n=1}^{\infty}\dfrac{1+\dfrac{1}{2}+\cdots+\dfrac{1}{n}}{(n+1)(n+2)}$ 的敛散性,若收敛,则求其和.
\end{example}
\begin{solution}
    令 $\displaystyle H_n=\sum_{k=1}^{n}\dfrac{1}{k}~,S_n=\sum_{k=1}^{n}\dfrac{H_k}{(k+1)(k+2)}=\sum_{\substack{1\leqslant k<n+1\\k\in \mathbb{N}_+}}\dfrac{H_k}{(k+1)(k+2)}=\sum\nolimits_1^{n+1}\dfrac{H_x}{(x+1)(x+2)}\delta x$,
    那么记 $$u(x)=H_x~,\varDelta v(x)=\dfrac{1}{(x+1)(x+2)}=-\dfrac{1}{x+2}-\qty(-\dfrac{1}{x+1})$$
    因此 $$\varDelta u(x)=x^{\underline{-1}},~v(x)=-\dfrac{1}{x+1}=-x^{\underline{-1}},~\mathrm{E}v(x)=-(x+1)^{\underline{-1}}$$
    于是 \begin{flalign*}
        S_n & =\sum\nolimits_1^{n+1}\dfrac{H_x}{(x+1)(x+2)}\delta x=\eval{-x^{\underline{-1}}H_x}_{1}^{n+1}+\sum\nolimits_{1}^{n+1}(x+1)^{\underline{-1}}x^{\underline{-1}}\delta x          \\
            & =-(n+1)^{\underline{-1}}H_{n+1}+1^{\underline{-1}}H_1+\sum\nolimits_{1}^{n+1}x^{\underline{-2}}\delta x=\dfrac{1}{2}-\dfrac{H_{n+1}}{n+2} -\eval{x^{\underline{-1}}}_{1}^{n+1} \\
            & =1-\dfrac{1}{n+2}-\dfrac{H_{n+1}}{n+2}=1-\dfrac{1}{n+1}-\dfrac{H_n}{n+2}
    \end{flalign*}
    又因为 $$\lim_{n\to\infty}\dfrac{H_n}{n+2}=\lim_{n\to\infty}\dfrac{\ln n+\gamma+\alpha(n)}{n+2}=\lim_{n\to\infty}\dfrac{\ln n}{n+2}=0~~(\alpha(n)\to0)$$
    所以 $\displaystyle\lim_{n\to\infty}S_n=1$,因此级数收敛,且其和为 1.
\end{solution}

\subsection{幂级数的性质}

\begin{definition}[幂级数]
    形如 $\displaystyle  \sum_{n=0}^{\infty} a_{n}\left(x-x_{0}\right)^{n} $ 的函数项级数称为 $ x-x_{0} $ 的\textit{幂级数},
    其中 $ a_{n}(n=0,1,2 , \cdots) $ 称为幂级数的项 $ \left(x-x_{0}\right)^{n} $ 的系数. 当 $ x_{0}=0 $ 时,$\displaystyle \sum_{n=0}^{\infty} a_{n} x^{n} $ 称为 $ x $ 的幂级数.
\end{definition}

\begin{theorem}[Abel 定理]
    若幂级数 $\displaystyle  \sum_{n=1}^{\infty} a_{n} x^{n} $ 在 $ x=x_{0}\left(x_{0} \neq 0\right) $ 处收敛,
    则它在满足不等式 $ |x|<\left|x_{0}\right| $ 的一切 $ x $ 处绝对收敛;
    若幂级数 $\displaystyle \sum_{n=0}^{\infty} a_{n} x^{n} $ 在 $ x=x_{0} $ 处发散,则它在满足不等式 $ |x|>\left|x_{0}\right| $ 的一切 $ x $ 处发散.
    \index{Abel 定理}
\end{theorem}


\begin{definition}[收敛半径与收敛区间]
    由 Abel 定理可知,必存在 $ R>0 $,当 $ |x|<R $ 时,幂级数绝对收敛; 当 $ |x|>R $ 时,幂级数发散,则称 $ R $ 为幂级数 $\displaystyle \sum_{n=0}^{\infty} a_{n} x^{n} $ 的收敛半径,$(-R, R) $ 为其收敛区间.\\
    特别地,当 $ x=R $ 与 $ x=-R $ 时,幂级数可能收敛,也可能发散;
    若幂级数 $\displaystyle \sum_{n=0}^{\infty} a_{n} x^{n} $ 仅在 $ x=0 $ 处收敛,其他点处都发散,则 $ R=0 $,收敛域为 $ \qty{0}$;
    若幂级数 $\displaystyle \sum_{n=0}^{\infty} a_{n} x^{n} $ 处处收敛,则 $ R=\infty $,收敛域为 $ (-\infty,+\infty) $.
\end{definition}

\begin{theorem}
    设 $\displaystyle \sum_{n=0}^{\infty} a_{n} x^{n}\left(a_{n} \neq 0\right) $,其收敛半径为 $ R $,
    则 $\displaystyle  R=\lim _{n \rightarrow \infty}\left|\frac{a_{n}}{a_{n+1}}\right| $ 或 $\displaystyle R=\dfrac{1}{\displaystyle\lim _{n \rightarrow \infty} \sqrt[n]{\left|a_{n}\right|}} .$
\end{theorem}

\begin{theorem}[奇偶次项幂级数的收敛半径]
    对于只含有偶次项 $\displaystyle\sum_{n=1}^{\infty}(x-x_0)^{2n}$ 或只含有奇次项 $\displaystyle\sum_{n=1}^{\infty}(x-x_0)^{2n+1}$ 的幂级数的收敛半径为:
    若 $\displaystyle\lim_{n\to\infty}\qty|\dfrac{a_{n+1}}{a_n}|=\rho$,则 $R=\dfrac{1}{\sqrt{\rho}}$; 若 $\displaystyle\lim_{n\to\infty}\sqrt[n]{a_n}=\rho$,则 $R=\dfrac{1}{\sqrt{\rho}}.$
    \index{奇偶次项幂级数的收敛半径}
\end{theorem}

\begin{theorem}[幂级数的运算性质]
    若幂级数 $\displaystyle \sum_{n=0}^{\infty} a_{n} x^{n} $ 与 $\displaystyle \sum_{n=0}^{\infty} b_{n} x^{n} $ 的收敛半径分别为 $ R_{1} $ 和 $ R_{2} $,
    当 $ R_{1} \neq R_{2} $ 时,令 $ R   =\min \left\{R_{1},R_{2}\right\} $,则有
    \begin{enumerate}[label=(\arabic{*})]
        \item $\displaystyle k \sum_{n=1}^{\infty} a_{n} x^{n}=\sum_{n=1}^{\infty} k a_{n} x^{n},|x|<R_{1} $,其中 $ k $ 为常数;
        \item $\displaystyle \sum_{n=0}^{\infty} a_{n} x^{n} \pm \sum_{n=0}^{\infty} b_{n} x^{n}=\sum_{n=0}^{\infty}\left(a_{n} \pm b_{n}\right) x^{n},|x|<R $;
        \item $\displaystyle \left(\sum_{n=0}^{\infty} a_{n} x^{n}\right)\left(\sum_{n=0}^{\infty} b_{n} x^{n}\right)=\sum_{n=0}^{\infty} c_{n} x^{n},|x|<R $,其中 $\displaystyle c_{n}=\sum_{k=0}^{n} a_{k} b_{n-k} .$
    \end{enumerate}
    \index{幂级数的运算性质}
\end{theorem}

\begin{theorem}[幂级数的性质]
    设 $\displaystyle \sum_{n=0}^{\infty} a_{n} x^{n} $ 的收敛半径为 $ R(R>0) $,则有
    \begin{enumerate}[label=(\arabic{*})]
        \item 和函数 $\displaystyle s(x)=\sum_{n=0}^{\infty} a_{n} x^{n} $ 在收敛域上连续,若 $\displaystyle \sum_{n=0}^{\infty} a_{n} x^{n} $ 在端点 $ x=R $ (或 $ x=-R$)  处收敛,
              则 $ s(x) $ 在 $ x=R $ 处左连续 (或在 $ x=-R $ 处右连续);
        \item 和函数 $\displaystyle s(x)=\sum_{n=0}^{\infty} a_{n} x^{n} $ 在收敛区间上可导,且有逐项求导公式
              $$s^{\prime}(x)=\left(\sum_{n=0}^{\infty} a_{n} x^{n}\right)^{\prime}=\sum_{n=1}^{\infty} n a_{n} x^{n-1} $$
        \item 和函数 $\displaystyle s(x)=\sum_{n=0}^{\infty} a_{n} x^{n} $ 在收敛域上可积,且有逐项求积公式
              $$\int_{0}^{x}s(t)\dd t=\sum_{n=0}^{\infty}\int_{0}^{x}a_nt^n\dd t=\sum_{n=0}^{\infty}\dfrac{a_n}{n+1}x^{n+1}.$$
    \end{enumerate}
    \index{幂级数的性质}
\end{theorem}

\subsection{函数展开为幂级数}

\subsubsection{常见函数的幂级数展开式}

\setcounter{magicrownumbers}{0}
\label{table mijishu}
\begin{table}[H]
    \begin{minipage}{\textwidth}
        \centering
        \begin{tabular}{l l}
            (\rownumber{}) $\displaystyle\mathrm{e}^x=\sum_{n=0}^{\infty}\frac{x^n}{n!},~|x|<+\infty$                                                                                                                                                                                                                                                                                & (\rownumber{}) $\displaystyle a^x=\sum_{n=0}^{\infty}\ln ^na\frac{ x^n}{n!},~|x|<+\infty$                                                                                                                                                                                                                                                                          \\
            \midrule
            \multirow{2}{*}{(\rownumber{}) $\displaystyle\ln(1+x)=\sum_{n=0}^{\infty}(-1)^n\frac{x^{n+1}}{n+1},~x\in(-1,1]$}                                                                                                                                                                                                                                                         & (\rownumber{}) $\displaystyle-\ln(1-x)=\sum_{n=1}^{\infty}\frac{x^n}{n},~|x|<1$                                                                                                                                                                                                                                                                                    \\
                                                                                                                                                                                                                                                                                                                                                                                     & (\rownumber{}) $\displaystyle\ln\frac{1+x}{1-x}=2\sum_{n=0}^{\infty}\frac{x^{2n+1}}{2n+1},~|x|<1$                                                                                                                                                                                                                                                                  \\
            \midrule
            (\rownumber{}) $\displaystyle\sin x=\sum_{n=0}^{\infty}(-1)^n\frac{x^{2n+1}}{(2n+1)!},~|x|<+\infty$                                                                                                                                                                                                                                                                      & (\rownumber{}) $\displaystyle\cos x=\sum_{n=0}^{\infty}(-1)^n\frac{x^{2n}}{(2n)!},~|x|<+\infty$                                                                                                                                                                                                                                                                    \\
            (\rownumber{}) \tablefootnote{$\displaystyle \arctan x=\int _{0}^{x}\dfrac{\mathrm{d}t}{1+t^{2}}=\int _{0}^{x}\sum ^{\infty }_{n=0}\left( -t^{2}\right) ^{n}\mathrm{d}t=\sum ^{\infty }_{n=0}(-1)^{n}\int _{0}^{x}t^{2n}\mathrm{d}t=\sum ^{\infty }_{n=0}(-1)^{n}\dfrac{x^{2n+1}}{2n+1}$}$\displaystyle\arctan x=\sum_{n=0}^{\infty}(-1)^n\frac{x^{2n+1}}{2n+1},~|x|<1.$ & (\rownumber{}) \tablefootnote{由 $\displaystyle \left( \arcsin x\right) '=\left( 1-x^{2}\right) ^{-\frac{1}{2}}=1+\sum ^{\infty }_{n=0}\dfrac{\left( -\dfrac{1}{2}\right) \left( -\dfrac{1}{2}-1\right) \ldots \left( -\dfrac{1}{2}+n-1\right) }{n!}\left( -x^{2}\right) ^{n}=1+\sum ^{\infty }_{n=1}\dfrac{\left( 2n\right) !}{4^{n}\left( n!\right) ^{2}}x^{2n}$ \\
            则 $\displaystyle\arcsin x=\int _{0}^{x}\left[ 1+\sum ^{\infty }_{n=1}\dfrac{\left( 2n\right) !}{4^{n}\left( n!\right) ^{2}}t^{2n}\right] \mathrm{d}t=x+\sum ^{\infty }_{n=1}\dfrac{\left( 2n\right) !}{4^{n}\left( n!\right) ^{2}}\int _{0}^{x}t^{2n}\mathrm{d}t=\sum_{n=0}^{\infty}\frac{(2n)!}{4^n(n!)^2(2n+1)}x^{2n+1}$}$\displaystyle\arcsin x=\sum_{n=0}^{\infty}\frac{(2n)!}{4^n(n!)^2}\dfrac{x^{2n+1}}{(2n+1)},~|x|<1$                                                                                                                                                                                                                                                                                                                \\
            \midrule
            (\rownumber{}) $\displaystyle\frac{1}{1-x}=\sum_{n=0}^{\infty}x^n,~|x|<1$                                                                                                                                                                                                                                                                                                & (\rownumber{}) $\displaystyle\frac{1}{(1-x)^2}=\sum_{n=1}^{\infty}(n+1)x^n,~|x|<1$                                                                                                                                                                                                                                                                                 \\
            (\rownumber{}) $\displaystyle\dfrac{2}{(1-x)^3}=\sum_{n=0}^{\infty}(n+2)(n+1)x^n,~|x|<1$                                                                                                                                                                                                                                                                                 & (\rownumber{}) $\displaystyle\dfrac{m}{(1-x)^{m+1}}=\sum_{n=0}^{\infty}(n+m)^{\underline{m}}x^n,~|x|<1$                                                                                                                                                                                                                                                            \\
            (\rownumber{}) $\displaystyle \dfrac{1}{1+x}=\sum_{n=0}^{\infty}(-1)^nx^n,~|x|<1$                                                                                                                                                                                                                                                                                        & (\rownumber{}) $\displaystyle\dfrac{x^q}{1+x^p}=\sum_{n=0}^{\infty}(-1)^nx^{np+q},~|x|<1$                                                                                                                                                                                                                                                                          \\
            \midrule
            \multicolumn{2}{l}{(\rownumber{}) $\displaystyle(1+x)^\alpha=1+\sum_{n=1}^{\infty}\dfrac{\alpha(\alpha-1)\cdots(\alpha-n+1)}{n!} x^n=\sum_{n=0}^{\infty}\binom{\alpha}{n}x^n=\sum_{n=0}^{\infty}\dfrac{\alpha^{\underline{n}}}{n!}x^n,~|x|<1$}
        \end{tabular}
    \end{minipage}
\end{table}

\begin{example}
    写出下列函数关于 $x$ 的幂级数展开式
    \setcounter{magicrownumbers}{0}
    \begin{table}[H]
        \centering
        \begin{tabular}{l | l | l | l | l}
            (\rownumber{}) $\displaystyle\mathrm{e}^{-x^2}.$ & (\rownumber{}) $\cos ^2x.$                                  & (\rownumber{}) $\sin^3x.$              & (\rownumber{}) $\dfrac{x^{10}}{1-x}.$       & (\rownumber{}) $\dfrac{1}{(1-x)^2}.$                        \\
            (\rownumber{}) $\ln\sqrt{\dfrac{1+x}{1-x}}.$     & (\rownumber{}) $\dfrac{x^{2}}{\left( 1+x^{2}\right) ^{2}}.$ & (\rownumber{}) $\dfrac{x}{2+x-x^{2}}.$ & (\rownumber{}) $\arctan\dfrac{1-2x}{1+2x}.$ & (\rownumber{}) $\dfrac{1}{\left(1-x^2\right)\sqrt{1-x^2}}.$
        \end{tabular}
    \end{table}
\end{example}
\begin{solution}
    \begin{enumerate}[label=(\arabic{*})]
        \item $\displaystyle\mathrm{e}^{-x^2}=\sum ^{\infty }_{n=0}\dfrac{\left( -x^{2}\right) ^{n}}{n!}=\sum ^{\infty }_{n=0}(-1) ^{n}\dfrac{x^{2n}}{n!}~~( \left| x\right|  < +\infty ) .$
        \item $\displaystyle \cos ^{2}x=\dfrac{1+\cos 2x}{2}=\dfrac{1}{2}+\dfrac{1}{2}\sum ^{\infty }_{n=0}\dfrac{(-1) ^{n}}{\left( 2n\right) !}\left( 2x\right) ^{2n}=1+\sum ^{\infty }_{n=1}(-1) ^{n}\dfrac{2^{2n-1}}{\left( 2n\right) !}x^{2n}~~( \left| x\right|  < +\infty ) .$
        \item 注意到 $\sin^3x=\dfrac{3}{4}\sin x-\dfrac{1}{4}\sin 3x$,于是
              \begin{flalign*}
                  \sin ^{3}x & =\dfrac{3}{4}\sin x-\dfrac{1}{4}\sin 3x=\dfrac{3}{4}\sum ^{\infty }_{n=0}\dfrac{(-1) ^{n}}{\left( 2n+1\right) !}x^{2n+1}-\dfrac{1}{4}\sum ^{\infty }_{n=0}\dfrac{(-1) ^{n}}{\left( 2n+1\right) !}\left( 3x\right) ^{2n+1} \\
                             & =\dfrac{3}{4}\sum ^{\infty }_{n=0}(-1) ^{n+1}\dfrac{3^{2n}-1}{\left( 2n+1\right) !}x^{2n+1}~~( \left| x\right|  < +\infty ) .
              \end{flalign*}
        \item $\displaystyle \dfrac{x^{10}}{1-x}=x^{10}\sum ^{\infty }_{n=0}x^{n}=\sum ^{\infty }_{n=10}x^{n}~~( \left| x\right|  < 1) .$
        \item $\displaystyle\dfrac{1}{\left( 1-x\right) ^{2}}=\left( 1-x\right) ^{-2}=1+\sum ^{\infty }_{n=1}\dfrac{\left( -2\right) (-2-1) \cdots \left( -2+n-1\right) }{n!}\left( -x\right) ^{n}=\sum ^{\infty }_{n=1}\left( n+1\right) x^{n}~~(|x|<1).$
        \item $\displaystyle\ln\sqrt{\dfrac{1+x}{1-x}}=\dfrac{1}{2}\ln \dfrac{1+x}{1-x}=\sum ^{\infty }_{n=0}\dfrac{x^{2n+1}}{2n+1}~~( \left| x\right|  < 1). $
        \item 由 $(1+x)^\alpha$ 展开式得 $$\left( 1+x^{2}\right) ^{-2}=1+\sum ^{\infty }_{n=1}\dfrac{\left( -2\right) (-2-1) \ldots \left( -2+n-1\right) }{n!}\left( x^{2}\right) ^{n}=1+\sum ^{\infty }_{n=1}(-1) ^{n}\left( n+1\right) x^{2n}$$
              故 $\displaystyle\dfrac{x^{2}}{\left( 1+x^{2}\right) ^{2}}=x^{2}+\sum ^{\infty }_{n=1}(-1) ^{n}\left( n+1\right) x^{2n+2}=\sum ^{\infty }_{n=1}(-1) ^{n-1}x^{2n}~~( \left| x\right|  < 1) .$
        \item 因为 $\displaystyle \dfrac{x}{2+x-x^{2}}=\dfrac{x}{3}\left( \dfrac{1}{1+x}+\dfrac{1}{2-x}\right) =\dfrac{x}{3}\left( \dfrac{1}{1+x}+\dfrac{1}{2}\cdot \dfrac{1}{1-\dfrac{x}{2}}\right) $,所以
              \begin{flalign*}
                  \dfrac{x}{2+x-x^{2}}=\dfrac{x}{3}\left[ \sum ^{\infty }_{n=0}(-1) ^{n}x^{n}+\dfrac{1}{2}\sum ^{\infty }_{n=0}\left( \dfrac{x}{2}\right) ^{n}\right] =\dfrac{1}{3}\sum ^{\infty }_{n=0}\left[ (-1) ^{n}+\dfrac{1}{2^{n+1}}\right] x^{n+1}~~(\left| x\right|  <1).
              \end{flalign*}
        \item 因为 $\arctan \dfrac{A-B}{1+AB}=\arctan A-\arctan B$,所以
              \begin{flalign*}
                  \arctan\dfrac{1-2x}{1+2x}=\arctan 1-\arctan 2x=\dfrac{\pi}{4}-\sum_{n=0}^{\infty}\dfrac{(-1)^n}{2n+1}(2x)^{2n+1}~~\left(x\neq-\dfrac{1}{2}\right).
              \end{flalign*}
        \item $\displaystyle\left( 1-x^{2}\right) ^{-\frac{3}{2}}=\sum ^{\infty }_{n=0}\dfrac{\left( -\dfrac{3}{2}\right) \left( -\dfrac{3}{2}-1\right) \ldots \left( -\dfrac{3}{2}-n+1\right) }{n!}\left( -x^{2}\right) ^{n}=\sum ^{\infty }_{n=0}\dfrac{\left( 2n+1\right) !!}{\left( 2n\right) !!}x^{2n}~~( \left| x\right|  < 1) .$
    \end{enumerate}
\end{solution}

\begin{example}
    写出下列函数关于 $x$ 的幂级数展开式
    \setcounter{magicrownumbers}{0}
    \begin{table}[H]
        \centering
        \begin{tabular}{l | l | l}
            (\rownumber{}) $(1+x)\ln(1+x).$            & (\rownumber{}) $\dfrac{1}{4}\ln\dfrac{1+x}{1-x}+\dfrac{1}{2}\arctan x.$ & (\rownumber{}) $\arctan\dfrac{2-2x}{1+4x}.$             \\
            (\rownumber{}) $\arctan\dfrac{2x}{2-x^2}.$ & (\rownumber{}) $x\arctan x-\ln\sqrt{1+x^2}.$                            & (\rownumber{}) $x\ln\qty(x+\sqrt{1+x^2})-\sqrt{1+x^2}.$
        \end{tabular}
    \end{table}
\end{example}
\begin{solution}
    \begin{enumerate}[label=(\arabic{*})]
        \item $\displaystyle f(x)=(1+x)\ln(1+x)=(1+x)\sum_{n=0}^{\infty}\dfrac{(-1)^n}{n+1}x^{n+1}~  |x|<1$,又 $|x|=1$ 时,级数收敛,因此原级数的收敛域为 $|x|\leqslant 1$.
        \item $\displaystyle f(x)=\dfrac{1}{4}\cdot 2\sum_{n=0}^{\infty}\dfrac{x^{2n+1}}{2n+1}+\dfrac{1}{2}\cdot\sum_{n=0}^{\infty}\dfrac{(-1)^n}{2n+1}x^{2n+1}=\sum_{n=0}^{\infty}\dfrac{x^{4n+1}}{4n+1}~  (|x|<1).$
        \item \textbf{法一: }由 $\arctan\dfrac{A-B}{1+AB}=\arctan A-\arctan B$,于是 $$\displaystyle \arctan \dfrac{2-2x}{1+4x}=\arctan 2-\arctan(2x)=\arctan 2-\sum_{n=0}^{\infty}\dfrac{(-1)^n}{2n+1}(2x)^{2n+1}~  \qty(|x|<\dfrac{1}{2})$$
              当 $|x|=\dfrac{1}{2}$ 时,级数为交错级数,且满足 Leibniz 判别法的条件,因此原级数的收敛域为 $|x|\leqslant\dfrac{1}{2}.$\\
              \textbf{法二: }$f'(x)=\qty(\arctan\dfrac{2-2x}{1+4x})'=-\dfrac{2}{1+4x^2}$,于是
              \begin{flalign*}
                  \arctan\dfrac{2-2x}{1+4x} & =-2\int_{0}^{x}\dfrac{\dd t}{1+4t^2}+\arctan 2=-2\int_{0}^{x}\qty[\sum_{n=0}^{\infty}(-1)^n(4t^2)^n]\dd t+\arctan 2                \\
                                            & =-2\sum_{n=0}^{\infty}(-1)^n4^n\int_{0}^{x}t^{2n}\dd t+\arctan 2=\arctan 2+\sum_{n=1}^{\infty}(-1)^n\dfrac{2^{2n-1}}{2n-1}x^{2n-1}
              \end{flalign*}
              收敛域的求法同法一.
        \item 由于 $\displaystyle f'(x)=\qty(\arctan\dfrac{2x}{2-x^2})'=\dfrac{4+2x^2}{4+x^4}$,故
              \begin{flalign*}
                  \arctan\dfrac{2x}{2-x^2} & =\int_{0}^{x}\dfrac{4+2t^2}{4+t^4}\dd t=\int_{0}^{x}\qty(1+\dfrac{t^2}{2})\sum_{n=0}^{\infty}(-1)^n\qty(\dfrac{t^4}{4})^n\dd t                                                 \\
                                           & =\int_{0}^{x}\qty[\sum_{n=0}^{\infty}(-1)^{\qty[\frac{n}{2}]}\dfrac{t^{2n}}{2^n}]\dd t=\sum_{n=0}^{\infty}(-1)^{\qty[\frac{n}{2}]}\dfrac{x^{2n+1}}{2^n(2n+1)}~  (|x|<\sqrt{2})
              \end{flalign*}
              当 $|x|=\sqrt{2}$ 时,级数为 $$\sqrt{2}\sum_{n=0}^{\infty}(-1)^{\qty[\frac{n}{2}]}\dfrac{1}{2n+1}~  \text{及}~  -\sqrt{2}\sum_{n=0}^{\infty}(-1)^{\qty[\frac{n}{2}]}\dfrac{1}{2n+1}$$
              它们均由两收敛级数 $$\sum_{n=0}^{\infty}(-1)^{n-1}\dfrac{1}{4n+1}~ \text{及}~  \sum_{n=0}^{\infty}(-1)^{n-1}\dfrac{1}{4n+3}$$
              逐项相加并乘以常数 $\sqrt{2}$ 及 $-\sqrt{2}$ 而得,故它们收敛,因此原级数的收敛域为 $|x|\leqslant\sqrt{2}.$
        \item $\displaystyle f(x)=x\cdot\sum_{n=0}^{\infty}\dfrac{(-1)^n}{2n+1}x^{2n+1}-\dfrac{1}{2}\sum_{n=0}^{\infty}\dfrac{(-1)^n}{n+1}(x^2)^{n+1}=\sum_{n=0}^{\infty}\dfrac{(-1)^nx^{2n+2}}{(2n+1)(2n+2)}~ (|x|<1)$
              当 $|x|=1$ 时,级数 $\displaystyle\sum_{n=0}^{\infty}\dfrac{(-1)^n}{(2n+1)(2n+2)}$ 收敛,因此原级数的收敛域为 $|x|\leqslant 1$.
        \item 因为 $\qty[\ln\qty(x+\sqrt{1+x^2})]'=\dfrac{1}{\sqrt{1+x^2}}$,所以
              \begin{flalign*}
                  \ln\qty(x+\sqrt{1+x^2}) & =\int_{0}^{x}\dfrac{\dd t}{\sqrt{1+t^2}}=\int_{0}^{x}\qty[1+\sum_{n=1}^{\infty}(-1)^n\dfrac{(2n-1)!!}{(2n)!!}t^{2n}]\dd t                                  \\
                                          & =x+\sum_{n=1}^{\infty}(-1)^n\dfrac{(2n-1)!!}{(2n)!!}\int_{0}^{x}t^{2n}\dd t=x+\sum_{n=1}^{\infty}(-1)^n\dfrac{(2n-1)!!}{(2n)!!}\cdot\dfrac{x^{2n+1}}{2n+1}
              \end{flalign*}
              于是,
              \begin{flalign*}
                  \text{原式} & =x\qty[x+\sum_{n=1}^{\infty}(-1)^n\dfrac{(2n-1)!!}{(2n)!!}\cdot\dfrac{x^{2n+1}}{2n+1}]-\qty[1+\dfrac{1}{2}x^2+\sum_{n=1}^{\infty}(-1)^n\dfrac{(2n-1)!!}{(2n+2)!!}x^{2n+2}] \\
                              & =-1+\dfrac{x^2}{2}+\sum_{n=1}^{\infty}\qty[(-1)^n\dfrac{(2n-1)!!}{(2n+2)!!}\dfrac{x^{2n+2}}{2n+1}]~  (|x|\leqslant 1).
              \end{flalign*}
    \end{enumerate}
\end{solution}

\begin{example}[2003 数一]
    将函数 $f(x)=\arctan \dfrac{1-2x}{1+2x}$ 展开成 $x$ 的幂级数,并求级数 $\displaystyle\sum_{n=0}^{\infty}\dfrac{(-1)^n}{2n+1}$ 的和.
\end{example}
\begin{solution}
    由 $\arctan \dfrac{A-B}{1+AB}=\arctan A-\arctan B$,所以
    $$f(x)=\arctan \dfrac{1-2x}{1+2x}=\arctan 1-\arctan 2x=\dfrac{\pi}{4}-\sum_{n=0}^{\infty}(-1)^n\dfrac{(2x)^{2n+1}}{2n+1}$$
    于是 \begin{flalign*}
        \sum_{n=0}^{\infty}\dfrac{(-1)^n}{2n+1} & =\sum_{n=0}^{\infty}(-1)^n\dfrac{(2x)^{2n+1}}{2n+1}\biggl |_{x=\frac{1}{2}}=\dfrac{\pi}{4}-f\qty(\dfrac{1}{2})=\dfrac{\pi}{4}-\arctan \dfrac{1-1}{1+1}=\dfrac{\pi}{4}.
    \end{flalign*}
\end{solution}

\begin{example}[2007 数三]
    将函数 $f(x)=\dfrac{1}{x^2-3x-4}$ 展开成 $x-1$ 的幂级数,并指出其收敛区间.
\end{example}
\begin{solution}
    因为 $\dfrac{1}{x^2-3x-4}=\dfrac{1}{5}\qty(\dfrac{1}{x-4}-\dfrac{1}{x+1})$,其中
    \begin{flalign*}
        \dfrac{1}{x-4} & =\dfrac{1}{(x-1)-3}=-\dfrac{1}{3}\cdot\dfrac{1}{1-\dfrac{x-1}{3}}=-\sum_{n=0}^{\infty}\dfrac{(x-1)^n}{3^{n+1}}~  x\in(-2,4)   \\
        \dfrac{1}{x+1} & =\dfrac{1}{x-1+2}=\dfrac{1}{2}\cdot\dfrac{1}{\dfrac{x-1}{2}+1}=(-1)^n\sum_{n=0}^{\infty}\dfrac{(x-1)^n}{2^{n+1}}~  x\in(-1,3)
    \end{flalign*}
    故 $\displaystyle\dfrac{1}{x^2-3x-4}=-\dfrac{1}{5}\sum_{n=0}^{\infty}\qty[\dfrac{1}{3^{n+1}}+\dfrac{(-1)^n}{2^{n+1}}](x-1)^n~  x\in(-1,3).$
\end{solution}

\begin{example}
    求函数 $$f(x)=\arctan\qty(\dfrac{2x}{2-x^2})+\dfrac{1}{4}\ln\qty|x^2-2x+2|-\dfrac{1}{4}\ln\qty|x^2+2x+2|-\dfrac{1}{2}\arctan(x-1)-\dfrac{1}{2}\arctan(x-1)$$
    关于 $x$ 的幂级数展开式和收敛半径.
\end{example}
\begin{solution}
    显然 $x^2\pm 2x+2>0$,故
    $$f(x)=\arctan\qty(\dfrac{2x}{2-x^2})+\dfrac{1}{4}\ln\qty(x^2-2x+2)-\dfrac{1}{4}\ln\qty(x^2+2x+2)-\dfrac{1}{2}\arctan(x-1)-\dfrac{1}{2}\arctan(x-1)$$
    对 $f(x)$ 求导,整理可得 $f'(x)=\dfrac{2x^2}{x^4+4}$,由于
    $$\dfrac{1}{4+x^4}=\dfrac{1}{4}\cdot\dfrac{1}{1+\dfrac{x^4}{4}}=\dfrac{1}{4}\sum_{n=0}^{\infty}(-1)^n\dfrac{x^{4n}}{4^n}$$
    其中 $\dfrac{x^4}{4}<1$,故得其收敛半径为 $R=\sqrt{2}$,
    于是 $$f'(x)=\dfrac{2x^2}{x^4+4}=\displaystyle \dfrac{1}{2}\sum_{n=0}^{\infty}(-1)^n\dfrac{x^{4n+2}}{4^n}$$
    对上式两端在 $[0,x]$ 上积分,且 $f(0)=0$,得
    \begin{flalign*}
        f(x)-f(0) & =\int_{0}^{x}\dfrac{1}{2}\sum_{n=0}^{\infty}(-1)^n\dfrac{t^{4n+2}}{4n}\dd t                                                                        \\
        f(x)      & =\dfrac{1}{2}\sum_{n=0}^{\infty}\int_{0}^{x}(-1)^n\dfrac{t^{4n+2}}{4n}\dd t=\dfrac{1}{2}\sum_{n=0}^{\infty}(-1)^n\dfrac{x^{4n+3}}{(4n+3)2^{2n+1}}.
    \end{flalign*}
\end{solution}

\subsection{幂级数的收敛域及和函数}

\subsubsection{求幂级数的收敛域}

求幂级数 $\displaystyle\sum_{n=0}^{\infty}a_nx^n$ 的收敛域的一般方法是,先由公式
$$R=\lim_{n\to\infty}\qty|\dfrac{a_n}{a_{n+1}}|$$
求出收敛半径,若 $R=0$,则幂级数仅在 $x=0$ 处收敛; 若 $R=+\infty$,则幂级数的收敛域为 $(-\infty,+\infty)$;
若 $0<R<+\infty$,则再考察幂级数在 $x=-R$ 和 $x=R$ 处的敛散性,进而确定幂级数的收敛域是 $(-R,R),[-R,R),(-R,R]$ 或 $[-R,R]$ 之中的某一区间.

\begin{example}
    求幂级数 $\displaystyle\sum_{n=1}^{\infty}\dfrac{n^2}{n^3+1}x^n$ 的收敛域.
\end{example}
\begin{solution}
    令 $a_n=\dfrac{n^2}{n^3+1}$,则收敛半径
    $$R=\displaystyle\lim_{n\to\infty}\qty|\dfrac{a_n}{a_{n+1}}|=\lim_{n\to\infty}\dfrac{n^2}{n^3+1}\cdot\dfrac{(n+1)^3+1}{(n+1)^2}=1$$
    在 $x=-1$ 处,级数为 $\displaystyle\sum_{n=1}^{\infty}(-1)^n\dfrac{n^2}{n^3+1}$,这是交错级数,并且 $\displaystyle\lim_{n\to\infty}\dfrac{n^2}{n^3+1}=0$,故由 Leibniz 判别法可知级数收敛,
    在 $x=1$ 处,级数为 $\displaystyle\sum_{n=1}^{\infty}\dfrac{n^2}{n^3+1}$,因为 $\dfrac{n^2}{n^3+1}\geqslant \dfrac{1}{n+1}$,
    而 $\displaystyle\sum_{n=1}^{\infty}\dfrac{1}{n+1}$ 发散,故由比较判别法知级数发散,因此,级数的收敛域为 $[-1,1).$
\end{solution}

求幂级数 $\displaystyle\sum_{n=0}^{\infty}a_n(x-x_0)^n$ 的收敛域的方法有两种: 一种是作变量代换: $y=x-x_0$,
得到级数 $\displaystyle\sum_{n=0}^{\infty}a_ny^n$,可按前述方法求得其收敛域,进而可求得原级数的收敛域,另一种是使用比值法.

\begin{example}
    求级数 $\displaystyle\sum_{n=1}^{\infty}n!\dfrac{(x-5)^n}{n^n}$ 的收敛域.
\end{example}
\begin{solution}
    \textbf{法一: }令 $y=x-5$,得级数 $\displaystyle\sum_{n=1}^{\infty}n!\qty(\dfrac{y}{n})^n$,其收敛半径
    $$R=\lim_{n\to\infty}\qty|\dfrac{a_n}{a_{n+1}}|=\lim_{n\to\infty}\dfrac{n!}{n^n}\cdot\dfrac{(n+1)^{n+1}}{(n+1)!}=\lim_{n\to\infty}\qty(1+\dfrac{1}{n})^n=\e$$
    当 $y=-\e$ 时,级数为 $\displaystyle\sum_{n=1}^{\infty}(-1)^n\dfrac{n!\e^n}{n^n}$,这是交错级数,可以证明: $\dfrac{n!\e}{n^n}$ 不趋于零 $(n\to\infty)$,级数发散;
    当 $y=\e$ 时,级数成为 $\displaystyle\sum_{n=1}^{\infty}\dfrac{n!\e^n}{n^n}$,显然也发散,故级数 $\displaystyle\sum_{n=1}^{\infty}n!\qty(\dfrac{y}{n})^n$ 的收敛域为 $(-\e,\e)$,
    从而可知原级数的收敛域为 $(5-\e ,5+\e)$.\\
    \textbf{法二: }因为 $$\rho=\lim_{n\to\infty}\qty|\dfrac{(n+1)!\dfrac{(x-5)^{n+1}}{(n+1)^{n+1}}}{n!\dfrac{(x-5)^n}{n^n}}|=\lim_{n\to\infty}\qty(\dfrac{1}{n+1})^n|x-5|=\dfrac{|x-5|}{\e}$$
    根据正项级数的比值判别法,当 $\rho<1$,即 $|x-5|<\e,~5-\e<x<5+\e$ 时,所给级数绝对收敛;
    当 $\rho\geqslant 1$,即 $x\geqslant 5+\e$ 或 $x\leqslant 5-\e$ 时,由于级数的一般项不趋于零,故级数发散,综上,级数的收敛域为 $(5-\e,5+\e).$
\end{solution}

% \begin{example}[2002 数三]
%     设幂级数 $\displaystyle\sum_{n=1}^{\infty}a_nx^n$ 与 $\displaystyle\sum_{n=1}^{\infty}b_nx^n$ 的收敛半径分别为 $\dfrac{\sqrt{5}}{3}\text{ 与 }\dfrac{1}{3}$,
%     则幂级数 $\displaystyle\sum_{n=1}^{\infty}\dfrac{a_n^2}{b_n^2}x^n$ 的收敛半径为 
%     \begin{tasks}(4)
%         \task $5$
%         \task $\dfrac{\sqrt{5}}{3}$
%         \task $\dfrac{1}{3}$
%         \task $\dfrac{1}{5}$
%     \end{tasks}
% \end{example}
% \begin{solution}
%     所求的收敛半径为 $R=\displaystyle\lim_{n\to\infty}\dfrac{\qty(\dfrac{a_n}{a_{n+1}})^2}{\qty(\dfrac{b_n}{b_{n+1}})^2}=\dfrac{\lim\limits_{n\to\infty}\qty(\dfrac{a_n}{a_{n+1}})^2}{\lim\limits_{n\to\infty}\qty(\dfrac{b_n}{b_{n+1}})^2}=\dfrac{5}{9}\cdot 9=5$,故选 A.
% \end{solution}

\begin{example}[2009 数三]
    求幂级数 $\displaystyle\sum_{n=1}^{\infty}\dfrac{\e^n-(-1)^n}{n^2}x^n$ 的收敛半径.
\end{example}
\begin{solution}
    令 $a_n=\dfrac{\e^n-(-1)^n}{n^2}$,则
    $$\rho=\lim_{n\to\infty}\qty|\dfrac{a_{n+1}}{a_n}|=\lim_{n\to\infty}\dfrac{n^2}{(n+1)^2}\cdot\dfrac{\e^{n+1}-(-1)^{n+1}}{\e^{n}-(-1)^n}=\lim_{n\to\infty}\dfrac{\e+\qty(\dfrac{-1}{\e})^n}{1-\qty(\dfrac{-1}{\e})^n}=\e$$
    于是收敛半径为 $R=\dfrac{1}{\rho}=\dfrac{1}{\e}.$
\end{solution}

\begin{example}[2020 数三]
    设幂级数 $\displaystyle\sum_{n=1}^{\infty}na_n(x-2)^n$ 的收敛区间为 $(-2,6)$,则 $\displaystyle\sum_{n=1}^{\infty}a_n(x+1)^{2n}$ 的收敛区间为
    \begin{tasks}(4)
        \task $(-2,6)$
        \task $(-3,1)$
        \task $(-5,3)$
        \task $(-17,15)$
    \end{tasks}
\end{example}
\begin{solution}
    因为幂级数 $\displaystyle\sum_{n=1}^{\infty}na_n(x-2)^n$ 收敛区间为 $(-2,6)$,所以幂级数 $\displaystyle\sum_{n=1}^{\infty}na_n(x-2)^n$ 的收敛半径为 4,令 $t=x-2$,则
    $$\sum_{n=1}^{\infty}na_n(x-2)^n=(x-2)\sum_{n=1}^{\infty}na_n(x-2)^{n-1}=t\sum_{n=1}^{\infty}na_nt^{n-1}=t\sum_{n=1}^{\infty}\qty(a_nt^n)'$$
    因而幂级数 $\displaystyle\sum_{n=1}^{\infty}a_nt^n$ 的收敛半径为 4,又因为 $$\sum_{n=1}^{\infty}a_n(x+1)^{2n}=\sum_{n=1}^{\infty}a_n\qty[(x+1)^2]^n$$
    则 $(x+1)^2<4$,解得 $-3<x<1$,故选 B.
\end{solution}

\subsubsection{求幂级数的和函数}

求幂级数在其收敛域内的和函数的基本方法是: 利用幂级数的四则运算、幂级数在其收敛域内可以逐项微分、逐项积分的性质以及幂级数的和函数在其收敛区间端点处的单侧连续性等.

\begin{example}
    求幂级数 $\displaystyle\sum_{n=1}^{\infty}\dfrac{1}{n}(3+x)^{2n}$ 的和函数.
\end{example}
\begin{solution}
    因为 $\displaystyle \sum_{n=0}^{\infty}(-1)^{n+1}\dfrac{x^{n+1}}{n+1}=\ln(1+x)$,所以有
    $\displaystyle\ln(1-x)=-\sum_{n=1}^{\infty}\dfrac{x^n}{n}$,
    于是 $$\displaystyle\sum_{n=1}^{\infty}\dfrac{(3+x)^{2n}}{n}=\sum_{n=1}^{\infty}\dfrac{\qty[(3+x)^{2}]^{n}}{n}=-\ln\qty[1-(x+3)^2]~ (-4<x<-2).$$
\end{solution}

\begin{example}
    求幂级数 $\displaystyle\sum_{n=1}^{\infty}\dfrac{x^{3n}}{n\cdot 8^n}$ 的和函数.
\end{example}
\begin{solution}
    设 $S(x)=\displaystyle\sum_{n=1}^{\infty}\dfrac{x^{3n}}{n\cdot 8^n}=\sum_{n=1}^{\infty}\dfrac{1}{n}\cdot\qty(\dfrac{x}{2})^{3n}$,那么
    $$S'(x)=\dfrac{3}{2}\sum_{n=1}^{\infty}\qty(\dfrac{x}{2})^{3n-1}\Rightarrow \dfrac{x}{2}S'(x)=\dfrac{3}{2}\sum_{n=0}^{\infty}\qty(\dfrac{x}{2})^{3n}-\dfrac{3}{2}=\dfrac{3}{2}\cdot\dfrac{8}{8-x^3}-\dfrac{3}{2}$$
    解得 $S'(x)=\dfrac{24}{8x-x^4}-\dfrac{3}{x}$,两边对 $x$ 积分得
    $$\int S(x)\dd x=24\int\dfrac{\dd x}{8x-x^4}-3\int\dfrac{\dd x}{x}=-\ln\qty(8-x^3)+C$$
    又 $S(0)=0$,解得 $C=\ln8$,于是幂级数的和函数 $S(x)=\ln\dfrac{8}{8-x^3}.$
\end{solution}

\begin{example}
    求幂级数 $\displaystyle \sum_{n=1}^{\infty}(-1)^{n-1}\dfrac{n^2-n+1}{n\cdot 2^{2n+1}}(x-1)^n$ 的和函数.
\end{example}
\begin{solution}
    易求得幂级数的收敛区间为 $(-3,5)$,并将原幂级数化为 $\displaystyle\dfrac{1}{2}\sum_{n=1}^{\infty}(-1)^{n-1}\dfrac{n^2-n+1}{n}\cdot\qty(\dfrac{x-1}{4})^n$,令 $y=\dfrac{x-1}{4}$,有
    $$S(y)=\dfrac{1}{2}\sum_{n=1}^{\infty}(-1)^{n-1}\dfrac{n^2-n+1}{n}\cdot y^n=\dfrac{1}{2}y^2\sum_{n=1}^{\infty}(-1)^{n-1}(n-1)y^{n-2}+\dfrac{1}{2}\sum_{n=1}^{\infty}\dfrac{(-1)^{n-1}}{n}y^n:=\dfrac{y^2}{2}S_1(y)+\dfrac{1}{2}S_2(y)$$
    于是 $$\int_{0}^{y}S_1(t)\dd t=\sum_{n=1}^{\infty}(-1)^{n-1}(n-1)\int_{0}^{y}t^{n-2}\dd t=\sum_{n=1}^{\infty}(-1)^{n-1}y^{n-1}=\dfrac{1}{1+y}\Rightarrow S_1(y)=-\dfrac{1}{(1+y)^2}~ (|y|<1)$$
    又 $\displaystyle S_2(y)=\displaystyle\sum_{n=1}^{\infty}(-1)^{n-1}\dfrac{y^n}{n}=\ln(1+y)~ (-1<x\leqslant 1)$,故
    $$S(y)=\dfrac{y^2}{2}S_1(y)+\dfrac{1}{2}S_2(y)=-\dfrac{1}{2}\qty(\dfrac{y}{1+y})^2+\dfrac{1}{2}\ln(1+y)~ (|y|<1)$$
    于是,所求幂级数的和函数为
    $$\sum_{n=1}^{\infty}(-1)^{n-1}\dfrac{n^2-n+1}{n\cdot 2^{2n+1}}(x-1)^n=S\qty(\dfrac{x-1}{4})=\dfrac{1}{2}\ln\dfrac{x+3}{4}-\dfrac{1}{2}\qty(\dfrac{x-1}{x+3})^2~ (-3<x<5).$$
\end{solution}

\begin{example}
    求 $\displaystyle \sum_{n=0}^{\infty}\dfrac{(-1)^nn^3}{(n+1)!}x^n$ 的收敛区间及和函数.
\end{example}
\begin{solution}
    易得级数的收敛区间为 $(-\infty,+\infty)$,又因为 $n^3=(n+1)\qty(n^2-n+1)-1$,于是 $\dfrac{n^3}{(n+1)!}=\dfrac{1}{(n-2)!}+\dfrac{1}{n!}-\dfrac{1}{(n+1)!}$,那么
    \begin{flalign*}
        \sum_{n=0}^{\infty}\dfrac{(-1)^nn^3}{(n+1)!}x^n & =\sum_{n=0}^{\infty}\dfrac{n^3}{(n+1)!}(-x)^n=-\dfrac{x}{2}+\sum_{n=2}^{\infty}\dfrac{(-x)^n}{(n-2)!}+\sum_{n=2}^{\infty}\dfrac{(-x)^n}{n!}-\sum_{n=2}^{\infty}\dfrac{(-x)^n}{(n+1)!} \\
                                                        & =-\dfrac{x}{2}+x^2\sum_{n=2}^{\infty}\dfrac{(-x)^{n-2}}{(n-2)!}+\sum_{n=2}^{\infty}\dfrac{(-x)^{n}}{n!}-\dfrac{1}{x}\sum_{n=2}^{\infty}\dfrac{(-x)^{n+1}}{(n+1)!}                     \\
                                                        & =-\dfrac{x}{2}+x^2\e^{-x}+\qty(\e^{-x}-1+x)+\dfrac{1}{x}\qty(\e^{-x}-1+x-\dfrac{x^2}{2})=\e\qty(x^2+1+\dfrac{1}{x})-\dfrac{1}{x}~~x\neq 0
    \end{flalign*}
    显然,当 $x=0$ 时,和函数为 0,因此 $\displaystyle \sum_{n=0}^{\infty}\dfrac{(-1)^nn^3}{(n+1)!}x^n=\begin{cases}
            \e^{-x}\qty(x^2-1+\dfrac{1}{x})-\dfrac{1}{x}, & x\neq 0 \\
            0,                                            & x=0.
        \end{cases}$
\end{solution}

\begin{example}
    求 $\displaystyle\sum_{k=1}^{\infty}\dfrac{k+2}{k!+(k+1)!+(k+2)!}$.
\end{example}
\begin{solution}
    因为 $\dfrac{k+2}{k!+(k+1)!+(k+2)!}=\dfrac{1}{k!(k+2)}$,于是考虑 $S(x)=\displaystyle\sum_{n=0}^{\infty}\dfrac{x^{n+2}}{n!(n+2)}$,
    则 $$S'(x)=\sum_{n=0}^{\infty}\dfrac{x^{n+1}}{n!}=x\sum_{n=0}^{\infty}\dfrac{x^n}{n!}=x\e^x$$
    所以 $$S(x)=S(0)+\int_{0}^{x}S'(t)\dd t=\int_{0}^{x}t\e^t\dd t=\eval{(t-1)\e^t}_{0}^{x}=(x-1)\e^x+1$$
    则 $\displaystyle\sum_{k=1}^{\infty}\dfrac{k+2}{k!+(k+1)!+(k+2)!}=S(1)-\dfrac{1}{2}=\dfrac{1}{2}.$
\end{solution}

对于有些系数含有阶乘符号的幂级数,求其和函数的一种行之有效的方法是: 首先设法建立关于和函数的微分方程,然后求解微分方程,即得所求和函数.

\begin{example}
    求幂级数 $\displaystyle\sum_{n=0}^{\infty}\dfrac{(2n)!!}{(2n+1)!!}x^{2n+1}$ 的和函数.
\end{example}
\begin{solution}
    设 $S(x)=\displaystyle\sum_{n=0}^{\infty}\dfrac{(2n)!!}{(2n+1)!!}x^{2n+1}$,易求得收敛区间为 $(-1,1)$,那么
    \begin{flalign*}
        S'(x)=1+\sum_{n=1}^{\infty}\dfrac{(2n)!!}{(2n-1)!!}x^{2n+2}=1+x\sum_{n=1}^{\infty}\dfrac{(2n)!!}{(2n-1)!!}x^{2n+1}
        =1+x\qty[x\sum_{n=0}^{\infty}\dfrac{(2n)!!}{(2n+1)!!}x^{2n+1}]'=1+xS(x)+x^2S'(x)
    \end{flalign*}
    即 $S'(x)-\dfrac{x}{1-x^2}S(x)=\dfrac{1}{1-x^2}$,该方程为一阶非齐次线性微分方程,由通解公式
    $$S(x)=\mathrm{e}^{\int\frac{x}{1-x^2}\dd x}\qty[\int\dfrac{1}{1-x^2}\cdot\mathrm{e}^{-\int\frac{x}{1-x^2}\dd x}\dd x+C]=\dfrac{1}{\sqrt{1-x^2}}\qty[\int\dfrac{\dd x}{\sqrt{1-x^2}}+C]=\dfrac{\arcsin x}{\sqrt{1-x^2}}+\dfrac{C}{\sqrt{1-x^2}}$$
    又 $S(0)=0$,解得 $C=0$,于是幂级数的和函数 $S(x)=\dfrac{\arcsin x}{\sqrt{1-x^2}}.$
\end{solution}

% \subsubsection{由递推关系式求幂级数的和函数}
% 
% \begin{example}
%     设 $a_1=1,~a_2=1,~a_{n+2}=2a_{n+1}+3a_{n}~ (n\geqslant 1)$,求 $\displaystyle\sum_{n=1}^{\infty}a_nx^n$ 的收敛域及和函数.
% \end{example}
% \begin{solution}
%     
% \end{solution}

\subsubsection{综合运用}

\begin{example}
    求幂级数 $\displaystyle\sum_{n=1}^\infty\frac{n+1}{n}x^n$ 的收敛域及和函数.
\end{example}
\begin{solution}
    记 $\displaystyle u_n=\frac{n+1}{n}$,则由 $\displaystyle\lim_{n\to\infty}\left|\frac{u_{n+1}}{u_n}\right|=\lim_{n\to\infty}\frac{n(n+2)}{(n+1)^2}=1$ 知,收敛半径 $R=1$.
    又当 $x=\pm 1$ 时,
    $$\displaystyle\sum_{n=1}^\infty\frac{n+1}{n}x^n=\sum_{n=1}^\infty\frac{n+1}{n}(\pm 1)^n=\sum_{n=1}^\infty(\pm 1)^n+\sum_{n=1}^\infty(\pm 1)^n\frac{1}{n}$$
    发散,故原级数的收敛域为 $(-1,1)$.
    设和函数为 $S(x)$,即 $$\displaystyle S(x)=\sum_{n=1}^\infty\frac{n+1}{n}x^n,~x\in(-1,1)$$
    两边在 $(0,x)$ 上积分,得
    $$\int_0^xS(t)\mathrm{d}t=\int_0^x\left(\sum_{n=1}^\infty\frac{n+1}{n}t^n\right)\mathrm{d}t=\sum_{n=1}^\infty\int_0^x\frac{n+1}{n}t^n\mathrm{d}t=\sum_{n=1}^\infty\frac{x^{n+1}}{n}$$
    即 $\displaystyle\frac{1}{x}\int_0^xS(t)\mathrm{d}t=\sum_{n=1}^\infty\frac{x^n}{n}$,然后两边分别对 $x$ 求导,得
    $$\left(\frac{1}{x}\int_0^xS(t)\mathrm{d}t\right)'=\left(\sum_{n=1}^\infty\frac{x^n}{n}\right)'=\sum_{n=1}^\infty x^{n-1}=\frac{1}{1-x}$$
    于是$$\frac{1}{x}\int_0^xS(t)\mathrm{d}t=\int_0^x\frac{\mathrm{d}x}{1-x}=-\ln(1-x)+C$$
    即 $\displaystyle\int_0^xS(t)\mathrm{d}t=-x\ln(1-x)+Cx$,两边再次求导,得
    $$S(x)=\frac{x}{1-x}-\ln(1-x)+C$$
    因为 $S(x)$ 在 $(-1,1)$ 内连续,且 $S(0)=0$,所以 $C=0$,所以原级数的和函数为
    $$S(x)=\frac{x}{1-x}-\ln(1-x),~x\in(-1,1).$$
\end{solution}

\begin{example}
    求幂级数 $\displaystyle\sum_{n=0}^\infty(n+1)x^n$ 的收敛域及和函数,并求 $\displaystyle\sum_{n=1}^\infty\frac{n}{2^n}$ 的和.
\end{example}
\begin{solution}
    记 $u_n=n+1$,则由 $\displaystyle\lim_{n\to\infty}\left|\frac{u_{n+1}}{u_n}\right|=\lim_{n\to\infty}\frac{n+2}{n+1}=1$ 知,收敛半径 $R=1$.
    又当 $x=\pm 1$ 时,$$\displaystyle\sum_{n=0}^\infty(n+1)(\pm 1)^n=\sum_{n=0}^\infty(\pm 1)^n+\sum_{n=0}^\infty n(\pm 1)^n$$
    发散,故原级数的收敛域为 $(-1,1)$. 设和函数为 $S(x)$,即 $\displaystyle S(x)=\sum_{n=0}^\infty(n+1)x^n,~x\in(-1,1)$.两边在 $(0,x)$ 上积分,得
    $$\int_0^xS(t)\mathrm{d}t=\int_0^x\left(\sum_{n=0}^\infty(n+1)t^n\right)\mathrm{d}t=\sum_{n=0}^\infty\int_0^x(n+1)t^n\mathrm{d}t=\sum_{n=0}^\infty x^{n+1}$$
    即 $\displaystyle\frac{1}{x}\int_0^xS(t)\mathrm{d}t=\sum_{n=0}^\infty x^n=\frac{1}{1-x}\Rightarrow \int_0^xS(t)\mathrm{d}t=\frac{x}{1-x}\overset{\text{求导}}{\Rightarrow}S(x)=\frac{1}{(1-x)^2},~x\in(-1,1)$.
    所以 $$\displaystyle\sum_{n=1}^\infty\frac{n}{2^n}=\frac{1}{2}\sum_{n=1}^\infty n \frac{1}{2^{n-1}}=x\sum_{n=0}^\infty(n+1)x^n\Biggl |_{x=\frac{1}{2}}=\frac{x}{(1-x)^2}\Biggl |_{x=\frac{1}{2}}=2$$.
\end{solution}

\begin{example}[2014 数学 (三)]
    求幂级数 $\displaystyle\sum_{n=0}^\infty(n+1)(n+3)x^n$ 的收敛域及和函数.
\end{example}
\begin{solution}
    记 $\displaystyle u_n=(n+1)(n+3)$,则由 $\displaystyle\lim_{n\to\infty}\left|\frac{u_{n+1}}{u_n}\right|=\lim_{n\to\infty}\frac{(n+2)(n+4)}{(n+1)(n+3)}=1$ 知,收敛半径 $R=1$.
    又当 $x=\pm 1$ 时,$\displaystyle\sum_{n=0}^\infty(n+1)(n+3)(\pm 1)^n$ 发散,故原级数的收敛域为 $(-1,1)$.\\
    设和函数为 $S(x)$,即 $\displaystyle S(x)=\sum_{n=0}^\infty(n+1)(n+3)x^n,~x\in(-1,1)$.两边在 $(0,x)$ 上积分,得
    \begin{flalign*}
        \int_0^xS(t)\mathrm{d}t & =\int_0^x\left(\sum_{n=0}^\infty(n+1)(n+3)t^n\right)\mathrm{d}t=\sum_{n=0}^\infty\int_0^x(n+1)(n+3)t^n\mathrm{d}t=\sum_{n=0}^\infty(n+3)x^{n+1} \\
                                & =\sum_{n=0}^\infty(n+2)x^{n+1}+\sum_{n=0}^\infty x^{n+1}=\left(\sum_{n=0}^\infty\int_0^x (n+2)t^{n+1}\mathrm{d}t\right)'+\frac{x}{1-x}          \\
                                & =\left(\frac{x^2}{1-x}\right)'+\frac{x}{1-x}=\frac{3x-2x^2}{(1-x)^2}
    \end{flalign*}
    于是 $\displaystyle S(x)=\left[\frac{3x-2x^2}{(1-x)^2}\right]'=\frac{x+3}{(1-x)^3},~x\in(-1,1)$.
\end{solution}
\begin{example}[首届数学竞赛预赛]
    已知 $u_n(x)$ 满足 $u_n'(x)=u_n(x)+x^{n-1}\mathrm{e}^x$ ($n$ 为正整数),且 $\displaystyle u_n(1)=\frac{\mathrm{e}}{n}$,求函数项级数 $\displaystyle\sum_{n=1}^\infty u_n(x)$ 之和.
\end{example}
\begin{solution}
    由 $u_n'(x)-u_n(x)=x^{n-1}\mathrm{e}^x$ 得,$\displaystyle u_n(x)=\mathrm{e}^{\int\mathrm{d}x}\left[\int x^{n-1}\mathrm{e}^x\cdot\mathrm{e}^{-\int\mathrm{d}x}\mathrm{d}x+C\right]=\frac{x^n}{n}\mathrm{e}^x+C\mathrm{e}^x$,
    又因为 $\displaystyle u_n(1)=\frac{\mathrm{e}}{n}$,所以 $C=0$,所以 $\displaystyle\sum_{n=1}^\infty u_n(x)=\sum_{n=1}^\infty\frac{x^n}{n}\mathrm{e}^x=\mathrm{e}^x\sum_{n=1}^\infty\frac{x^n}{n}=-\mathrm{e}^x\ln(1-x),~x\in(-1,1)$.
\end{solution}

\begin{example}
    设 $a_0=1,a_1=-2,a_2=\dfrac{7}{2}$,$$a_{n+1}=-\qty(1+\dfrac{1}{n+1})a_n~ (n\geqslant 2)$$
    求幂级数 $\displaystyle\sum_{n=0}^{\infty}a_nx^n$ 在 $(-1,1)$ 内的和函数 $S(x)$.
\end{example}
\begin{solution}
    收敛半径 $R=\displaystyle\lim_{n\to\infty}\dfrac{|a_n|}{|a_{n+1}|}=\lim_{n\to\infty}\dfrac{1}{1+\dfrac{1}{n+1}}=1$,所以 $|x|<1$ 时幂级数收敛,
    \begin{flalign*}
        S(x)=\sum_{n=0}^{\infty}a_nx^n=a_0+a_1x+a_2x^2+\sum_{n=3}^{\infty}a_nx^n=1+2x+\dfrac{7}{2}x^2+\sum_{n=0}^{\infty}a_nx^n
    \end{flalign*}
    当 $n\geqslant 2$ 时,由
    \begin{flalign*}
        a_{n+1} & =(-1)\dfrac{n+2}{n+1}a_n=(-1)^2\dfrac{n+2}{n+1}\cdot\dfrac{n+1}{n}a_{n-1}=\cdots=(-1)^{n-1}\dfrac{n+2}{n+1}\cdot\dfrac{n+1}{n}\cdots\dfrac{4}{3}a_{2} \\
                & =(-1)^{n-1}\dfrac{n+2}{3}\cdot\dfrac{7}{2}=(-1)^{n-1}\dfrac{7}{6}(n+2)
    \end{flalign*}
    故 $$S(x)=1-2x+\dfrac{7}{2}x^2+\dfrac{7}{6}\sum_{n=3}^{\infty}(-1)^n(n+1)x^n$$
    其中 \begin{flalign*}
        \sum_{n=3}^{\infty}(-1)^n(n+1)x^n=\sum_{n=3}^{\infty}(-1)^n\qty(x^{n+1})'=\qty[\sum_{n=3}^{\infty}(-1)^nx^{n+1}]'=-\qty[\dfrac{x^4}{1+x}]'=-\dfrac{4x^3+3x^4}{(1+x)^2}
    \end{flalign*}
    故 $S(x)=1-2x+\dfrac{7}{2}x^2-\dfrac{7}{6}\dfrac{4x^3+3x^4}{(1+x)^2}.$
\end{solution}

\begin{example}[2019 江苏年竞赛题]
    求幂级数 $\displaystyle \sum_{n=1}^\infty\frac{n}{8^n(2n-1)}x^{3n-1}$ 的收敛域与和函数.
\end{example}
\begin{solution}
    记 $\displaystyle a_n=\frac{n}{8^n(2n-1)}x^{3n-1}$,那么
    $$\lim_{n\to\infty}\left |\frac{a_{n+1}}{a_n}\right |=\lim_{n\to\infty}\left |\frac{(n+1)x^{3n+2}}{8^{n+1}(2n+1)}\cdot\frac{8^n(2n-1)}{nx^{3n-1}}\right |=\left |\frac{x}{2}\right |^3$$
    $|x|<2$ 时原级数收敛,$|x|>2$ 时原级数发散;当 $x=\pm 2$ 时,
    $$\lim_{n\to\infty}\sum_{n=1}^\infty\frac{n}{8^n(2n-1)}(\pm 2)^{3n-1}=\lim_{n\to\infty}\frac{1}{2}\sum_{n=1}^\infty\frac{(\pm 1)^nn}{2n-1}=\frac{1}{2}\not=0$$
    所以原级数发散,因此原级数的收敛域为 $(-2,2)$.
    \begin{flalign*}
        \sum_{n=1}^\infty\frac{n}{8^n(2n-1)}x^{3n-1}=\frac{1}{2}\sum_{n=1}^\infty\frac{2n-1+1}{8^n(2n-1)}x^{3n-1}
        =\frac{1}{2x}\sum_{n=1}^\infty\left(\frac{x}{2}\right)^{3n}+\frac{1}{2}\sum_{n=1}^\infty\frac{x^{3n-1}}{8^n(2n-1)}
    \end{flalign*}
    其中,$\displaystyle \lim_{n\to\infty}\frac{1}{2x}\sum_{k=1}^n\left(\frac{x}{2}\right)^{3k}=\frac{x^2}{2(8-x^3)}$,现求第二项的和.\\
    当 $x\in (0,2)$ 时,令 $\displaystyle\frac{x}{2}=t^{\frac{2}{3}}$,则
    \begin{flalign*}
        \frac{1}{2}\sum_{n=1}^\infty\frac{x^{3n-1}}{8^n(2n-1)} & =\frac{t}{2x}\sum_{n=1}^\infty\frac{t^{2n-1}}{2n-1}=\frac{t}{2x}\left(\sum_{n=1}^\infty\int_0^t\frac{t^{2n-1}}{2n-1}\mathrm{d}t\right)'                                                                                                                    \\
                                                               & =\frac{t}{2x}\int_0^t\sum_{n=1}^\infty\frac{\left(t^{2n-1}\right)'}{2n-1}\mathrm{d}t=\frac{t}{2x}\int_0^t\frac{\mathrm{d}t}{1-t^2}=\frac{t}{4x}\ln\frac{1+t}{1-t}=\frac{\sqrt{x}}{8\sqrt{2}}\ln\frac{2\sqrt{2}+x^{\frac{3}{2}}}{2\sqrt{2}-x^{\frac{3}{2}}}
    \end{flalign*}
    当 $x\in(-2,0)$ 时,令 $\displaystyle \frac{x}{2}=-t^{\frac{3}{2}}~~(t>0)$,则
    \begin{flalign*}
        \frac{1}{2}\sum_{n=1}^\infty\frac{x^{3n-1}}{8^n(2n-1)} & =\frac{t}{2x}\sum_{n=1}^\infty\frac{(-1)^n}{2n-1}t^{2n-1}=\frac{t}{2x}\int_0^t\sum_{n=1}^\infty\frac{(-1)^n}{2n-1}\left(t^{2n-1}\right)'\mathrm{d}t \\
                                                               & =-\frac{t}{2x}\int_0^t\frac{\mathrm{d}t}{1+t^2}=\frac{t}{2x}\arctan t=-\frac{\sqrt{-x}}{4\sqrt{2}}\arctan\frac{x\sqrt{-x}}{2\sqrt{2}}
    \end{flalign*}
    综上,所求和函数为
    $$S(x)=\left\{\begin{matrix}
            \displaystyle\frac{x^2}{2(8-x^3)}+\frac{\sqrt{x}}{8\sqrt{2}}\ln\frac{2\sqrt{2}+x^{\frac{3}{2}}}{2\sqrt{2}-x^{\frac{3}{2}}} , & 0\leqslant x<2 \\
            \displaystyle\frac{x^2}{2(8-x^3)}-\frac{\sqrt{-x}}{4\sqrt{2}}\arctan\frac{x\sqrt{-x}}{2\sqrt{2}}                          ,  & -2<x<0
        \end{matrix}\right.$$
\end{solution}

\begin{example}
    设 $\alpha =\lim\limits_{x\to0^+}\dfrac{x^2\tan\dfrac{x}{2}}{1-(1+x)^{\int_{0}^{x}\sin^2\sqrt{t}\dd t}}$,求常数项级数 $\displaystyle\sum_{n=1}^{\infty}n^2\sin ^{n-1}\alpha$ 的和.
\end{example}
\begin{solution}
    对极限式做恒等变换,有
    \begin{flalign*}
        \alpha & = -\lim_{x\to0^+}\dfrac{x^2\tan\dfrac{x}{2}}{\exp\displaystyle\int_{0}^{x}\sin^2\sqrt{t}\dd t\ln(1+x)-1}=-\lim_{x\to0^+}\dfrac{\dfrac{x^3}{2}}{\displaystyle\int_{0}^{x}\sin^2\sqrt{t}\dd t\ln(1+x)} \\
               & =-\lim_{x\to0^+}\dfrac{\dfrac{1}{2}x^2}{\displaystyle\int_{0}^{x}\sin^2\sqrt{t}\dd t}\xlongequal[]{L'}-\lim_{x\to0^+}\dfrac{x}{\sin^2\sqrt{x}}=-1
    \end{flalign*}
    为求 $\displaystyle\sum_{n=1}^{\infty}n^2\sin^{n-1}\alpha$ 的和,先求幂级数 $\displaystyle\sum_{n=1}^{\infty}n^2x^{n-1}$ 的和函数 $S(x)$,显然它的收敛域为 $(-1,1)$,当 $x\in(-1,1)$ 时,由幂级数在收敛区间的解析性质,有
    \begin{flalign*}
        S(x) & =\sum_{n=1}^{\infty}n^2x^{n-1}=\sum_{n=2}^{\infty}n(n-1)x^{n-1}+\sum_{n=1}^{\infty}nx^{n-1}                                  \\
             & =x\sum_{n=2}^{\infty}n(n-1)x^{n-2}+\sum_{n=1}^{\infty}nx^{n-1}=x\qty(\sum_{n=2}^{\infty}x^n)''+\qty(\sum_{n=1}^{\infty}x^n)' \\
             & =x\qty(x^2\sum_{n=0}^{\infty}x^n)''+\qty(x\sum_{n=0}^{\infty}x^n)'=x\qty(\dfrac{x^2}{1-x})''+\qty(\dfrac{x}{1-x})'           \\
             & =\dfrac{2x}{(1-x)^3}+\dfrac{1}{(1-x)^2}=\dfrac{1+x}{(1-x)^3}
    \end{flalign*}
    由于 $|\sin\alpha|=|\sin(-1)|<1$,故
    $$\sum_{n=1}^{\infty}n^2\sin^{n-1}\alpha=S(\sin\alpha)=\dfrac{1-\sin 1}{(1+\sin 1)^3}.$$
\end{solution}

\begin{example}
    设函数 $\displaystyle f(x)=\dfrac{1}{1-x-x^2}$,$a_n=\dfrac{1}{n!}f^{(n)}(0)~ (n\geqslant 0)$,证明: 级数 $\displaystyle\sum_{n=0}^{\infty}\dfrac{a_{n+1}}{a_na_{n+2}}$ 收敛并求它的和.
\end{example}
\begin{proof}[{\songti \textbf{证法一}}]
    令 $\displaystyle f(x)=\sum_{n=0}^{\infty}a_nx^n$,那么
    \begin{flalign*}
        1 & =(1-x-x^2)f(x)=(1-x-x^2)\sum_{n=0}^{\infty}a_nx^n=\sum_{n=0}^{\infty}a_nx^n-\sum_{n=0}^{\infty}a_nx^{n+1}-\sum_{n=0}^{\infty}a_nx^{n+2} \\
          & =a_0+(a_1-a_0)x+\sum_{n=0}^{\infty}(a_{n+2}-a_{n+1}-a_n)x^{n+2}
    \end{flalign*}
    比较系数得 $a_0=1,~a_1=a_0,~a_{n+2}=a_{n+1}+a_n$,从而有 $a_n\geqslant n$,故 $\displaystyle \lim_{n\to+\infty}a_n=+\infty$,故
    $$\dfrac{a_{n+1}}{a_na_{n+2}}=\dfrac{a_{n+2}-a_{n}}{a_na_{n+2}}=\dfrac{1}{a_n}-\dfrac{1}{a_{n+2}}$$
    所以 $\displaystyle \lim_{m\to+\infty}\sum_{n=0}^{m}\qty(\dfrac{1}{a_n}-\dfrac{1}{a_{n+2}})=\lim_{m\to+\infty}\qty(\dfrac{1}{a_0}+\dfrac{1}{a_1}-\dfrac{1}{a_{m+1}}-\dfrac{1}{a_{m+2}})=1+1=2.$
\end{proof}
\begin{proof}[{\songti \textbf{证法二}}]
    对 $(x^2+x-1)f(x)=-1$ 求 $n+2$ 阶导,
    $$\sum_{k=0}^{n+2}\mathrm{C}_{n+2}^k(x^2+x-1)^{(k)}f^{(n+2-k)}(x)=0$$
    故 $$(x^2+x-1)f^{(n+2)}(x)+(n+2)(2x+1)f^{(n+1)}(x)+(n+2)(n+1)f^{(n)}(x)=0$$
    将 $x=0$,代入上式,并化简得
    $$f^{(n+1)}(0)=\dfrac{1}{n+2}\qty[f^{(n+2)}(0)-(n+2)(n+1)f^{(n)}(0)]$$
    于是有
    \begin{flalign*}
        \sum_{n=0}^{\infty}\dfrac{a_{n+1}}{a_na_{n+2}} & =\sum_{n=0}^{\infty}\dfrac{n!(n+2)!}{(n+1)!}\cdot\dfrac{f^{(n+1)}(0)}{f^{(n)}(0)f^{(n+2)}(0)}                                                                                     \\
                                                       & =\sum_{n=0}^{\infty}\qty[\dfrac{n!(n+2)!}{(n+1)!}\cdot\dfrac{1}{n+2}\dfrac{f^{(n+2)}(0)-(n+2)(n+1)f^{(n)}(0)}{f^{(n)}(0)f^{(n+2)}(0)}]                                            \\
                                                       & =\sum_{n=0}^{\infty}n!\dfrac{f^{(n+2)}(0)-(n+2)(n+1)f^{(n)}(0)}{f^{(n)}(0)f^{(n+2)}(0)}=\sum_{n=0}^{\infty}\dfrac{n!}{f^{(n)}(0)}-\sum_{n=0}^{\infty}\dfrac{(n+2)!}{f^{(n+2)}(0)} \\
                                                       & =\sum_{n=0}^{\infty}\dfrac{n!}{f^{(n)}(0)}-\sum_{n=2}^{\infty}\dfrac{n!}{f^{(n)}(0)}=\dfrac{1}{f(0)}+\dfrac{1}{f'(0)}=1+1=2.
    \end{flalign*}
\end{proof}

\subsection{幂级数的应用}

\subsubsection{证明不等式}

\begin{example}
    设 $x>0,x\neq1$,证明 $0<\dfrac{x\ln x}{x^2-1}<\dfrac{1}{2}.$
\end{example}
\begin{proof}[{\songti \textbf{证}}]
    当 $0<x<1$ 或 $x>1$ 时,$x^{2}-1$ 与 $x\ln x$ 同号,所以 $\dfrac{x\ln x}{x^{2}-1}>0$,
    令 $x=\dfrac{1+t}{1-t}$,则 $\dfrac{x\ln x}{x^{2}-1}=\dfrac{1}{4}\left(\dfrac{1}{t}-t\right)\cdot\ln\dfrac{1+t}{1-t},~0<|t|<1$,
    而由 $\ln\dfrac{1+t}{1-t}$ 的幂级数展开式,有
    \begin{flalign*}
        \dfrac{1}{4}\left( \dfrac{1}{t}-t\right) \ln \dfrac{1+t}{1-t} & =\dfrac{1}{2}\left( \dfrac{1}{t}-t\right) \sum ^{\infty }_{n=0}\dfrac{t^{2n+1}}{2n+1}=\dfrac{1}{2}\left( \sum ^{\infty }_{n=0}\dfrac{t^{2n}}{2n+1}-\sum ^{\infty }_{n=0}\dfrac{t^{2n+2}}{2n+1}\right) \\
                                                                      & =\dfrac{1}{2}-\dfrac{1}{3}t^{2}+\sum ^{\infty }_{n=2}\left( \dfrac{1}{2n+1}-\dfrac{1}{2n-1}\right) t^{2n}=\dfrac{1}{2}-\dfrac{1}{3}t^{2}-\sum ^{\infty }_{n=2}\dfrac{t^{2n}}{4n^{2}-1}
    \end{flalign*}
    记 $\displaystyle u_{n}=\dfrac{t^{2n}}{4n^{2}-1}\text{,则 } \lim _{n\rightarrow \infty }\dfrac{u_{n}+1}{u_{n}}=t^{2}\Rightarrow \sum ^{\infty }_{n=2}\dfrac{t^{2n}}{4n^{2}-1}$ 收敛,且和大于 $0$,
    于是 $0<\dfrac{x\ln x}{x^2-1}<\dfrac{1}{2}.$
\end{proof}

\begin{example}
    设 $x>0,x\neq1$,证明 $\dfrac{\ln x}{x-1}\leqslant \dfrac{1}{\sqrt{x}}.$
\end{example}
\begin{proof}[{\songti \textbf{证}}]
    要证 $\dfrac{\ln x}{x-1}\leqslant \dfrac{1}{\sqrt{x}}$,即证 $I=\dfrac{\ln x}{x-1}-\dfrac{1}{\sqrt{x}}\leqslant 0$,为此
    \begin{flalign*}
        I=\xlongequal[0<|t|<1]{x=\left(\dfrac{1+t}{1-t}\right)^2}\dfrac{1}{t}\ln \dfrac{1+t}{1-t}-\dfrac{2}{1-t^{2}}=\dfrac{2}{t}\sum ^{\infty }_{n=0}\dfrac{t^{2n+1}}{2n+1}-2\sum ^{\infty }_{n=0}t^{2n}=\sum ^{\infty }_{n=0}\dfrac{-4nt^{2n}}{2n+1}\leqslant 0
    \end{flalign*}
    故得证 $\dfrac{\ln x}{x-1}\leqslant \dfrac{1}{\sqrt{x}}.$
\end{proof}

\begin{example}
    设 $0<x<1$,证明: $3\ln \dfrac{1+x}{1-x} <\dfrac{6x-4x^{3}}{1-x^{2}}.$
\end{example}
\begin{proof}[{\songti \textbf{证}}]
    因为 $3\ln\dfrac{1+x}{1-x}=6\displaystyle\sum_{n=0}^{\infty}\dfrac{x^{2n+1}}{2n+1},~|x|<1$,于是要证 $$6\displaystyle\sum_{n=0}^{\infty}\dfrac{x^{2n+1}}{2n+1}<\dfrac{6x-4x^3}{1-x^2}$$
    又 $6x+2x^3\leqslant 6\displaystyle\sum_{n=0}^{\infty}\dfrac{x^{2n+1}}{2n+1}$,只需证
    \begin{flalign*}
        6x+2x^3<4x+\dfrac{2x}{1-x^2}\Rightarrow 2x+2x^3-\dfrac{2x}{1-x^2}>0\Rightarrow 1-x^4>1
    \end{flalign*}
    而 $0<x<1$,故上式成立,即得待证不等式.
\end{proof}

\subsubsection{求和问题}

% begin{example}
%    求级数 $\displaystyle\sum_{n=1}^{\infty}(-1)^{n-1}\dfrac{n\cdot 2^{2n+1}-2n-1}{n(2n-1)\cdot 2^{n+1}}$ 的和.
% end{example}
% begin{solution}
%    
% end{solution}

% \begin{example}
%     (南开大学) 求和: $\displaystyle\sum_{n=1}^{\infty}\dfrac{(-1)^{n-1}(n-2)}{n(n+1)}.$
% \end{example}

\begin{example}\scriptsize\linespread{0.8}
    证明: $\displaystyle\lim_{m,n\to+\infty}\sum_{i=1}^m\sum_{j=1}^n\frac{(-1)^{i+j}}{i+j}$ 存在,并求之.
\end{example}
\begin{solution}\scriptsize\linespread{0.8}
    因为$$\int_{-1}^0x^{i+j-1}\mathrm{d}x=\frac{1}{i+j}x^{i+j}\Big |_{-1}^0=-\frac{(-1)^{i+j}}{i+j}$$
    所以
    \begin{flalign*}
        S_{m,n} & =\sum_{i=1}^m\sum_{j=1}^n\frac{(-1)^{i+j}}{i+j}=-\sum_{i=1}^m\sum_{j=1}^n\int_{-1}^0x^{i+j-1}\mathrm{d}x
        =-\sum_{i=1}^m\int_{-1}^0\left(\sum_{j=1}^nx^{i+j-1}\right)\mathrm{d}x                                                                                   \\
                & =-\sum_{i=1}^m\int_{-1}^0\frac{x^i\left(1-x^n\right)}{1-x}\mathrm{d}x=-\int_{-1}^0\frac{1}{1-x}\sum_{i=1}^m\left(x^i-x^{n+i}\right)\mathrm{d}x \\
                & =-\int_{-1}^0\frac{x-x^{m+1}}{(1-x)^2}\mathrm{d}x+\int_{-1}^0\frac{x^{n+1}-x^{n+m+1}}{(1-x)^2}\mathrm{d}x
    \end{flalign*}
    而 $x\in[-1,0],~(1-x)^2\ge1$,所以 $\displaystyle\int_{-1}^0\frac{x^i}{(1-x)^2}\mathrm{d}x\leqslant \int_{-1}^0x^i\mathrm{d}x$,又 $\displaystyle\lim_{i\to+\infty}\int_{-1}^0x^i\mathrm{d}x=0$,
    所以 $\displaystyle \lim_{i\to+\infty}\int_{-1}^0\frac{x^i}{(1-x)^2}\mathrm{d}x=0$,所以 $\displaystyle\lim_{m,n\to+\infty}S_{m,n}=-\int_{-1}^0\frac{x}{(1-x)^2}\mathrm{d}x=\ln2-\frac{1}{2}.$
\end{solution}

\begin{example}
    求级数 $\displaystyle\sum_{n=1}^{\infty}\dfrac{(-1)^{n-1}}{3n-1}$ 与 $\displaystyle\sum_{n=1}^{\infty}\dfrac{(-1)^{n-1}}{(3n-1)(2n-1)}$ 的和.
\end{example}
\begin{solution}
    令 $S(x)=\displaystyle\sum_{n=1}^{\infty}\dfrac{(-1)^{n-1}}{3n-1}x^{3n-1}$,那么 $S'(x)=\displaystyle\sum_{n=1}^{\infty}(-1)^{n-1}x^{3n-2}=\sum_{n=0}^{\infty}(-1)^nx^{3n+1}=\dfrac{x}{1+x^3},|x|<1$,
    于是 \begin{flalign*}
        S(1) & =\int_{0}^{1}S'(x)\dd x=\int_{0}^{1}\dfrac{x}{1+x^3}\dd x=\int_{0}^{1}\dfrac{x}{(x+1)\qty(x^2-x+1)}\dd x=\dfrac{1}{3}\int_{0}^{1}\qty(\dfrac{1}{2}\dfrac{2x-1}{x^2-x+1}+\dfrac{1}{2}\dfrac{3}{x^2-x+1}-\dfrac{1}{x+1})\dd x \\
             & =\eval{\dfrac{1}{6}\ln\qty(x^2-x+1)-\dfrac{1}{3}\ln(x+1)+\dfrac{1}{\sqrt{3}}\arctan\dfrac{2x-1}{\sqrt{3}}}_{0}^{1}=\dfrac{\pi}{3\sqrt{3}}-\dfrac{1}{3}\ln 2
    \end{flalign*}
    因为 $\dfrac{1}{(3n-1)(2n-1)}=\dfrac{2}{2n-1}-\dfrac{3}{2n-1}$,所以 $$\displaystyle\sum_{n=1}^{\infty}\dfrac{(-1)^{n-1}}{(3n-1)(2n-1)}=\sum_{n=1}^{\infty}(-1)^{n-1}\qty(\dfrac{2}{2n-1}-\dfrac{3}{3n-1})=2\sum_{n=1}^{\infty}\dfrac{(-1)^{n-1}}{2n-1}-3\sum_{n=1}^{\infty}\dfrac{(-1)^{n-1}}{3n-1}$$
    令 $\displaystyle Q(x)=\sum_{n=1}^{\infty}\dfrac{(-1)^{n-1}}{2n-1}$,那么 $$Q'(x)=\sum_{n=1}^{\infty}(-1)^{n-1}x^{2n-2}=\sum_{n=0}^{\infty}(-1)^nx^{2n}=\dfrac{1}{1+x^2}~~|x|<1$$
    因此 $\displaystyle 2\sum_{n=1}^{\infty}\dfrac{(-1)^{n-1}}{2n-1}=2Q(1)=2Q(0)+2\int_{0}^{1}\dfrac{\dd x}{1+x^2}=\eval{2\arctan x}_{0}^{1}=\dfrac{\pi}{2},~3\sum_{n=1}^{\infty}\dfrac{(-1)^{n-1}}{3n-1}=3S(1)=\dfrac{\pi}{\sqrt{3}}-\ln 2$,
    故 $$\sum_{n=1}^{\infty}\dfrac{(-1)^{n-1}}{(3n-1)(2n-1)}=\dfrac{\pi}{2}-\dfrac{\pi}{\sqrt{3}}+\ln 2.$$
\end{solution}

\begin{example}
    设 $a_n=\displaystyle\int_{0}^{+\infty}x^n\e^{-x}\dd x~~(n=0,1,2,\cdots)$,
    \begin{enumerate*}[label=(\arabic{*})]
        \item 求 $\displaystyle\lim_{n\to\infty}\dfrac{a_{n-1}}{a_n}$;
        \item 求级数 $\displaystyle\sum_{n=0}^{\infty}\dfrac{1+n^2}{a_n}$ 的和.
    \end{enumerate*}
\end{example}
\begin{solution}
    \begin{enumerate}[label=(\arabic{*})]
        \item 因为 $\displaystyle \Gamma(\alpha)=\int_{0}^{+\infty}x^{\alpha-1}\e^{-x}\dd x~~(\alpha>0)$,且 $\Gamma(\alpha+1)=\Gamma(\alpha)$,因此 $$\lim_{n\to\infty}\dfrac{a_{n-1}}{a_n}=\lim_{n\to\infty}\dfrac{\Gamma(n)}{\Gamma(n+1)}=\lim_{n\to\infty}\dfrac{\Gamma(n)}{n\Gamma(n)}=0.$$
        \item 又因为 $a_n=\Gamma(n+1)=n!$,因此级数转化为 $\displaystyle\sum_{n=0}^{\infty}\dfrac{1+n^2}{n!}$,为此构造幂级数 $S(x)=\displaystyle\sum_{n=0}^{\infty}\dfrac{1+n^2}{n!}x^n$,那么
              \begin{flalign*}
                  S(x) & =\sum_{n=0}^{\infty}\dfrac{x^n}{n!}+\sum_{n=0}^{\infty}\dfrac{n^2x^n}{n!}=\e^{x}+\sum_{n=1}^{\infty}\dfrac{n}{(n-1)!}x^n=\e^x+x\qty(\sum_{n=1}^{\infty}\dfrac{x^n}{(n-1)!})'=\e^x+x\qty(x\sum_{n=1}^{\infty}\dfrac{x^{n-1}}{(n-1)!})' \\
                       & =\e^{x}+x\qty(x\e^{x})'=\e^x+x\e^{x}+x^2\e^{x}\Rightarrow \sum_{n=0}^{\infty}\dfrac{1+n^2}{a_n}=S(1)=3\e.
              \end{flalign*}
    \end{enumerate}
\end{solution}

\begin{example}
    设 $a_n=\displaystyle\int_{0}^{n\pi}x|\sin x|\dd x,~n=1,2,\cdots$,
    \begin{enumerate*}[label=(\arabic{*})]
        \item 证明: $\displaystyle\lim_{n\to\infty}\sum_{k=1}^{n}\dfrac{a_k}{2^k}=\pi\sum_{n=1}^{\infty}\dfrac{n^2}{2^n}$;
        \item 求级数 $\displaystyle\sum_{n=1}^{\infty}\dfrac{n^2}{2^n}$ 的和.
    \end{enumerate*}
\end{example}
\begin{solution}
    \begin{enumerate}[label=(\arabic{*})]
        \item 化积分为求和形式
              \begin{flalign*}
                  a_n & =\sum_{k=0}^{n-1}\int_{k \pi}^{(k+1)\pi}x|\sin x|\dd x\xlongequal{t=x-k\pi}\sum_{k=0}^{n-1}\int_{0}^{\pi}(k\pi+t)|\sin(k\pi+t)|\dd t \\
                      & =\sum_{k=0}^{n-1}\int_{0}^{\pi}(k\pi+t)\sin t\dd t=\pi\sum_{k=0}^{n-1}(2k+1)=n^2\pi
              \end{flalign*}
              因此 $\displaystyle\lim_{n\to\infty}\sum_{k=1}^{n}\dfrac{a_k}{2^k}=\pi\sum_{n=1}^{\infty}\dfrac{n^2}{2^n}.$
        \item 令 $S(x)=\displaystyle\sum_{n=1}^{\infty}n^2x^n$,即求 $S\qty(\dfrac{1}{2})$ 的值,为此,
              \begin{flalign*}
                  S(x)   & =x\sum_{n=1}^{\infty}n^2x^{n-1}=xS_1(x)                                                                                                  \\
                  S_1(x) & =\qty(\sum_{n=1}^{\infty}\int_{0}^{x}n^2t^{n-1}\dd t)'=\qty(\sum_{n=1}^{\infty}nx^n)'=\qty(x\sum_{n=1}^{\infty}nx^{n-1})'=\qty(xS_2(x))' \\
                  S_2(x) & =\qty(\sum_{n=1}^{\infty}\int_{0}^{x}nt^{n-1}\dd t)'=\qty(\sum_{n=1}^{\infty}x^n)'=\qty(\dfrac{1}{1-x})'=\dfrac{1}{(1-x)^2}              \\
                         & \Rightarrow S_1(x)=\dfrac{1+x}{(1-x)^3}\Rightarrow S(x)=\dfrac{x^2+x}{(1-x)^3}\Rightarrow S\qty(\dfrac{1}{2})=6.
              \end{flalign*}
    \end{enumerate}
\end{solution}

\subsubsection{计算积分}

\begin{example}
    用幂级数展开方式求例 \ref{sin2xlnsinxlncosx}.
    \label{jishusin2xlnsinxlncosx}
\end{example}
\begin{solution}
    同样地,换元得到 $\displaystyle I=\dfrac{1}{4}\int_{0}^{1}\ln t\ln(1-t)\dd t$,并且 $\displaystyle -\ln(1-x)=\sum_{n=1}^{\infty}\dfrac{x^n}{n}~ |x|<1$,所以\footnote{
        $\displaystyle\sum_{n=1}^{\infty}\qty[\dfrac{1}{n}-\dfrac{1}{n+1}-\dfrac{1}{(n+1)^2}]=\sum_{n=1}^{\infty}\qty(\dfrac{1}{n}-\dfrac{1}{n+1})-\sum_{n=1}^{\infty}\dfrac{1}{(n+1)^2}
            =\lim_{n\to\infty}\dfrac{n}{n+1}-\qty[\sum_{n=1}^{\infty}\dfrac{1}{n^2}-1]=2-\dfrac{\pi^2}{6}.$
    }
    \begin{flalign*}
        I & =-\dfrac{1}{4}\int_{0}^{1}\ln t\sum_{n=1}^{\infty}\dfrac{t^n}{n}\dd t=-\dfrac{1}{4}\sum_{n=1}^{\infty}\int_{0}^{1}\dfrac{t^n\ln t}{n}\dd t=-\dfrac{1}{4}\sum_{n=1}^{\infty}\int_{0}^{1}\dfrac{\ln t\dd \qty(t^{n+1})}{n(n+1)} \\
          & =-\dfrac{1}{4}\sum_{n=1}^{\infty}\qty[\dfrac{t^{n+1}\ln t}{n(n+1)}\bigg |_0^1-\dfrac{1}{n(n+1)}\int_{0}^{1}t^n\dd t]=\dfrac{1}{4}\sum_{n=1}^{\infty}\dfrac{1}{n(n+1)^2}                                                       \\
          & =\dfrac{1}{4}\sum_{n=1}^{\infty}\qty[\dfrac{1}{n}-\dfrac{1}{n+1}-\dfrac{1}{(n+1)^2}]=\dfrac{1}{4}\left(2-\dfrac{\pi^2}{6}\right)=\dfrac{1}{2}-\dfrac{\pi^2}{24}.
    \end{flalign*}
\end{solution}
