\section{Fourier 级数}

Fourier 级数是函数项级数的一种特殊形式. 它是研究和表示周期函数的有力工具.

\subsection{Fourier 系数}

\begin{definition}[Fourier 系数]
    形如 $\displaystyle\dfrac{a_{0}}{2}+\sum_{n=1}^{\infty}\left(a_{n} \cos n x+b_{n} \sin n x\right) $ 的级数
    称为 \textit{Fourier 级数}或\textit{三角级数}.
    设 $ f(x) $ 是周期为 $ 2 \pi $ 的周期函数, 且能展开为 Fourier 级数, 即
    $$f(x) \sim \frac{a_{0}}{2}+\sum_{n=1}^{\infty}\left(a_{n} \cos n x+b_{n} \sin n x\right) .$$
    若积分
    \begin{flalign*}
        a_{n} & =\frac{1}{\pi} \int_{-\pi}^{\pi} f(x) \cos n x \dd x x, n=0,1,2 \cdots \\
        b_{n} & =\frac{1}{\pi} \int_{-\pi}^{\pi} f(x) \sin n x \dd x x, n=1,2 \cdots
    \end{flalign*}
    都存在, 则称 $\displaystyle \frac{a_{0}}{2}+\sum_{n=1}^{\infty}\left(a_{n} \cos n x+b_{n} \sin n x\right) $ \textit{为} $f(x)$ \textit{的 Fourier 级数}, 其中 $a_0,~a_n,~b_n,~n=1,2,\cdots$
    称为 $f(x)$ 的\textit{ Fourier 系数}.
\end{definition}

\begin{theorem}[Dirichlet 收敛定理]
    设 $ f(x) $ 是以 $ 2 l $ 为周期的周期函数, 如果它满足
    \begin{enumerate}[label=(\arabic{*})]
        \item 一个周期内除有限个第一类间断点外都连续;
        \item 一个周期内至多只有有限个极值点.
    \end{enumerate}
    则 $ f(x) $ 的 Fourier 级数收敛, 并且
    当 $ x $ 为 $ f(x) $ 的连续点时, 级数收敛于 $ f(x) $;
    当 $ x $ 为 $ f(x) $ 的间断点时, 级数收敛于 $\displaystyle \frac{f\left(x^{-}\right)+f\left(x^{+}\right)}{2} .$
    \index{Dirichlet 收敛定理}
\end{theorem}

\begin{example}
    设 $f(x)=\begin{cases}
        -1, & -\pi <x\leqslant 0\\ 1+x^2,&0<x<\pi
    \end{cases}$ 则以 $2\pi$ 为周期的 Fourier 级数在 $x=\pi$ 处收敛于 
    \begin{tasks}(4)
        \task $1+\pi^2$.
        \task $-1$.
        \task $\dfrac{\pi}{2}$.
        \task $\dfrac{\pi^2}{2}$.
    \end{tasks}
\end{example}
\begin{solution}
    根据收敛定理可知, $f(x)$ 的 Fourier 级数在 $x=\pi$ 处收敛于 $$
    \dfrac{f\qty(\pi^-)+f\qty(\pi^+)}{2}=\dfrac{1}{2}\qty[\lim_{x \to \pi^-}\qty(1+x^2)+\lim_{x \to (-\pi)^+}f(x)]=\dfrac{1}{2}\qty[1+\pi^2-1]=\dfrac{\pi^2}{2}
    $$
    因此选 D.
\end{solution}

\subsection{周期为 \texorpdfstring{$2l$}. 的 Fourier 展开}

\begin{definition}[正弦级数与余弦级数]
    \index{正弦级数与余弦级数}
    设 $ f(x) $ 是周期为 $ 2 \pi $ 的周期函数, 且能展开成 Fourier 系数, 即
    $$f(x) \sim \frac{a_{0}}{2}+\sum_{n=1}^{\infty}\left(a_{n} \cos n x+b_{n} \sin n x\right)$$
    \begin{enumerate}[label=(\arabic{*})]
        \item 当 $ f(x) $ 为奇函数时
              $$a_{0}=\frac{1}{\pi} \int_{-\pi}^{\pi} f(x) \dd  x=0,~ a_{n}=\frac{1}{\pi} \int_{-\pi}^{\pi} f(x) \cos n x \dd  x=0$$
              此时 $ f(x) $ 的 Fourier 级数表达式为 $\displaystyle \sum_{n=1}^{\infty} b_{n} \sin n x $ 仅含正弦项, 称为\textit{正弦级数};
        \item 当 $ f(x) $ 为偶函数时
              $$b_{n}=\frac{1}{\pi} \int_{-\pi}^{\pi} f(x) \sin n x \dd  x=0$$
              此时 $ f(x) $ 的 Fourier 级数表达式为: $\displaystyle \frac{a_{0}}{2}+\sum_{n=1}^{\infty} a_{n} \cos n x$, 此式仅含余弦项, 
              称为\textit{余弦级数}.
    \end{enumerate}
\end{definition}

\begin{theorem}[Fourier 展开]
    设 $ f(x) $ 的周期为 $ 2 l$, 且满足收敛定理的条件, 则其在 $ [-l, l] $ 上的 Fourier 级数展开式为
    $$f(x) \sim \frac{a_{0}}{2}+\sum_{n=1}^{\infty}\left(a_{n} \cos \frac{n \pi}{l} x+b_{n} \sin \frac{n \pi}{l} x\right) $$
    其中 Fourier 系数 $ a_{0},~ a_{n},~ b_{n}~~(n=1,2,3, \cdots) $ 取值为:
    \begin{flalign*}
        a_{0} & =\frac{1}{l} \int_{-l}^{l} f(x) \dd  x,~a_{n}=\frac{1}{l} \int_{-l}^{l} f(x) \cos \frac{n \pi x}{l} \dd  x \\
        b_{n} & =\frac{1}{l} \int_{-l}^{l} f(x) \sin \frac{n \pi x}{l} \dd  x
    \end{flalign*}
    特别地, 进行变量代换 $ z=\dfrac{\pi}{l} x$, 于是区间 $ -l \leqslant x<l$ 变换为 $ -\pi \leqslant z<\pi $, 存在周期为 $ 2 \pi $ 的函数 $F(z) $ 使得 $ f(x)=f\left(\dfrac{l x}{\pi}\right)=F(z) $, 
    对 $ F(z) $ 进行 Fourier 展开, 最后将 $ z=\dfrac{\pi}{l} x $ 回代即可.
    \index{Fourier 展开}
\end{theorem}

\begin{definition}[周期延拓与奇偶延拓]
    若 $ f(x) $ 仅在 $ [-\pi, \pi) $ 上有定义, 此时需要将其展开为 Fourier 级数, 则需要在 $ [-\pi, \pi) $ 外补充 $ f(x) $ 的定义, 使其成为周期为 $ 2 \pi $ 的周期函数. 这种方法称为\textit{周期延拓}.
    若 $ f(x) $ 仅给出 $ [0, \pi] $ 上的定义, 若要将其展开为正弦级数, 则需补充 $ [-\pi, 0] $ 的定义, 使其成为奇函数, 这种方法称为\textit{函数的奇延拓};
    若要将其展开为余弦级数, 则需补充 $ [-\pi, 0] $ 的定义, 使其成为偶函数, 这种方法称为\textit{函数的偶延拓}. 进行奇偶延拓后, 补充周期延拓即可将函数 $ f(x) $ 进行 Fourier 展开.
    给出函数 $ f(x) $ 在 $ [0, l] $ 的定义同理.
\end{definition}

\begin{example}
    在所指定的区间内把下列函数展开为 Fourier 级数.
    \setcounter{magicrownumbers}{0}
    \begin{table}[H]
        \centering
        \begin{tabular}{l | l}
            (\rownumber{}) 在区间 $(0,2l)$ 内展开 $f(x)=\begin{cases}A &,0<x<l \\0 & ,l<x<2l\end{cases}$. & (\rownumber{}) 在区间 $(-\pi,\pi)$ 内展开 $f(x)=x$    \\
            (\rownumber{}) 在区间 $(0,2\pi)$ 内展开 $f(x)=\dfrac{\pi-x}{2}.$                              & (\rownumber{}) 在区间 $(-\pi,\pi)$ 内展开 $f(x)=|x|.$ \\
            %(\rownumber{}) 在区间 $(-\pi,\pi)$ 内展开 $f(x)=\pi^2-x^2.$                                   & (\rownumber{}) 在区间 $(a,a+2l)$ 内展开 $f(x)=x.$
        \end{tabular}
    \end{table}
\end{example}
\begin{solution}
    \begin{enumerate}[label=(\arabic{*})]
        \item 由于
              \begin{flalign*}
                  a_0 & =\dfrac{1}{l}\int_{0}^{2l}f(x)\dd x=\dfrac{1}{l}\int_{0}^{l}A\dd x=A;                                                                  \\
                  a_n & =\dfrac{1}{l}\int_{0}^{2l}f(x)\cos\dfrac{n\pi x}{l}\dd x=\dfrac{1}{l}\int_{0}^{l}A\cos\dfrac{n\pi x}{l}\dd x=0;                        \\
                  b_n & =\dfrac{1}{l}\int_{0}^{2l}f(x)\sin\dfrac{n\pi x}{l}\dd x=\dfrac{1}{l}\int_{0}^{l}A\sin\dfrac{n\pi x}{l}\dd x=\dfrac{A}{n\pi}[1-(-1)^n]
              \end{flalign*}
              故按展开定理, $f(x)$ 可展开为 $$\dfrac{A}{2}+\dfrac{2A}{\pi}\sum_{k=0}^{\infty}\dfrac{1}{2k+1}\sin(2k+1)\dfrac{\pi x}{l}=\begin{cases}
                      A            & ,0<x<l   \\
                      \dfrac{A}{2} & , x=l    \\
                      0            & , l<x<2l
                  \end{cases}$$
        \item 因为 $f(x)=x$ 是奇函数, 从而 $a_0=a_n=0$, 且
              $$b_n=\dfrac{1}{\pi}\int_{-\pi}^{\pi}x\sin nx\dd x=\dfrac{2}{\pi}\int_{0}^{\pi}x\sin nx\dd x=(-1)^{n-1}\dfrac{2}{n}$$
              故按展开定理, $f(x)$ 可展开为 $\displaystyle 2\sum_{n=1}^{\infty}(-1)^{n-1}\dfrac{\sin nx}{n}=x.$
        \item 由于
              \begin{flalign*}
                  a_0 & =\dfrac{1}{\pi}\int_{0}^{2\pi}\dfrac{\pi-x}{2}\dd x=\dfrac{1}{2\pi}\int_{0}^{2\pi}(\pi-x)\dd x=0;                                                                 \\
                  a_n & =\dfrac{1}{\pi}\int_{0}^{2\pi}\dfrac{\pi-x}{2}\cos nx\dd x=\dfrac{\pi-x}{2n\pi}\sin nx\biggl |_0^{2\pi}+\dfrac{1}{2n\pi}\int_{0}^{2\pi}\sin nx\dd x=0;            \\
                  b_n & =\dfrac{1}{\pi}\int_{0}^{2\pi}\dfrac{\pi-x}{2}\sin nx\dd x=-\dfrac{\pi-x}{2n\pi}\cos nx\biggl |_0^{2\pi}-\dfrac{1}{2n\pi}\int_{0}^{2\pi}\cos nx\dd x=\dfrac{1}{n}
              \end{flalign*}
              故按展开定理, $f(x)$ 可展开为 $\displaystyle \sum_{n=1}^{\infty}\dfrac{\sin nx}{n}=\dfrac{\pi-x}{2}.$
        \item 因为 $f(x)=|x|$ 为偶函数, 从而 $b_n=0$, 且
              \begin{flalign*}
                  a_0 & =\dfrac{2}{\pi}\int_{0}^{\pi}x\dd x=\pi;                                                                                                                   \\
                  a_n & =\dfrac{2}{\pi}\int_{0}^{\pi}x\cos nx\dd x=\dfrac{2}{n\pi}x\sin nx\biggl |_{0}^{\pi}-\dfrac{2}{n\pi}\int_{0}^{\pi}\sin nx\dd x=\dfrac{2}{n^2\pi}[(-1)^n-1]
              \end{flalign*}
              故按展开定理, $f(x)$ 可展开为 $\displaystyle \dfrac{\pi}{2}-\dfrac{4}{\pi}\sum_{n=0}^{\infty}\dfrac{\cos (2n+1)x}{(2n+1)^2}=|x|.$
    \end{enumerate}
\end{solution}

\begin{example}
    将函数 $f(x)=1-x^2~~(0\leqslant x\leqslant \pi)$ 展开为余弦级数, 并求级数 $\displaystyle\sum_{n=1}^{\infty}\dfrac{(-1)^{n-1}}{n^2}$ 的和.
\end{example}
\begin{solution}
    由题意 $l=\pi,~b_n=0~~(n=1,2,\cdots)$, 那么 $\displaystyle a_0=\dfrac{2}{\pi}\int_{0}^{\pi}\qty(1-x^2)\dd x=2-\dfrac{2\pi^2}{3}$, 
    \begin{flalign*}
        a_n & =\dfrac{1}{l}\displaystyle\int_{-l}^{l}f(x)\cos nx\dd x=\dfrac{2}{\pi}\int_{0}^{\pi}f(x)\cos nx\dd x=\dfrac{2}{\pi}\int_{0}^{\pi}\qty(1-x^2)\cos nx\dd x \\
            & =\eval{\dfrac{2}{\pi}\qty(\dfrac{1-x^2}{n}\sin nx-\dfrac{2x}{n^2}\cos nx+\dfrac{2}{n^3}\sin nx)}_{0}^{\pi}=-\dfrac{4}{n^2}(-1)^n
    \end{flalign*}
    于是 $\displaystyle f(x)=1-\dfrac{\pi^2}{3}+4\sum_{n=1}^{\infty}\dfrac{(-1)^{n+1}}{n^2}$, 将 $x=0$ 代入可得 $f(0)=1-\dfrac{\pi^2}{3}+4\displaystyle\sum_{n=1}^{\infty}\dfrac{(-1)^{n+1}}{n^2}$ 又由 $f(x)=1-x^2$ 知 $f(0)=1$, 则 $\displaystyle\sum_{n=1}^{\infty}\dfrac{(-1)^{n-1}}{n^2}=\dfrac{\pi^2}{12}.$
\end{solution}

\begin{example}
    求级数 $\displaystyle \sum_{n=1}^{\infty}\frac{1}{n^2}.$ 的和.
\end{example}
\begin{solution}
    \textbf{法一: }由 $(1+x)^{\alpha}$ 的二项展开式, 取 $\alpha=-\dfrac{1}{2},~x=-t^2$, 则有 
    $$\frac{1}{\sqrt{1-t^{2}}}=1+\sum_{n=1}^{\infty} \frac{(2 n-1)!!}{(2 n)!!} t^{2 n}, \quad-1<t<1 $$
    $\forall x \in(-1,1) $, 上式两边分别在 $ (0, x) $ 或 $ (x, 0) $ 上积分, 有
    $$\arcsin x=x+\sum_{n=1}^{\infty} \frac{(2 n-1)!!}{(2 n)!!} \cdot \frac{1}{2 n+1} x^{2 n+1}$$
    又因为 $\displaystyle  \forall k \geqslant 1, \frac{2 k-1}{2 k}<\frac{2 k}{2 k+1} $, 所以有 $\displaystyle  \frac{(2 n-1)!!}{(2 n)!!}<\frac{1}{\sqrt{2 n+1}} $, 于是 $$\displaystyle  \frac{(2 n-1)!!}{(2 n)!!} \cdot \frac{1}{2 n+1}<\frac{1}{(2 n+1)^{\frac{3}{2}}} $$ 从而 $\displaystyle  \sum_{n=1}^{\infty} \frac{(2 n-1)!!}{(2 n)!!} \cdot \frac{1}{2 n+1} $ 收敛, 故当 $ x= \pm 1 $ 时, 上述关于 $ \arcsin x$ 的等式也成立, 即有 $$ \forall x \in[-1,1] ,  \arcsin x=x+\sum_{n=1}^{\infty} \frac{(2 n-1)!!}{(2 n)!!} \cdot \frac{1}{2 n+1} x^{2 n+1} $$
    令 $ x=\sin u $, 则上式化为
    $$u=\sin u+\sum_{n=1}^{\infty} \frac{(2 n-1)!!}{(2 n+1) \cdot(2 n)!!} \sin ^{2 n+1} u, \quad-\frac{\pi}{2}<u \leqslant \frac{\pi}{2}$$
    再将上式两端从 $0$ 到 $\displaystyle  \frac{\pi}{2} $ 积分, 得
    \begin{flalign*}
    \frac{\pi^{2}}{8} & =1+\sum_{n=1}^{\infty} \frac{(2 n-1)!!}{(2 n+1) \cdot(2 n)!!} \cdot \int_{0}^{\frac{\pi}{2}} \sin ^{2 n+1} u \mathrm{~d} u=1+\sum_{n=1}^{\infty} \frac{(2 n-1)!!}{(2 n+1) \cdot(2 n)!!} \cdot \frac{(2 n)!!}{(2 n+1)!!} \\
    & =1+\sum_{n=1}^{\infty} \frac{1}{(2 n+1)^{2}}=\sum_{n=1}^{\infty} \frac{1}{(2 n-1)^{2}},
    \end{flalign*}
    于是 $\displaystyle  \sum_{n=1}^{\infty} \frac{1}{n^{2}}=\sum_{n=1}^{\infty} \frac{1}{(2 n-1)^{2}}+\sum_{n=1}^{\infty} \frac{1}{(2 n)^{2}}=\frac{\pi^{2}}{8}+\frac{1}{4} \sum_{n=1}^{\infty} \frac{1}{n^{2}} $, 故得 $\displaystyle  \sum_{n=1}^{\infty} \frac{1}{n^{2}}=\frac{\pi^{2}}{6} $.\\ 
    \textbf{法二: }将函数 $f(x)=x^2$ 在 $(-\pi,\pi]$ 上展开成 Fourier 级数, 因为 $f(x)$ 在 $(-\pi,\pi)$ 上为偶函数, 所以其 Fourier 系数为
    $$b_{0}=0,~a_0=\frac{2}{\pi}\int_{0}^{\pi}x^2\dd x=\frac{2\pi^2}{3},~a_n=\frac{2}{\pi}\int_{0}^{\pi} x^2\cos nx \dd x=\dfrac{4\cdot}{n^2},~(n=1,2,\cdots)$$
    于是由收敛定理有 $$\displaystyle x^2=\frac{\pi^2}{3}+\sum_{n=1}^{\infty}\dfrac{4(-1)^n}{n^2}\cdot\cos nx~(-\pi< x\leqslant \pi)$$ 当 $x=\pi$ 时, 有 $\displaystyle \pi^{2}=\frac{\pi^2}{3}+4\sum_{n=1}^{\infty}\frac{1}{n^2}$, 即得 $\displaystyle \sum_{n=1}^{\infty}\dfrac{1}{n^2}=\dfrac{\pi^2}{6}.$
\end{solution}

\subsection{Fourier 级数综合性问题}

\subsubsection{已知 Fourier 展开求 Fourier 系数}

\begin{example}
    设 $f(x)$ 是在周期为 2 的周期函数且 $f(x)=\begin{cases}
            x, & 0\leqslant x\leqslant 1 \\
            0, & 1<x<2
        \end{cases}$ $f(x)$ 的 Fourier 级数为 $\displaystyle \dfrac{a_0}{2}+\sum_{n=1}^{\infty}(a_n\cos n\pi x+b_n\sin n\pi x)$, 则 $n\geqslant 1$ 时, 求 $a_n.$
\end{example}
\begin{solution}
    对于周期为 $2l$ 的周期函数的 Fourier 级数为 $\displaystyle \dfrac{a_0}{2}+\sum_{n=1}^{\infty}\qty(a_n\cos \dfrac{n\pi x}{l})+b_n\sin\dfrac{n\pi x}{l}$ 系数 $a_n=\displaystyle\dfrac{1}{l}\int_{-l}^{l}f(x)\cos\dfrac{n\pi x}{l}\dd x,~n=1,2,\cdots$, 
    由题意知, $f(x)$ 的周期为 2, 故 $l=1$, 所以 $a_n=\displaystyle\int_{-1}^{1}f(x)\cos n\pi x\dd x$, 即
    \begin{flalign*}
        a_n & =\int_{0}^{2}f(x)\cos n\pi x\dd x=\int_{0}^{1}f(x)\cos n\pi x\dd x+\int_{1}^{2}f(x)\cos n\pi x\dd x=\int_{0}^{1}x\cos n\pi x\dd x \\
            & =\eval{\dfrac{x}{n\pi}\sin n\pi x+\dfrac{1}{n^2\pi^2}\cos n\pi x}_{0}^{1}=\dfrac{(-1)^n-1}{n^2\pi^2}.
    \end{flalign*}
\end{solution}

\subsubsection{抽象函数求 Fourier 系数}

\begin{example}
    设函数 $f(x)$ 连续且满足 $f(x+\pi)+f(x)=0$, 则 $f$ 以 $2\pi$ 为周期的 Fourier 系数 ($n=1, 2, \cdots $)
    \begin{tasks}(2)
        \task $a_{2n}=0, b_{2n}=0$.
        \task $a_{2n}=0, b_{2n-1}=0$.
        \task $a_{2n-1}=0, b_{2n-1}=0$.
        \task $a_{2n-1}=0, b_{2n}=0$.
    \end{tasks}
\end{example}
\begin{solution}
    计算 $a_n$ 与 $b_n$,
    \begin{flalign*}
        a_n& =\dfrac{1}{\pi}\int_{-\pi}^{\pi}f(x)\cos nx \dd x=\dfrac{1}{\pi}\qty[\int_{-\pi}^{0}f(x)\cos nx \dd x+\int_{0}^{\pi}f(x) \cos nx \dd x]\\ 
        & =\dfrac{1}{\pi}\qty[\int_{-\pi}^{0}f(x)\cos nx \dd x+\int_{0}^{\pi}f(u+x) \cos n(u+\pi) \dd u]=\dfrac{1}{\pi}\int_{-\pi}^{0}\qty[f(x)-(-1)^{n}f(x)]\cos nx \dd x
    \end{flalign*}
    同理可得 $\displaystyle b_n=\dfrac{1}{\pi}\int_{-\pi}^{0} \qty[f(x)-(-1)^{n}f(x)]\sin nx \dd x$, 代入选项 ABCD, 可判断 A 正确.
\end{solution}

\subsubsection{平均收敛与均方逼近}

\begin{definition}[平均收敛]
    若函数列 $f_n(x)$ 满足 $\displaystyle\lim_{n\to\infty}\qty(\dfrac{1}{n}\int_{-\pi}^{\pi}|f_n(x)-S(x)|^2\dd x)^{\frac{1}{2}}=0$, 
    则称 $f_n(x)$ 在 $[-\pi,\pi]$ 平均收敛于 $S(x)$.
\end{definition}

\begin{lemma}
    若 $f(x),g(x)$ 在 $[-\pi,\pi]$ 平方可积, 则
    \begin{enumerate}[label=(\arabic{*})]
        \item $f(x)g(x)$ 在 $[-\pi,\pi]$ 可积, 且
        $$\qty|\int_{-\pi}^{\pi}f(x)g(x)\dd x|\leqslant \int_{-\pi}^{\pi}|f(x)g(x)|\dd x\leqslant \qty(\int_{-\pi}^{\pi}|f(x)|^2\dd x)^{\frac{1}{2}}\qty(\int_{-\pi}^{\pi}|g(x)|^2\dd x)^{\frac{1}{2}}$$
        \item $f(x)+g(x)$ 在 $[-\pi,\pi]$ 可积, 且 
        $$\qty(\int_{-\pi}^{\pi}|f(x)+g(x)|^2\dd x)^{\frac{1}{2}}\leqslant \qty(\int_{-\pi}^{\pi}|f(x)|^2\dd x)^{\frac{1}{2}}+\qty(\int_{-\pi}^{\pi}|g(x)|^2\dd x)^{\frac{1}{2}}$$
    \end{enumerate}
\end{lemma}

\begin{theorem}[均方逼近]
    \index{均方逼近}若 $f(x)$ 以 $2\pi$ 为周期, 在 $[-\pi,\pi]$ 上平方可积, 则在所有 $n$ 阶多项式 $\displaystyle T_n(x)=\dfrac{a_0}{2}+\sum_{k=1}^{n}(\alpha_k\cos kx+\beta_k\sin kx)$ 中, 
    当 $T_n(x)$ 取 $f(x)$ 的 Fourier 级数的 $n$ 阶部分和 $\displaystyle S_n(x)=\dfrac{a_0}{2}+\sum_{k=1}^{n}(a_k\cos kx+b_k\sin kx)$ 时, 
    $f(x)$ 与 $T_n(x)$ 的均方误差最小, 即 
    $$\min\norm*{f(x)-T_n(x)}=\norm*{f(x)-S_n(x)}.$$
\end{theorem}