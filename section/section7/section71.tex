\section{常数项级数}

% \subsection{求和问题}
% 
% \begin{example}
%     计算 $\displaystyle\frac{1}{2}+\frac{3}{2^2}+\frac{5}{2^3}+\cdots+\frac{2n-1}{2^n}+\cdots.$
% \end{example}
% \begin{solution}
%     \begin{flalign*}
%         S_n & =2S_n-S_n                                                                                                                                    \\
%             & =1+\frac{3}{2}+\frac{5}{2^2}+\cdots+\frac{2n-1}{2^{n-1}}-\left(\frac{1}{2}+\frac{3}{2^2}+\cdots+\frac{2n-3}{2^{n-2}}+\frac{2n-1}{2^n}\right) \\
%             & =1+1+\frac{1}{2}+\frac{1}{2^2}+\cdots+\frac{1}{2^{n-1}}-\frac{2n-1}{2^n}=1+\frac{1-\frac{1}{2^{n-1}}}{1-\frac{1}{2}}-\frac{2n-1}{2^n},
%     \end{flalign*}
%     故原级数的和为 $\displaystyle S=\lim_{n\to\infty}S_n=3.$
% \end{solution}
% \begin{example}
%     计算 $\displaystyle\sum_{n=1}^\infty n\mathrm{e}^{-nx}.$
% \end{example}
% \begin{solution}
%     \begin{flalign*}
%         \left(1-\mathrm{e}^{-x}\right)\sum_{k=1}^nk\mathrm{e}^{-kx}=\sum_{k=1}^nk\mathrm{e}^{-kx}-\sum_{k=1}^nk\mathrm{e}^{-(k+1)x}
%         =\sum_{k=1}^n\mathrm{e}^{-kx}-n\mathrm{e}^{-(n+1)x}\to\frac{\mathrm{e}^{-x}}{1-\mathrm{e}^{-x}}~~(n\to\infty).
%     \end{flalign*}
%     故 $\displaystyle\lim_{n\to\infty}\sum_{k=1}^nk\mathrm{e}^{-kx}=\frac{\mathrm{e}^{-x}}{\left(1-\mathrm{e}^{-x}\right)^2}.$
% \end{solution}
% \begin{example}
%     求如下级数之和
%     \setcounter{magicrownumbers}{0}
%     \begin{table}[H]
%         \centering
%         \begin{tabular}{l | l}
%             (\rownumber{}) $\displaystyle\sum_{k=1}^\infty \arctan\frac{1}{2k^2}.$ & (\rownumber{}) $\displaystyle\sum_{k=2}^\infty \arctan\frac{2}{4k^2-4k+1}.$
%         \end{tabular}
%     \end{table}
% \end{example}
% \begin{solution}
%     \begin{enumerate}[label=(\arabic*)]
%         \item 由 $$\arctan\frac{1}{2k^2}=\arctan\frac{1}{2k-1}-\arctan\frac{1}{2k+1}$$
%               所以\begin{flalign*}
%                   \displaystyle\sum_{k=1}^\infty \arctan\frac{1}{2k^2} & =\sum_{k=1}^\infty\left(\arctan\frac{1}{2k-1}-\arctan\frac{1}{2k+1}\right) \\
%                                                                        & =\arctan1-\lim_{n\to\infty}\arctan\frac{1}{2n+1}=\frac{\pi}{4}.
%               \end{flalign*}
%         \item 由 $$\arctan x-\arctan y=\arctan\frac{x-y}{1+xy}$$
%               令 $\displaystyle\begin{cases}
%                       x-y=2 \\
%                       4k^2-4k=xy
%                   \end{cases}$, 解得 $\displaystyle\begin{cases}
%                       x=2k \\
%                       y=2k-2
%                   \end{cases}$, 或 $\displaystyle\begin{cases}
%                       x=2-2k \\
%                       y=-2k
%                   \end{cases}$, \\
%               当 $x=2k$ 时, \begin{flalign*}
%                   \sum_{k=2}^\infty \arctan\frac{2}{4k^2-4k+1} & =\sum_{k=2}^\infty[\arctan2k-\arctan(2k-2)]=-\arctan2+\lim_{n\to\infty}\arctan2n \\
%                                                                & =\frac{\pi}{2}-\arctan2.
%               \end{flalign*}
%               当 $x=2-2k$ 时, \begin{flalign*}
%                   \sum_{k=2}^\infty \arctan\frac{2}{4k^2-4k+1} & =\sum_{k=2}^{\infty}[\arctan(2-2k)-\arctan(-2k)]                    \\
%                                                                & =\sum_{k=2}^\infty[\arctan2k-\arctan(2k-2)]=\frac{\pi}{2}-\arctan2.
%               \end{flalign*}
%               综上, 原级数的和为 $\displaystyle\frac{\pi}{2}-\arctan2.$
%     \end{enumerate}
% \end{solution}
% \begin{example}
%     设 $x\in[0,\pi]$, 试求级数 $\displaystyle\sum_{n=1}^{\infty}\frac{\sin nx}{n}$ 的和函数.
% \end{example}
% \begin{solution}
%     若 $x=0$ 或 $\pi$, 显然级数和为 0.\\
%     现设 $0<x\le \pi$, 记 $\displaystyle S_n(x)=\sum_{k=1}^{n}\frac{\sin kx}{k}$, 则
%     \begin{flalign*}
%         S_n'(x) & =\left(\sum_{k=1}^{n}\dfrac{\sin kx}{k}\right)'=\sum_{k=1}^{n}\cos kx=\dfrac{1}{2\sin\dfrac{x}{2}}\sum_{k=1}^{n}2\sin\dfrac{x}{2}\cos kx \\
%                 & =\dfrac{1}{2\sin\dfrac{x}{2}}\sum_{k=1}^{n}\left[\sin\left(k+\dfrac{1}{2}\right)x-\sin\left(k-\dfrac{1}{2}\right)x\right]                \\
%                 & =\dfrac{1}{2\sin\dfrac{x}{2}}\left(\sin\dfrac{2n+1}{2}x-\sin\dfrac{x}{2}\right)
%         =\dfrac{\sin\left(n+\dfrac{1}{2}\right)x}{2\sin\dfrac{x}{2}}-\dfrac{1}{2}.
%     \end{flalign*}
%     于是
%     \begin{flalign*}
%         S_n(x) & =S_n(x)-S_n(\pi)=-\int_{x}^{\pi}S_n'(t)\mathrm{d}t                                                                 \\
%                & =-\frac{1}{2}\int_{x}^{\pi}\frac{1}{\sin\frac{t}{2}}\sin\left(n+\frac{1}{2}\right)t\mathrm{d}t+\frac{1}{2}(\pi-x).
%     \end{flalign*}
%     利用 Riemann 引理, $n\to\infty$ 时上式第一项趋向零, 所以级数和
%     $$S(x)=\begin{cases}
%             0                               & ,x=0,\pi  \\
%             \displaystyle\frac{1}{2}(\pi-x) & ,0<x<\pi.
%         \end{cases}$$
% \end{solution}
% 
% \begin{example}
%     求级数 $\displaystyle\sum_{n=1}^{\infty}\dfrac{1}{n^2}$ 的和.
% \end{example}
% \begin{solution}
%     \textbf{法一: }因为 $$\arcsin x=\sum_{n=0}^{\infty}\dfrac{(2n)!}{4^n(n!)^2(2n+1)}=x+\sum_{n=1}^{\infty}\dfrac{(2n-1)!!}{(2n)!!}\cdot\dfrac{x^{2n+1}}{2n+1}~  |x|\leqslant 1$$
%     令 $x=\sin u$, 则上式化为 $$u=\sin u+\sum_{n=1}^{\infty}\dfrac{(2n-1)!!}{(2n)!!}\cdot\dfrac{\sin^{2n+1}u}{2n+1}~  |u|\leqslant\dfrac{\pi}{2}$$
%     再将上式两端从 $0$ 到 $\dfrac{\pi}{2}$ 积分, 得
%     \begin{flalign*}
%         \dfrac{\pi^2}{8}&=1+\sum_{n=1}^{\infty}\dfrac{(2n-1)!!}{(2n+1)\cdot(2n)!!}\cdot\int_{0}^{\frac{\pi}{2}}\sin^{2n+1}u\dd u=1+\sum_{n=1}^{\infty}\dfrac{(2n-1)!!}{(2n+1)\cdot(2n)!!}\cdot\dfrac{(2n)!!}{(2n+1)!!}\\
%         &=1+\sum_{n=1}^{\infty}\dfrac{1}{(2n+1)^2}=\sum_{n=1}^{\infty}\dfrac{1}{(2n-1)^2}
%     \end{flalign*}
%     于是 $\displaystyle \sum_{n=1}^{\infty}\dfrac{1}{n^2}=\sum_{n=1}^{\infty}\dfrac{1}{(2n-1)^2}+\sum_{n=1}^{\infty}\dfrac{1}{(2n)^2}=\dfrac{\pi^2}{8}+\dfrac{1}{4}\sum_{n=1}^{\infty}\dfrac{1}{n^2}\Rightarrow \sum_{n=1}^{\infty}\dfrac{1}{n^2}=\dfrac{\pi^2}{6}.$\\
%     \textbf{法二: }由复数运算的 De~  Moivre 公式: $$(\cos x+\mathrm{i}\sin x)^m=\cos mx+\mathrm{i}\sin mx$$
%     对比等号两边的虚部, 则有 $$\sin mx=\mathrm{C}_m^1\cos^{m-1}x\sin x-\mathrm{C}_m^3\cos^{m-3}x\sin ^3x+\cdots$$
%     令 $m=2n+1$, 得到 $$\sin(2n+1)x=\mathrm{C}_{2n+1}^1\cos^{2n}x\sin x-\mathrm{C}_{2n+1}^3\cos^{2n-2}x\sin ^3x+\cdots$$
%     当 $x=\dfrac{k\pi}{2n+1}~ (1\leqslant k\leqslant n)$ 时, $\sin(2n+1)x=0,~\sin x\neq 0$, 此时有
%     $$\mathrm{C}_{2n+1}^1\cot^{2n}x-\mathrm{C}_{2n+1}^3\cot^{2n-2}x+\cdots=0$$
%     即 $\cot^2\dfrac{\pi}{2n+1},~\cot^2\dfrac{2\pi}{2n+1},~\cdots,~\cot^2\dfrac{n\pi}{2n+1}$ 是方程 $\mathrm{C}_{2n+1}^1u^n-\mathrm{C}_{2n+1}^3u^{n-1}+\cdots=0$ 的根, 
%     由根与系数的关系, 有 $$\sum_{k=1}^{n}\cot^2\dfrac{k\pi}{2n+1}=-\dfrac{\mathrm{C}_{2n+1}^3}{\mathrm{C}_{2n+1}^1}=\dfrac{n(2n-1)}{3}$$
%     又因为 $\csc^2\alpha=\cot^2\alpha+1$, 于是
%     $$\sum_{k=1}^{n}\csc^2\dfrac{k\pi}{2n+1}=\dfrac{n(2n-1)}{3}+n=\dfrac{n(2n+2)}{3}$$
%     再由 $\sin x<x<\tan x ~ \qty(0<x<\dfrac{\pi}{2})$, 得 $\cot x<\dfrac{1}{x}<\csc x$, 于是得
%     $$\dfrac{n(2n-1)}{3}\leqslant \dfrac{(2n+1)^2}{\pi^2}\sum_{k=1}^{n}\dfrac{1}{k^2}\leqslant \dfrac{n(2n+2)}{3}\Rightarrow \dfrac{n(2n-1)\pi^2}{3(2n+1)^2}\leqslant \sum_{k=1}^{n}\dfrac{1}{k^2}\leqslant \dfrac{n(2n+2)\pi^2}{3(2n+1)^2}$$
%     令 $n\to\infty$, 由夹逼准则得 $\displaystyle\sum_{n=1}^{\infty}\dfrac{1}{n^2}=\dfrac{\pi^2}{6}.$
% \end{solution}

\subsection{常数项级数的敛散性及其判别法}

\begin{definition}[常数项级数]
    设有数列 $ \left\{u_{n}\right\}: u_{1}, u_{2}, \cdots, u_{n}, \cdots $, 将其各项依次累加所得的式子 $ u_{1}+u_{2}+\cdots+u_{n}+\cdots$ 称为 \textit{(常数项) 无穷级数}, 简称 \textit{(常数项) 级数}, 记作 $\displaystyle \sum_{n=1}^{\infty} u_{n} $, 即 $\displaystyle \sum_{n=1}^{\infty} u_{n}=u_{1}+u_{2}+\cdots+   u_{n}+\cdots .$
    \index{常数项级数}
\end{definition}

\begin{definition}[常数项级数收敛]
    设给定常数项级数
    \begin{equation}
        \sum_{n=1}^{\infty} u_{n}=u_{1}+u_{2}+\cdots+u_{n}+\cdots
        \tag{1}
    \end{equation}
    称 $\displaystyle S_{n}=\sum_{i=1}^{n} u_{i}=u_{1}+u_{2}+\cdots+u_{n} $ 为级数 $\displaystyle \sum_{n=1}^{\infty} u_{n} $ 的前 $ n $ 项部分和, 若 $\displaystyle \lim _{n \rightarrow \infty} S_{n}=S $ (有限值), 则称级数 (1) \textit{收敛}; 若 $\displaystyle \lim _{n \rightarrow \infty} S_{n} $ 不存在, 则称级数 (1) \textit{发散}.
    \index{常数项级数收敛}
\end{definition}

常数项级数的基本性质:
\begin{enumerate}[label=(\arabic{*})]
    \item 级数 $\displaystyle \sum_{n=1}^{\infty} u_{n} $ 与 $\displaystyle \sum_{n=1}^{\infty} k u_{n} $ 有相同的敛散性 ($k $ 是不为零的常数).
    \item 若级数 $\displaystyle \sum_{n=1}^{\infty} u_{n}, \sum_{n=1}^{\infty} v_{n} $ 均收敛, 则级数 $\displaystyle \sum_{n=1}^{\infty}\left(u_{n} \pm v_{n}\right) $ 亦收敛, 且有
          $$\sum_{n=1}^{\infty}\left(u_{n} \pm v_{n}\right)=\sum_{n=1}^{\infty} u_{n} \pm \sum_{n=1}^{\infty} v_{n} .$$
    \item 在级数中去掉或添加有限项, 不影响级数的敛散性 (但收敛时, 级数和一般会改变).
    \item 收敛级数任意加括号后所成的级数仍收敛. 如果正项级数加括号后所成的级数收敛, 则原级数收敛.
    \item 级数收敛的必要条件: 若 $\displaystyle \sum_{n=1}^{\infty} u_{n} $ 收敛, 则 $\displaystyle \lim _{n \rightarrow \infty} u_{n}=0 $.
\end{enumerate}

\begin{example}
    级数 $\displaystyle\sum_{n=1}^{\infty}\dfrac{1}{n^{1+\frac{1}{n}}}$ 的敛散性的下列结论中, 正确的是
    \begin{tasks}(2)
        \task 因 $1+\dfrac{1}{n}>1$ 故级数收敛
        \task 因 $\displaystyle\lim_{n\to\infty}\dfrac{1}{n^{1+\frac{1}{n}}}=0$ 故级数收敛
        \task 因 $\dfrac{1}{n^{1+\frac{1}{n}}}<\dfrac{1}{n}$ 故级数收敛
        \task 级数发散
    \end{tasks}
\end{example}
\begin{solution}
    p-级数 中的 $p$ 表示为常数, 与 $n$ 无关, 所以选项 A 错误, 事实上, $\displaystyle \lim_{n\to\infty}\dfrac{\dfrac{1}{n^{1+\frac{1}{n}}}}{\dfrac{1}{n}}=\lim_{n\to\infty}\dfrac{1}{n^{\frac{1}{n}}}=1$, 其中 $\displaystyle\lim_{n\to\infty}n^{\frac{1}{n}}=1$ (重要极限),
    则原级数的敛散性与 $\displaystyle\sum_{n=1}^{\infty}\dfrac{1}{n}$ 同敛散, 又 $\displaystyle\sum_{n=1}^{\infty}\dfrac{1}{n}$ 发散, 故级数 $\displaystyle\sum_{n=1}^{\infty}\dfrac{1}{n^{1+\frac{1}{n}}}$ 发散, 选 D.
\end{solution}

% 此外, 级数还有 4 个基本性质:
% \begin{enumerate}[label=(\arabic{*})]
%     \item 去掉级数前面有限项或者在级数前面添加有限项, 不影响级数的敛散性;
%     \item 对于任意非零常数 $c$, 级数 $\displaystyle\sum_{n=1}^{\infty}a_n\text{ 与 }\sum_{n=1}^{\infty}ca_n$ 具有相同的敛散性;
%     \item 若级数 $\displaystyle\sum_{n=1}^{\infty}a_n\text{ 与 }\sum_{n=1}^{\infty}b_n$ 都收敛, 它们的和分别为 $s\text{ 与 }t$, 则 $\displaystyle\sum_{n=1}^{\infty}(a_n\pm b_n)$ 收敛且和为 $s\pm t$;
%     \item 收敛级数任意加括号后所成的级数仍然收敛, 且其和不变.
% \end{enumerate}

\begin{example}
    判别级数的敛散性 $\displaystyle\sum_{n=1}^{\infty}\ln\qty(1+2^n)\ln\dfrac{n+1}{n}.$
\end{example}
\begin{solution}
    将所给级数的一般项作适当变形, 有
    \begin{flalign*}
        a_n & =\ln\qty[2^n\qty(\dfrac{1}{2^n}+1)]\ln\qty(1+\dfrac{1}{n})=\qty[n\ln 2+\ln\qty(\dfrac{1}{2^n}+1)]\ln\qty(1+\dfrac{1}{n}) \\
            & =\ln 2\ln\qty(1+\dfrac{1}{n})^n+\ln\qty(\dfrac{1}{2^n}+1)\ln\qty(1+\dfrac{1}{n})
    \end{flalign*}
    可见, $\displaystyle\lim_{n\to\infty}a_n=\ln 2$, 即 $\displaystyle a_n\lim_{n\to\infty}\not\to0$, 不满足级数收敛的必要条件, 故级数发散.
\end{solution}

\subsection{正项级数的审敛法}

为了迅速而又准确地判断一个级数的敛散性, 掌握以下几个已知其敛散性的级数往往是有益的.

\begin{definition}[几何级数]
    若有几何级数 $\displaystyle\sum_{n=1}^{\infty}ar^n$, 那么当 $|r|<1$ 时收敛, 且其和为 $\dfrac{a}{1-r}$; 当 $|r|\geqslant 1$ 时发散.
    \index{几何级数}
\end{definition}

\begin{definition}[调和级数]
    称 $\displaystyle\sum_{n=1}^{\infty}\dfrac{1}{n}$ 为调和级数, 并且该级数是发散的.
    \index{调和级数}
\end{definition}

\begin{definition}[p-级数]
    若有 p-级数 $\displaystyle\sum_{n=1}^{\infty}\dfrac{1}{n^p}$, 那么当 $p>1$ 时收敛, 当 $p\leqslant 1$ 时发散.
    \index{p-级数}
\end{definition}

\begin{theorem}[正项级数收敛的充要条件]
    正项级数 $\displaystyle  \sum_{n=1}^{\infty} u_{n} $ 收敛的充要条件是其部分和数列 $ \left\{s_{n}\right\} $ 有上界.
    \index{正项级数收敛的充要条件}
\end{theorem}

\begin{theorem}[比较审敛法]
    设 $\displaystyle  \sum_{n=1}^{\infty} u_{n}, \sum_{n=1}^{\infty} v_{n} $ 是两个正项级数, 且 $ u_{n} \leqslant v_{n}(n=1,2, \cdots) $,
    若 $\displaystyle  \sum_{n=1}^{\infty} v_{n} $ 收敛, 则 $\displaystyle  \sum_{n=1}^{\infty} u_{n} $ 收敛; 若 $\displaystyle  \sum_{n=1}^{\infty} u_{n}  $发散, 则 $\displaystyle \sum_{n=1}^{\infty} v_{n} $ 发散 (“大收敛, 小发散”).\\
    特别地, 条件 $ u_{n} \leqslant v_{n}(n=1,2, \cdots) $ 可改为 $ u_{n} \leqslant c v_{n}(n \geqslant k, c>0)$, 结论仍然成立.
    \index{比较审敛法}
\end{theorem}

\begin{theorem}[比较审敛法的极限形式]
    设 $\displaystyle  \sum_{n=1}^{\infty} u_{n}, \sum_{n=1}^{\infty} v_{n} $ 是两个正项级数, 若 $\displaystyle  \lim _{n \rightarrow \infty} \frac{u_{n}}{v_{n}}=l $, 则有
    \begin{enumerate}[label=(\arabic{*})]
        \item 当 $ 0<l<+\infty  $时, $\displaystyle \sum_{n=1}^{\infty} u_{n} $ 与 $\displaystyle  \sum_{n=1}^{\infty} v_{n} $ 同收敛或同发散;
        \item 当 $ l=0 $ 且 $\displaystyle  \sum_{n=1}^{\infty} v_{n} $ 收敛时, $\displaystyle  \sum_{n=1}^{\infty} u_{n} $ 也收敛;
        \item 当 $ l=+\infty $ 且 $\displaystyle  \sum_{n=1}^{\infty} v_{n} $ 发散时, $\displaystyle \sum_{n=1}^{\infty} u_{n} $ 也发散.
    \end{enumerate}
    \index{比较审敛法的极限形式}
\end{theorem}

\begin{example}[2009 数一]
    设有两个数列 $ \left\{a_{n}\right\},\left\{b_{n}\right\}$, 若 $\displaystyle \lim _{n \to \infty} a_{n}=0 $ 则
    \begin{tasks}(2)
        \task 当 $\displaystyle \sum_{n=1}^{\infty} b_{n} $ 收敛时, $\displaystyle \sum_{n=1}^{\infty} a_{n} b_{n} $ 收敛
        \task 当 $\displaystyle \sum_{n=1}^{\infty} b_{n} $ 发散时, $\displaystyle \sum_{n=1}^{\infty} a_{n} b_{n} $ 发散
        \task 当 $\displaystyle \sum_{n=1}^{\infty}\left|b_{n}\right| $ 收敛时, $\displaystyle \sum_{n=1}^{\infty} a_{n}^{2} b_{n}^{2} $ 收敛
        \task 当 $\displaystyle \sum_{n=1}^{\infty}\left|b_{n}\right| $ 发散时, $\displaystyle \sum_{n=1}^{\infty} a_{n}^{2} b_{n}^{2} $ 发散
    \end{tasks}
\end{example}
\begin{solution}
    若令 $ a_{n}=b_{n}=\dfrac{(-1)^{n}}{\sqrt{n}}$, 则 $\displaystyle \lim _{n \rightarrow \infty} a_{n}=0, \sum_{n=1}^{\infty} b_{n} $ 收敛,
    却有 $\displaystyle  \sum_{n=1}^{\infty} a_{n} b_{n}=\sum_{n=1}^{\infty} \frac{1}{n} $ 发散 $\displaystyle \sum_{n=1}^{\infty} a_{n}^{2} b_{n}^{2}=\sum_{n=1}^{\infty} \frac{1}{n^{2}} $ 收敛,
    故排除 A, D;
    若取 $ a_{n}=b_{n}=\dfrac{1}{n} $, 则 $\displaystyle  \lim _{n \rightarrow \infty} a_{n}=0, \sum_{n=1}^{\infty}\left|b_{n}\right| $ 发散,
    却有 $\displaystyle \sum_{n=1}^{\infty} a_{n} b_{n}=\sum_{n=1}^{\infty} \frac{1}{n^{2}} $ 收敛, 故排除 B;
    又 $\displaystyle \sum_{n=1}^{\infty}\left|b_{n}\right| $ 与 $\displaystyle \sum_{n=1}^{\infty} a_{n}^{2} b_{n}^{2} $ 均为正项级数, 且
    $$\displaystyle\lim _{n \rightarrow \infty} a_{n}=0, \lim _{n \rightarrow \infty}\left|b_{n}\right|=0 ,~\lim_{n\to\infty}\dfrac{a_n^2b_n^2}{|b_n|}=\lim_{n\to\infty}an^2\cdot\lim_{n\to\infty}|b_n|=0$$
    由正项级数比较判别法的极限形式知: 当 $\displaystyle\sum_{n=1}^{\infty}|b_n|$ 收敛时, $\displaystyle\sum_{n=1}^{\infty}a_n^2b_n^2$ 收敛, 因此选 C.
\end{solution}

\begin{theorem}[比值审敛法]
    设 $\displaystyle  \sum_{n=1}^{\infty} u_{n} $ 为正项级数, 若 $\displaystyle  \lim _{n \rightarrow \infty} \frac{u_{n+1}}{u_{n}}=\rho $, 则当 $ \rho<1 $ 时, 级数收敛;
    当 $ \rho>1 $ 时, 级数发散; 当 $ \rho=1 $ 时, 级数可能收敛, 也可能发散.\\
    特别地, 当 $ \dfrac{u_{n+1}}{u_{n}}>1 $ 时, 级数 $\displaystyle  \sum_{n=1}^{\infty} u_{n} $ 一定发散; 当 $ \dfrac{u_{n+1}}{u_{n}}<1 $ 时, 级数可能收敛, 也可能发散.
    \index{比值审敛法}
\end{theorem}

\begin{theorem}[根值审敛法]
    设 $\displaystyle \sum_{n=1}^{\infty} u_{n} $ 为正项级数, 若 $\displaystyle \lim _{n \rightarrow \infty} \sqrt[n]{u_{n}}=\rho $, 则当 $ \rho<1 $ 时, 级数收敛;
    当 $ \rho>1 $ 时, 级数发散; 当 $ \rho=1 $ 时, 级数可能收敛, 也可能发散.
    \index{根值审敛法}
\end{theorem}

\begin{theorem}[积分判别法]
    设 $ f(x) $ 为 $ [1,+\infty) $ 上的非负递减函数, 则正项级数 $\displaystyle  \sum_{n=1}^{\infty} f(n) $ 与反常积分 $\displaystyle \int_{1}^{+\infty} f(x) \dd x $ 同时收敛或同时发散.
    \index{积分判别法}
\end{theorem}

\begin{example}
    判别级数 $\displaystyle  \sum_{n=1}^{\infty} \frac{n^{n}}{n!} $ 的敛散性.
\end{example}
\begin{solution}
    \textbf{法一: }利用级数收敛的必要条件, 因为 $\displaystyle  \frac{n^{n}}{n!}=\frac{n \cdot n \cdots n}{1 \cdot 2 \cdots n}>1(n>1) $, 所以 $\displaystyle  \lim _{n \rightarrow \infty} \frac{n^{n}}{n!} \neq 0 $, 故由级数收敛的必要条件知, 级数 $\displaystyle  \sum_{n=1}^{\infty} \frac{n^{n}}{n!} $ 发散.\\ 
    \textbf{法二: }利用比较判别法, 因为当 $ n>2 $ 时, 有 $\displaystyle  \frac{n^{n}}{n!}=\frac{n \cdot n \cdot n \cdots n}{1 \cdot 2 \cdot 3 \cdots n}>n $, 且级数 $\displaystyle  \sum_{n=1}^{\infty} n $ 发散, 所以由比较判别法知, 级数 $\displaystyle  \sum_{n=1}^{\infty} \frac{n^{n}}{n!} $ 发散.\\ 
    \textbf{法三: }利用比值判别法, 记 $\displaystyle  u_{n}=\frac{n^{n}}{n!} $, 则 $\displaystyle  \lim _{n \rightarrow \infty} \frac{u_{n+1}}{u_{n}}=\lim _{n \rightarrow \infty} \frac{(n+1)^{n+1}}{(n+1)!} \cdot \frac{n!}{n^{n}}=\lim _{n \rightarrow \infty}\left(1+\frac{1}{n}\right)^{n}=\mathrm{e}>1 $, 由比值判别法知, 级数 $\displaystyle  \sum_{n=1}^{\infty} \frac{n^{n}}{n!} $ 发散.\\ 
    \textbf{法四: }利用根值判别法, 由 Stirling 公式: $\displaystyle  n!\sim \sqrt{2 n \pi} \cdot\left(\frac{n}{\mathrm{e}}\right)^{n}(n \rightarrow \infty) $, 有
    $$
    \lim _{n \rightarrow \infty} \sqrt[n]{u_{n}}=\lim _{n \rightarrow \infty} \sqrt[n]{\frac{n^{n}}{n!}}=\lim _{n \rightarrow \infty} \frac{\mathrm{e}}{\sqrt[n]{\sqrt{2 n \pi}}}=\mathrm{e}>1 
    $$
    故由根值判别法知, 级数 $\displaystyle  \sum_{n=1}^{\infty} \frac{n^{n}}{n!} $ 发散.
\end{solution}

\begin{example}
    判别级数 $\displaystyle \sum_{n=1}^{\infty} \frac{7^{n}}{8^{n}-5^{n}} $ 的敛散性.
\end{example}
\begin{solution}
    \textbf{法一: }利用比值判别法, 记 $\displaystyle u_{n}=\frac{7^{n}}{8^{n}-5^{n}} $, 则 $$\displaystyle \lim _{n \rightarrow \infty} \frac{u_{n+1}}{u_{n}}=\lim _{n \rightarrow \infty} \dfrac{7^{n+1}}{8^{n+1}-5^{n+1}} \cdot \dfrac{8^{n}-5^{n}}{7^{n}}=7 \lim _{n \rightarrow \infty} \frac{1-\left(\dfrac{5}{8}\right)^{n}}{8-5\left(\dfrac{5}{8}\right)^{n}}=\frac{7}{8}<1 $$
    故由比值判别法知原级数收敛.\\
    \textbf{法二: }利用根值判别法, 记 $\displaystyle u_{n}=\frac{7^{n}}{8^{n}-5^{n}} $, 则
    $$\lim _{n \rightarrow \infty} \sqrt[n]{u_{n}}=\lim _{n \rightarrow \infty} \sqrt[n]{\frac{7^{n}}{8^{n}-5^{n}}}=\frac{7}{8} \lim _{n \rightarrow \infty} \frac{1}{\sqrt{1-\left(\dfrac{5}{8}\right)^{n}}}=\frac{7}{8}<1$$
    故由根值判别法知原级数收敛.\\
    \textbf{法三: }利用比较判别法的极限形式, 记 $\displaystyle u_{n}=\frac{7^{n}}{8^{n}-5^{n}} $, 则 $\displaystyle u_{n} \sim \frac{7^{n}}{8^{n}}=\left(\frac{7}{8}\right)^{n}(n \rightarrow \infty) $, 而 $\displaystyle \sum_{n=1}^{\infty}\left(\frac{7}{8}\right)^{n} $ 收敛, 故由比较判别法的极限形式知原级数收敛.
\end{solution}

% \begin{example}
%     判断级数 $\displaystyle \sum_{n=1}^\infty\dfrac{1}{\sqrt{n(n+1)}\qty(\sqrt{n+1}+\sqrt{n})}$ 的敛散性.
% \end{example}
% \begin{solution}
%     记 $u_n=\dfrac{1}{\sqrt{n(n+1)}\qty(\sqrt{n+1}+\sqrt{n})}$, 则有 
%     $$\frac{1}{\sqrt{n(n+1)}\qty(\sqrt{n+1}+\sqrt{n})}=\frac{1}{\sqrt{n(n+1)}} \cdot \frac{1}{\sqrt{n+1}+\sqrt{n}}=\frac{\sqrt{n+1}-\sqrt{n}}{\sqrt{n(n+1)}}=\frac{1}{\sqrt{n}}-\frac{1}{\sqrt{n+1}}$$
%     于是通项的部分和 $$S_{n}=\sum_{k=1}^{n}u_{k}=\sum_{k=1}^{n}\qty(\frac{1}{\sqrt{k}}-\frac{1}{\sqrt{k+1}})=1-\frac{1}{\sqrt{n+1}}\rightarrow1~(n\rightarrow\infty)$$
%     故原级数收敛.
% \end{solution}

\begin{example}
    判别级数 $\displaystyle \sum_{n=1}^{\infty}\qty(\sqrt{n+2}-2 \sqrt{n+1}+\sqrt{n}) $ 的敛散性.
\end{example}
\begin{solution}
    \textbf{法一: }利用级数敛散的定义, 分别记 $ u_{n}, S_{n} $ 为级数 $ \displaystyle\sum_{n=1}^{\infty}\qty(\sqrt{n+2}-2 \sqrt{n+1}+\sqrt{n}) $ 的通项和部分和, 则
    $$u_{n}=\sqrt{n+2}-2 \sqrt{n+1}+\sqrt{n}=\qty(\sqrt{n+2}-\sqrt{n+1})-\qty(\sqrt{n+1}-\sqrt{n}) $$
    于是
    \begin{flalign*}
        S_{n} & =\sum_{k=1}^{n} u_{k}=\sum_{k=1}^{n}\qty[\qty(\sqrt{k+2}-\sqrt{k+1})-\qty(\sqrt{k+1}-\sqrt{k})]=\sqrt{n+2}-\sqrt{n+1}-\sqrt{2}+1 \\
              & =\frac{1}{\sqrt{n+2}+\sqrt{n+1}}-\sqrt{2}+1 \rightarrow-\sqrt{2}+1 \quad(n \rightarrow \infty)
    \end{flalign*}
    故原级数收敛.\\
    \textbf{法二: }令 $ f(x)=\sqrt{x}, u_{n} $ 为原级数的通项, 将 $ f(x) $ 分别在 $ [n, n+1] $ 和 $ [n+1, n+2] $ 上用 Lagrange 中值定理, 有 $$\sqrt{n+1}-\sqrt{n}=\frac{1}{2 \sqrt{\xi_{1}}}, \quad \sqrt{n+2}-\sqrt{n+1}=\frac{1}{2 \sqrt{\xi_{2}}}$$
    其中 $ n<\xi_{1}<n+1, n+1<\xi_{2}<n+2 $, 于是
    \begin{flalign*}
        0<-u_{n} & =\qty(\sqrt{n+1}-\sqrt{n})-\qty(\sqrt{n+2}-\sqrt{n+1})=\frac{1}{2 \sqrt{\xi_{1}}}-\frac{1}{2 \sqrt{\xi_{2}}}=\frac{\sqrt{\xi_{2}}-\sqrt{\xi_{1}}}{2 \sqrt{\xi_{1}} \sqrt{\xi_{2}}}                                               \\
                 & =\frac{1}{2} \cdot \frac{\xi_{2}-\xi_{1}}{\sqrt{\xi_{1}} \sqrt{\xi_{2}}\left(\sqrt{\xi_{1}}+\sqrt{\xi_{2}}\right)}<\frac{2}{2 \xi_{1} \cdot 2 \sqrt{\xi_{1}}}=\frac{1}{2 \xi_{1}{ }^{\frac{3}{2}}}<\frac{1}{2 n^{\frac{3}{2}}} .
    \end{flalign*}
    又 $\displaystyle \sum_{n=1}^{\infty} \frac{1}{n^{\frac{3}{2}}} $ 收敛, 故由比较判别法知级数 $\displaystyle \sum_{n=1}^{\infty}\left(-u_{n}\right) $ 收敛, 从而原级数 $\displaystyle \sum_{k=1}^{n} u_{n} $ 收敛.\\
    \textbf{法三: }利用比较判别法的极限形式, 记 $ u_{n} $ 为原级数的通项, 则
    \begin{flalign*}
        u_{n} & =\sqrt{n+2}-2 \sqrt{n+1}+\sqrt{n}=\qty(\sqrt{n+2}-\sqrt{n+1})-(\sqrt{n+1}-\sqrt{n})                                                                             \\
              & =\frac{1}{\sqrt{n+2}+\sqrt{n+1}}-\frac{1}{\sqrt{n+1}+\sqrt{n}}=\frac{\sqrt{n}-\sqrt{n+2}}{\qty(\sqrt{n+2}+\sqrt{n+1}) \cdot\qty(\sqrt{n+1}+\sqrt{n})}           \\
              & =-\frac{2}{\qty(\sqrt{n+2}+\sqrt{n+1}) \cdot\qty(\sqrt{n+1}+\sqrt{n})\qty(\sqrt{n+2}+\sqrt{n})} \sim-\frac{1}{4\qty(\sqrt{n})^{3}} \quad(n \rightarrow \infty),
    \end{flalign*}
    所以 $ 0<-u_{n} \sim \frac{1}{4 n^{\frac{3}{2}}} $, 又 $\displaystyle \sum_{n=1}^{\infty} \frac{1}{n^{\frac{3}{2}}} $ 收敛, 故由比较判别法的极限形式知, 级数 $\displaystyle \sum_{n=1}^{\infty}\left(-u_{n}\right)$ 收敛, 从而原级数 $\displaystyle \sum_{k=1}^{n} u_{n} $ 收敛.
\end{solution}

\begin{example}
    判别级数 $\displaystyle \sum_{n=1}^{\infty} \dfrac{n \cos ^2 \dfrac{n \pi}{2}}{2^n}$ 的敛散性.
\end{example}
\begin{solution}
    \textbf{法一: }因为 $0 \leqslant \dfrac{n \cos ^2 \dfrac{n \pi}{2}}{2^n}<\dfrac{n}{2^n}$, 所以转而考虑级数 $\displaystyle \sum_{n=1}^{\infty} \dfrac{n}{2^n}$ 的收敛珄, 记 $u_n=\dfrac{n}{2^n}$, 则
    $$\lim _{n \rightarrow \infty} \dfrac{u_{n+1}}{u_n}=\lim _{n \rightarrow \infty} \dfrac{n+1}{2^{n+1}} \cdot \dfrac{2^n}{n}=\dfrac{1}{2}<1$$
    于是由比值判别法知级数 $\displaystyle \sum_{n=1}^{\infty} \dfrac{n}{2^n}$ 收敛, 从而由比较判别法知级数 $\displaystyle \sum_{n=1}^{\infty} \dfrac{n \cdot \cos ^2 \dfrac{n \pi}{2}}{2^n}$ 收敛.\\
    \textbf{法二: }同法一, 先考虑级数 $\displaystyle \sum_{n=1}^{\infty} \dfrac{n}{2^n}$ 的敛散性, 因为
    $$\lim _{n \rightarrow \infty} \sqrt[n]{u_n}=\lim _{n \rightarrow \infty} \sqrt[n]{\dfrac{n}{2^n}}=\lim _{n \rightarrow \infty} \dfrac{\sqrt[n]{n}}{2}=\dfrac{1}{2}<1$$
    所以由根值判别法知级数 $\displaystyle \sum_{n=1}^{\infty} \dfrac{n}{2^2}$ 收敛, 进而由比较判别法知级数 $\displaystyle \sum_{n=1}^{\infty} \dfrac{n \cdot \cos ^2 \dfrac{n \pi}{2}}{2^n}$ 收敛.\\
    \textbf{法三: }因为 $\cos ^2 \dfrac{n \pi}{2}=\dfrac{1+\cos n \pi}{2}=\dfrac{1+(-1)^n}{2}$, 所以
    $$
        \lim _{n \rightarrow \infty} \dfrac{n \cos ^2 \dfrac{n \pi}{2}}{2^n}=\sum_{n=1}^{\infty} \dfrac{n}{2^n} \cdot \dfrac{1+(-1)^n}{2}=\sum_{k=1}^{\infty} \dfrac{2 k}{2^{2 k}} \text {. }
    $$
    记 $u_k=\dfrac{2 k}{2^{2 k}}$, 则 $\displaystyle \lim _{k \rightarrow \infty} \sqrt[k]{u_k}=\lim _{k \rightarrow \infty} \sqrt[k]{\dfrac{2 k}{2^{2 k}}}=\lim _{k \rightarrow \infty} \dfrac{\sqrt[k]{2 k}}{4}=\dfrac{1}{4}<1$, 故级数 $\displaystyle \sum_{k=1}^{\infty} \dfrac{2 k}{2^{2 k}}$ 收敛, 从而 $\displaystyle \sum_{n=1}^\infty \dfrac{n\cos^2\dfrac{n\pi}{2}}{2^n}$ 收敛.
\end{solution}

\begin{example}
    判别级数 $\displaystyle  \sum_{n=1}^{\infty} \ln \left(1+\frac{1}{n}\right) $ 的敛散性.
\end{example}
\begin{solution}
    \textbf{法一: }因为 $\displaystyle  \lim _{x \rightarrow 0} \dfrac{\ln (1+x)}{x}=1 $, 所以 $\displaystyle  \lim _{n \rightarrow \infty} \dfrac{\ln \left(1+\dfrac{1}{n}\right)}{\dfrac{1}{n}}=1 $, 而级数 $\displaystyle  \sum_{n=1}^{\infty} \dfrac{1}{n} $ 发散, 故由比较判别法的极限形式知, 级数 $\displaystyle  \sum_{n=1}^{\infty} \ln \left(1+\dfrac{1}{n}\right) $ 发散.\\
    \textbf{法二: }利用微分中值定理, 因为 $ \ln x $ 在 $ [1,+\infty) $ 上满足 Lagrange 定理的条件, 所以由 Lagrange 中值定理知, 存在 $ \xi_{n} \in(n, n+1)(n \geqslant 1) $, 有 $ \ln \left(1+\dfrac{1}{n}\right)=\ln (n+1)-\ln n=\dfrac{1}{\xi_{n}} $, 从而有 $ \dfrac{1}{n+1}<   \ln \left(1+\dfrac{1}{n}\right)<\dfrac{1}{n} $, 又级数 $\displaystyle  \sum_{n=1}^{\infty} \dfrac{1}{n+1} $ 发散, 故由比较判别法知, 原级数 $\displaystyle  \sum_{n=1}^{\infty} \ln \left(1+\dfrac{1}{n}\right) $ 发散.\\
    \textbf{法三: }由 Taylor 公式, 有 $ \ln \left(1+\dfrac{1}{n}\right)=\dfrac{1}{n}-\dfrac{1}{2 n^{2}}+o\left(\dfrac{1}{n^{2}}\right) $,
    又 $ \dfrac{1}{2 n^{2}}-o\left(\dfrac{1}{n^{2}}\right) \sim \dfrac{1}{2 n^{2}}(n \rightarrow \infty) $, 所以 $ \displaystyle \sum_{n=1}^{\infty}\left(\dfrac{1}{2 n^{2}}-o\left(\dfrac{1}{n^{2}}\right)\right) $ 收敛. 而级数 $ \displaystyle \sum_{n=1}^{\infty} \dfrac{1}{n} $ 发散, 故由级数的性质, 得原级数 $ \displaystyle \sum_{n=1}^{\infty} \ln \left(1+\dfrac{1}{n}\right) $ 发散.\\
    \textbf{法四: }利用级数敛散的定义, 记原级数的部分和为 $\displaystyle  S_{n}=\sum_{k=1}^{n} \ln \left(1+\dfrac{1}{k}\right) $, 则
    $$S_{n}=\sum_{k=1}^{n}\left[\ln \left(1+\dfrac{1}{k}\right)-\ln k\right]=\ln (1+n) \rightarrow+\infty \quad(n \rightarrow \infty)$$
    故级数 $\displaystyle  \sum_{n=1}^{\infty} \ln \left(1+\dfrac{1}{n}\right) $ 发散.\\
    \textbf{法五: }利用广义积分与级数的联系
    $$ \int_{1}^{+\infty} \dfrac{1}{x} \dd x=\sum_{n=1}^{\infty} \int_{n}^{n+1} \dfrac{1}{x} \dd x=\sum_{n=1}^{\infty}[\ln (n+1)-\ln n]=\sum_{n=1}^{\infty} \ln \left(1+\dfrac{1}{n}\right) $$
    因为 $\displaystyle  \int_{1}^{+\infty} \dfrac{1}{x} \dd x=\lim _{b \rightarrow+\infty} \int_{1}^{b} \dfrac{1}{x} \dd x=\lim _{b \rightarrow+\infty} \ln b=+\infty $, 所以级数 $\displaystyle  \sum_{n=1}^{\infty} \ln \left(1+\dfrac{1}{n}\right) $ 发散.
\end{solution}

\begin{example}
    设级数 $\displaystyle \sum_{n=1}^{\infty}\dfrac{a_n}{n}$ 收敛, 证明 $\displaystyle \lim_{n \to \infty}\dfrac{a_1+a_2+\cdots+a_n}{n}=0.$
\end{example}
\begin{proof}[{\songti \textbf{证}}]
    记 $\displaystyle S_n=\sum_{k=1}^{n}\dfrac{a_k}{k},~S_0=0,~S=\sum_{n=1}^{\infty}\dfrac{a_n}{n}$, 则 $S_n-S_{n-1}=\dfrac{a_n}{n}$, 且 $$\lim_{n \to \infty}S_n=S\Rightarrow S_n=S+\varepsilon _n,~(n\geqslant 1,~\varepsilon_n\to 0,~n\to \infty )$$
    令 $b_n=\dfrac{a_1+a_2+\cdots+a_n}{n}$, 则 $\forall n\in \mathbb{N}$, 有
    \begin{flalign*}
        b_n & =\dfrac{S_1+(S_2-S_1)+\cdots+(S_n-s_{n-1})}{n}=\dfrac{\displaystyle \sum_{k=1}^{n}k(S_k-S_{k-1})}{n}=\dfrac{\displaystyle S+\varepsilon_1+\sum_{k=2}^{n}k(\varepsilon_k-\varepsilon_{k-1})}{n} \\
            & =\dfrac{S}{n}+\dfrac{\varepsilon_1}{n}+\dfrac{1}{n}\sum_{k=2}^{n}k(\varepsilon_k-\varepsilon_{k-1})=\dfrac{S}{n}+\varepsilon_n-\dfrac{1}{n}\sum_{k=1}^{n-1}\varepsilon_k
    \end{flalign*}
    而因为 $\varepsilon_k \to 0~(k\to \infty)$, 所以 $\displaystyle \lim_{n \to \infty}\dfrac{1}{n}\sum_{k=1}^{n-1}\varepsilon_k=\lim_{n \to \infty}\dfrac{\displaystyle \sum_{k=1}^{n-1}}{n-1}\cdot\dfrac{n-1}{n}=0$, 故 $\displaystyle \lim_{n \to \infty}b_n=0.$
\end{proof}

\begin{example}
    设数列 $\qty{u_n}$ 满足 $u_1=3,~u_2=5$, 当 $n\geqslant 3$ 时, $u_n=u_{n-1}+u_{n-2}$, 证明级数 $\displaystyle \sum_{n=1}^{\infty} \dfrac{1}{u_n}$ 收敛.
\end{example}
\begin{proof}[{\songti \textbf{证法一}}]
    由已知得 $ u_{n}>0 $, 且 $ \left\{u_{n}\right\} $ 是单调增加的, 所以 $ u_{n}=u_{n-2}+u_{n-1}<   2 u_{n-1} $, 于是得 $$\displaystyle  u_{n}=u_{n-2}+u_{n-1}>\frac{1}{2} u_{n-1}+u_{n-1}=\frac{3}{2} u_{n-1} $$ 从而得 $$\displaystyle  u_{n}>\frac{3}{2} u_{n-1}>   \left(\frac{3}{2}\right)^{2} u_{n-2}>\cdots>\left(\frac{3}{2}\right)^{n-1} u_{1}=3 \cdot\left(\frac{3}{2}\right)^{n-1} $$ 即 $\displaystyle  \frac{1}{u_{n}}<\frac{1}{3} \cdot\left(\frac{2}{3}\right)^{n-1} $, 而级数 $\displaystyle  \sum_{n=1}^{\infty}\left(\frac{2}{3}\right)^{n-1} $ 收敛, 故由比较判别法知级数 $\displaystyle  \sum_{n=1}^{\infty} \frac{1}{u_{n}} $ 收敛.
\end{proof}
\begin{proof}[{\songti \textbf{证法二}}]
    由已知得 $ u_{n}>0 $, 且 $ \left\{u_{n}\right\} $ 是单调增加的, 所以 $ u_{n}=u_{n-2}+u_{n-1}<   2 u_{n-1} $, 于是得 $$\displaystyle  u_{n}=u_{n-2}+u_{n-1}>\frac{1}{2} u_{n-1}+u_{n-1}=\frac{3}{2} u_{n-1} $$ 从而得 $\displaystyle  \frac{u_{n}^{-1}}{u_{n-1}{ }^{-1}}=\frac{u_{n-1}}{u_{n}}<\frac{2}{3}<1$, 故由达朗贝尔判别法知级数 $\displaystyle  \sum_{n=1}^{\infty} \frac{1}{u_{n}} $ 收敛.
\end{proof}
\begin{proof}[{\songti \textbf{证法三}}]
    将递推关系式改写成矩阵形式:
    $$\begin{pmatrix} u_n \\ u_{n-1} \\\end{pmatrix}=\begin{pmatrix} 1 & 1 \\ 1 & 0 \\\end{pmatrix}\begin{pmatrix} u_{n-1} \\ u_{n-2} \\\end{pmatrix}=\cdots=\begin{pmatrix} 1 & 1 \\ 1 & 0 \\\end{pmatrix}^{n-2}\begin{pmatrix} u_2 \\ u_1 \\\end{pmatrix}$$
    记 $\vb*{A}=\begin{pmatrix} 1 & 1 \\ 1 & 0 \\\end{pmatrix}$, 由 $\left\vert \lambda\vb*{E}-\vb*{A} \right\vert =0$, 求得 $\vb*{A}$ 的特征值为 $\lambda_1=\dfrac{1+\sqrt{5}}{2},~\lambda_2=\dfrac{1-\sqrt{5}}{2}$, 其对应的特征向量为 $\vb*{X}=\begin{pmatrix} \lambda_1 \\ 1 \\\end{pmatrix},~\vb*{X}_2=\begin{pmatrix} \lambda_2 \\ 1 \\\end{pmatrix}$, 取 $\vb*{X}=\mqty(\vb*{X}_1,\vb*{X}_2)=\begin{pmatrix} \lambda_1 & \lambda_2 \\ 1 & 1 \\\end{pmatrix}$, 则有 $\vb*{X}^{-1}=\dfrac{1}{\sqrt{5}}\begin{pmatrix} 1 & -\lambda_2 \\ -1 & \lambda_1 \\\end{pmatrix},~\vb*{A}=\vb*{X}\begin{pmatrix} \lambda_1 &  \\  & \lambda_2 \\\end{pmatrix}\vb*{X}^{-1}$,
    $$
        \vb*{A}^{n-2}=\vb*{X}\begin{pmatrix} \lambda_1 &  \\  & \lambda_2 \\\end{pmatrix}^{n-2}\vb*{X}^{-1}=\dfrac{1}{\sqrt{5}}\begin{pmatrix} \lambda_1^{n-1}-\lambda_2^{n-1} & -\lambda_2\lambda_1^{n-1}+\lambda_1\lambda_2^{n-1} \\ \lambda_1^{n-2}-\lambda_2^{n-2} & -\lambda_2\lambda_1^{n-2}+\lambda_1\lambda_2^{n-2} \\\end{pmatrix}
    $$
    于是由 $\begin{pmatrix} u_n\\ u_{n-1} \\\end{pmatrix}=\vb*{A}^{n-2}\begin{pmatrix} u_2 \\ u_1 \\\end{pmatrix}$ 得 
    $$
    u_n=\dfrac{1}{\sqrt{5}}\qty[\qty(\lambda_1^{n-1}-\lambda_2^{n-1})u_2+\qty(-\lambda_2\lambda_1^{n-1}+\lambda_1\lambda_2^{n-1})u_1]
    $$
    从而 
    \begin{flalign*}
    \dfrac{u_n}{u_{n+1}}=\dfrac{u_2\qty[\dfrac{1}{\lambda_1}-\qty(\dfrac{\lambda_2}{\lambda_1})^{n-1}]+u_1\qty[-\dfrac{\lambda_2}{\lambda_1}+\qty(\dfrac{\lambda_2}{\lambda_1})^{n-1}]}{u_2\qty[1-\qty(\dfrac{\lambda_2}{\lambda_1})^{n}]+u_1\qty[-\lambda_2+\lambda_1\qty(\dfrac{\lambda_2}{\lambda_1})^{n}]} 
    \end{flalign*}
    因为 $\left\vert \dfrac{\lambda_2}{\lambda_1} \right\vert <1$, 所以 $ \displaystyle \lim_{n \to \infty}\dfrac{u_n}{u_{n+1}}=\dfrac{2}{\sqrt{5}} $, 即 $\displaystyle \lim_{n \to \infty}\dfrac{\dfrac{1}{u_{n+1}}}{\dfrac{1}{u_n}}=\dfrac{2}{\sqrt{5}}<1$, 故由正项级数的比值判别法, 级数 $ \displaystyle \sum_{n=1}^{\infty} \dfrac{1}{u_n} $ 收敛.
\end{proof}

\subsection{交错级数}

\begin{theorem}[Leibniz 定理]
    若 $\qty{u_n}$ 单调递减, 且 $\displaystyle\lim_{n\to\infty}u_n=0$, 则 $\displaystyle\sum_{n=0}^{\infty}(-1)^nu_n$ 收敛.
    \index{Leibniz 定理}
\end{theorem}

\begin{example}
    判别级数 $\displaystyle  \sum_{n=2}^{\infty} \frac{(-1)^{n}}{\sqrt{n}+(-1)^{n}} $ 的敛散性.
\end{example}
\begin{solution}
    \textbf{法一: }将通项化简为
    $$\frac{(-1)^{n}}{\sqrt{n}+(-1)^{n}}=\frac{(-1)^{n} \sqrt{n}}{\sqrt{n} \cdot\left(\sqrt{n}+(-1)^{n}\right)}=\frac{(-1)^{n}}{\sqrt{n}}-\frac{1}{\sqrt{n}\left(\sqrt{n}+(-1)^{n}\right)}$$
    于是 $$\displaystyle  \sum_{n=2}^{\infty} \frac{(-1)^{n}}{\sqrt{n}+(-1)^{n}}=\sum_{n=2}^{\infty} \frac{(-1)^{n}}{\sqrt{n}}-\sum_{n=2}^{\infty} \frac{1}{\sqrt{n}\left(\sqrt{n}+(-1)^{n}\right)} $$
    对于 $\displaystyle  \sum_{n=2}^{\infty} \frac{(-1)^{n}}{\sqrt{n}} $, 由 Leibniz 判别法知收敛, 对于 $\displaystyle  \sum_{n=2}^{\infty} \frac{1}{\sqrt{n}\left(\sqrt{n}+(-1)^{n}\right)} $, 该级数为正项级数, 且 $\displaystyle  \frac{1}{\sqrt{n}\left(\sqrt{n}+(-1)^{n}\right)} \sim \frac{1}{n}(n \rightarrow \infty) $, 而 $\displaystyle  \sum_{n=2}^{\infty} \frac{1}{n} $ 发散, 所以 $\displaystyle  \sum_{n=2}^{\infty} \frac{1}{\sqrt{n}\left(\sqrt{n}+(-1)^{n}\right)} $ 发散, 从而知原级数发散.\\ 
    \textbf{法二: }将通项的分母有理化, 原级数可化为
    $$
    \sum_{n=2}^{\infty} \frac{(-1)^{n}}{\sqrt{n}+(-1)^{n}}=\sum_{n=2}^{\infty} \frac{(-1)^{n} \sqrt{n}}{n-1}-\sum_{n=2}^{\infty} \frac{1}{n} 
    $$
    而 $\displaystyle  \sum_{n=2}^{\infty} \frac{1}{n} $ 发散, 由 Leibniz 判别法知 $\displaystyle  \sum_{n=2}^{\infty} \frac{(-1)^{n} \sqrt{n}}{n-1} $ 收敛, 从而得原级数发散.\\ 
    \textbf{法三: }对原级数重新分组, 依次每两项加括号, 得到一个新级数 $\displaystyle  \sum_{k=1}^{\infty} u_{k} $:
    $$\left(\frac{1}{\sqrt{2}+1}-\frac{1}{\sqrt{3}-1}\right)+\cdots+\left(\frac{1}{\sqrt{2 k}+1}-\frac{1}{\sqrt{2 k+1}-1}\right)+\cdots$$
    其通项 $\displaystyle  u_{k}=\frac{1}{\sqrt{2 k}+1}-\frac{1}{\sqrt{2 k+1}-1}=\frac{\sqrt{2 k+1}-\sqrt{2 k}-2}{(\sqrt{2 k}+1)(\sqrt{2 k+1}-1)}<0 $, 所以新级数 $\displaystyle  \sum_{k=1}^{\infty} u_{k} $ 是负项级数, 又由
    $$
    u_{k}=\frac{\sqrt{2 k+1}-\sqrt{2 k}-2}{2 k}=\frac{\frac{1}{\sqrt{2 k+1}+\sqrt{2 k}}-2}{2 k} \sim-\frac{1}{k}(k \rightarrow \infty) 
    $$
    知 $\displaystyle  \left(-u_{k}\right) \sim \frac{1}{k}   (k \rightarrow \infty) $, 所以级数 $\displaystyle  \sum_{k=1}^{\infty} u_{k} $ 发散, 故由级数收敛的性质得原级数发散.
\end{solution}

原级数通项 $\displaystyle  \frac{(-1)^{n}}{\sqrt{n}+(-1)^{n}} \sim \frac{(-1)^{n}}{\sqrt{n}}(n \rightarrow \infty) $, 但原级数发散, 级数 $\displaystyle  \sum_{n=2}^{\infty} \frac{(-1)^{n}}{\sqrt{n}} $ 却是收敛的. 由此可见: 对于变号级数不能用比较判别法的极限形式判别.

\begin{example}
    设 $a_n=\displaystyle\int_{0}^{1}x^n\sqrt{1-x^2}\dd x,~b_n=\displaystyle\int_{0}^{\frac{\pi}{2}}\sin^nt\dd t~~(n=1,2,\cdots)$,
    \begin{enumerate}[label=(\arabic{*})]
        \item 求极限 $\displaystyle\lim_{n\to\infty}\dfrac{a_n}{b_n}$;
        \item 证明: 级数 $\displaystyle\sum_{n=1}^{\infty}(-1)^{n-1}\dfrac{a_n}{b_n}$ 收敛, 并求其和.
    \end{enumerate}
\end{example}
\begin{solution}
    \begin{enumerate}[label=(\arabic{*})]
        \item 由推论 \ref{inference:Wallis} 知, $b_n=\displaystyle\int_{0}^{\frac{\pi}{2}}\sin ^nt\dd t=\begin{cases}
                      \dfrac{(n-1)!!}{n!!}\cdot\dfrac{\pi}{2}, & n\text{ 为正偶数}      \\[6pt]
                      \dfrac{(n-1)!!}{n!!},                    & n\text{ 大于 1 的奇数}
                  \end{cases}$, 并且
              \begin{flalign*}
                  a_n\xlongequal{x=\cos t}\int_{0}^{\frac{\pi}{2}}\cos^nt\cdot\sin^2t\dd t=\begin{cases}
                                                                                               \dfrac{(n-1)!!}{(n+2)!!}\cdot\dfrac{\pi}{2}, & n\text{ 为正偶数}      \\[6pt]
                                                                                               \dfrac{(n-1)!!}{(n+2)!!},                    & n\text{ 大于 1 的奇数}
                                                                                           \end{cases}
              \end{flalign*}
              因此 $$\displaystyle\lim_{n\to\infty}\dfrac{a_n}{b_n}=\begin{cases}
                      \dfrac{(n-1)!!}{(n+2)!!}\cdot\dfrac{\pi}{2}\cdot\dfrac{n!!}{(n-1)!!}\cdot\dfrac{2}{\pi}=\dfrac{1}{n+2}\to0, & n\text{ 为正偶数}      \\[6pt]
                      \dfrac{(n-1)!!}{(n+2)!!}\cdot\dfrac{n!!}{(n-1)!!}=\dfrac{1}{n+2}\to0,                                       & n\text{ 大于 1 的奇数}
                  \end{cases}$$
              故 $\displaystyle\lim_{n\to\infty}\dfrac{a_n}{b_n}=0.$
        \item 由 (1) 可知 $\displaystyle\sum_{n=1}^{\infty}(-1)^{n-1}\dfrac{1}{n+2}$ 为交错级数, 由 Leibniz 定理, 知级数收敛, 并令 $S(x)=\displaystyle\sum_{n=1}^{\infty}(-1)^{n-1}\dfrac{x^{n+2}}{n+2}$, 则
              $$S'(x)=\sum_{n=1}^{\infty}(-1)^{n-1}x^{n+1}=\sum_{n=0}^{\infty}(-1)^nx^{n+2}=\dfrac{x^2}{1+x}~~(-1<x<1)$$
              即 $S(x)=S(0)+\displaystyle\int_{0}^{x}S'(t)\dd t=\int_{0}^{x}\dfrac{t^2}{1+t}\dd t=\eval{\dfrac{1}{2}t^2-t+\ln|1+t|}_{0}^{x}=\dfrac{1}{2}x^2-x+\ln(1+x),x\in(-1,1]$,
              故 $$\displaystyle\sum_{n=1}^{\infty}(-1)^{n-1}\dfrac{a_n}{b_n}=\sum_{n=1}^{\infty}(-1)^{n-1}\dfrac{1}{n+2}=S(1)=\ln 2-\dfrac{1}{2}.$$
    \end{enumerate}
\end{solution}

\subsection{一般数项级数的判别法}

\begin{definition}[绝对收敛与条件收敛]
    若级数 $\displaystyle \sum_{n=1}^{\infty}\left|u_{n}\right| $ 收敛, 则称级数 $\displaystyle \sum_{n=1}^{\infty} u_{n} $ 绝对收敛,
    若级数 $\displaystyle \sum_{n=1}^{\infty} u_{n} $ 收敛, 而级数 $\displaystyle  \sum_{n=1}^{\infty}\left|u_{n}\right| $ 发散, 则称级数 $\displaystyle \sum_{n=1}^{\infty} u_{n} $ 条件收敛.
\end{definition}

判别任意项级数收敛的常用方法:
\begin{enumerate}[label=(\arabic{*})]
    \item 绝对值判别法: 若 $\displaystyle  \sum_{n=1}^{\infty}\left|u_{n}\right| $ 收敛, 则 $\displaystyle  \sum_{n=1}^{\infty} u_{n} $ 绝对收敛;
    \item 交错级数一般使用 Leibniz 判别法;
    \item 将级数分解成两个收敛级数之和.
\end{enumerate}

判别级数发散的常用方法:
\begin{enumerate}[label=(\arabic{*})]
    \item 证明级数的一般项的极限不存在或不为零;
    \item 将级数按某种方式加括号后所得级数发散;
    \item 将级数分解为一个收敛级数与一个发散级数之和.
\end{enumerate}

\begin{theorem}
    设 $\displaystyle\sum_{n=1}^{\infty}u_n$ 是任意项级数, 其中 $\sum$ 的上标都是 $\infty$, 下标并非总是 $1$,
    \setcounter{magicrownumbers}{0}
    \begin{table}[H]
        \centering
        \begin{tabular}{l l}
            (\rownumber{}) 若 $\displaystyle \sum_{n=1}^{\infty}|u_n|$ 收敛, 则 $\displaystyle\sum_{n=1}^{\infty}u_n$ 收敛.            & (\rownumber{}) 若 $\displaystyle\sum_{n=1}^{\infty}u_n$ 发散, 则 $\displaystyle\sum_{n=1}^{\infty}|u_n|$ 发散.                \\
            (\rownumber{}) 若 $\displaystyle\sum_{n=1}^{\infty}u_n$ 收敛, 则 $\displaystyle\sum_{n=1}^{\infty}|u_n|$ 不一定收敛.       & (\rownumber{}) 若 $\displaystyle\sum_{n=1}^{\infty}u^2_n$ 收敛, 则 $\displaystyle\sum_{n=1}^{\infty}\dfrac{u_n}{n}$ 绝对收敛. \\
            (\rownumber{}) 若 $\displaystyle\sum_{n=1}^{\infty}u_n$ 收敛, 则 $\displaystyle\sum_{n=1}^{\infty}u^2_n$ 不一定收敛.       & (\rownumber{}) 若 $\displaystyle\sum_{n=1}^{\infty}u_n$ 收敛, 则 $\displaystyle\sum_{n=1}^{\infty}u_{2n}$ 不一定收敛.         \\
            (\rownumber{}) 若 $\displaystyle\sum_{n=1}^{\infty}u_n$ 收敛, 则 $\displaystyle\sum_{n=1}^{\infty}(u_{2n-1}+u_{2n})$ 收敛. & (\rownumber{}) 若 $\displaystyle\sum_{n=1}^{\infty}u_n$ 收敛, 则 $\displaystyle\sum_{n=1}^{\infty}(u_{n}\pm u_{n+1})$ 收敛.
        \end{tabular}
    \end{table}
\end{theorem}

\subsection{其他判别法}

\begin{theorem}[Kummer 判别法]\index{Kummer 判别法}
    设 $a_n>0,~b_n>0~(n=1,2, \cdots )$\begin{enumerate}[label=(\arabic{*})]
        \item 若存在常数 $\alpha>0$, 使得 $\dfrac{b_n}{b_{n+1}}a_n-a_{n+1}\geqslant \alpha~(n=1,2, \cdots )$, 则级数 $\displaystyle \sum_{n=1}^{\infty}b_n$ 收敛;
        \item 若 $\displaystyle \sum_{n=1}^{\infty}\dfrac{1}{a_n}$ 发散, 且 $\dfrac{b_n}{b_{n+1}}a_n-a_{n+1}\leqslant 0~(n=1,2, \cdots )$, 则级数 $\displaystyle \sum_{n=1}^{\infty}b_n$ 发散.
    \end{enumerate}
\end{theorem}

\begin{theorem}[Raabe 判别法]\index{Raabe 判别法}
    设 $a_n>0,~n=1,2, \cdots $\begin{enumerate}[label=(\arabic{*})]
        \item 若存在 $r>1$, 使得当 $n>N$ 时, 有 $n\qty(\dfrac{a_n}{a_{n+1}}-1)\geqslant r$, 则 级数 $\displaystyle \sum_{n=1}^{\infty}a_n$ 收敛;
        \item 若对充分大的 $n$ 都有 $n\qty(\dfrac{a_n}{a_{n+1}}-1)\leqslant 1$, 则级数 $\displaystyle \sum_{n=1}^{\infty}a_n$ 发散.
    \end{enumerate}
\end{theorem}

\begin{theorem}[Raabe 判别法的极限形式]\index{Raabe 判别法的极限形式}
    设 $a_n>0,~n=1,2, \cdots $, 且 $$\dfrac{a_n}{a_{n+1}}=1+\dfrac{l}{n}+o\qty(\dfrac{1}{n}) ~(n\to \infty)$$, 则当 $l>1$ 时级数 $\displaystyle \sum_{n=1}^{\infty} a_n$ 收敛; 当 $l<1$ 时, $\displaystyle \sum_{n=1}^{\infty} a_n$ 发散.
\end{theorem}

\begin{theorem}[Gauss 判别法]\index{Gauss 判别法}
    设 $a_n>0,~n=1,2, \cdots $, 且 $$
        \dfrac{a_n}{a_{n+1}}=1+\dfrac{1}{n}+\dfrac{\beta}{n\ln n}+o\qty(\dfrac{1}{n\ln n})~(n\to \infty)
    $$
    则当 $\beta>1$ 时, 级数 $\displaystyle \sum_{n=1}^{\infty} a_n$ 收敛; 当 $\beta<1$ 时, 级数 $\displaystyle \sum_{n=1}^{\infty} a_n$ 发散.
\end{theorem}

\begin{theorem}[Frink 判别法]\index{Frink 判别法}
    设 $\displaystyle \sum_{n=1}^{\infty} a_n$ 为正项级数, $\displaystyle \lim_{n \to \infty}\qty(\dfrac{a_n}{a_{n+1}})^{n}=\rho$,
    则当 $\rho<\dfrac{1}{\e}$ 时, 级数 $\displaystyle \sum_{n=1}^{\infty} a_n$ 收敛; 当 $\rho>\dfrac{1}{\e}$ 时, 级数 $\displaystyle \sum_{n=1}^{\infty} a_n$ 发散.
\end{theorem}

% \subsubsection{Cauchy 准则及应用}
% 
% \paragraph{Cauchy 准则} 级数 $\displaystyle\sum_{n=1}^{\infty}a_n$ 收敛的充要条件是: $\forall \varepsilon>0,~\exists N>0$, 当 $n>N$ 时
% $$\left| \sum ^{n+p}_{k=n+1}a_{k}\right|  <\varepsilon ~ \left( \forall p\in \mathbf{N}\right). $$
% 
% \paragraph{Cauchy 准则的否定形式} 级数 $\displaystyle\sum_{n=1}^{\infty}a_n$ 发散的充要条件是: $\exists \varepsilon_0>0,~\forall N>0$, $\exists n>N$ 及某个自然数 $p$, 
% 使得 $\displaystyle\left| \sum ^{n+p}_{k=n+1}a_{k}\right| \geq \varepsilon _{0}.$
% 
% \subsubsection{正项级数敛散性的判定}
% 
% 判断级数 $\displaystyle\sum_{n=1}^{\infty}a_n$ 的敛散性, 通常有如下方法:
% 
% \paragraph{定义法} 若通项 $a_n\not\to0~ (n\to\infty)$, 则 $\displaystyle\sum_{n=1}^{\infty}a_n$ 发散.
% 
% \paragraph{判阶法} 如果 $a\to0~ (n\to+\infty)$, 并且相对 $\dfrac{1}{n}$ 来讲, 它是 $p$ 阶的无穷小量, 那么当 $p>1$ 时, 级数 $\displaystyle\sum_{n=1}^{\infty}a_n$ 收敛; 当 $p\leqslant1$ 时, $\displaystyle\sum_{n=1}^{\infty}a_n$ 发散.
% 
% \paragraph{D' Alembert 判别法 (比值判别法)} 对正项级数 $\displaystyle\sum_{n=1}^{\infty}a_n$, \\
% 若 $\exists q>0,~N>0\text{, 使得 }\forall n>N\text{, 有 }\dfrac{a_{n+1}}{a_n}\leqslant q<1$, 则级数 $\displaystyle\sum_{n=1}^{\infty}a_n$ 收敛; \\
% 若 $\exists N>0,~n>N$ 时, 恒有 $\dfrac{a_{n+1}}{a_n}\geqslant 1$, 则级数 $\displaystyle\sum_{n=1}^{\infty}a_n$ 发散. \\
% 特别地, 若 $\displaystyle\lim_{n\to\infty}\dfrac{a_{n+1}}{a_n}=l$, 则当 $l<1$ 时, 级数 $\displaystyle\sum_{n=1}^{\infty}a_n$ 收敛; 当 $l>1$ 时, $\displaystyle\sum_{n=1}^{\infty}a_n$ 发散.
% 
% \paragraph{Cauchy 判别法 (根式判别法)} 对正项级数 $\displaystyle\sum_{n=1}^{\infty}a_n$, \\
% 若 $a_n\geqslant 0,~\exists q,N>0,~\forall n>N$, 有 $\sqrt[n]{a_n}\leqslant q<1$, 则级数 $\displaystyle\sum_{n=1}^{\infty}a_n$ 收敛; \\
% 若 $\exists N>0,~\forall n>N\text{, 有 }\sqrt[n]{a_n}\geqslant1$, 则级数 $\displaystyle\sum_{n=1}^{\infty}a_n$ 发散.\\
% 特别地, 若 $\displaystyle\lim_{n\to\infty}\sqrt[n]{a_n}=l$, 则当 $l<1$ 时, $\displaystyle\sum_{n=1}^{\infty}a_n$ 收敛;当 $l>1$ 时, $\displaystyle\sum_{n=1}^{\infty}a_n$ 发散.
% 
% \paragraph{比较判别法} $\displaystyle\sum_{n=1}^{\infty}a_n$ 和 $\displaystyle\sum_{n=1}^{\infty}b_n$ 是正项级数, 若从某项开始恒有 $a_n\leqslant b_n$, 
% 若 $\displaystyle\sum_{n=1}^{\infty}b_n$ 收敛, 则 $\displaystyle\sum_{n=1}^{\infty}a_n$ 收敛; 反之, 若 $\displaystyle\sum_{n=1}^{\infty}a_n$ 发散, 则 $\displaystyle\sum_{n=1}^{\infty}b_n$ 也发散. (小发散, 大收敛).
% 
% \paragraph{Cauchy 积分判别法} 若在区间 $[1,+\infty)$ 上, $f(x)\searrow$, 且 $f(x)\geqslant0$, 则级数 $\displaystyle\sum_{n=1}^{\infty}f(n)$ 与 $\displaystyle\int_{1}^{+\infty}f(x)\mathrm{d}x$ 同时敛散.
% 
% \paragraph{部分和有界} 考虑部分和 $\displaystyle\sum_{k=1}^{n}a_k$ 是否关于 $n$ 有界, 有界则收敛, 无界则发散.
% 
% \subsubsection{变号级数敛散性的判断}
% 
% \subsection{级数敛散性的应用}