\begin{flushright}
    \begin{tabular}{r|}
        \textit{“数学主要的目标是公众的利益和自然现象的解释。”}\\
        ——\textit{傅立叶}
    \end{tabular}
\end{flushright}

无穷级数是指一个包含无限个项的级数. 这些项可以是数字、变量或函数,它们按照一定的规律相加或相乘. 
无穷级数在数学中有着重要的应用,例如在微积分中用于求和、逼近和积分等问题. 

无穷级数可以是收敛的,也可以是发散的. 如果一个无穷级数的部分和随着项数的增加趋于一个有限的数,那么这个级数是收敛的;
反之,如果部分和趋于无穷大或没有确定的极限,那么这个级数是发散的. 

一些著名的无穷级数包括调和级数、几何级数、幂级数等. 无穷级数的性质和收敛性通常可以通过级数的收敛定理、比较判别法、积分判别法等方法来确定. 

无穷级数也是高等数学的基本内容之一,它包含常数项级数与函数项级数,而在函数项级数中,我
们讨论的是幂级数和傅里叶级数. 每个常数项级数都与数列之间存在着一一对应关系,但二者讨论的侧
重点不同,常数项级数侧重于讨论敛散性,这也是研究生入门考试主要考察的内容之一. 幂级数的收敛
域及和函数是本章的另一个重点内容.