\section{向量组的线性相关性}

\subsection{向量的概念}

\subsection{向量的线性组合}

\begin{definition}[线性组合]
    由 $ m $ 个 $ n $ 维向量 $ \vb*{\alpha}_{1}, \vb*{\alpha}_{2}, \cdots, \vb*{\alpha}_{m} $ 和 $ m $ 个常数 $ k_{1}, k_{2}, \cdots, k_{m} $ 所构成的向量
    $$k_{1} \vb*{\alpha}_{1}+k_{2} \vb*{\alpha}_{2}+\cdots+k_{m} \vb*{\alpha}_{m}$$
    称为向量组 $ \vb*{\alpha}_{1}, \vb*{\alpha}_{2}, \cdots, \vb*{\alpha}_{m} $ 的一个线性组合.
\end{definition}
\begin{definition}[线性表示]
    对于向量组 $ \vb*{\alpha}_{1}, \vb*{\alpha}_{2}, \cdots, \vb*{\alpha}_{m} $ 和向量 $ \vb*{\beta}$,若存在 $ m $ 个常数 $ k_{1}, k_{2}, \cdots, k_{m} $,使得
    $$\vb*{\beta}=k_{1} \vb*{\alpha}_{1}+k_{2} \vb*{\alpha}_{2}+\cdots+k_{m} \vb*{\alpha}_{m}$$
    则称向量 $ \vb*{\beta} $ 是向量组 $ \vb*{\alpha}_{1}, \vb*{\alpha}_{2}, \cdots, \vb*{\alpha}_{m} $ 的线性组合,或者说向量 $ \vb*{\beta} $
    可以由向量组 $ \vb*{\alpha}_{1}, \vb*{\alpha}_{2}, \cdots, \vb*{\alpha}_{m} $ 线性表出,$k_{1}, k_{2}, \cdots, k_{m} $ 称为这个线性组合的系数.
\end{definition}
\begin{definition}[向量组间的线性表示]
    设有两个 $ n $ 维向量组
    $\vb*{A}: \vb*{\alpha}_{1}, \vb*{\alpha}_{2}, \cdots, \vb*{\alpha}_{s} \text { 和 } \vb*{B}: \vb*{\beta}_{1}, \vb*{\beta}_{2}, \cdots, \vb*{\beta}_{t}$,
    若 $ \vb*{B} $ 组中的每个向量 $ \vb*{\beta}_{i}~~(i=1,2, \cdots, t) $ 都能由
    向量组 $ \vb*{A} $ 中的向量 $ \vb*{\alpha}_{1}, \vb*{\alpha}_{2}, \cdots, \vb*{\alpha}_{s} $ 线性表示,则称
    向量组 $ \vb*{B} $ 能由向量组 $ \vb*{A} $ 线性表示,对应矩阵方程 $\vb*{AX}=\vb*{B}$,
    其中,$\vb*{A}=\left(\vb*{\alpha}_{1}, \vb*{\alpha}_{2}, \cdots, \vb*{\alpha}_{s}\right),~\vb*{B}=\left(\vb*{\beta}_{1}, \vb*{\beta}_{2}, \cdots, \vb*{\beta}_{t}\right).$
\end{definition}
\begin{definition}[线性相关、线性无关]
    设 $ \vb*{\alpha}_{1}, \vb*{\alpha}_{2}, \cdots, \vb*{\alpha}_{m} $ 是 $m$ 个向量,对于方程
    $$\lambda_1\vb*{\alpha}_{1}+\lambda_2\vb*{\alpha}_{2}+\cdots+\lambda_m\vb*{\alpha}_{m}=\vb*{0}$$
    若有其非零解 $(\lambda_1,\lambda_2,\cdots,\lambda_m)\neq\vb*{0}$,则称 $\vb*{\alpha}_{1}, \vb*{\alpha}_{2}, \cdots, \vb*{\alpha}_{m}$ 线性相关;
    若其只有唯一解 $(\lambda_1,\lambda_2,\cdots,\lambda_m)$ $=\vb*{0}$,则称 $\vb*{\alpha}_{1}, \vb*{\alpha}_{2}, \cdots, \vb*{\alpha}_{m}$ 线性无关.
\end{definition}

\begin{theorem}[向量组线性相关的若干结论]
    \begin{enumerate}[label=(\arabic{*})]
        \item 向量组 $ \vb*{\alpha}_{1}, \vb*{\alpha}_{2}, \cdots, \vb*{\alpha}_{m}(m \geqslant 2) $ 线性相关的充分必要条件是 $ \vb*{\alpha}_{1}, \vb*{\alpha}_{2}, \cdots, \vb*{\alpha}_{m} $ 中至少有一 个向量可由其余 $ m-1 $ 个向量线性表示;
        \item 若向量组 $ \vb*{\alpha}_{1}, \vb*{\alpha}_{2}, \cdots, \vb*{\alpha}_{m} $ 线性无关,而向量组 $ \vb*{\beta}, \vb*{\alpha}_{1}, \vb*{\alpha}_{2}, \cdots, \vb*{\alpha}_{m} $ 线性相关,则 $ \vb*{\beta} $ 可由 \\$ \vb*{\alpha}_{1}, \vb*{\alpha}_{2}, \cdots, \vb*{\alpha}_{m} $ 线性表示,且表示式是唯一的;
        \item 对于 $ m $ 个 $ n $ 维向量 $ \vb*{\alpha}_{1}, \vb*{\alpha}_{2}, \cdots, \vb*{\alpha}_{m}$, 若 $ m>n $,则这 $ m $ 个向量必线性相关;
        \item 若 $ \vb*{\alpha}_{1}, \vb*{\alpha}_{2}, \cdots, \vb*{\alpha}_{r} $ 线性相关, 则 $ \vb*{\alpha}_{1}, \cdots, \vb*{\alpha}_{r}, \vb*{\alpha}_{r+1}, \cdots, \vb*{\alpha}_{m} $ 也线性相关; (部分相关,整体必相关.)
        \item 若 $ r $ 维向量组线性无关,则将它的每个向量添加 $ n-r $ 个分量所成的 $ n $ 维向量组也线性无关. (整体无关,部分必无关.)
    \end{enumerate}
\end{theorem}

\begin{example}[2014 数一]
    设 $\vb*{\alpha}_1,~\vb*{\alpha}_2,~\vb*{\alpha}_3$ 均为 3 维向量,则对任意常数 $k,~l$,向量组 $\vb*{\alpha}_1+k\vb*{\alpha}_3,~\vb*{\alpha}_2+l\vb*{\alpha}_3$ 线性无关是向量组 $\vb*{\alpha}_1,~\vb*{\alpha}_2,~\vb*{\alpha}_3$ 线性无关的
    \begin{tasks}(2)
        \task 必要非充分条件
        \task 充分非必要条件
        \task 充分必要条件
        \task 既非充分也非必要条件
    \end{tasks}
\end{example}
\begin{solution}
    \textbf{整体无关,部分必无关; 部分相关,整体必相关. }
    设 $\lambda_1(\vb*{\alpha}_1+k\vb*{\alpha}_3)+\lambda_2(\vb*{\alpha}_2+l\vb*{\alpha}_3)=\vb*{0}$ 则 $$\lambda_1\vb*{\alpha}_1+\lambda_2\vb*{\alpha}_2+(k\lambda_1+l\lambda_2)\vb*{\alpha}_3=\vb*{0}$$
    若 $\vb*{\alpha}_1,~\vb*{\alpha}_2,~\vb*{\alpha}_3$ 线性无关,则 $\lambda_1=\lambda_2=0$,所以 $\vb*{\alpha}_1+k\vb*{\alpha}_3,~\vb*{\alpha}_2+l\vb*{\alpha}_3$ 线性无关;
    反之,若 $\vb*{\alpha}_1+k\vb*{\alpha}_3,~\vb*{\alpha}_2+l\vb*{\alpha}_3$ 线性无关,则不一定有 $\vb*{\alpha}_1,~\vb*{\alpha}_2,~\vb*{\alpha}_3$ 线性无关,例如: 设
    $$\vb*{\alpha}_1=(1,0,0)^\top,~\vb*{\alpha}_2=(0,1,0)^\top,~\vb*{\alpha}_3=(0,0,0)$$
    显然,对任意常数 $k,~l$,向量组 $\vb*{\alpha}_1+k\vb*{\alpha}_3,~\vb*{\alpha}_2+l\vb*{\alpha}_3$ 线性无关,但 $\vb*{\alpha}_1,~\vb*{\alpha}_2,~\vb*{\alpha}_3$ 线性相关,因此是必要非充分条件,选 A.
\end{solution}

\begin{example}
    设 $\vb*{A}_{n\times n}=(\vb*{\alpha}_1,\vb*{\alpha}_2,\cdots,\vb*{\alpha}_n)$ 的前 $n-1$ 个向量线性相关,后 $n-1$ 个向量线性无关,且 $\vb*{\beta}=\vb*{\alpha}_1+\vb*{\alpha}_2+\cdots+\vb*{\alpha}_n$,证明: $\vb*{Ax}=\vb*{\beta}$ 有无穷多解.
\end{example}
\begin{proof}[{\songti \textbf{证法一}}]
    要证 $\vb*{Ax}=\vb*{\beta}$ 有无穷多解,即证 $\rank\vb*{A}=\rank(\vb*{A},\vb*{\beta})<n$,因为 $\vb*{\alpha}_2,\cdots,\vb*{\alpha}_n$ 线性无关,所以 $\rank\vb*{A}\geqslant n-1$,又因为 $\vb*{\alpha}_1,\cdots,\vb*{\alpha}_{n-1}$ 线性相关,所以 $\rank\vb*{A}<n$,
    于是 $$n-1\leqslant \rank\vb*{A}<n\Rightarrow \rank\vb*{A}=n-1$$
    又因为 $\vb*{\beta}=\vb*{\alpha}_1+\vb*{\alpha}_2+\cdots+\vb*{\alpha}_n$,所以
    $$\rank(\vb*{A},\vb*{\beta})=\rank(\vb*{\alpha}_1,\vb*{\alpha}_2,\cdots,\vb*{\alpha}_n,\vb*{\alpha}_1+\vb*{\alpha}_2+\cdots+\vb*{\alpha}_n)\to\rank(\vb*{\alpha}_1,\vb*{\alpha}_2,\cdots,\vb*{\alpha}_n,\vb*{0})=\rank\vb*{A}=n-1$$
    故得证 $\rank\vb*{A}=\rank(\vb*{A},\vb*{\beta})<n$,因此 $\vb*{Ax}=\vb*{\beta}$ 有无穷多解成立.
\end{proof}
\begin{proof}[{\songti \textbf{证法二}}]
    因为 $\rlap{$\underbrace{\phantom{\vb*{\alpha}_1,\vb*{\alpha}_2,\cdots,\vb*{\alpha}_{n-1}}}_{\text{线性相关}}$} \vb*{\alpha}_1,\overbrace{\vb*{\alpha}_2,\cdots,\vb*{\alpha}_{n-1},\vb*{\alpha}_n}^{\text{线性无关}}$,
    所以 $\vb*{\alpha}_2,\cdots,\vb*{\alpha}_{n-1}$ 线性无关,因此 $\vb*{\alpha}_1$ 可由 $\vb*{\alpha}_2,\cdots,\vb*{\alpha}_{n-1}$ 线性表示,即 $\vb*{\alpha}_1=l_2\vb*{\alpha}_2+l_3\vb*{\alpha}_3+\cdots+l_{n-1}\vb*{\alpha}_{n-1}+0\cdot\vb*{\alpha}_n$,将其代入 $\vb*{\beta}$ 得
    $$\vb*{\beta}=l_2\vb*{\alpha}_2+l_3\vb*{\alpha}_3+\cdots+l_{n-1}\vb*{\alpha}_{n-1}+0\cdot\vb*{\alpha}_n+(\vb*{\alpha}_2+\cdots+\vb*{\alpha}_n)=(l_2+1)\vb*{\alpha}_2+(l_3+1)\vb*{\alpha}_3+\cdots+(l_{n-1}+1)\vb*{\alpha}_{n-1}+\vb*{\alpha}_n$$
    那么 $\vb*{Ax}=\vb*{\beta}$ 的一个解为 $(0,l_2+1,l_3+1,\cdots,l_{n-1}+1,1)^\top$,又因为 $\vb*{\beta}=\vb*{\alpha}_1+\vb*{\alpha}_2+\cdots+\vb*{\alpha}_n$,所以 $\vb*{Ax}=\vb*{\beta}$ 的另一解为 $(1,1,\cdots,1)^\top$,而这两个解不相关,则 $\vb*{Ax}=\vb*{\beta}$ 有无穷多解.
\end{proof}

\subsection{向量组的极大无关组}

\begin{definition}[极大线性无关组]
    设 $\vb*{A}$ 是一个 $n$ 维向量组 (它可以包含无限多个向量),如果在 $\vb*{A}$ 中取出 $r$ 个向量 $\vb*{\alpha}_1,\vb*{\alpha}_2,\cdots,\vb*{\alpha}_r$ 满足条件:
    \begin{enumerate}[label=(\arabic{*})]
        \item 向量组 $\vb*{\alpha}_1,\vb*{\alpha}_2,\cdots,\vb*{\alpha}_r$ 线性无关;
        \item 对于 $\vb*{A}$ 中任意的向量 $\vb*{\beta}$,向量组 $\vb*{\alpha}_1,\vb*{\alpha}_2,\cdots,\vb*{\alpha}_r,\vb*{\beta}$ 线性相关.
    \end{enumerate}
    则称向量组 $\vb*{\alpha}_1,\vb*{\alpha}_2,\cdots,\vb*{\alpha}_r$ 为向量组 $\vb*{A}$ 的一个极大线性无关组,简称极大无关组.
\end{definition}

\begin{example}
    求向量组 $$\vb*{\beta}_1=(1,2,1,3)^\top,~
        \vb*{\beta}_2=(1,1,-1,1)^\top,~
        \vb*{\beta}_3=(1,3,3,5)^\top,~$$
    $$\vb*{\beta}_4=(4,5,-3,6)^\top,~
        \vb*{\beta}_5=(-3,-5,-2,-7)^\top$$
    的秩、极大线性无关组,并将其余的向量用极大线性无关组表示出.
\end{example}
\begin{solution}
    令 $\vb*{B}=\qty(\vb*{\beta}_1,\vb*{\beta}_2,\vb*{\beta}_3,\vb*{\beta}_4,\vb*{\beta}_5)$ 对列向量组作初等行变换,有
    \begin{flalign*}
        \vb*{B}\xrightarrow[r_4-3r_1]{\substack{r_2-2r_1 \\r_3-r_1}}\mqty(1&1&1&4&-3\\0&-1&1&-3&1\\0&-2&2&-7&1\\0&-2&2&-6&2)\xrightarrow[\substack{r_3+2r_2\\r_4+2r_2}]{\substack{r_1+r_2\\r_2\times(-1)\\r_3\times(-1)}}
        \begin{pNiceMatrix}
            \Block[borders={bottom,tikz=dashed}]{1-1}{}        1 & 0                                                 & 2  & 1                                                 & -2 \\
            0                                                    & 1\Block[borders={bottom,left,tikz=dashed}]{1-2}{} & -1 & 3                                                 & -1 \\
            0                                                    & 0                                                 & 0  & 1\Block[borders={bottom,left,tikz=dashed}]{1-2}{} & 1  \\
            0                                                    & 0                                                 & 0  & 0                                                 & 0
        \end{pNiceMatrix}=\mqty(\vb*{\beta}_1',\vb*{\beta}_2',\vb*{\beta}_3',\vb*{\beta}_4',\vb*{\beta}_5')
    \end{flalign*}
    于是 $\rank(\vb*{B})=3$,那么 $\vb*{\beta}_1',\vb*{\beta}_2',\vb*{\beta}_4'$ 为 $\vb*{B}$ 的一个极大无关组,且
    \begin{flalign*}
        \vb*{\beta}_3' & =2\vb*{\beta}_1'-\vb*{\beta}_2'                  \\
        \vb*{\beta}_5' & =-3\vb*{\beta}_1'-4\vb*{\beta}_2'+\vb*{\beta}_4'
    \end{flalign*}
\end{solution}

\begin{example}[2006 数三]
    设 4 维向量组
    \begin{flalign*}
        \vb*{\alpha}_1 & =(1+a,1,1,1)^\top,~\vb*{\alpha}_2=(2,2+a,2,2)^\top,~ \\
        \vb*{\alpha}_3 & =(3,3,3+a,3)^\top,~\vb*{\alpha}_4=(4,4,4,4+a)^\top
    \end{flalign*}
    问 $a$ 为何值时,$\vb*{\alpha}_i~ (i=1,2,3,4)$ 线性相关? 当 $\vb*{\alpha}_i~ (i=1,2,3,4)$ 时,求出一个极大线性无关组,并将其余的向量用极大线性无关组表示出.
\end{example}
\begin{solution}
    记 $\vb*{A}=(\vb*{\alpha}_1,\vb*{\alpha}_2,\vb*{\alpha}_3,\vb*{\alpha}_4)$,那么其行列式
    $$\mqty|1+a&2&3&4\\1&2+a&3&4\\1&2&3+a&4\\1&2&3&4+a| \xlongequal[j=2,3,4]{c_1+c_j}\mqty|10+a &2 &3 &4\\10+a &2+a &3 &4\\10+a &2 &3+a &4\\10+a &2 &3 &4+a|\xlongequal[i=2,3,4]{r_i-r_1}(10+a)\mqty|1&2&3&4\\0&a&0&0\\0&0&a&0\\0&0&0&a|=(10+a)a^3$$
    于是当 $a=0\text{ 或 }-10$ 时,$\vb*{\alpha}_i~ (i=1,2,3,4)$ 线性相关,
    当 $a=0$ 时,$\vb*{\alpha}_1$ 为 $\vb*{\alpha}_i~ (i=1,2,3,4)$ 的一个极大线性无关组,且 $\vb*{\alpha}_2=2\vb*{\alpha}_1,\vb*{\alpha}_3=3\vb*{\alpha}_1,\vb*{\alpha}_4=4\vb*{\alpha}_1$;
    当 $a=-10$ 时,对 $\vb*{A}$ 实施初等行变换,有
    \begin{flalign*}
        \mqty(-9 & 2 & 3 & 4 \\1&-8&3&4\\1&2&-7&4\\1&2&3&-6)\xrightarrow[i=2,3,4]{r_i-r_1}\mqty(-9&2&3&4\\10&-10&0&0\\10&0&-10&0\\10&0&0&-10)\xrightarrow[r_1+2r_2+3r_3+4r_4]{r_i\times\frac{1}{10}~~i=2,3,4}\mqty(0&0&0&0\\1&-1&0&0\\1&0&-1&0\\1&0&0&-1)=(\vb*{\beta}_1,\vb*{\beta}_2,\vb*{\beta}_3,\vb*{\beta}_4)
    \end{flalign*}
    由于 $\vb*{\beta}_i~ (i=2,3,4)$ 是 $\vb*{\beta}_i~ (i=1,2,3,4)$ 的一个极大线性无关组,且 $\vb*{\beta}_1=-\vb*{\beta}_2-\vb*{\beta}_3-\vb*{\beta}_4$,
    故 $\vb*{\alpha}_i~ (i=2,3,4)$ 是 $\vb*{\alpha}_i~ (i=1,2,3,4)$ 的一个极大线性无关组,且 $\vb*{\alpha}_1=-\vb*{\alpha}_2-\vb*{\alpha}_3-\vb*{\alpha}_4$.
\end{solution}

\begin{example}
    已知向量组 $ \vb*{\alpha}_{1}=(1,4,0,2)^\top, \vb*{\alpha}_{2}=(2,7,1,3)^\top ,\vb*{\alpha}_{3}=(0,1,-1, a)^\top, \vb*{\alpha}_{4}=(3,10, b, 4)^\top $ 线性相关.
    \begin{enumerate}[label=(\arabic{*})]
        % \item 求 $ a, b $ 的值;
        \item 判断 $ \vb*{\alpha}_{4} $ 能否由 $ \vb*{\alpha}_{1}, \vb*{\alpha}_{2}, \vb*{\alpha}_{3} $ 线性表示? 如能就写出表达式;
        \item 求向量组 $ \vb*{\alpha}_{1}, \vb*{\alpha}_{2}, \vb*{\alpha}_{3}, \vb*{\alpha}_{4} $ 的一个极大线性无关组.
    \end{enumerate}
\end{example}
\begin{solution}
    \begin{enumerate}[label=(\arabic{*})]
        \item $\mqty|\vb*{\alpha}_1,\vb*{\alpha}_2,\vb*{\alpha}_3,\vb*{\alpha}_4|=\mqty|1&2&0&3\\4&7&1&10\\0&1&-1&b\\2&3&a&4| \xlongequal[\substack{r_2+r_3\\r_4+r_3}]{\substack{r_2-4r_1\\r_4-2r_1}}\mqty|1&2&0&3\\0&0&0&b-2\\0&1&-1&b\\0&0&a-1&b-2|=\mqty|0&0&b-2\\1&-1&b\\0&a-1&b-2|=(a-1)(b-2)=0$,
              所以当 $a=1$ 或 $b=2$ 时,向量组 $\vb*{\alpha}_1,\vb*{\alpha}_2,\vb*{\alpha}_3,\vb*{\alpha}_4$ 线性相关.\\
              当 $a=1$ 时,$\begin{pNiceArray}{c|c}
                      \vb*{\alpha}_1,\vb*{\alpha}_2,\vb*{\alpha}_3&\vb*{\alpha}_4
                  \end{pNiceArray}=\begin{pNiceArray}{ccc|c}
                      1&2&0&3\\
                      4&7&1&10\\
                      0&1&-1&b\\
                      2&3&1&4
                  \end{pNiceArray}\xlongequal[\substack{r_4-r_2\\r_3+r_2\\r_2\times (-1)}]{\substack{r_2-4r_1\\r_4-2r_1}}
                  \begin{pNiceArray}{cccc}
                      1\Block[borders={bottom,tikz=dashed}]{1-1}{}&2&0&3\\
                      0&1\Block[borders={bottom,left,tikz=dashed}]{1-2}{}&-1&2\\
                      0&0&0&b-2\Block[borders={bottom,left,tikz=dashed}]{1-1}{}\\
                      0&0&0&0
                  \end{pNiceArray}$,当 $b\neq2$ 时,增广矩阵秩为 3,系数矩阵秩为 2,因此 $\vb*{\alpha}_4$ 不能由 $\vb*{\alpha}_1,\vb*{\alpha}_2,\vb*{\alpha}_3$ 线性表示;
              当 $a=1~,b=2$ 时,$\vb*{\alpha}_4$ 可以由$\vb*{\alpha}_1,\vb*{\alpha}_2,\vb*{\alpha}_3$ 线性表示,$$\vb*{\alpha}_4=(-1-2t)\vb*{\alpha}_1+(2+t)\vb*{\alpha}_2+t\vb*{\alpha}_3$$
              当 $b=2$ 时,$\begin{pNiceArray}{c|c}
                      \vb*{\alpha}_1,\vb*{\alpha}_2,\vb*{\alpha}_3&\vb*{\alpha}_4
                  \end{pNiceArray}=\begin{pNiceArray}{ccc|c}
                      1&2&0&3\\
                      4&7&1&10\\
                      0&1&-1&2\\
                      2&3&a&4
                  \end{pNiceArray}\xlongequal[\substack{r_3+r_2\\r_4-r_2\\r_2\times(-1)\\r_3\leftrightarrow r_4}]{\substack{r_2-4r_1\\r_4-2r_1\\r_1+2r_2}}
                  \begin{pNiceArray}{cccc}
                      1\Block[borders={bottom,tikz=dashed}]{1-1}{}&  2&  0&3\\
                      0&  1\Block[borders={bottom,left,tikz=dashed}]{1-1}{}&  -1&2\\
                      0&  0&  a-1\Block[borders={bottom,left,tikz=dashed}]{1-2}{}&0\\
                      0&  0&  0&0
                  \end{pNiceArray}$,当 $a\neq1,~b=2$ 时,有 $\vb*{\alpha}_4=-\vb*{\alpha}_1+2\vb*{\alpha_2}$,当 $a=1,~b=2$,有
              $\vb*{\alpha}_4=(-1-2t)\vb*{\alpha}_1+(2+t)\vb*{\alpha}_2+t\vb*{\alpha}_3$,其中 $t$ 为任意常数.
        \item 当 $a=1,~b=2$ 时,$\rank(\vb*{\alpha}_1,\vb*{\alpha}_2,\vb*{\alpha}_3,\vb*{\alpha}_4)=\rank\mqty(1&2&0&3\\4&7&1&10\\0&1&-1&2\\2&3&1&4)=\rank\mqty(1&2&0&3\\0&-1&1&-2\\0&0&0&0\\0&0&0&0)=2$,
              极大无关组为 $\vb*{\alpha}_1,\vb*{\alpha}_2$;
              当 $a=1,~b\neq 2$ 时,$\rank(\vb*{\alpha}_1,\vb*{\alpha}_2,\vb*{\alpha}_3,\vb*{\alpha}_4)=3$,极大无关组为 $\vb*{\alpha}_1,\vb*{\alpha}_2,\vb*{\alpha}_4$;
              当 $a\neq 1,~b=2$ 时,$\rank(\vb*{\alpha}_1,\vb*{\alpha}_2,\vb*{\alpha}_3,\vb*{\alpha}_4)=3$,极大无关组为 $\vb*{\alpha}_1,\vb*{\alpha}_2,\vb*{\alpha}_3$.
    \end{enumerate}
\end{solution}

\subsection{Cramer 法则的应用}

\begin{theorem}[Cramer 法则]
    如果数域 $ K $ 上的含有 $ n $ 个末知量 $ n $ 个方程的线性方程组
    $$\begin{cases}
            a_{11} x_{1}+a_{12} x_{2}+\cdots+a_{1 n} x_{n}=b_{1} \\
            a_{21} x_{1}+a_{22} x_{2}+\cdots+a_{2 n} x_{n}=b_{2} \\
            \cdots \cdots \cdots \cdots                          \\
            a_{n 1} x_{1}+a_{n 2} x_{2}+\cdots+a_{n n} x_{n}=b_{n}
        \end{cases}$$
    的系数行列式 $ D \neq 0$,那么此方程组有唯一解 $ \left(\dfrac{D_{1}}{D}, \dfrac{D_{2}}{D}, \cdots, \dfrac{D_{n}}{D}\right)$,
    其中 $ D_{j} $ 是把 $ D $ 中第 $ j $ 列的元素换成方程组的常数项 $ b_{1}, b_{2}, \cdots, b_{n} $ 而成的行列式 $ (j=1,2, \cdots, n)$,即
    $$\mqty|a_{11}  & \cdots & a_{1, j-1} & b_{1}  & a_{1, j+1} & \cdots & a_{1 n} \\
        a_{21}  & \cdots & a_{2, j-1} & b_{2}  & a_{2, j+1} & \cdots & a_{2 n} \\
        \vdots  &        & \vdots     & \vdots & \vdots     &        & \vdots  \\
        a_{n 1} & \cdots & a_{n, j-1} & b_{n}  & a_{n, j+1} & \cdots & a_{n n}|$$
    特别地,当 $ b_{i}=0(i=1,2, \cdots, n) $ 时,如果系数行列式 $ D \neq 0 $,那么方程组只有零解;
    反之,如果方程组有非零解,那么必有 $ D=0 .$
\end{theorem}

\begin{example}[2005 华中科技大学]
    解线性方程组 $\begin{cases}
            x_1+ax_2+a^2x_3=a^3 \\
            x_1+bx_2+b^2x_3=b^3 \\
            x_1+cx_2+c^2x_3=c^3 \\
        \end{cases}$
    其中 $a,b,c$ 是互不相等的常数.
\end{example}
\begin{solution}
    利用 Cramer 法则求解,因为系数行列式 $$D=\mqty|1&a&a^2\\1&b&b^2\\1&c&c^2|=(c-b)(c-a)(b-a)\neq0$$
    所以方程组有唯一解,又因为
    $$D_1=\mqty|a^3&a&a^2\\b^3&b&b^2\\b^3&b&b^2|=abcD,~D_2=\mqty|1&a^3&a^2\\1&b^3&b^2\\1&c^3&c^2|=-(ab+ac+bc)D,~D_3=\mqty|1&a&a^3\\1&b&b^3\\1&c&c^3|=(a+b+c)D$$
    因此 $x_1=\dfrac{D_1}{D}=abc,~x_2=\dfrac{D_2}{D}=-ab-ac-bc,~x_3=\dfrac{D_3}{D}=a+b+c.$
\end{solution}
