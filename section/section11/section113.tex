\section{齐次线性方程组}

\subsection{齐次方程组的一般解}

\begin{theorem}
    齐次方程组 $\vb*{Ax}=\vb*{0}$ 有非零解的充分必要条件为 $\rank\vb*{A}<n$,此时它的通解为
    $$\vb*{x}=k_1\vb*{\xi}_1+k_2\vb*{\xi}_2+\cdots+k_{n-r}\vb*{\xi}_{n-r}$$
    其中 $r=\rank\vb*{A}$,$k_i\in K\text{ 为任意常数 }(i=1,2,\cdots,n-1)$,$\vb*{\xi}_1,\vb*{\xi}_2,\cdots,\vb*{\xi}_{n-r}$ 为方程组的一个基础解系.
\end{theorem}

\begin{example}
    讨论当 $a,~b$ 取何值时,线性方程组 $$\left\{\begin{matrix}
            x_1  & + & x_2  & + & x_3  & + & x_4  & + & x_5  & =1 \\
            3x_1 & + & 2x_2 & + & x_3  & + & x_4  & - & 3x_5 & =a \\
                 &   & x_2  & + & 2x_3 & + & 2x_4 & + & 6x_5 & =3 \\
            5x_1 & + & 4x_2 & + & 3x_3 & + & 3x_4 & - & x_5  & =b
        \end{matrix}\right.$$有解? 无解? 并求出有解时的一般解.
\end{example}
\begin{solution}
    对方程组的增广矩阵施行初等行变换,化为简化的行阶梯形矩阵:
    \begin{flalign*}
        \begin{pNiceArray}{ccccc:c}
            1 & 1 & 1 & 1 & 1  & 1 \\
            3 & 2 & 1 & 1 & -3 & a \\
            0 & 1 & 2 & 2 & 6  & 3 \\
            5 & 4 & 3 & 3 & -1 & b \\
        \end{pNiceArray}
        \xrightarrow[\substack{r_1-r_2 \\r_{3,4}+r_2}]{\substack{r_2 \leftrightarrow r_3 \\r_3-3r_1\\r_4-5r_1}}
        \begin{pNiceArray}{ccccc:c}
            1 & 0 & -1 & -1 & -5 & -2  \\
            0 & 1 & 2  & 2  & 6  & 3   \\
            0 & 0 & 0  & 0  & 0  & a   \\
            0 & 0 & 0  & 0  & 0  & b-2 \\
        \end{pNiceArray}
    \end{flalign*}
    \begin{enumerate}[label=(\arabic{*})]
        \item 当 $a\neq0$ 或 $b\neq 2$ 时,方程组无解;
        \item 当 $a=0$ 且 $b=2$ 时,方程组有解,此时,易知方程组的一般解为
              $$\vb*{x}=\begin{pmatrix}
                      -2 \\
                      3  \\
                      0  \\
                      0  \\
                      0
                  \end{pmatrix}+k_1\begin{pmatrix}
                      1  \\
                      -2 \\
                      1  \\
                      0  \\
                      0
                  \end{pmatrix}+k_2\begin{pmatrix}
                      1  \\
                      -2 \\
                      0  \\
                      1  \\
                      0
                  \end{pmatrix}+k_3\begin{pmatrix}
                      5  \\
                      -6 \\
                      0  \\
                      0  \\
                      1
                  \end{pmatrix}$$
              其中 $k_1,~k_2,~k_3$ 为任意常数.
    \end{enumerate}
\end{solution}

\begin{example}
    求 $\lambda$ 的值,使齐次线性方程组
    $$\left\{\begin{matrix}
            (\lambda +3)x_1  & + & x_2             & + & 2x_3            & = & 0 \\
            \lambda x_1      & + & (\lambda -1)x_2 & + & x_3             & = & 0 \\
            3(\lambda +1)x_1 & + & \lambda x_2     & + & (\lambda +3)x_3 & = & 0
        \end{matrix}\right.$$ 有非零解,并求出其一般解.
\end{example}
\begin{solution}
    方程组的系数行列式为 $$D=\mqty|\lambda +3&1&2\\\lambda&\lambda-1&1\\3\lambda+3&\lambda&\lambda +3|=\lambda^2(\lambda-1)$$
    所以 $\lambda=0$ 或 $\lambda=1$ 时齐次方程组有非零解,进一步,当 $\lambda=0$ 时,对方程组的系数矩阵施行初等行变换,得
    $\mqty(1&0&1\\0&1&-1\\0&0&0)$,此时,方程的一般解为 $\vb*{x}=k(-1,1,1)$; 当 $\lambda=1$ 时,系数矩阵可化为
    $\mqty(1&0&1\\0&1&-2\\0&0&0)$,于是,可求得方程的一般解为 $\vb*{x}=k(-1,2,1)$,其中 $k$ 为任意常数.
\end{solution}

\begin{example}
    设向量 $\vb*{\alpha}=(1,2,1)^\top,~\vb*{\beta}=\qty(1,\dfrac{1}{2},0)^\top,~\vb*{\gamma}=(0,0,8)^\top$,记
    $$\vb*{A}=\vb*{\alpha\beta}^\top,~\vb*{B}=\vb*{\beta}^\top\vb*{\alpha}$$
    求线性方程组 $2\vb*{B}^2\vb*{A}^2\vb*{x}=\vb*{A}^4\vb*{x}+\vb*{B}^4\vb*{x}+\vb*{\gamma}.$
\end{example}
\begin{solution}
    由题意可知,$$\vb*{A}=\mqty(1\\2\\1)\qty(1,\dfrac{1}{2},0)=\mqty(1&\dfrac{1}{2}&0\\[6pt]2&1&0\\1&\dfrac{1}{2}&0),~\vb*{B}=\qty(1,\dfrac{1}{2},0)\mqty(1\\2\\1)=2$$
    于是
    \begin{flalign*}
        \vb*{A}^2 & =\qty(\vb*{\alpha\beta}^\top)\qty(\vb*{\alpha\beta}^\top)=\vb*{\alpha}\qty(\vb*{\beta}^\top\vb*{\alpha})\vb*{\beta}^\top=\qty(\vb*{\beta}^\top\vb*{\alpha})\vb*{\alpha\beta}^\top=2\vb*{A} \\
        \vb*{A}^4 & =\qty(\vb*{A}^2)^2=(2\vb*{A})^2=4\vb*{A}^2=8\vb*{A}
    \end{flalign*}
    代入方程组,得 $$16\vb*{Ax}=8\vb*{Ax}+16\vb*{x}+\vb*{\gamma}\Rightarrow (\vb*{A}-2\vb*{E}_3)\vb*{x}=\dfrac{1}{8}\vb*{\gamma}\Rightarrow \mqty(-1&\dfrac{1}{2}&0\\[6pt]2&-1&0\\1&\dfrac{1}{2}&-2)\vb*{x}=(0,0,1)^\top$$
    则有增广矩阵 $\vb*{M}=\mqty(-1&\dfrac{1}{2}&0&0\\[6pt]2&-1&0&0\\1&\dfrac{1}{2}&-2&1)$,并对 $\vb*{M}$ 实行初等行变换,有
    $$\mqty(-1&\dfrac{1}{2}&0&0\\[6pt]2&-1&0&0\\1&\dfrac{1}{2}&-2&1)\xrightarrow[\substack{r_2-2r_1\\r_3-r_1}]{r_1\times(-1)}\mqty(1&-\dfrac{1}{2}&0&0\\[6pt]0&0&0&0\\0&1&-2&1)$$
    所以线性方程组的通解为 $$(x_1,x_2,x_3)^\top=k(1,2,1)^\top+\qty(\dfrac{1}{2},1,0)^\top~  k\in\mathbb{R} .$$
\end{solution}

\subsection{齐次方程组的基础解系}

\begin{example}[2005 南开大学]
    设齐次线性方程组 $$\left\{\begin{matrix}
                 &   & x_2  & + & ax_3 & + & bx_4 & = & 0 \\
            -x_1 &   &      & + & cx_3 & + & dx_4 & = & 0 \\
            ax_1 & + & cx_2 &   &      & - & ex_4 & = & 0 \\
            bx_1 & + & dx_2 &   &      & - & ex_3 & = & 0
        \end{matrix}\right.$$
    的一般解以 $x_3,~x_4$ 为未知量,
    \begin{enumerate}[label=(\arabic{*})]
        \item 求 $a,b,c,d,e$ 满足的条件;
        \item 求齐次线性方程组的基础解系.
    \end{enumerate}
\end{example}
\begin{solution}
    \begin{enumerate}[label=(\arabic{*})]
        \item 对方程组的系数矩阵施行初等行变换,得 $$\mqty(0&1&a&b\\-1&0&c&d\\a&c&0&-e\\b&d&-e&0)\to\mqty(1&0&-c&-d\\0&1&a&b\\0&0&bc-e-ad&0\\0&0&0&ad-e-bc)$$
              显然,欲使方程组的一般解以 $x_3,~x_4$ 为自由未知量,必需 $\begin{cases}
                      bc-ad-e=0 \\
                      ad-bc-e=0
                  \end{cases}$,即 $\begin{cases}
                      ad=bc \\e=0
                  \end{cases}$.
        \item 当 $a,,b,c,d,e$ 满足 (1) 的条件时,根据上述初等变换结果可得方程组的基础解系为
              $$\vb*{\xi}_1=(c,-a,1,0)^\top,~\vb*{\xi}_2=(d,-b,0,1)^\top.$$
    \end{enumerate}
\end{solution}

\begin{example}[2005 西安电子科技大学]
    设四元齐次线性方程组为
    \begin{equation*}
        \left\{\begin{matrix}
            2x_1 & + & 3x_2 & - & x_3 &   &     & = & 0 \\
            x_1  & + & 2x_2 & + & x_3 & - & x_4 & = & 0
        \end{matrix}\right.
        \tag{I}
    \end{equation*}
    已知另一四元齐次线性方程组 (II) 的基础解系为 $$\vb*{\alpha}_1=(2,-1,a+2,1)^\top,~\vb*{\alpha}_2=(-1,2,4,a+8)^\top$$
    \begin{enumerate}[label=(\arabic{*})]
        \item 求方程组 (I) 的基础解系;
        \item 问当 $a$ 为何值时,方程组 (I) 与 (II) 有非零公共解? 在有非零公共解时,求出全部非零公共解.
    \end{enumerate}
\end{example}
\begin{solution}
    \begin{enumerate}[label=(\arabic{*})]
        \item 方程组 (I) 的系数矩阵为 $\mqty(2&3&-1&0\\1&2&1&-1)$,可用初等行变换化为 $\mqty(1&0&-5&3\\0&1&3&-2)$,故 (I) 的基础解系为
              $$\vb*{\beta}_1=(5,-3,1,0)^\top,~\vb*{\beta}_2=(-3,2,0,1)^\top.$$
        \item 显然,方程组 (I) 和方程组 (II) 有非零公共解等价于 $$k_1\vb*{\beta}_1+k_2\vb*{\beta}_2=k_3\vb*{\alpha}_1+k_4\vb*{\alpha}_2$$
              其中 $k_1$ 与 $k_2$ 不同时为 0,这归结于关于 $k_i,~i=1,2,3,4$ 的齐次线性方程组
              \begin{equation*}
                  k_1\vb*{\beta}_1+k_2\vb*{\beta}_2-k_3\vb*{\alpha}_1-k_4\vb*{\alpha}_2=\vb*{0}
                  \tag{III}
              \end{equation*}
              有非零解,且 $k_1,~k_2$ 不同时为零,对 (III) 的系数矩阵施行初等行变换,得
              $$\vb*{A}=\mqty(5&-3&-2&1\\-3&2&1&-2\\1&0&-a-2&-4\\0&1&-1&-a-8)\to\mqty(1&0&-a-2&-4\\0&1&-1&-a-8\\0&0&a+1&-a-1\\0&0&0&a+1)$$
              可见,当 $a\neq-1$ 时,$\rank\vb*{A}=4$,方程组 (III) 仅有零解; 当 $a=-1$ 时,$\rank\vb*{A}=2$ 方程组 (III) 有非零解,其通解为
              $$\vb*{k}=c_1(1,1,1,0)+c_2(4,7,0,1)$$
              注意到 $\begin{cases}
                      k_1=c_1+4c_2 \\k_2=c_1+7c_2
                  \end{cases}$ 所以 $k_1,~k_2$ 同时为零当且仅当 $c_1,~c_2$ 同时为零,
              综上所述,当且仅当 $a=-1$ 时,方程组 (I) 和方程组 (II) 有非零公共解,且全部非零公共解为 $k_1+\vb*{\beta}_1+k_2\vb*{\beta}_2$,
              其中 $k_1,~k_2$ 取不同时为零的任意常数.
    \end{enumerate}
\end{solution}
