\section{向量组的秩}

\subsection{向量组的线性表示}

\begin{definition}[向量组的秩]
    向量组 $ \vb*{\alpha}_{1}, \vb*{\alpha}_{2}, \cdots, \vb*{\alpha}_{s} $ 的极大线性无关组所含向量 个数称为向量组的秩,记为 
    $$ \rank\left(\vb*{\alpha}_{1}, \vb*{\alpha}_{2}, \cdots, \vb*{\alpha}_{k}\right) .$$
\end{definition}

\begin{theorem}
    若向量组 $\vb*{\alpha}_i~ (i=1,2,\cdots,s)$ 可以由向量组 $\vb*{\beta}_j~ (j=1,2,\cdots,t)$ 表示,则
    $$\rank(\vb*{\alpha}_1,\vb*{\alpha}_2,\cdots,\vb*{\alpha}_s)\leqslant \rank(\vb*{\beta}_1,\vb*{\beta}_2,\cdots,\vb*{\beta}_t).$$
\end{theorem}
\begin{proof}[{\songti \textbf{证}}]
    由于 $\vb*{\alpha}_i$ 可由 $\vb*{\beta}_j$ 线性表示,不妨设 $$\vb*{\alpha}_i=\sum_{k=1}^{t}x_{ik}\vb*{\beta}_k$$
    故 $$(\vb*{\alpha}_1,\vb*{\alpha}_2,\cdots,\vb*{\alpha}_s)=(\vb*{\beta}_1,\vb*{\beta}_2,\cdots,\vb*{\beta}_t)\mqty(x_{11}&x_{12}&\cdots&x_{1t}\\x_{21}&x_{22}&\cdots&x_{2t}\\\vdots &\vdots&&\vdots\\x_{s1}&x_{s2}&\cdots&x_{st})$$
    由于矩阵乘积的秩小于等于每一个,则
    $$\rank(\vb*{\alpha}_1,\vb*{\alpha}_2,\cdots,\vb*{\alpha}_s)\leqslant \rank\min\qty{(\vb*{\beta}_1,\vb*{\beta}_2,\cdots,\vb*{\beta}_t),\mqty(x_{11}&x_{12}&\cdots&x_{1t}\\x_{21}&x_{22}&\cdots&x_{2t}\\\vdots &\vdots&&\vdots\\x_{s1}&x_{s2}&\cdots&x_{st})}$$
    即 $\rank(\vb*{\alpha}_1,\vb*{\alpha}_2,\cdots,\vb*{\alpha}_s)\leqslant \rank(\vb*{\beta}_1,\vb*{\beta}_2,\cdots,\vb*{\beta}_t).$
\end{proof}

\begin{example}
    设 $n$ 维向量组 $(I):\vb*{\alpha}_1,\vb*{\alpha}_2,\cdots,\vb*{\alpha}_k~ (k<n)$ 线性无关,则 $n$ 维向量组 $(II):\vb*{\beta}_1,\vb*{\beta}_2,\cdots,\vb*{\beta}_k$ 也线性无关的充要条件为
    \begin{tasks}
        \task $\vb*{\beta}_1,\vb*{\beta}_2,\cdots,\vb*{\beta}_k$ 可由 $\vb*{\alpha}_1,\vb*{\alpha}_2,\cdots,\vb*{\alpha}_k$ 线性表示
        \task $\vb*{\alpha}_1,\vb*{\alpha}_2,\cdots,\vb*{\alpha}_k$ 可由 $\vb*{\beta}_1,\vb*{\beta}_2,\cdots,\vb*{\beta}_k$ 线性表示
        \task 向量组 $(I)$ 与向量组 $(II)$ 等价
        \task 矩阵 $(\vb*{\alpha}_1,\vb*{\alpha}_2,\cdots,\vb*{\alpha}_k)$ 与矩阵 $(\vb*{\beta}_1,\vb*{\beta}_2,\cdots,\vb*{\beta}_k)$ 等价
    \end{tasks}
\end{example}
\begin{solution}
    对于 A 选项,因为 $\vb*{\beta}_1,\vb*{\beta}_2,\cdots,\vb*{\beta}_k$ 可由 $\vb*{\alpha}_1,\vb*{\alpha}_2,\cdots,\vb*{\alpha}_k$ 线性表示,所以 $\rank(II)\leqslant \rank(I)$,而向量组 $(I)$ 线性无关,
    所以 $\rank(I)=k$,即 $\rank(II)\leqslant k$,由于无法保证 $\rank(II)=k$,故不能得出 $(II)$ 线性无关;\\
    对于 B 选项,因为 $\vb*{\alpha}_1,\vb*{\alpha}_2,\cdots,\vb*{\alpha}_k$ 可由 $\vb*{\beta}_1,\vb*{\beta}_2,\cdots,\vb*{\beta}_k$ 线性表示,所以 $\rank(I)\leqslant \rank(II)$,同样地,
    $$k=\rank(I)\leqslant\rank(II)\leqslant k\Rightarrow \rank(II)=k$$
    因此 $(II)$ 线性无关,充分条件成立,但不是必要条件,如 $(I):\mqty(1\\0)$ 与 $(II):\mqty(0\\1)$ 均线性无关,但 $(I)$ 不能由 $(II)$ 线性表示;\\
    对于 C 选项,由 $(I)$ 与 $(II)$ 等价知 $(I)$ 与 $(II)$ 可相互线性表示,则 $\rank(I)=\rank(II)=k$,故 $(II)$ 线性无关,充分条件成立,同 B 选项知必须条件不成立;\\
    因为矩阵 $\vb*{A}=(\vb*{\alpha}_1,\vb*{\alpha}_2,\cdots,\vb*{\alpha}_k)$ 与矩阵 $\vb*{B}=(\vb*{\beta}_1,\vb*{\beta}_2,\cdots,\vb*{\beta}_k)$ 等价,则 $\rank\vb*{A}=\rank\vb*{B}=k$,
    又 $(I):\vb*{\alpha}_1,\vb*{\alpha}_2,\cdots,\vb*{\alpha}_k~ (k<n)$ 线性无关,即 $\rank\vb*{A}=\rank\vb*{B}=k$,故 $\vb*{\beta}_1,\vb*{\beta}_2,\cdots,\vb*{\beta}_k$ 线性无关,故选 D.
\end{solution}

\begin{example}
    已知 $n$ 维向量组 $$(I):\vb*{\alpha}_1,~\vb*{\alpha}_2,\cdots,\vb*{\alpha}_s,~(II): \vb*{\beta}_1,\vb*{\beta}_2,~\cdots,\vb*{\beta}_t$$
    且 $\rank(\vb*{\alpha}_1,\vb*{\alpha}_2,\cdots,\vb*{\alpha}_s)=\rank(\vb*{\beta}_1,\vb*{\beta}_2,\cdots,\vb*{\beta}_t)=r$,则 
    \begin{tasks}
        \task 当 $s=r$ 时,向量组 $(I)$ 与 $(II)$ 等价
        \task 当 $s=t=r$ 时,向量组 $(I)$ 与 $(II)$ 等价
        \task 当 $\rank(\vb*{\alpha}_1,\cdots,\vb*{\alpha}_s,\vb*{\beta}_1,\cdots,\vb*{\beta}_t)=r$ 时,向量组 $(I)$ 与 $(II)$ 等价
        \task 当 $\rank(\vb*{\alpha}_1,\cdots,\vb*{\alpha}_s,\vb*{\beta}_1,\cdots,\vb*{\beta}_t)=2r$ 时,向量组 $(I)$ 与 $(II)$ 等价
    \end{tasks}
\end{example}
\begin{solution}
    记向量组 $(III):\vb*{\alpha}_1,\cdots,\vb*{\alpha}_s,\vb*{\beta}_1,\cdots,\vb*{\beta}_t$,取 ${I}$ 和 ${II}$ 的一个极大无关组 $(I_0): \vb*{\alpha}_{i_1},\vb*{\alpha}_{i_2},\cdots,\vb*{\alpha}_{i_r}$ 与 $(II_0):\vb*{\beta}_{j_1},\vb*{\beta}_{j_2},\cdots,\vb*{\beta}_{j_r}$,
    那么向量组 $(I)$ 与 $(I_0)$ 等价,向量组 $(II)$ 与 $(II_0)$ 的等价,如果 $\rank(III)=r$,那么向量组 $(I_0)$ 与 $(II_0)$ 都是 $(III)$ 的一个极大无关组,这表明向量组 $(I_0)$ 与 $(II_0)$ 等价,因此,向量组 $(I)$ 与 $(II)$ 等价.\\
    另外,考虑向量组 $$(I):\vb*{\alpha}_1=(1,0,0,0)^\top,~\vb*{\alpha}_2=(0,1,0,0)^\top,~(II):\vb*{\beta}_1=(0,0,1,0)^\top,~\vb*{\beta}_2=(0,0,0,1)^\top$$
    则 $\rank(\vb*{\alpha}_1,\vb*{\alpha}_2)=\rank(\vb*{\beta}_1,\vb*{\beta}_2)=2,~\rank(\vb*{\alpha}_1,\vb*{\alpha}_2,\vb*{\beta}_1,\vb*{\beta}_2)=4$,但选项 A、B、D 均不成立.
\end{solution}

\subsection{向量组相互表示}

\begin{example}[2000 数二]
    已知向量组 $\vb*{\beta}_1=\mqty(0\\1\\-1),~\vb*{\beta}_2=\mqty(a\\2\\1),~\vb*{\beta}_3=\mqty(b1\\1\\0)$ 与向量组 $\vb*{\alpha}_1=\mqty(1\\2\\-3),\\~\vb*{\alpha}_2=\mqty(3\\0\\1),~\vb*{\alpha}_3=\mqty(9\\6\\-7)$ 具有相同的秩,且 $\vb*{\beta}_3$ 可由 $\vb*{\alpha}_1,\vb*{\alpha}_2,\vb*{\alpha}_3$ 线性表示,求 $a,b$ 的值.
\end{example}
\begin{solution}
    易知 $\vb*{\alpha}_1,~\vb*{\alpha}_2$ 线性无关,$\vb*{\alpha}_3=3\vb*{\alpha}_1+2\vb*{\alpha}_2$,即 $\vb*{\alpha}_3$ 可由 $\vb*{\alpha}_1,~\vb*{\alpha}_2$ 线性表示,所以向量组 $\vb*{\alpha}_1,\vb*{\alpha}_2,\vb*{\alpha}_3$ 的秩为 2,且 $\vb*{\alpha}_1,~\vb*{\alpha}_2$ 是它的一个极大无关组,
    由于向量组 $\vb*{\beta}_1,~\vb*{\beta}_2,~\vb*{\beta}_3$ 与 $\vb*{\alpha}_1,~\vb*{\alpha}_2,~\vb*{\alpha}_3$ 具有相同的秩,故 $\vb*{\beta}_1,~\vb*{\beta}_2,~\vb*{\beta}_3$ 线性相关,从而行列式 
    $$|\vb*{\beta}_1,\vb*{\beta}_2,\vb*{\beta}_3|=\mqty|0&a&b\\1&2&1\\-1&1&0|=0\Rightarrow a=3b$$
    又 $\vb*{\beta}_3$ 可由 $\vb*{\alpha}_1,~\vb*{\alpha}_2,~\vb*{\alpha}_3$ 线性表示,而 $\vb*{\alpha}_3$ 可由 $\vb*{\alpha}_1,~\vb*{\alpha}_2$ 线性表示,所以 $\vb*{\alpha}_1,~\vb*{\alpha}_2,~\vb*{\beta}_3$ 线性相关,于是 
    $$|\vb*{\alpha}_1,\vb*{\alpha}_2,\vb*{\beta}_3|=\mqty|1&3&b\\2&0&1\\-3&1&0|=0\Rightarrow b=5\Rightarrow a=15.$$
\end{solution}

\begin{example}[2016 南京航空航天大学]
    设由向量组 $$(I):\vb*{\alpha}_1=\mqty(1\\1\\a),~\vb*{\alpha}_2=\mqty(-2\\a\\4),~\vb*{\alpha}_3=\mqty(-2\\a\\a),~(II):\vb*{\beta}_1=\mqty(1\\1\\a),~\vb*{\beta}_2=\mqty(1\\a\\1),~\vb*{\beta}_3=\mqty(a\\1\\1)$$
    \begin{enumerate}[label=(\arabic{*})]
        \item 求 $a$ 的值,使得向量组 $(I)$ 线性相关;
        \item 求 $a$ 的值,使得向量组 $(I)$ 不能由向量组 $(II)$ 线性表示;
        \item 在题 (1) 和 (2) 同时成立的条件下,将向量 $\vb*{\gamma}=(1,-2,-5)^\top$ 用 $\vb*{\beta}_1,~\vb*{\beta}_2,\vb*{\alpha}_3$ 线性表示.
    \end{enumerate}
\end{example}
\begin{solution}
    \begin{enumerate}[label=(\arabic{*})]
        \item 令 $\vb*{A}=(\vb*{\alpha}_1,\vb*{\alpha}_2,\vb*{\alpha}_3)$,则行列式 $\det\vb*{A}=0$ 等价于向量组 $(I)$ 线性相关,则
              $$\mqty|1&-2&-2\\1&a&a\\a&4&a|=a^2-2a-8=0\Rightarrow a=4\text{ 或 }-2.$$
        \item 令 $\vb*{B}=(\vb*{\beta}_1,\vb*{\beta}_2,\vb*{\beta}_3)$,由于向量组 $(I)$ 不能由 $(II)$ 线性表示,说明向量组 $(II)$ 线性相关,如若不然,则任一 $\vb*{\alpha}_i$,可由向量组 $(II)$ 线性表示,矛盾,
              因此,$\det\vb*{B}=0$,则
              $$\mqty|1&1&a\\1&a&1\\a&1&1|=(a-1)^2(a+2)=0\Rightarrow a=1\text{ 或 }a=-2$$
              当 $a=1$ 时,$\vb*{\alpha}_3$ 显然不能由 $\vb*{\beta}_j~~(j=1,2,3)$ 线性表示,符合题意; 当 $a=-2$ 时,将 $\begin{pNiceArray}{c:c}
                      \vb*{B} & \vb*{A}
                  \end{pNiceArray}$ 化简为行阶梯形,有
              $$\begin{pNiceArray}{c:c}\vb*{B} & \vb*{A}\end{pNiceArray}=\begin{pNiceArray}{ccc:ccc}
                      1&1&-2&1&-2&-2\\1&-2&1&1&-2&-2\\-2&1&1&-2&4&-2
                  \end{pNiceArray}\xrightarrow[\substack{r_3\times(-\frac{1}{6})\\r_2\times(-\frac{1}{3})\\r_1-r_2}]{\substack{r_2-r_1\\r_3+2r_1\\r_3+r_2}}\begin{pNiceArray}{ccc:ccc}
                      1\Block[borders={bottom,tikz=dashed}]{1-1}{}&0&-1&1&-2&-2\\
                      0&1\Block[borders={left,bottom,tikz=dashed}]{1-4}{}&-1&0&0&0\\
                      0&0&0&0&0&1\Block[borders={left,bottom,tikz=dashed}]{1-1}{}
                  \end{pNiceArray}$$
              由此可见,$\vb*{\alpha}_3$ 不能由 $\vb*{\beta}_i~~(i=1,2,3)$ 线性表示,所以 $a=-2$ 也符合题意.
        \item 若 (1) 和 (2) 同时成立,则 $a=-2$,此时,把 $(\vb*{\beta}_1,\vb*{\beta}_2,\vb*{\alpha}_3,\vb*{\gamma})$ 化为行最简形,有
              $$\begin{pNiceArray}{c:c}\vb*{\beta}_1,\vb*{\beta}_2,\vb*{\alpha}_3 & \vb*{\gamma}\end{pNiceArray}=\begin{pNiceArray}{ccc:c}
                      1&1&-2&1\\
                      1&-2&-2&-2\\
                      -2&1&-2&-5
                  \end{pNiceArray}\to\begin{pNiceArray}{ccc:c}
                      1&0&0&2\\
                      0&1&0&1\\
                      0&0&1&1
                  \end{pNiceArray}$$
              因为初等行变换不改变列向量之间的线性关系,所以 $\vb*{\gamma}=2\vb*{\beta}_1+\vb*{\beta}_2+\vb*{\alpha}_3.$
    \end{enumerate}
\end{solution}
