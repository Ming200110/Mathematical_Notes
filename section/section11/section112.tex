\section{向量的关系}

\subsection{向量之间的线性关系}

\begin{definition}[线性表示]
    对于向量 $\vb*{\beta},\vb*{\alpha}_1,\vb*{\alpha}_2,\cdots,\vb*{\alpha}_m$, 如果存在一组数 $k_1,k_2,\cdots, k_m$, 使得
    $$\vb*{\beta}=k_{1} \vb*{\alpha}_{1}+k_{2} \vb*{\alpha}_{2}+\cdots+k_{m} \vb*{\alpha}_{m}$$
    成立, 则称 $ \vb*{\beta} $ 是 $ \vb*{\alpha}_{1}, \vb*{\alpha}_{2}, \cdots, \vb*{\alpha}_{m} $ 的\textit{线性组合}, 或称 $ \vb*{\beta} $ 可由 $ \vb*{\alpha}_{1}, \vb*{\alpha}_{2}, \cdots \vb*{\alpha}_{m} $ \textit{线性表示}.
\end{definition}

\begin{definition}[线性相关与线性无关]
    设 $ \vb*{\alpha}_{1}, \vb*{\alpha}_{2}, \cdots, \vb*{\alpha}_{m} $ 为一组向量, 如果存在一组不全为零的数 $ k_{1}$, $k_{2}, \cdots, k_{m} $, 使得
    $$k_{1} \vb*{\alpha}_{1}+k_{2} \vb*{\alpha}_{2}+\cdots+k_{m} \vb*{\alpha}_{m}=\mathbf{0}$$
    成立, 则称向量组 $ \vb*{\alpha}_{1}, \vb*{\alpha}_{2}, \cdots, \vb*{\alpha}_{m} $ \textit{线性相关}; 当且仅当  $k_{1}=k_{2}=\cdots=k_{m}=0 $ 时等式成立, 则称向量组 $ \vb*{\alpha}, \vb*{\alpha}_{2}, \cdots, \vb*{\alpha}_{m} $ \textit{线性无关}.
\end{definition}

设
$$\begin{array}{c}
        \vb*{\alpha}_{1}=\left(a_{11}, a_{12}, \cdots, a_{1 n}\right)^{\top}, \quad \vb*{\alpha}_{2}=\left(a_{21}, a_{22}, \cdots, a_{2 n}\right)^{\top}, \cdots, \vb*{\alpha}_{m}=\left(a_{m 1}, a_{m 2}, \cdots, a_{m n}\right)^{\top}, \\
        \vb*{\beta}=\left(b_{1}, b_{2}, \cdots, b_{n}\right)^{\top},
    \end{array}$$
这里 $ m \leqslant n $.
\begin{enumerate}[label=(\arabic{*})]
    \item $\vb*{\beta} $ 可由 $ \vb*{\alpha}_{1}, \vb*{\alpha}_{2}, \cdots, \vb*{\alpha}_{m} $ 线性表示的充要条件是线性方程组 $ x_{1} \vb*{\alpha}_{1}+x_{2} \vb*{\alpha}_{2}+\cdots+   x_{m} \vb*{\alpha}_{m}=\vb*{\beta} $ 有解, 即下列线性方程组有解
          $$\left\{\begin{array}{l}
                  a_{11} x_{1}+a_{21} x_{2}+\cdots+a_{m 1} x_{m}=b_{1},                        \\
                  a_{12} x_{1}+a_{22} x_{2}+\cdots+a_{m 2} x_{m}=b_{2},                        \\
                  \cdots \cdots \cdots \cdots \cdots \cdots \cdots \cdots \cdots \cdots \cdots \\
                  a_{1 n} x_{1}+a_{2 n} x_{2}+\cdots+a_{m n} x_{m}=b_{n} .
              \end{array}\right.$$
    \item \begin{enumerate}
              \item 令 $ \vb*{A}=\left(\vb*{\alpha}_{1}, \vb*{\alpha}_{2}, \cdots, \vb*{\alpha}_{m}\right), \vb*{B}=\left(\vb*{\alpha}_{1}, \vb*{\alpha}_{2}, \cdots, \vb*{\alpha}_{m}, \vb*{\beta}\right) $, 则 $ \vb*{\beta} $ 可由 $ \vb*{\alpha}_{1}, \vb*{\alpha}_{2}, \cdots, \vb*{\alpha}_{m} $ 线性表示的充要条件是以 $ \vb*{\alpha}_{1}, \vb*{\alpha}_{2}, \cdots, \vb*{\alpha}_{m} $ 为列向量的矩阵和以 $ \vb*{\alpha}_{1}, \vb*{\alpha}_{2}, \cdots, \vb*{\alpha}_{m}, \vb*{\beta} $ 为列向量的矩阵有相同的秩, 即 $\rank(\vb*{A})=\rank(\vb*{B}) $.
              \item $ \vb*{\beta} $ 可由 $ \vb*{\alpha}_{1}, \vb*{\alpha}_{2}, \cdots, \vb*{\alpha}_{m} $ 唯一线性表示的充要条件是 $ \rank(\vb*{A})=\rank(\vb*{B})=m $.
              \item $\vb*{\beta} $ 不能由 $ \vb*{\alpha}_{1}, \vb*{\alpha}_{2}, \cdots, \vb*{\alpha}_{m} $ 线性表示的充要条件是 $ \rank(\vb*{A})<\rank(\vb*{B}) $.
          \end{enumerate}
    \item 向量组 $ \vb*{\alpha}_{1}, \vb*{\alpha}_{2}, \cdots, \vb*{\alpha}_{m} $ 线性相关的充要条件是齐次线性方程组
          $$\left\{\begin{array}{l}
                  a_{11} x_{1}+a_{21} x_{2}+\cdots+a_{m 1} x_{m}=0,                            \\
                  a_{12} x_{1}+a_{22} x_{2}+\cdots+a_{m 2} x_{m}=0,                            \\
                  \cdots \cdots \cdots \cdots \cdots \cdots \cdots \cdots \cdots \cdots \cdots \\
                  a_{1 n} x_{1}+a_{2 n} x_{2}+\cdots+a_{m n} x_{m}=0
              \end{array}\right.$$
          有非零解, 且当 $ m=n $ 时, 其线性相关的充要条件是
          $$|\vb*{A}|=\begin{vmatrix}a_{11}  & a_{21}  & \cdots & a_{n 1} \\
               a_{12}  & a_{22}  & \cdots & a_{n 2} \\
               \vdots  & \vdots  &        & \vdots  \\
               a_{1 n} & a_{2 n} & \cdots & a_{n n}\end{vmatrix}=0 .$$
    \item 向量组 $ \vb*{\alpha}_{1}, \vb*{\alpha}_{2}, \cdots, \vb*{\alpha}_{m} $ 线性无关的充要条件是齐次线性方程组
          $$\left\{\begin{array}{l}
                  a_{11} x_{1}+a_{21} x_{2}+\cdots+a_{m 1} x_{m}=0,                            \\
                  a_{12} x_{1}+a_{22} x_{2}+\cdots+a_{m 2} x_{m}=0,                            \\
                  \cdots \cdots \cdots \cdots \cdots \cdots \cdots \cdots \cdots \cdots \cdots \\
                  a_{1 n} x_{1}+a_{2 n} x_{2}+\cdots+a_{n m} x_{m}=0
              \end{array}\right.$$
          只有零解, 且当 $ m=n $ 时, 其线性无关的充要条件是
          $$|\vb*{A}|=\begin{vmatrix}a_{11}  & a_{21}  & \cdots & a_{n 1} \\
               a_{12}  & a_{22}  & \cdots & a_{n 2} \\
               \vdots  & \vdots  &        & \vdots  \\
               a_{1 n} & a_{2 n} & \cdots & a_{n n}\end{vmatrix}\neq 0 .$$
    \item 向量组 $ \vb*{\alpha}_{1}, \vb*{\alpha}_{2}, \cdots, \vb*{\alpha}_{m} $ 线性相关的充要条件是以  $\vb*{\alpha}_{1}, \vb*{\alpha}_{2}, \cdots, \vb*{\alpha}_{m} $ 为列向量的矩阵的秩小于向量个数 $ m $.
    \item 向量组 $ \vb*{\alpha}_{1}, \vb*{\alpha}_{2}, \cdots, \vb*{\alpha}_{m} $ 线性无关的充要条件是以  $\vb*{\alpha}_{1}, \vb*{\alpha}_{2}, \cdots, \vb*{\alpha}_{m} $ 为列向量的矩阵的秩等于向量个数 $ m $.
    \item 向量组 $ \vb*{\alpha}_{1}, \vb*{\alpha}_{2}, \cdots, \vb*{\alpha}_{m}(m \geqslant 2) $ 线性相关的充要条件是向量组中至少有一个向量是其余向量的线性组合; 向量组 $ \vb*{\alpha}_{1}, \vb*{\alpha}_{2}, \cdots, \vb*{\alpha}_{m}(m \geqslant 2) $ 线性无关的充要条件是向量组中任一个向量都不能由其余向量线性表示.
    \item 如果向量组 $ \vb*{\alpha}_{1}, \vb*{\alpha}_{2}, \cdots, \vb*{\alpha}_{m} $ 线性无关, 而向量组  $\vb*{\alpha}_{1}, \vb*{\alpha}_{2}, \cdots, \vb*{\alpha}_{m}, \vb*{\beta} $ 线性相关, 则 $ \vb*{\beta}$ 可以由 $ \vb*{\alpha}_{1}$, $\vb*{\alpha}_{2}, \cdots, \vb*{\alpha}_{m} $ 线性表示, 且表达式唯一.
    \item 如果向量组 $ \vb*{\alpha}_{1}, \vb*{\alpha}_{2}, \cdots, \vb*{\alpha}_{m} $ 可以由向量组 $ \vb*{\beta}_{1}, \vb*{\beta}_{2} \cdots, \vb*{\beta}_{t} $ 线性表示, 并且 $ m>t $, 则向量组 $ \vb*{\alpha}_{1}$, $\vb*{\alpha}_{2}, \cdots, \vb*{\alpha}_{m} $ 线性相关; 或者说, 如果向量组 $ \vb*{\alpha}_{1}, \vb*{\alpha}_{2}, \cdots, \vb*{\alpha}_{m} $ 线性无关, 并且可以由 $ \vb*{\beta}_{1}, \vb*{\beta}_{2}, \cdots, \vb*{\beta}_{t} $ 线性表示, 则 $ m \leqslant t $.
    \item 在向量组 $ \vb*{\alpha}_{1}, \vb*{\alpha}_{2}, \cdots, \vb*{\alpha}_{m} $ 中, 如果有一个部分组线性相关, 则整个向量组线性相关; 如果整个向量组 $ \vb*{\alpha}_{1}, \vb*{\alpha}_{2}, \cdots, \vb*{\alpha}_{m} $ 线性无关, 则其任一部分组也一定线性无关.
    \item 设 $ r $ 维向量组 $ \vb*{\alpha}_{i}=\left(a_{i 1}, a_{i 2}, \cdots, a_{i r}\right)(i=1,2, \cdots, m) $ 线性无关, 则在每个向量上再添加 $ n-r $ 个分量所得到的 $ n $ 维向量组 $ \vb*{\alpha}^{\prime}{ }_{i}=\left(a_{i 1}, a_{i 2}, \cdots, a_{i r}, a_{i, r+1}, \cdots, a_{i n}\right)(i=1,2, \cdots , m)$ 也线性无关.
    \item $n+1 $ 个 $ n $ 维向量必线性相关.
    \item 一个零向量线性相关; 一个非零向量线性无关; 两个非零向量线性相关的充要条件是对应分量成比例; 含有零向量的向量组必线性相关.
    \item 设 $ \vb*{\varepsilon}_{1}=(1,0, \cdots, 0), \vb*{\varepsilon}_{2}=(0,1, \cdots, 0), \cdots, \vb*{\varepsilon}_{n}=(0,0, \cdots, 1) $, 称 $ \vb*{\varepsilon}_{1}, \vb*{\varepsilon}_{2}, \cdots, \vb*{\varepsilon}_{n} $ 为 $ n $ 维单位向量组, 且
          \begin{enumerate}
              \item $\vb*{\varepsilon}_{1}, \vb*{\varepsilon}_{2}, \cdots, \vb*{\varepsilon}_{n} $ 线性无关;
              \item 任意 $ n $ 维向量 $ \vb*{\alpha}=\left(a_{1}, a_{2}, \cdots, a_{n}\right) $ 都可由 $ \vb*{\varepsilon}_{1}, \vb*{\varepsilon}_{2}, \cdots, \vb*{\varepsilon}_{n} $ 线性表示, 即
                    $$\vb*{\alpha}=a_{1} \vb*{\varepsilon}_{1}+a_{2} \vb*{\varepsilon}_{2}+\cdots+a_{n} \vb*{\varepsilon}_{n}.$$
          \end{enumerate}
    \item 初等行变换不改变矩阵的列向量组之间的线性关系; 初等列变换不改变矩阵的行向量组之间的线性关系.
\end{enumerate}

\begin{example}
    设 $\vb*{\alpha}_1=(0,1,2,3), \vb*{\beta}_1=(2,2,3,1), \vb*{\beta}_2=(-1,2,1,2), \vb*{\beta}_3=(2,1,-1,-2)$, 问 $\vb*{\alpha}_1$ 是否可表示成 $\vb*{\beta}_1, \vb*{\beta}_2, \vb*{\beta}_3$ 的线性表示.
\end{example}
\begin{solution}
    由线性表示秩的关系:
    \begin{flalign*}
        \qty(\vb*{\beta}_1^{\top}, \vb*{\beta}_2^{\top}, \vb*{\beta}_3^{\top}, \vb*{\alpha}_1^{\top}) & =\begin{pmatrix} 2 & -1 & 2 & 0 \\ 2 & 2 & 1 & 1 \\ 3 & 1 & -1 & 2 \\ 1 & 2 & -2 & 3 \\\end{pmatrix}\xrightarrow[\substack{r_3-2r_1 \\ r_4-2r_1}]{\substack{r_1\leftrightarrow r_4\\ r_2-2r_1}}\begin{pmatrix} 1 & 2 & -2 & 3 \\ 0 & -2 & 5 & -5 \\ 0 & -5 & 5 & -7 \\ 0 & -5 & 6 & -6 \\\end{pmatrix}\\
                                                                                                      & \xrightarrow[\substack{r_3-r_4                                                                                                      \\ r_4+5r_2}]{r_2\times\qty(-\frac{1}{2})}\begin{pmatrix} 1 & 2 & -2 & 3 \\ 0 & 1 & -\dfrac{5}{2} & \dfrac{5}{2} \\ 0 & 0 & -1 & -1 \\ 0 & 0 & -\dfrac{13}{2} & \dfrac{13}{2} \\\end{pmatrix}\xrightarrow[\substack{r_4\times\qty(-\frac{2}{13})-r_3\\ r_4\times\qty(-\frac{1}{2})}]{r_3\times(-1)}\begin{pmatrix} 1 & 2 & -2 & 3 \\ 0 & 1 & -\dfrac{5}{2} & \dfrac{5}{2} \\ 0 & 0 & 1 & 1 \\ 0 & 0 & 0 & 1 \\\end{pmatrix}
    \end{flalign*}
    由 $\rank(\vb*{\beta}_1, \vb*{\beta}_2, \vb*{\beta}_3)=3<\rank(\vb*{\beta}_1, \vb*{\beta}_2, \vb*{\beta}_3, \vb*{\alpha}_1)=4$, 知 $\vb*{\alpha}_1$ 不能表示为 $\vb*{\beta}_1, \vb*{\beta}_2, \vb*{\beta}_3$ 的线性组合.
\end{solution}

\begin{example}[1998 数二]
    已知 $\vb*{\alpha}_1=(1,4,0,2)^\top, \vb*{\alpha}_2=(2,7,1,3)^\top, \vb*{\alpha}_3=(0,1,-1,a)^\top, \vb*{\beta}=(3,10,b,4)^\top$,
    \begin{enumerate}[label=(\arabic{*})]
        \item $a,b$ 取何值时, $\vb*{\beta}$ 不能由 $\vb*{\alpha}_i~(i=1,2,3)$ 线性表示;
        \item $a,b$ 取何值时, $\vb*{\beta}$ 可由 $\vb*{\alpha}_i~(i=1,2,3)$ 线性表示, 并写出表达式.
    \end{enumerate}
\end{example}
\begin{solution}
    易知 $\vb*{\beta}$ 不能由 $\vb*{\alpha}_i~(i=1,2,3)$ 线性表示当且仅当 $$\rank(\vb*{\alpha}_1,\vb*{\alpha}_2,\vb*{\alpha}_3)\neq \rank(\vb*{\alpha}_1,\vb*{\alpha}_2,\vb*{\alpha}_3,\vb*{\beta})$$
    $\vb*{\beta}$ 可由 $\vb*{\alpha}_i~(i1,2,3)$ 线性表示当且仅当
    $$
        \rank(\vb*{\alpha}_1,\vb*{\alpha}_2,\vb*{\alpha}_3)= \rank(\vb*{\alpha}_1,\vb*{\alpha}_2,\vb*{\alpha}_3,\vb*{\beta})
    $$
    那么
    \begin{flalign*}
        \begin{pmatrix} 1 & 2 & 0 & 3 \\ 4 & 7 & 1 & 10 \\ 0 & 1 & -1 & b \\ 2 & 3 & a & 4 \\\end{pmatrix}\xrightarrow[r_4-2r_1]{r_2-4r_1}
        \begin{pmatrix} 1 & 2 & 0 & 3 \\ 0 & -1 & 1 & -2 \\ 0 & 1 & -1 & b \\ 0 & -1 & a & -2 \\\end{pmatrix}\xrightarrow[r_4-r_2]{r_3+r_2}
        \begin{pmatrix} 1 & 2 & 0 & 3 \\ 0 & -1 & 1 & -2 \\ 0 & 0 & 0 & b-2 \\ 0 & 0 & a-1 & 0 \\\end{pmatrix}\xrightarrow[r_2\times(-1)]{r_1+2r_2}
        \begin{pmatrix} 1 & 0 & 2 & -1 \\ 0 & 1 & -1 & 2 \\ 0 & 0 & 0 & b-2 \\ 0 & 0 & a-1 & 0 \\\end{pmatrix}
    \end{flalign*}
    \begin{enumerate}[label=(\arabic{*})]
        \item 当 $b\neq 2$ 时, 即 $\rank(\vb*{\alpha}_1,\vb*{\alpha}_2,\vb*{\alpha}_3)\neq \rank(\vb*{\alpha}_1,\vb*{\alpha}_2,\vb*{\alpha}_3,\vb*{\beta})$ 时, 从而 $\vb*{\beta}$ 不能用 $\vb*{\alpha}_i~(i=1,2,3)$ 线性表示.
        \item 当 $b= 2$ 时, 即 $\rank(\vb*{\alpha}_1,\vb*{\alpha}_2,\vb*{\alpha}_3)\neq \rank(\vb*{\alpha}_1,\vb*{\alpha}_2,\vb*{\alpha}_3,\vb*{\beta})$ 时, 从而 $\vb*{\beta}$ 可由用 $\vb*{\alpha}_i~(i=1,2,3)$ 线性表示, 表达式如下:
              \begin{enumerate}
                  \item 当 $a=1$ 时, 表达式为 $\vb*{\beta}=(-2c-1)\vb*{\alpha}_1+(c+2)\vb*{\alpha}_2+c\vb*{\alpha}_3$, 其中 $c$ 为任意常数.
                  \item 当 $a\neq1$ 时, 表达式为 $\vb*{\beta}=-\vb*{\alpha}_1+2\vb*{\alpha}_2.$
              \end{enumerate}
    \end{enumerate}
\end{solution}

\subsection{极大线性无关组}

\begin{definition}[极大线性无关组]
    \index{极大线性无关组}
    设向量组 $ \vb*{\alpha}_{i 1}, \vb*{\alpha}_{i 2}, \cdots, \vb*{\alpha}_{i r} $ 为向量组 $ \vb*{\alpha}_{1}, \vb*{\alpha}_{2}, \cdots, \vb*{\alpha}_{m} $ 的一个部分组, 且满足
    \begin{enumerate}[label=(\arabic{*})]
        \item $\vb*{\alpha}_{i 1}, \vb*{\alpha}_{i 2}, \cdots, \vb*{\alpha}_{i} $ 线性无关;
        \item 向量组 $ \vb*{\alpha}_{1}, \vb*{\alpha}_{2}, \cdots, \vb*{\alpha}_{m} $ 中任一向量均可由 $ \vb*{\alpha}_{i 1}, \vb*{\alpha}_{i 2}, \cdots, \vb*{\alpha}_{i r} $ 线性表示,
    \end{enumerate}
    则称\textit{向量组} $\vb*{\alpha}_{i 1}, \vb*{\alpha}_{i 2}, \cdots, \vb*{\alpha}_{1} $ \textit{为向量组} $ \vb*{\alpha}_{1}, \vb*{\alpha}_{2}, \cdots, \vb*{\alpha}_{m} $ \textit{的一个极大线性无关组}, 简称\textit{极大无关组}.
\end{definition}

\begin{example}
    向量组 $\vb*{\alpha}_1=(1,3,5,-1)^\top,~\vb*{\alpha}_2=(2,-1,-3,-4)^\top,~\vb*{\alpha}_3=(6,4,4,6)^\top,~\vb*{\alpha}_4=(7,7,9,1)^\top,~\vb*{\alpha}_5=(3,2,2,3)^\top$ 的极大线性无关组是 
    \begin{tasks}(4)
        \task $\vb*{\alpha}_1,\vb*{\alpha}_2,\vb*{\alpha}_5$
        \task $\vb*{\alpha}_1,\vb*{\alpha}_3,\vb*{\alpha}_5$
        \task $\vb*{\alpha}_2,\vb*{\alpha}_3,\vb*{\alpha}_4$
        \task $\vb*{\alpha}_3,\vb*{\alpha}_4,\vb*{\alpha}_5$
    \end{tasks}
\end{example}
\begin{solution}
    列向量作行变换最后得 $\begin{pmatrix} 1 & 2 & 6 & 7 & 3 \\ 0 & 1 & 2 & 2 & 1 \\ 0 & 0 & 0 & -4 & 0 \\ 0 & 0 & 0 & 0 & 0\end{pmatrix}$
    因为三阶子式 $\begin{vmatrix} 2 & 6 & 7 \\ 1 & 2 & 2 \\ 0 & 0 & -4 \\\end{vmatrix}\neq0$, 所以 $\vb*{\alpha}_2,\vb*{\alpha}_3,\vb*{\alpha}_4$ 是极大线性无关组, 故选 C.
\end{solution}

\subsection{向量组等价}

\begin{definition}[向量组等价]
    如果向量组 $(I),(II)$ 可以相互线性表示, 则称向量组 $(I)$ 与向量组 $(II)$ \textit{等价}.
    \index{向量组等价}
\end{definition}

向量组等价的结论:
\begin{enumerate}[label=(\arabic{*})]
    \item 任一向量组和它的极大无关组等价;
    \item 向量组的任意两个极大无关组等价;
    \item 两个等价的线性无关的向量组所含向量的个数相同;
    \item 两个向量组等价的充要条件是它们的极大无关组等价;
    \item 等价的两个向量组有相同的秩.
\end{enumerate}

\begin{example}
    设 $n$ 维向量组 $(I):\vb*{\alpha}_1,\vb*{\alpha}_2,\cdots,\vb*{\alpha}_k~ (k<n)$ 线性无关, 则 $n$ 维向量组 $(II):\vb*{\beta}_1,\vb*{\beta}_2,\cdots,\vb*{\beta}_k$ 也线性无关的充要条件为
    \begin{tasks}
        \task $\vb*{\beta}_1,\vb*{\beta}_2,\cdots,\vb*{\beta}_k$ 可由 $\vb*{\alpha}_1,\vb*{\alpha}_2,\cdots,\vb*{\alpha}_k$ 线性表示
        \task $\vb*{\alpha}_1,\vb*{\alpha}_2,\cdots,\vb*{\alpha}_k$ 可由 $\vb*{\beta}_1,\vb*{\beta}_2,\cdots,\vb*{\beta}_k$ 线性表示
        \task 向量组 $(I)$ 与向量组 $(II)$ 等价
        \task 矩阵 $(\vb*{\alpha}_1,\vb*{\alpha}_2,\cdots,\vb*{\alpha}_k)$ 与矩阵 $(\vb*{\beta}_1,\vb*{\beta}_2,\cdots,\vb*{\beta}_k)$ 等价
    \end{tasks}
\end{example}
\begin{solution}
    对于 A 选项, 因为 $\vb*{\beta}_1,\vb*{\beta}_2,\cdots,\vb*{\beta}_k$ 可由 $\vb*{\alpha}_1,\vb*{\alpha}_2,\cdots,\vb*{\alpha}_k$ 线性表示, 所以 $\rank(II)\leqslant \rank(I)$, 而向量组 $(I)$ 线性无关,
    所以 $\rank(I)=k$, 即 $\rank(II)\leqslant k$, 由于无法保证 $\rank(II)=k$, 故不能得出 $(II)$ 线性无关;\\
    对于 B 选项, 因为 $\vb*{\alpha}_1,\vb*{\alpha}_2,\cdots,\vb*{\alpha}_k$ 可由 $\vb*{\beta}_1,\vb*{\beta}_2,\cdots,\vb*{\beta}_k$ 线性表示, 所以 $\rank(I)\leqslant \rank(II)$, 同样地,
    $$k=\rank(I)\leqslant\rank(II)\leqslant k\Rightarrow \rank(II)=k$$
    因此 $(II)$ 线性无关, 充分条件成立, 但不是必要条件, 如 $(I):\mqty(1\\0)$ 与 $(II):\mqty(0\\1)$ 均线性无关, 但 $(I)$ 不能由 $(II)$ 线性表示;\\
    对于 C 选项, 由 $(I)$ 与 $(II)$ 等价知 $(I)$ 与 $(II)$ 可相互线性表示, 则 $\rank(I)=\rank(II)=k$, 故 $(II)$ 线性无关, 充分条件成立, 同 B 选项知必须条件不成立;\\
    因为矩阵 $\vb*{A}=(\vb*{\alpha}_1,\vb*{\alpha}_2,\cdots,\vb*{\alpha}_k)$ 与矩阵 $\vb*{B}=(\vb*{\beta}_1,\vb*{\beta}_2,\cdots,\vb*{\beta}_k)$ 等价, 则 $\rank\vb*{A}=\rank\vb*{B}=k$,
    又 $(I):\vb*{\alpha}_1,\vb*{\alpha}_2,\cdots,\vb*{\alpha}_k~ (k<n)$ 线性无关, 即 $\rank\vb*{A}=\rank\vb*{B}=k$, 故 $\vb*{\beta}_1,\vb*{\beta}_2,\cdots,\vb*{\beta}_k$ 线性无关, 故选 D.
\end{solution}

\begin{example}
    已知 $n$ 维向量组 $$(I):\vb*{\alpha}_1,~\vb*{\alpha}_2,\cdots,\vb*{\alpha}_s,~(II): \vb*{\beta}_1,\vb*{\beta}_2,~\cdots,\vb*{\beta}_t$$
    且 $\rank(\vb*{\alpha}_1,\vb*{\alpha}_2,\cdots,\vb*{\alpha}_s)=\rank(\vb*{\beta}_1,\vb*{\beta}_2,\cdots,\vb*{\beta}_t)=r$, 则
    \begin{tasks}
        \task 当 $s=r$ 时, 向量组 $(I)$ 与 $(II)$ 等价
        \task 当 $s=t=r$ 时, 向量组 $(I)$ 与 $(II)$ 等价
        \task 当 $\rank(\vb*{\alpha}_1,\cdots,\vb*{\alpha}_s,\vb*{\beta}_1,\cdots,\vb*{\beta}_t)=r$ 时, 向量组 $(I)$ 与 $(II)$ 等价
        \task 当 $\rank(\vb*{\alpha}_1,\cdots,\vb*{\alpha}_s,\vb*{\beta}_1,\cdots,\vb*{\beta}_t)=2r$ 时, 向量组 $(I)$ 与 $(II)$ 等价
    \end{tasks}
\end{example}
\begin{solution}
    记向量组 $(III):\vb*{\alpha}_1,\cdots,\vb*{\alpha}_s,\vb*{\beta}_1,\cdots,\vb*{\beta}_t$, 取 ${I}$ 和 ${II}$ 的一个极大无关组 $(I_0): \vb*{\alpha}_{i_1},\vb*{\alpha}_{i_2},\cdots,\vb*{\alpha}_{i_r}$ 与 $(II_0):\vb*{\beta}_{j_1},\vb*{\beta}_{j_2},\cdots,\vb*{\beta}_{j_r}$,
    那么向量组 $(I)$ 与 $(I_0)$ 等价, 向量组 $(II)$ 与 $(II_0)$ 的等价, 如果 $\rank(III)=r$, 那么向量组 $(I_0)$ 与 $(II_0)$ 都是 $(III)$ 的一个极大无关组, 这表明向量组 $(I_0)$ 与 $(II_0)$ 等价, 因此, 向量组 $(I)$ 与 $(II)$ 等价.\\
    另外, 考虑向量组 $$(I):\vb*{\alpha}_1=(1,0,0,0)^\top,~\vb*{\alpha}_2=(0,1,0,0)^\top,~(II):\vb*{\beta}_1=(0,0,1,0)^\top,~\vb*{\beta}_2=(0,0,0,1)^\top$$
    则 $\rank(\vb*{\alpha}_1,\vb*{\alpha}_2)=\rank(\vb*{\beta}_1,\vb*{\beta}_2)=2,~\rank(\vb*{\alpha}_1,\vb*{\alpha}_2,\vb*{\beta}_1,\vb*{\beta}_2)=4$, 但选项 A、B、D 均不成立.
\end{solution}

\begin{example}
    设有向量组 $(I):\vb*{\alpha}_1=(1,0,2)^\top,\vb*{\alpha}_2=(1,1,3)^\top,\vb*{\alpha}_3=(1,-1,a+2)^\top$ 和向量组 $(II):\vb*{\beta}_1=(1,2,a+3)^\top,\vb*{\beta}_2=(2,1,a+6)^\top,\vb*{\beta}_3=(2,1,a+4)^\top$ 试问:
    \begin{enumerate}[label=(\arabic{*})]
        \item 当 $a $ 为何值时, 向量组 $(I)$ 与向量组 $(II)$ 等价;
        \item 当 $a$ 为何值时, 向量组 $(I)$ 与向量组 $(II)$ 不等价.
    \end{enumerate}
\end{example}
\begin{solution}
    \begin{enumerate}[label=(\arabic{*})]
        \item 对 $\begin{pNiceArray}{ccc:ccc}
                      \vb*{\alpha}_1 & \vb*{\alpha}_2 & \vb*{\alpha}_3 & \vb*{\beta}_1 & \vb*{\beta}_2 & \vb*{\beta}_3
                  \end{pNiceArray}$ 作初等行变换, 有
              $$
                  \begin{pNiceArray}{ccc:ccc}
                      1 & 1 & 1 & 1 & 2 & 2 \\
                      0 & 1 & -1 & 2 & 1 & 1 \\
                      2 & 3 & a+2 & a+3 & a+6 & a+4 \\
                  \end{pNiceArray}\xrightarrow[r_1-r_2]{r_3-2r_1-r_2}
                  \begin{pNiceArray}{ccc:ccc}
                      1 & 0 & 2 & -1 & 1 & 1 \\
                      0 & 1 & -1 & 2 & 1 & 1 \\
                      0 & 0 & a+1 & a-1 & a+1 & a-1 \\
                  \end{pNiceArray}
              $$
              当 $a\neq -1$ 时, $$\rank(\vb*{\alpha}_1,\vb*{\alpha}_2,\vb*{\alpha}_3)=3=\rank(\vb*{\alpha}_1,\vb*{\alpha}_2,\vb*{\alpha}_3,\vb*{\beta}_i)~(i=1, 2, 3)$$
              因此 $\vb*{\beta}_i=(k_1, k_2, k_3)\cdot (\vb*{\alpha}_1, \vb*{\alpha}_2, \vb*{\alpha}_3)$ 有唯一解, 从而 $\vb*{\beta}_1, \vb*{\beta}_2, \vb*{\beta}_3$ 可由 $\vb*{\alpha}_1, \vb*{\alpha}_2, \vb*{\alpha}_3$ 线性表示,
              又因为 $\det(\vb*{\beta}_1, \vb*{\beta}_2, \vb*{\beta}_3)=6\neq 0$, 所以 $$
                  \rank(\vb*{\beta}_1, \vb*{\beta}_2, \vb*{\beta}_3)=3=\rank(\vb*{\beta}_1, \vb*{\beta}_2, \vb*{\beta}_3,\vb*{\alpha}_i)~(i=1, 2, 3)
              $$
              因此 $\vb*{\alpha}_i=(l_1, l_2, l_3)\cdot (\vb*{\beta}_1, \vb*{\beta}_2, \vb*{\beta}_3)$ 有唯一解, 从而 $\vb*{\alpha}_1, \vb*{\alpha}_2, \vb*{\alpha}_3$ 可由 $\vb*{\beta}_1, \vb*{\beta}_2, \vb*{\beta}_3$ 线性表示.
        \item 当 $a=-1$ 时, $$\rank(\vb*{\alpha}_1, \vb*{\alpha}_2, \vb*{\alpha}_3)=2\neq \rank(\vb*{\alpha}_1, \vb*{\alpha}_2, \vb*{\alpha}_3,\vb*{\beta}_1)=3$$
              因此 $\vb*{\beta}_1$ 不可由 $\vb*{\alpha}_1, \vb*{\alpha}_2, \vb*{\alpha}_3$ 线性表示, 从而向量组 $(I)$ 与 $(II)$ 不等价.
    \end{enumerate}
\end{solution}

\begin{example}[2011 数一]
    设向量组 $\vb*{\alpha}_1=(1,0,1)^\top,\vb*{\alpha}_2=(0,1,1)^\top,\vb*{\alpha}_3=(1,3,5)^\top$ 不能由向量组 $\vb*{\beta}_1=(1,1,1)^\top,\vb*{\beta}_2=(1,2,3)^\top,\vb*{\beta}_3=(3,4,a)^\top$ 线性表示,
    \begin{enumerate}[label=(\arabic{*})]
        \item 求 $a$ 的值;
        \item 将 $\vb*{\beta}_1, \vb*{\beta}_2, \vb*{\beta}_3$ 用 $\vb*{\alpha}_1, \vb*{\alpha}_2, \vb*{\alpha}_3$ 线性表示.
    \end{enumerate}
\end{example}
\begin{solution}
    \begin{enumerate}[label=(\arabic{*})]
        \item 因为四个 $3$ 维向量 $\vb*{\beta}_1, \vb*{\beta}_2, \vb*{\beta}_3,\vb*{\alpha}$ 必定线性相关, 所以若 $\vb*{\beta}_1, \vb*{\beta}_2, \vb*{\beta}_3$ 线性无关, 那么 $\vb*{\alpha}_1, \vb*{\alpha}_2, \vb*{\alpha}_3$ 必可由 $\vb*{\beta}_1, \vb*{\beta}_2, \vb*{\beta}_3$ 线性表示, 与题设矛盾, 故 $\vb*{\beta}_1, \vb*{\beta}_2, \vb*{\beta}_3$ 一定相关, 那么
              $$
                  |\vb*{\beta}_1, \vb*{\beta}_2, \vb*{\beta}_3|=\mqty|1&1&3\\1&2&4\\1&3&a|=\mqty|1&1&3\\0&1&1\\0&2&a-3|=a-5=0\Rightarrow a=5.
              $$
        \item 对 $\begin{pNiceArray}{ccc:ccc}\vb*{\alpha}_1& \vb*{\alpha}_2& \vb*{\alpha}_3&\vb*{\beta}_1& \vb*{\beta}_2& \vb*{\beta}_3\end{pNiceArray}$ 作初等行变换, 有
              \begin{flalign*}
                  \begin{pNiceArray}{ccc:ccc}1&0&1&1&1&3\\0&1&3&1&2&4\\1&1&5&1&3&5\end{pNiceArray}\xrightarrow[r_2-3r_3]{r_3-r_1-r_2}\begin{pNiceArray}{ccc:ccc}1&0&0&2&1&5\\0&1&0&4&2&10\\0&0&1&-1&0&-2\end{pNiceArray}
              \end{flalign*}
              所以 $\vb*{\beta}_1=2\vb*{\alpha}_1+4\vb*{\alpha}_2-\vb*{\alpha}_3,~\vb*{\beta}_2=\vb*{\alpha}_1+2\vb*{\alpha}_2,~\vb*{\beta}_3=5\vb*{\alpha}_1+10\vb*{\alpha}_2-2\vb*{\alpha}_3$.
    \end{enumerate}
\end{solution}

\begin{example}[2000 数二]
    已知向量组 $\vb*{\beta}_1=\mqty(0\\1\\-1),~\vb*{\beta}_2=\mqty(a\\2\\1),~\vb*{\beta}_3=\mqty(b1\\1\\0)$ 与向量组 $\vb*{\alpha}_1=\mqty(1\\2\\-3),\\~\vb*{\alpha}_2=\mqty(3\\0\\1),~\vb*{\alpha}_3=\mqty(9\\6\\-7)$ 具有相同的秩, 且 $\vb*{\beta}_3$ 可由 $\vb*{\alpha}_1,\vb*{\alpha}_2,\vb*{\alpha}_3$ 线性表示, 求 $a,b$ 的值.
\end{example}
\begin{solution}
    易知 $\vb*{\alpha}_1,~\vb*{\alpha}_2$ 线性无关, $\vb*{\alpha}_3=3\vb*{\alpha}_1+2\vb*{\alpha}_2$, 即 $\vb*{\alpha}_3$ 可由 $\vb*{\alpha}_1,~\vb*{\alpha}_2$ 线性表示, 所以向量组 $\vb*{\alpha}_1,\vb*{\alpha}_2,\vb*{\alpha}_3$ 的秩为 2, 且 $\vb*{\alpha}_1,~\vb*{\alpha}_2$ 是它的一个极大无关组,
    由于向量组 $\vb*{\beta}_1,~\vb*{\beta}_2,~\vb*{\beta}_3$ 与 $\vb*{\alpha}_1,~\vb*{\alpha}_2,~\vb*{\alpha}_3$ 具有相同的秩, 故 $\vb*{\beta}_1,~\vb*{\beta}_2,~\vb*{\beta}_3$ 线性相关, 从而行列式
    $$|\vb*{\beta}_1,\vb*{\beta}_2,\vb*{\beta}_3|=\mqty|0&a&b\\1&2&1\\-1&1&0|=0\Rightarrow a=3b$$
    又 $\vb*{\beta}_3$ 可由 $\vb*{\alpha}_1,~\vb*{\alpha}_2,~\vb*{\alpha}_3$ 线性表示, 而 $\vb*{\alpha}_3$ 可由 $\vb*{\alpha}_1,~\vb*{\alpha}_2$ 线性表示, 所以 $\vb*{\alpha}_1,~\vb*{\alpha}_2,~\vb*{\beta}_3$ 线性相关, 于是
    $$|\vb*{\alpha}_1,\vb*{\alpha}_2,\vb*{\beta}_3|=\mqty|1&3&b\\2&0&1\\-3&1&0|=0\Rightarrow b=5\Rightarrow a=15.$$
\end{solution}

\begin{example}[2016 南京航空航天大学]
    设由向量组 $$(I):\vb*{\alpha}_1=\mqty(1\\1\\a),~\vb*{\alpha}_2=\mqty(-2\\a\\4),~\vb*{\alpha}_3=\mqty(-2\\a\\a),~(II):\vb*{\beta}_1=\mqty(1\\1\\a),~\vb*{\beta}_2=\mqty(1\\a\\1),~\vb*{\beta}_3=\mqty(a\\1\\1)$$
    \begin{enumerate}[label=(\arabic{*})]
        \item 求 $a$ 的值, 使得向量组 $(I)$ 线性相关;
        \item 求 $a$ 的值, 使得向量组 $(I)$ 不能由向量组 $(II)$ 线性表示;
        \item 在题 (1) 和 (2) 同时成立的条件下, 将向量 $\vb*{\gamma}=(1,-2,-5)^\top$ 用 $\vb*{\beta}_1,~\vb*{\beta}_2,\vb*{\alpha}_3$ 线性表示.
    \end{enumerate}
\end{example}
\begin{solution}
    \begin{enumerate}[label=(\arabic{*})]
        \item 令 $\vb*{A}=(\vb*{\alpha}_1,\vb*{\alpha}_2,\vb*{\alpha}_3)$, 则行列式 $\det\vb*{A}=0$ 等价于向量组 $(I)$ 线性相关, 则
              $$\mqty|1&-2&-2\\1&a&a\\a&4&a|=a^2-2a-8=0\Rightarrow a=4\text{ 或 }-2.$$
        \item 令 $\vb*{B}=(\vb*{\beta}_1,\vb*{\beta}_2,\vb*{\beta}_3)$, 由于向量组 $(I)$ 不能由 $(II)$ 线性表示, 说明向量组 $(II)$ 线性相关, 如若不然, 则任一 $\vb*{\alpha}_i$, 可由向量组 $(II)$ 线性表示, 矛盾,
              因此, $\det\vb*{B}=0$, 则
              $$\mqty|1&1&a\\1&a&1\\a&1&1|=(a-1)^2(a+2)=0\Rightarrow a=1\text{ 或 }a=-2$$
              当 $a=1$ 时, $\vb*{\alpha}_3$ 显然不能由 $\vb*{\beta}_j~~(j=1,2,3)$ 线性表示, 符合题意; 当 $a=-2$ 时, 将 $\begin{pNiceArray}{c:c}
                      \vb*{B} & \vb*{A}
                  \end{pNiceArray}$ 化简为行阶梯形, 有
              $$\begin{pNiceArray}{c:c}\vb*{B} & \vb*{A}\end{pNiceArray}=\begin{pNiceArray}{ccc:ccc}
                      1&1&-2&1&-2&-2\\1&-2&1&1&-2&-2\\-2&1&1&-2&4&-2
                  \end{pNiceArray}\xrightarrow[\substack{r_3\times(-\frac{1}{6})\\r_2\times(-\frac{1}{3})\\r_1-r_2}]{\substack{r_2-r_1\\r_3+2r_1\\r_3+r_2}}\begin{pNiceArray}{ccc:ccc}
                      1\Block[borders={bottom,tikz=dashed}]{1-1}{}&0&-1&1&-2&-2\\
                      0&1\Block[borders={left,bottom,tikz=dashed}]{1-4}{}&-1&0&0&0\\
                      0&0&0&0&0&1\Block[borders={left,bottom,tikz=dashed}]{1-1}{}
                  \end{pNiceArray}$$
              由此可见, $\vb*{\alpha}_3$ 不能由 $\vb*{\beta}_i~~(i=1,2,3)$ 线性表示, 所以 $a=-2$ 也符合题意.
        \item 若 (1) 和 (2) 同时成立, 则 $a=-2$, 此时, 把 $(\vb*{\beta}_1,\vb*{\beta}_2,\vb*{\alpha}_3,\vb*{\gamma})$ 化为行最简形, 有
              $$\begin{pNiceArray}{c:c}\vb*{\beta}_1,\vb*{\beta}_2,\vb*{\alpha}_3 & \vb*{\gamma}\end{pNiceArray}=\begin{pNiceArray}{ccc:c}
                      1&1&-2&1\\
                      1&-2&-2&-2\\
                      -2&1&-2&-5
                  \end{pNiceArray}\to\begin{pNiceArray}{ccc:c}
                      1&0&0&2\\
                      0&1&0&1\\
                      0&0&1&1
                  \end{pNiceArray}$$
              因为初等行变换不改变列向量之间的线性关系, 所以 $\vb*{\gamma}=2\vb*{\beta}_1+\vb*{\beta}_2+\vb*{\alpha}_3.$
    \end{enumerate}
\end{solution}

\subsection{向量组的秩}

\begin{definition}[向量组的秩]
    向量组 $ \vb*{\alpha}_{1}, \vb*{\alpha}_{2}, \cdots, \vb*{\alpha}_{m} $ 的极大无关组中所含向量的个数称为该\textit{向量组的秩}, 记为 $ \rank\left(\vb*{\alpha}_{1}, \vb*{\alpha}_{2}, \cdots, \vb*{\alpha}_{m}\right) $, 如果一个向量组仅合存焉问量, 则规定它的秩为零.
\end{definition}

向量组的秩的性质:
\begin{enumerate}[label=(\arabic{*})]
    \item 若 $ \rank\left(\vb*{\alpha}_{1}, \vb*{\alpha}_{2}, \cdots, \vb*{\alpha}_{m}\right)=r $, 则
          \begin{enumerate}
              \item $\vb*{\alpha}_{1}, \vb*{\alpha}_{2}, \cdots, \vb*{\alpha}_{m} $ 的任何含存多于 $ r $ 个向量的部分组一定线性相关;
              \item $ \vb*{\alpha}_{1}, \vb*{\alpha}_{2}, \cdots, \vb*{\alpha}_{m} $ 的任何含 $ r $ 个向量的线性无关部分组一定是极大无关组.
          \end{enumerate}
    \item $\rank\left(\vb*{\alpha}_{1}, \vb*{\alpha}_{2}, \cdots, \vb*{\alpha}_{m}\right) \leqslant m $, 且 $ \rank\left(\vb*{\alpha}_{1}, \vb*{\alpha}_{2}, \cdots, \vb*{\alpha}_{m}\right)=m \Leftrightarrow \vb*{\alpha}_{1}, \vb*{\alpha}_{2}, \cdots, \vb*{\alpha}_{m} $ 线性无关.
    \item 向量 $ \vb*{\beta} $ 可用 $ \vb*{\alpha}_{1}, \vb*{\alpha}_{2}, \cdots, \vb*{\alpha}_{m} $ 线性表示 $ \Leftrightarrow \rank\left(\vb*{\alpha}_{1}, \vb*{\alpha}_{2}, \cdots, \vb*{\alpha}_{m}, \vb*{\beta}\right)=\rank\left(\vb*{\alpha}_{1}, \vb*{\alpha}_{2}, \cdots, \vb*{\alpha}_{m}\right) $.
    \item 向量 $ \vb*{\beta} $ 可用 $ \vb*{\alpha}_{1}, \vb*{\alpha}_{2}, \cdots, \vb*{\alpha}_{m} $ 唯一线性表示 $ \Leftrightarrow \rank\left(\vb*{\alpha}_{1}, \vb*{\alpha}_{2}, \cdots, \vb*{\alpha}_{m}, \vb*{\beta}\right)=\rank\left(\vb*{\alpha}_{1},\vb*{\alpha}_{2}, \cdots,\right.$\\ $\left.  \vb*{\alpha}_{m}\right)=m $.
    \item 向量组 $ \vb*{\beta}_{1}, \vb*{\beta}_{2}, \cdots, \vb*{\beta}_{t} $ 可以用 $ \vb*{\alpha}_{1}, \vb*{\alpha}_{2}, \cdots, \vb*{\alpha}_{s} $ 线性表示 $ \Leftrightarrow \rank\left(\vb*{\alpha}_{1}, \vb*{\alpha}_{2}, \cdots, \vb*{\alpha}_{s}, \vb*{\beta}_{1}, \vb*{\beta}_{2}, \cdots,\right.$\\$ \left.\vb*{\beta}_{t}\right)=\rank\left(\vb*{\alpha}_{1}, \vb*{\alpha}_{2}, \cdots, \vb*{\alpha}_{s}\right) .$
    \item 向量组 $ \vb*{\alpha}_{1}, \vb*{\alpha}_{2}, \cdots, \vb*{\alpha}_{s} $ 与 $ \vb*{\beta}_{1}, \vb*{\beta}_{2}, \cdots, \vb*{\beta}_{t} $ 等价 $ \Leftrightarrow \rank\left(\vb*{\alpha}_{1}, \vb*{\alpha}_{2}, \cdots, \vb*{\alpha}_{s}\right)=\rank\left(\vb*{\alpha}_{1}, \vb*{\alpha}_{2}, \cdots, \vb*{\alpha}_{s},\right.$\\$\left. \vb*{\beta}_{1}, \vb*{\beta}_{2}, \cdots, \vb*{\beta}_{t}\right)=\rank\left(\vb*{\beta}_{1}, \vb*{\beta}_{2}, \cdots, \vb*{\beta}_{t}\right) $.
    \item 设 $ \vb*{A} $ 是一个 $ m \times n $ 矩阵, 记 $ \vb*{\alpha}_{1}, \vb*{\alpha}_{2}, \cdots, \vb*{\alpha}_{n} $ 是 $ \vb*{A} $ 的列向量组 ($m$ 维), $ \vb*{\beta}_{1}, \vb*{\beta}_{2}, \cdots ,  \vb*{\beta}_{m} $ 是 $ \vb*{A} $ 的行向量组 ($n$ 维), 则 $ \rank(\vb*{A})=r\left(\vb*{\alpha}_{1}, \vb*{\alpha}_{2}, \cdots, \vb*{\alpha}_{n}\right)=r\left(\vb*{\beta}_{1}, \vb*{\beta}_{2}, \cdots, \vb*{\beta}_{m}\right) $.
\end{enumerate}

\begin{theorem}
    若向量组 $\vb*{\alpha}_i~ (i=1,2,\cdots,s)$ 可以由向量组 $\vb*{\beta}_j~ (j=1,2,\cdots,t)$ 表示, 则
    $$\rank(\vb*{\alpha}_1,\vb*{\alpha}_2,\cdots,\vb*{\alpha}_s)\leqslant \rank(\vb*{\beta}_1,\vb*{\beta}_2,\cdots,\vb*{\beta}_t).$$
\end{theorem}
\begin{proof}[{\songti \textbf{证}}]
    由于 $\vb*{\alpha}_i$ 可由 $\vb*{\beta}_j$ 线性表示, 不妨设 $$\vb*{\alpha}_i=\sum_{k=1}^{t}x_{ik}\vb*{\beta}_k$$
    故 $$(\vb*{\alpha}_1,\vb*{\alpha}_2,\cdots,\vb*{\alpha}_s)=(\vb*{\beta}_1,\vb*{\beta}_2,\cdots,\vb*{\beta}_t)\mqty(x_{11}&x_{12}&\cdots&x_{1t}\\x_{21}&x_{22}&\cdots&x_{2t}\\\vdots &\vdots&&\vdots\\x_{s1}&x_{s2}&\cdots&x_{st})$$
    由于矩阵乘积的秩小于等于每一个, 则
    $$\rank(\vb*{\alpha}_1,\vb*{\alpha}_2,\cdots,\vb*{\alpha}_s)\leqslant \rank\min\qty{(\vb*{\beta}_1,\vb*{\beta}_2,\cdots,\vb*{\beta}_t),\mqty(x_{11}&x_{12}&\cdots&x_{1t}\\x_{21}&x_{22}&\cdots&x_{2t}\\\vdots &\vdots&&\vdots\\x_{s1}&x_{s2}&\cdots&x_{st})}$$
    即 $\rank(\vb*{\alpha}_1,\vb*{\alpha}_2,\cdots,\vb*{\alpha}_s)\leqslant \rank(\vb*{\beta}_1,\vb*{\beta}_2,\cdots,\vb*{\beta}_t).$
\end{proof}
