\section{向量间的线性关系}

\subsection{基本概念}

\begin{definition}[线性表示]
    对于向量 $\vb*{\beta},\vb*{\alpha}_1,\vb*{\alpha}_2,\cdots,\vb*{\alpha}_m$, 如果存在一组数 $k_1,k_2,\cdots, k_m$, 使得
    $$\vb*{\beta}=k_{1} \vb*{\alpha}_{1}+k_{2} \vb*{\alpha}_{2}+\cdots+k_{m} \vb*{\alpha}_{m}$$
    成立, 则称 $ \vb*{\beta} $ 是 $ \vb*{\alpha}_{1}, \vb*{\alpha}_{2}, \cdots, \vb*{\alpha}_{m} $ 的\textit{线性组合}, 或称 $ \vb*{\beta} $ 可由 $ \vb*{\alpha}_{1}, \vb*{\alpha}_{2}, \cdots \vb*{\alpha}_{m} $ \textit{线性表示}.
\end{definition}

\begin{definition}[线性相关与线性无关]
    设 $ \vb*{\alpha}_{1}, \vb*{\alpha}_{2}, \cdots, \vb*{\alpha}_{m} $ 为一组向量, 如果存在一组不全为零的数 $ k_{1}$, $k_{2}, \cdots, k_{m} $, 使得
    $$k_{1} \vb*{\alpha}_{1}+k_{2} \vb*{\alpha}_{2}+\cdots+k_{m} \vb*{\alpha}_{m}=\mathbf{0}$$
    成立, 则称向量组 $ \vb*{\alpha}_{1}, \vb*{\alpha}_{2}, \cdots, \vb*{\alpha}_{m} $ \textit{线性相关}; 当且仅当  $k_{1}=k_{2}=\cdots=k_{m}=0 $ 时等式成立, 则称向量组 $ \vb*{\alpha}, \vb*{\alpha}_{2}, \cdots, \vb*{\alpha}_{m} $ \textit{线性无关}.
\end{definition}

\subsection{常用结论}

设
$$\begin{array}{c}
        \vb*{\alpha}_{1}=\left(a_{11}, a_{12}, \cdots, a_{1 n}\right)^{\top}, \quad \vb*{\alpha}_{2}=\left(a_{21}, a_{22}, \cdots, a_{2 n}\right)^{\top}, \cdots, \vb*{\alpha}_{m}=\left(a_{m 1}, a_{m 2}, \cdots, a_{m n}\right)^{\top}, \\
        \vb*{\beta}=\left(b_{1}, b_{2}, \cdots, b_{n}\right)^{\top},
    \end{array}$$
这里 $ m \leqslant n $.
\begin{enumerate}[label=(\arabic{*})]
    \item $\vb*{\beta} $ 可由 $ \vb*{\alpha}_{1}, \vb*{\alpha}_{2}, \cdots, \vb*{\alpha}_{m} $ 线性表示的充要条件是线性方程组 $ x_{1} \vb*{\alpha}_{1}+x_{2} \vb*{\alpha}_{2}+\cdots+   x_{m} \vb*{\alpha}_{m}=\vb*{\beta} $ 有解, 即下列线性方程组有解
          $$\left\{\begin{array}{l}
                  a_{11} x_{1}+a_{21} x_{2}+\cdots+a_{m 1} x_{m}=b_{1},                        \\
                  a_{12} x_{1}+a_{22} x_{2}+\cdots+a_{m 2} x_{m}=b_{2},                        \\
                  \cdots \cdots \cdots \cdots \cdots \cdots \cdots \cdots \cdots \cdots \cdots \\
                  a_{1 n} x_{1}+a_{2 n} x_{2}+\cdots+a_{m n} x_{m}=b_{n} .
              \end{array}\right.$$
    \item \begin{enumerate}
              \item 令 $ \vb*{A}=\left(\vb*{\alpha}_{1}, \vb*{\alpha}_{2}, \cdots, \vb*{\alpha}_{m}\right), \vb*{B}=\left(\vb*{\alpha}_{1}, \vb*{\alpha}_{2}, \cdots, \vb*{\alpha}_{m}, \vb*{\beta}\right) $, 则 $ \vb*{\beta} $ 可由 $ \vb*{\alpha}_{1}, \vb*{\alpha}_{2}, \cdots, \vb*{\alpha}_{m} $ 线性表示的充要条件是以 $ \vb*{\alpha}_{1}, \vb*{\alpha}_{2}, \cdots, \vb*{\alpha}_{m} $ 为列向量的矩阵和以 $ \vb*{\alpha}_{1}, \vb*{\alpha}_{2}, \cdots, \vb*{\alpha}_{m}, \vb*{\beta} $ 为列向量的矩阵有相同的秩, 即 $\rank(\vb*{A})=\rank(\vb*{B}) $.
              \item $ \vb*{\beta} $ 可由 $ \vb*{\alpha}_{1}, \vb*{\alpha}_{2}, \cdots, \vb*{\alpha}_{m} $ 唯一线性表示的充要条件是 $ \rank(\vb*{A})=\rank(\vb*{B})=m $.
              \item $\vb*{\beta} $ 不能由 $ \vb*{\alpha}_{1}, \vb*{\alpha}_{2}, \cdots, \vb*{\alpha}_{m} $ 线性表示的充要条件是 $ \rank(\vb*{A})<\rank(\vb*{B}) $.
          \end{enumerate}
    \item 向量组 $ \vb*{\alpha}_{1}, \vb*{\alpha}_{2}, \cdots, \vb*{\alpha}_{m} $ 线性相关的充要条件是齐次线性方程组
          $$\left\{\begin{array}{l}
                  a_{11} x_{1}+a_{21} x_{2}+\cdots+a_{m 1} x_{m}=0,                            \\
                  a_{12} x_{1}+a_{22} x_{2}+\cdots+a_{m 2} x_{m}=0,                            \\
                  \cdots \cdots \cdots \cdots \cdots \cdots \cdots \cdots \cdots \cdots \cdots \\
                  a_{1 n} x_{1}+a_{2 n} x_{2}+\cdots+a_{m n} x_{m}=0
              \end{array}\right.$$
          有非零解, 且当 $ m=n $ 时, 其线性相关的充要条件是
          $$|\vb*{A}|=\left|\begin{array}{cccc}
                  a_{11}  & a_{21}  & \cdots & a_{n 1} \\
                  a_{12}  & a_{22}  & \cdots & a_{n 2} \\
                  \vdots  & \vdots  &        & \vdots  \\
                  a_{1 n} & a_{2 n} & \cdots & a_{n n}
              \end{array}\right|=0 .$$
    \item 向量组 $ \vb*{\alpha}_{1}, \vb*{\alpha}_{2}, \cdots, \vb*{\alpha}_{m} $ 线性无关的充要条件是齐次线性方程组
          $$\left\{\begin{array}{l}
                  a_{11} x_{1}+a_{21} x_{2}+\cdots+a_{m 1} x_{m}=0,                            \\
                  a_{12} x_{1}+a_{22} x_{2}+\cdots+a_{m 2} x_{m}=0,                            \\
                  \cdots \cdots \cdots \cdots \cdots \cdots \cdots \cdots \cdots \cdots \cdots \\
                  a_{1 n} x_{1}+a_{2 n} x_{2}+\cdots+a_{n m} x_{m}=0
              \end{array}\right.$$
          只有零解, 且当 $ m=n $ 时, 其线性无关的充要条件是
          $$|\vb*{A}|=\left|\begin{array}{cccc}
                  a_{11}  & a_{21}  & \cdots & a_{n 1} \\
                  a_{12}  & a_{22}  & \cdots & a_{n 2} \\
                  \vdots  & \vdots  &        & \vdots  \\
                  a_{1 n} & a_{2 n} & \cdots & a_{n n}
              \end{array}\right| \neq 0 .$$
    \item 向量组 $ \vb*{\alpha}_{1}, \vb*{\alpha}_{2}, \cdots, \vb*{\alpha}_{m} $ 线性相关的充要条件是以  $\vb*{\alpha}_{1}, \vb*{\alpha}_{2}, \cdots, \vb*{\alpha}_{m} $ 为列向量的矩阵的秩小于向量个数 $ m $.
    \item 向量组 $ \vb*{\alpha}_{1}, \vb*{\alpha}_{2}, \cdots, \vb*{\alpha}_{m} $ 线性无关的充要条件是以  $\vb*{\alpha}_{1}, \vb*{\alpha}_{2}, \cdots, \vb*{\alpha}_{m} $ 为列向量的矩阵的秩等于向量个数 $ m $.
    \item 向量组 $ \vb*{\alpha}_{1}, \vb*{\alpha}_{2}, \cdots, \vb*{\alpha}_{m}(m \geqslant 2) $ 线性相关的充要条件是向量组中至少有一个向量是其余向量的线性组合; 向量组 $ \vb*{\alpha}_{1}, \vb*{\alpha}_{2}, \cdots, \vb*{\alpha}_{m}(m \geqslant 2) $ 线性无关的充要条件是向量组中任一个向量都不能由其余向量线性表示.
    \item 如果向量组 $ \vb*{\alpha}_{1}, \vb*{\alpha}_{2}, \cdots, \vb*{\alpha}_{m} $ 线性无关, 而向量组  $\vb*{\alpha}_{1}, \vb*{\alpha}_{2}, \cdots, \vb*{\alpha}_{m}, \vb*{\beta} $ 线性相关, 则 $ \vb*{\beta}$ 可以由 $ \vb*{\alpha}_{1}$, $\vb*{\alpha}_{2}, \cdots, \vb*{\alpha}_{m} $ 线性表示, 且表达式唯一.
    \item 如果向量组 $ \vb*{\alpha}_{1}, \vb*{\alpha}_{2}, \cdots, \vb*{\alpha}_{m} $ 可以由向量组 $ \vb*{\beta}_{1}, \vb*{\beta}_{2} \cdots, \vb*{\beta}_{t} $ 线性表示, 并且 $ m>t $, 则向量组 $ \vb*{\alpha}_{1}$, $\vb*{\alpha}_{2}, \cdots, \vb*{\alpha}_{m} $ 线性相关; 或者说, 如果向量组 $ \vb*{\alpha}_{1}, \vb*{\alpha}_{2}, \cdots, \vb*{\alpha}_{m} $ 线性无关, 并且可以由 $ \vb*{\beta}_{1}, \vb*{\beta}_{2}, \cdots, \vb*{\beta}_{t} $ 线性表示, 则 $ m \leqslant t $.
    \item 在向量组 $ \vb*{\alpha}_{1}, \vb*{\alpha}_{2}, \cdots, \vb*{\alpha}_{m} $ 中, 如果有一个部分组线性相关, 则整个向量组线性相关; 如果整个向量组 $ \vb*{\alpha}_{1}, \vb*{\alpha}_{2}, \cdots, \vb*{\alpha}_{m} $ 线性无关, 则其任一部分组也一定线性无关.
    \item 设 $ r $ 维向量组 $ \vb*{\alpha}_{i}=\left(a_{i 1}, a_{i 2}, \cdots, a_{i r}\right)(i=1,2, \cdots, m) $ 线性无关, 则在每个向量上再添加 $ n-r $ 个分量所得到的 $ n $ 维向量组 $ \vb*{\alpha}^{\prime}{ }_{i}=\left(a_{i 1}, a_{i 2}, \cdots, a_{i r}, a_{i, r+1}, \cdots, a_{i n}\right)(i=1,2, \cdots , m)$ 也线性无关.
    \item $n+1 $ 个 $ n $ 维向量必线性相关.
    \item 一个零向量线性相关; 一个非零向量线性无关; 两个非零向量线性相关的充要条件是对应分量成比例; 含有零向量的向量组必线性相关.
    \item 设 $ \vb*{\varepsilon}_{1}=(1,0, \cdots, 0), \vb*{\varepsilon}_{2}=(0,1, \cdots, 0), \cdots, \vb*{\varepsilon}_{n}=(0,0, \cdots, 1) $, 称 $ \vb*{\varepsilon}_{1}, \vb*{\varepsilon}_{2}, \cdots, \vb*{\varepsilon}_{n} $ 为 $ n $ 维单位向量组, 且
          \begin{enumerate}
              \item $\vb*{\varepsilon}_{1}, \vb*{\varepsilon}_{2}, \cdots, \vb*{\varepsilon}_{n} $ 线性无关;
              \item 任意 $ n $ 维向量 $ \vb*{\alpha}=\left(a_{1}, a_{2}, \cdots, a_{n}\right) $ 都可由 $ \vb*{\varepsilon}_{1}, \vb*{\varepsilon}_{2}, \cdots, \vb*{\varepsilon}_{n} $ 线性表示, 即
                    $$\vb*{\alpha}=a_{1} \vb*{\varepsilon}_{1}+a_{2} \vb*{\varepsilon}_{2}+\cdots+a_{n} \vb*{\varepsilon}_{n}.$$
          \end{enumerate}
    \item 初等行变换不改变矩阵的列向量组之间的线性关系; 初等列变换不改变矩阵的行向量组之间的线性关系.
\end{enumerate}