\section{特殊构型积分补充}

\subsection{Euler 积分}\label{sec:eulerPoints}

Euler 积分在概率论和统计学中有重要的应用, 特别是在 Gauss 分布的概率密度函数中. 这个积分在实际计算中比较困难, 因为它没有一个基本的初等函数的原函数, 因此无法通过常规的积分技巧来计算. 

\subsubsection{Euler 积分及其基本变形}

\begin{definition}[第二型 Euler 积分]
    $\displaystyle \Gamma (\alpha)=\int_{0}^{+\infty}x^{\alpha-1}\e^{-x}\dd x~ (\alpha>0).$
\end{definition}
基本变形:
\begin{flalign*}
    \Gamma(\alpha)=2\int_{0}^{+\infty}t^{2\alpha-1}\e^{-t^2}\dd t~ (\alpha>0),~\Gamma(\alpha)=\int_{0}^{1}\ln^{\alpha-1}\dfrac{1}{t}\dd t~ (\alpha>0).
\end{flalign*}

\begin{theorem}[第二型递推性质]
    $\Gamma(\alpha+1)=\alpha\Gamma(\alpha)$, 并且 $\Gamma(1)=1,~\Gamma(n+1)=n!$.
    \label{gammaalpha1}
    \index{第二型递推性质}
\end{theorem}

又因为 $\Gamma\qty(\dfrac{1}{2})=\sqrt{\pi}$, 由定理 \ref{gammaalpha1} 进行递推, 可得
$$\Gamma\qty(n+\dfrac{1}{2})=\dfrac{(2n-1)!!}{2^n}\sqrt{\pi}.$$

$\Gamma$ 函数只在 $x$ 正半轴有定义, 利用递推性质可定义 $\alpha\leqslant 0$ 时的 $\Gamma(\alpha)$.

\begin{theorem}[第二型余元公式]
    $\Gamma(\alpha)\Gamma(1-\alpha)=\dfrac{\pi}{\sin\alpha\pi}~ (0<\alpha<1).$
    \index{第二型余元公式}
\end{theorem}

\begin{theorem}[第二型倍元公式]
    $\displaystyle \Gamma(2\alpha)=\dfrac{2^{2\alpha-1}}{\sqrt{\pi}}\Gamma(\alpha)\Gamma\qty(\alpha+\dfrac{1}{2})~ (\alpha>0).$
    \index{第二型倍元公式}
\end{theorem}

\begin{definition}[第一型 Euler 积分]
    $\displaystyle \B(p,q)=\int_{0}^{1}x^{p-1}(1-x)^{q-1}\dd x~ (p,q>0).$
\end{definition}

基本变形:
\begin{flalign*}
    \B(p,q)=2\int_{0}^{\frac{\pi}{2}}\cos^{2p-1}\theta\sin^{2q-1}\theta\dd \theta,~\B(p,q)=\int_{0}^{+\infty}\dfrac{u^{p-1}}{(1+u)^{p+q}}\dd u,~\B(p,q)=\int_{0}^{1}\dfrac{x^{p-1}+x^{q-1}}{(1+x)^{p+q}}\dd x.
\end{flalign*}

\begin{theorem}[对称性质]
    $\B(p,q)=\B(q,p).$
    \index{对称性质}
\end{theorem}

\begin{theorem}[第一型递推性质]
    $\B(p,q)=\dfrac{q-1}{p+q-1}\B(p,q-1)=\dfrac{p-1}{p+q-1}\B(p-1,q).$
    特别地, 对于正整数 $m,n$, 有 $$\B(m,n)=\dfrac{(n-1)!(m-1)!}{(m+n-1)!}.$$
    \index{第一型递推性质}
\end{theorem}

\begin{theorem}[第一型余元公式]
    $B(p,1-p)=\dfrac{\pi}{\sin p\pi}~ (0<p<1).$
    特别地, $B\qty(\dfrac{1}{2},\dfrac{1}{2})=\pi.$
    \index{第一型余元公式}
\end{theorem}

\begin{theorem}[Dirichlet 定理]
    $\B(p,q)=\dfrac{\Gamma(p)\Gamma(q)}{\Gamma(p+q)}.$
    \index{Dirichlet 定理}
\end{theorem}

\subsubsection{利用 Euler 积分表示其他积分}

\begin{example}
    已知 $\displaystyle I_n=\int_{1}^{+\infty}\dfrac{\sqrt{t-1}}{t^n}\dd t~~ (n\geqslant 2)$, 求 $\displaystyle\lim_{n\to\infty}\dfrac{I_{n+1}}{I_n}.$
\end{example}
\begin{solution}
    $\displaystyle I_n=\int_{1}^{+\infty}(t-1)^{\frac{1}{2}}t^{-n}\dd t$ 并令 $\dfrac{1}{t}=x$, 于是 
    \begin{flalign*}
        I_n=\int_{0}^{1}\qty(\dfrac{1}{x}-1)^{\frac{1}{2}}x^{n-2}\dd x=\int_{0}^{1}(1-x)^{\frac{1}{2}}x^{n-\frac{5}{2}}\dd x=\B\qty(n-\dfrac{3}{2},\dfrac{3}{2})
    \end{flalign*}
    则 $$\displaystyle\lim_{n\to\infty}\dfrac{I_{n+1}}{I_n}=\lim_{n\to\infty}\dfrac{\B\qty(n-\dfrac{1}{2},\dfrac{3}{2})}{\B\qty(n-\dfrac{3}{2},\dfrac{3}{2})}=\lim_{n\to\infty}\dfrac{n-\dfrac{3}{2}}{n}\cdot\dfrac{B\qty(n-\dfrac{3}{2},\dfrac{3}{2})}{B\qty(n-\dfrac{3}{2},\dfrac{3}{2})}=1.$$
\end{solution}

\subsection{Dirichlet 积分}

\begin{lemma}[Dirichlet 核]
    \label{Dirichlethe}
    在区间 $(0,\pi)$ 上定义 $D_n=\dfrac{\sin\dfrac{(2n+1)x}{2}}{2\sin\dfrac{x}{2}},~n\in\mathbb{N}_+$, 则有 $$\int_{0}^{\pi}D_n(x)\dd x=\dfrac{\pi}{2}.$$
\end{lemma}
\begin{proof}[{\songti \textbf{证}}]
    显然, $D_n$ 在 $x=0$ 处无定义, 但是 $\displaystyle\lim_{x\to0^+}D_n(x)=\dfrac{2n+1}{2}$, 因此 $D_n$ 在 $[0,\pi]$ 可积, 因为有三角恒等式
    $$2\sin\dfrac{x}{2}\qty(\dfrac{1}{2}+\sum_{k=1}^{n}\cos kx\dd x)=\sin\dfrac{(2n+1)x}{2}$$
    于是, $\displaystyle D_n=\dfrac{1}{2}+\sum_{k=1}^{n}\cos kx$, 所以, 不难得到
    $$\int_{0}^{\pi}D_n(x)\dd x=\int_{0}^{\pi}\qty(\dfrac{1}{2}+\sum_{k=1}^{n}\cos kx)\dd x=\dfrac{\pi}{2}.$$
\end{proof}

\begin{theorem}[Dirichlet 积分等式]
    $\displaystyle\int_{0}^{+\infty}\dfrac{\sin x}{x}\dd x=\dfrac{\pi}{2}.$
    \index{Dirichlet 积分等式}
\end{theorem}
\begin{proof}[{\songti \textbf{证}}]
    易知该反常积分条件收敛, 根据引理 \ref{Dirichlethe} 有, $\displaystyle\int_{0}^{\pi}\dfrac{\sin\dfrac{(2n+1)\pi}{2}}{2\sin\dfrac{x}{2}}\dd x=\dfrac{\pi}{2}$, 考虑将其分母换为 $x$ 所产生的影响, 
    由 L'Hospital 法则, 有 $$f(x)=\dfrac{1}{x}-\dfrac{1}{2\sin\dfrac{x}{2}}=O(x)~~ (x\to0)$$
    因此 $f$ 在 $[0,\pi]$ 上常义可积, 由 Riemann-Lebesgue 定理, 有
    $$\lim_{n\to\infty}f(x)\sin\qty(n+\dfrac{1}{2})x\dd x=0$$
    即 $$\lim_{n\to\infty}\int_{0}^{\pi}\dfrac{\sin\qty(n+\dfrac{1}{2})x}{x}\dd x=\lim_{n\to\infty}\int_{0}^{\pi}\dfrac{\sin\qty(n+\dfrac{1}{2})x}{2\sin\dfrac{x}{2}}\dd x=\dfrac{\pi}{2}$$
    最后, 作代换 $t=\qty(n+\dfrac{1}{2})x$, 即得 $$\int_{0}^{\pi}\dfrac{\sin\qty(n+\dfrac{1}{2})x}{x}\dd x=\int_{0}^{\qty(n+\frac{1}{2})\pi}\dfrac{\sin t}{t}\dd t=\dfrac{\pi}{2}.$$
\end{proof}

\subsection{Lobachevsky 积分法}

\begin{theorem}[Lobachevsky 积分法]
    若 $f(x)$ 在 $x\in[0,+\infty)$ 范围内满足 $f(x+\pi)=f(x)$ 及 $f(\pi-x)=f(x)$, 则
    $$\int_{0}^{+\infty}f(x)\dfrac{\sin x}{x}\dd x=\int_{0}^{\frac{\pi}{2}}f(x)\dd x.$$
    当 $f(x)=1$ 时, 便是 Dirichlet 积分.
    \index{Lobachevsky 积分法}
\end{theorem}

\subsection{Fresnel 积分与 Fej\texorpdfstring{$\acute{\text{e}}$}.r 积分}

\begin{theorem}[Fresnel 积分等式]
    $\displaystyle\int_{-\infty}^{+\infty}\sin x^2\dd x=\int_{-\infty}^{+\infty}\cos x^2\dd x=\sqrt{\dfrac{\pi}{2}}.$
    \index{Fresnel 积分等式}
\end{theorem}

\begin{theorem}[Fej$\acute{\text{e}}$r 积分]
    假设 $k\in\mathbb{N}^+$, 则有以下四种形式的 Fej$\acute{\text{e}}$r 积分:
    \begin{enumerate}[label=(\arabic{*})]
        \item $\displaystyle \int_{0}^{\pi}\dfrac{\sin nx}{\sin x}\dd x=\begin{cases}
                      0,   & n=2k    \\
                      \pi, & n=2k-1;
                  \end{cases}$
        \item $\displaystyle\int_{0}^{\frac{\pi}{2}}\dfrac{\sin nx}{\sin x}\dd x=\begin{cases}
                      2\displaystyle\sum_{i=1}^{k}\dfrac{(-1)^{k-1}}{2k-1}, & n=2k    \\[6pt]
                      \dfrac{\pi}{2},                                       & n=2k-1;
                  \end{cases}$
        \item $\displaystyle \int_{0}^{\pi}\qty(\dfrac{\sin nx}{\sin x})^2\dd x=n\pi;$
        \item $\displaystyle\int_{0}^{\frac{\pi}{2}}\qty(\dfrac{\sin nx}{\sin x})^2\dd x=\dfrac{n\pi  }{2}.$
    \end{enumerate}
\end{theorem}

\subsection{Laplace 积分}

\begin{theorem}[Laplace 积分等式]
    $\displaystyle \int_{0}^{+\infty}\dfrac{\cos bx}{a^2+x^2}\dd x=\dfrac{\pi}{2a}\e^{-ab},~\int_{0}^{+\infty}\dfrac{x\sin bx}{a^2+x^2}\dd x=\dfrac{\pi}{2}\e^{-ab}~~(a,b>0).$
    \index{Laplace 积分等式}
\end{theorem}
\begin{inference}
    当 $4q>p^2$ 时, 有
    \begin{flalign*}
        \int_{-\infty}^{+\infty}\dfrac{\cos x}{x^2+px+q}\dd x & =\dfrac{2\pi}{\sqrt{4q-p^2}}\e^{-\frac{\sqrt{4q-p^2}}{2}}\cos\dfrac{p}{2}   \\
        \int_{-\infty}^{+\infty}\dfrac{\sin x}{x^2+px+q}\dd x & =-\dfrac{2\pi}{\sqrt{4q-p^2}}\e^{-\frac{\sqrt{4q-p^2}}{2}}\sin\dfrac{p}{2}.
    \end{flalign*}
\end{inference}
