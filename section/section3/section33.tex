\section{反常积分}

反常积分是指在积分区间上函数有无穷大或无界的情况下的积分, 无法直接通过定积分来计算的积分. 反常积分分为两类:
第一类是无穷积分, 即积分区间为无穷的情况; 第二类是间断积分, 即积分函数在积分区间上有间断点的情况.

\subsection{反常积分的计算}

\begin{definition}[无穷区间上的反常积分]
    设函数 $ f(x) $ 在 $ [a,+\infty) $ 上有定义, 在 $ [a, b]~(b<+\infty) $ 上可积, 若极限 $\displaystyle  \lim _{b \to+\infty} \int_{a}^{b} f(x) \dd  x $ 存在, 则定义
    $$\int_{0}^{+\infty} f(x) \dd  x=\lim _{b \to \infty} \int_{a}^{b} f(x) \dd  x$$
    并称 $\displaystyle  \int_{0}^{+\infty} f(x) \dd  x $ 为 $ f(x) $ 在 $ [a,+\infty) $ 上的\textit{反常积分}, 这时也称反常积分 $\displaystyle  \int_{0}^{+\infty} f(x) \dd  x $ \textit{存在}或\textit{收敛}; 若上述极限不存在, 则称反常积分 $\displaystyle  \int_{a}^{+\infty} f(x) \dd  x $ \textit{不存在}或\textit{发散}.
    类似地, 定义
    \begin{flalign*}
        \int_{-\infty}^{b} f(x) \dd  x       & =\lim _{a \to-\infty} \int_{a}^{b} f(x) \dd  x,                                                                                                                  \\
        \int_{-\infty}^{+\infty} f(x) \dd  x & =\int_{-\infty}^c f(x) \dd  x+\int_{c}^{+\infty} f(x) \dd  x=\lim _{a \to-\infty} \int_{a}^c f(x) \dd  x+\lim _{b \to+\infty} \int_{c}^{b} f(x) \dd  x .
    \end{flalign*}
\end{definition}

\begin{definition}[无界函数的反常积分 (瑕积分)]
    设函数 $ f(x) $ 在 $ [a, b) $ 上连续, 而且 $\displaystyle  \lim _{x \to b^{-}} f(x)=   \infty $, 若极限 $\displaystyle  \lim _{\varepsilon \to 0^{+}} \int_{a}^{b-\varepsilon} f(x) \dd  x $ 存在, 则定义
    $$\int_{a}^{b} f(x) \dd  x=\lim _{\varepsilon \to 0^{+}} \int_{a}^{b-\varepsilon} f(x) \dd  x$$
    并称 $\displaystyle  \int_{a}^{b} f(x) \dd  x $ 为 $ f(x) $ 在 $ [a, b) $ 上的反常积分, 这时也称反常积分 $\displaystyle  \int_{a}^{b} f(x) \dd  x $ 存在或收敛; 若上述极限不存在, 则称反常积分 $\displaystyle  \int_{a}^{b} f(x) \dd  x $ 不存在或发散.\\
                    类似地, 若 $ f(x) $ 在 $ (a, b] $ 上连续, $\displaystyle  \lim _{x \to a^{+}} f(x)=\infty $, 则定义
    $$\int_{a}^{b} f(x) \dd  x=\lim _{\varepsilon \to 0^{+}} \int_{a+\varepsilon}^{b} f(x) \dd  x$$
    若 $ f(x) $ 在 $ (a, b) $ 内连续, $\displaystyle  \lim _{x \to a^{+}} f(x)=\infty, \lim _{x \to b^{-}} f(x)=\infty $, 则定义
    $$\int_{a}^{b} f(x) \dd  x=\lim _{\varepsilon_{1} \to 0^{+}} \int_{a+\varepsilon_{1}}^{c} f(x) \dd  x+\lim _{\varepsilon_{2} \to 0^{+}} \int_{c}^{b-\varepsilon_2} f(x) \dd  x .$$
\end{definition}

\begin{theorem}[反常积分的区间再现公式]
    \index{反常积分的区间再现公式}若 $\displaystyle\int_{0}^{+\infty}f(x)\dd x$ 收敛, 则对任意的正整数 $k$, 都有
    $$\int_{0}^{+\infty}f(x)\dd x=\int_{0}^{+\infty}\dfrac{k}{x^2}f\left(\dfrac{k}{x}\right)\dd x=\dfrac{1}{2}\int_{0}^{+\infty}\left[f(x)+\dfrac{k}{x^2}f\left(\dfrac{k}{x}\right)\right]\dd x.$$
\end{theorem}

\begin{example}
    计算下列反常积分.
    \setcounter{magicrownumbers}{0}
    \begin{table}[H]
        \centering
        \begin{tabular}{l | l | l}
            (\rownumber{}) $\displaystyle\int_{0}^{+\infty}\dfrac{\dd x}{\left(1+x^2\right)\left(1+x^\alpha\right)}~ (\alpha\in\mathbb{R} ).$ & (\rownumber{}) $\displaystyle\int_{0}^{+\infty}\dfrac{\arctan\sqrt[3]{x}}{1+x^2}\dd x.$                  & (\rownumber{}) $\displaystyle\int_{0}^{+\infty}\dfrac{x^2}{\left(1+x^2\right)^2}\dd x.$   \\
            (\rownumber{}) $\displaystyle\int_{0}^{+\infty}\frac{\ln x}{x^2+a^2}\dd x~  (a>0).$                                               & (\rownumber{}) $\displaystyle\int_{0}^{+\infty}\frac{\arctan x}{\left(1+x^2\right)^{\frac{5}{2}}}\dd x.$ & (\rownumber{}) $\displaystyle\int_{0}^{+\infty}\frac{x\ln x}{\left(1+x^2\right)^2}\dd x.$
        \end{tabular}
    \end{table}
\end{example}

\begin{solution}
    \begin{enumerate}[label=(\arabic{*})]
        \item \textbf{法一: }令 $ x=\tan t$, 则 $ \dd  x=\sec ^{2} t \dd  t$,
              \begin{flalign*}
                  \text{原式}  =\int_{0}^{\frac{\pi}{2}}\dfrac{\dd t}{1+\tan^\alpha t}\xlongequal[]{t=\frac{\pi}{2}-u}\int_{0}^{\frac{\pi}{2}}\dfrac{\dd u}{1+\cot^\alpha u}=\int_{0}^{\frac{\pi}{2}}\frac{\tan^\alpha u}{\tan^\alpha u+1}\dd u
                  =\frac{1}{2} \int_{0}^{\frac{\pi}{2}} \frac{\tan ^{\alpha} u+1}{\tan ^{\alpha} u+1} \dd  u=\frac{1}{2} \int_{0}^{\frac{\pi}{2}} \dd  u=\frac{\pi}{4} .
              \end{flalign*}
              \textbf{法二: }令 $ x=\dfrac{1}{t}$, 则 $ \dd  x=-\dfrac{1}{t^{2}} \dd  t$,
              \begin{flalign*}
                  \text{原式} & =\int_{+\infty}^{0} \dfrac{-\dfrac{1}{t^{2}} \dd  t}{\left(1+\dfrac{1}{t^{2}}\right)\left(1+\dfrac{1}{t^{\alpha}}\right)}=\int_{0}^{+\infty} \dfrac{t^{\alpha} \dd  t}{\left(1+t^{2}\right)\left(t^{\alpha}+1\right)}=\int_{0}^{+\infty} \dfrac{x^{\alpha} \dd  x}{\left(1+x^{2}\right)\left(x^{\alpha}+1\right)} \\
                              & =\dfrac{1}{2}\left[\int_{0}^{+\infty} \dfrac{\dd  x}{\left(1+x^{2}\right)\left(1+x^{\alpha}\right)}+\int_{0}^{+\infty} \dfrac{x^{\alpha} \dd  x}{\left(1+x^{2}\right)\left(1+x^{\alpha}\right)}\right]=\dfrac{1}{2} \int_{0}^{+\infty} \dfrac{\dd  x}{1+x^{2}}=\dfrac{\pi}{4}
              \end{flalign*}
              \textbf{法三: }$\displaystyle \int _{0}^{+\infty }\dfrac{\dd x}{\left( 1+x^{2}\right) \left( 1+x^{\alpha }\right) }=\int _{0}^{1}\dfrac{\dd x}{\left( 1+x^{2}\right) \left( 1+x^{\alpha }\right) }+\int _{1}^{+\infty }\dfrac{\dd x}{\left( 1+x^{2}\right) \left( 1+x^{\alpha }\right) }$, 于是
              $$\int_{1}^{+\infty} \dfrac{\dd  x}{\left(1+x^{2}\right)\left(1+x^{\alpha}\right)}=\int_{1}^{0} \dfrac{-\dfrac{1}{t^{2}} \dd  t}{\left(1+\dfrac{1}{t^{2}}\right)\left(1+\dfrac{1}{t^{\alpha}}\right)}=\int_{0}^{1} \dfrac{t^{\alpha} \dd  t}{\left(1+t^{2}\right)\left(1+t^{\alpha}\right)}$$
              于是原式=$\displaystyle\int_{0}^{1} \frac{\dd  x}{\left(1+x^{2}\right)\left(1+x^{\alpha}\right)}+\int_{0}^{1} \frac{x^{\alpha} \dd  x}{\left(1+x^{2}\right)\left(1+x^{\alpha}\right)}=\int_{0}^{1} \frac{1}{1+x^{2}} \dd  x=\frac{\pi}{4} .$\\
              \textbf{法四: }$\displaystyle\text{原式}=\dfrac{1}{2}\int_{0}^{+\infty}\dfrac{\left(1+x^\alpha\right)^{-1}+\left(1+\dfrac{1}{x^\alpha}\right)^{-1}}{1+x^2}\dd x=\dfrac{1}{2}\int_{0}^{+\infty}\frac{\dd x}{1+x^2}=\frac{\pi}{4}.$
        \item $\displaystyle\text{原式}=\dfrac{1}{2}\int_{0}^{+\infty}\dfrac{\arctan\sqrt[3]{x}+\arctan\dfrac{1}{\sqrt[3]{x}}}{1+x^2}\dd x=\frac{\pi}{4}\int_{0}^{+\infty}\dfrac{\dd x}{1+x^2}=\frac{\pi^2}{8}.$
        \item $\displaystyle\text{原式}=\dfrac{1}{2}\int_{0}^{+\infty}\dfrac{\dfrac{x^2}{1+x^2}+\dfrac{1}{1+x^2}}{1+x^2}\dd x=\dfrac{1}{2}\int_{0}^{+\infty}\dfrac{\dd x}{1+x^2}=\dfrac{\pi}{4}.$
        \item \textbf{法一: }令 $x=a\tan t$, 则
              \begin{flalign*}
                   & \text{原式}=\int_{0}^{\frac{\pi}{2}}\frac{\ln(a\tan t)}{a^2\tan^2t+a^2}\cdot a\sec^2t\dd t=\frac{1}{a}\int_{0}^{\frac{\pi}{2}}(\ln a+\ln\tan t)\dd t=\frac{\pi}{2a}\ln a+\frac{1}{a}\int_{0}^{\frac{\pi}{2}}\ln\tan t\dd t \\
                   & \int_{0}^{\frac{\pi}{2}}\ln\tan t\dd t\xlongequal[]{u=\frac{\pi}{2}-t}\int_{\frac{\pi}{2}}^{0}\ln\tan\left(\frac{\pi}{2}-u\right)(-\dd u)=\int_{0}^{\frac{\pi}{2}}\ln\cot u\dd u
              \end{flalign*}
              $\displaystyle\int_{0}^{\frac{\pi}{2}}\ln \tan t\dd t+\int_{0}^{\frac{\pi}{2}}\ln\cot u\dd u=\int_{0}^{\frac{\pi}{2}}\ln(\tan x\cot x)\dd x=0\Rightarrow\int_{0}^{\frac{\pi}{2}}\ln\tan x\dd x=0$,
              原式=$\dfrac{\pi}{2a}\ln a.$\\
              \textbf{法二: }$\displaystyle\int _{0}^{+\infty }\dfrac{\ln x}{x^{2}+a^{2}}\dd x=\int ^{a}_{0}\dfrac{\ln x}{x^{2}+a^{2}}\dd x+\int _{a}^{+\infty }\dfrac{\ln x}{x^{2}+a^{2}}\dd x$, 其中
              \begin{flalign*}
                  \int _{a}^{+\infty }\dfrac{\ln x}{x^{2}+a^{2}}\dd x\xlongequal[]{x=\frac{a^2}{t}}\int _{a}^{0}\dfrac{\ln \dfrac{a^{2}}{t}}{\dfrac{a^{4}}{t^{2}}+a^{2}}\left( -\dfrac{a^{2}}{t^{2}}\right) \dd t=\int _{0}^{a}\dfrac{\ln a^{2}-\ln t}{t^{2}+a^{2}}\dd t
              \end{flalign*}
              于是, $\displaystyle\int ^{+\infty }_{0}\dfrac{\ln x}{x^{2}+a^{2}}\dd x=\int _{0}^{a}\dfrac{\ln a^{2}}{x^{2}+a^{2}}\dd x=2\ln a\left( \dfrac{1}{a}\arctan \dfrac{x}{a}\right) _{0}^{a}=\dfrac{\pi }{2a}\ln a.$\\
              \textbf{法三: }$\displaystyle\text{原式}=\frac{1}{a^2}\int_{0}^{+\infty}\frac{\ln x}{\left(\dfrac{x}{a}\right)^2+1}\dd x\xlongequal[]{\frac{x}{a}=t}\frac{1}{a}\int_{0}^{+\infty}\frac{\ln at}{t^2+1}\dd t=\frac{1}{2a}\int_{0}^{+\infty}\frac{\ln at+\ln\dfrac{a}{t}}{t^2+1}\dd t=\frac{\pi}{2a}\ln a.$
        \item 令 $x=\tan t$, 则原式=$\displaystyle \int_{0}^{\frac{\pi}{2}}\frac{t\dd t}{\sec^3t}=\int_{0}^{\frac{\pi}{2}}t\cos^3t\dd t$, 由表格积分法:
              \begin{table}[H]
                  \centering
                  \begin{tabular}{l| c c c}
                      $f'$   & $t$         & $1$                                            & $0$                                       \\
                      \midrule
                             & $+\searrow$ & $-\searrow$                                                                                \\
                      \midrule
                      $\int$ & $\cos^3t$   & $\dfrac{1}{3}\cos^2t\sin t+\dfrac{2}{3}\sin t$ & $-\dfrac{1}{9}\cos^3t-\dfrac{2}{3}\cos t$
                  \end{tabular}
              \end{table}
              $\displaystyle\text{原式}=\frac{t}{3}\cos^2t\sin t\dfrac{2t}{3}\sin t+\dfrac{1}{9}\cos^3t+\dfrac{2}{3}\cos t\bigg |_0^{\frac{\pi}{2}}=\frac{\pi}{3}-\frac{7}{9}.$
        \item $\displaystyle\text{原式}=\dfrac{1}{2}\int_{0}^{+\infty}\dfrac{\dfrac{x\ln x}{1+x^2}+\dfrac{\frac{1}{x}\ln\frac{1}{x}}{1+\left(\frac{1}{x}\right)^2}}{1+x^2}\dd x=\dfrac{1}{2}\int_{0}^{+\infty}\dfrac{\dfrac{x\ln x}{1+x^2}+\dfrac{-x\ln x}{1+x^2}}{1+x^2}\dd x=0.$
    \end{enumerate}
\end{solution}

\begin{example}
    计算下列反常积分.
    \setcounter{magicrownumbers}{0}
    \begin{table}[H]
        \centering
        \begin{tabular}{l | l | l}
            (\rownumber{}) $\displaystyle\int_{3}^{+\infty}\dfrac{\dd x}{(x-1)^4\sqrt{x^2-2x}}.$ & (\rownumber{}) $\displaystyle\int_{0}^{+\infty}\dfrac{x\e ^{-x}}{\qty(1+\e ^{-x})^2}.$ & (\rownumber{}) $\displaystyle\int_{1}^{+\infty}\dfrac{\dd x}{\e ^{1+x}+\e ^{3-x}}.$ \\
        \end{tabular}
    \end{table}
\end{example}
\begin{solution}
    \begin{enumerate}[label=(\arabic{*})]
        \item 原式化为 $\displaystyle\int_{3}^{+\infty}\dfrac{\dd x}{(x-1)^4\sqrt{(x-1)^2-1}}$, 令 $x-1=\sec\theta$, 则
              \begin{flalign*}
                  \text{原式} & =\int_{\frac{\pi}{3}}^{\frac{\pi}{2}}\dfrac{\sec\theta\tan\theta}{\sec^4\theta\tan\theta}\dd \theta=\int_{\frac{\pi}{3}}^{\frac{\pi}{2}}\cos^3\theta\dd \theta=\dfrac{1}{3}\cos^2\sin\theta+\dfrac{2}{3}\sin\theta\biggl |_{\frac{\pi}{3}}^{\frac{\pi}{2}}=\dfrac{2}{3}-\dfrac{3\sqrt{3}}{8}.
              \end{flalign*}
        \item 由于 $\qty(\dfrac{1}{1+\e ^{-x}})'=\dfrac{\e ^{-x}}{\qty(\e ^{-x}+1)^2}$, 由分部积分法, 得
              \begin{flalign*}
                  F(x)  =\int\dfrac{x\e ^{-x}}{\qty(1+\e ^{-x})^2}\dd x=\int x\dd \qty(\dfrac{1}{1+\e ^{-x}})=\dfrac{x}{1+\e ^{-x}}-\int\dfrac{\dd x}{1+\e ^{-x}}
                  =\dfrac{x}{1+\e ^{-x}}-\ln\qty(\e ^{-x}+1)+C
              \end{flalign*}
              于是最终的积分化为 $$\int_{0}^{+\infty}\dfrac{x\e ^{-x}}{\qty(1+\e ^{-x})^2}\dd x=F(+\infty)-F(0)=\lim_{x\to+\infty}\qty[\dfrac{x}{1+\e ^{-x}}-\ln\qty(\e ^x+1)]+\ln 2$$
              \begin{flalign*}
                  F(+\infty) & =\lim_{x\to+\infty}\qty[\dfrac{x}{1+\e ^{-x}}-\ln\qty(\e ^x+1)]=\lim_{x\to+\infty}\dfrac{x\e ^x-\e ^x\ln\qty(\e ^x+1)-\ln\qty(\e ^x+1)}{\e ^x+1}            \\
                             & =\lim_{x\to+\infty}\dfrac{\e ^x\qty[x-\ln\qty(\e ^x+1)]}{\e ^x}=\lim_{x\to+\infty}\qty[x-\ln\e ^x\qty(1+\e ^{-x})]=-\lim_{x\to+\infty}\ln\qty(1+\e ^{-x})=0
              \end{flalign*}
              即 $\displaystyle\int_{0}^{+\infty}\dfrac{x\e ^x}{\qty(1+\e ^{-x})^2}\dd x=\ln 2.$
        \item 由题意可得
              \begin{flalign*}
                  I=\dfrac{1}{\e }\int_{1}^{+\infty}\dfrac{\dd x}{\e ^x+\e ^{2-x}}=\dfrac{1}{\e }\int_{1}^{+\infty}\dfrac{\dd \e ^x}{\e ^{2x}+\e ^2}=\dfrac{1}{\e ^2}\arctan\e ^{x-1}\biggl |_1^{+\infty}=\dfrac{\pi}{4}\e ^{-2}.
              \end{flalign*}
    \end{enumerate}
\end{solution}

\begin{example}
    已知 $\displaystyle\lim_{x\to\infty}\qty(\dfrac{x-a}{x+a})^x=\int_{a}^{+\infty}4x^2\e ^{-2x}\dd x$, 求常数 $a$ 的值.
\end{example}
\begin{solution}
    等式左边易化简为 $\e ^{-2a}$, 等式右边由表格积分法得 $$\e ^{-2a}=\e ^{-2x}\qty(-2x^2-2x-1)\biggl |_{a}^{+\infty}=\e ^{-2a}\qty(2a^2+2a+1)$$
    即 $2a^2+2a+1=1$, 解得 $a=0\text{ 或 }-1.$
\end{solution}

\begin{example}
    试证: $\displaystyle\int_{0}^{\frac{\pi}{2}}\dfrac{x\cos x\sin x}{\qty(a^2\cos^2 x+b^2\sin^2 x)^2}\dd x=\dfrac{\pi}{4ab^2(a+b)}~ (a,b>0,a\neq b).$
\end{example}
\begin{proof}[{\songti \textbf{证}}]
    注意到 $(\tan x)'=\sec^2 x$, 于是
    \begin{flalign*}
        I & =\int_{0}^{\frac{\pi}{2}}\dfrac{x\cos x\sin x}{\qty(a^2\cos^2x+b^2\sin ^2x)^2}\dd x=\int_{0}^{\frac{\pi}{2}}\dfrac{x\tan x}{\qty(a^2+b^2\tan^2x)^2}\dd (\tan x)\xlongequal{\tan x= t}\int_{0}^{+\infty}\dfrac{t\arctan t}{\qty(a^2+b^2t^2)^2}\dd t \\
          & =\dfrac{-1}{2b^2}\int_{0}^{+\infty}\arctan t\dd \qty(\dfrac{1}{a^2+b^2t^2})=\dfrac{-1}{2b^2}\qty[\dfrac{\arctan t}{a^2+b^2t^2}\biggl |_{0}^{+\infty}-\int_{0}^{+\infty}\dfrac{\dd t}{\qty(a^2+b^2t^2)\qty(1+t^2)}]                                 \\
          & =\dfrac{1}{2b^2\qty(b^2-a^2)}\int_{0}^{+\infty}\qty(\dfrac{b^2}{a^2+b^2t^2}-\dfrac{1}{1+t^2})\dd t=\dfrac{1}{2b^2\qty(b^2-a^2)}\qty(\dfrac{b}{a}\arctan\dfrac{bt}{a}-\arctan t)\biggl |_{0}^{+\infty}=\dfrac{\pi}{4ab^2(a+b)}.
    \end{flalign*}
\end{proof}

\begin{example}
    \label{fxdfracx12x1}设 $f(x)=\dfrac{(x+1)^2(x-1)}{x^3(x-2)}$, 求 $\displaystyle I=\int_{-1}^{3}\dfrac{f'(x)}{1+f^2(x)}\dd x.$
\end{example}
\begin{solution}
    易得 $x=0,2$ 为 $f(x)$ 的无穷间断点, 故积分为反常积分, 且有 $$I=\int_{-1}^{0}+\int_{0}^{2}+\int_{2}^{3}$$
    且 $\displaystyle\int\dfrac{f'(x)}{1+f^2(x)}\dd x=\arctan f(x)+C$, 于是
    \begin{flalign*}
        I=\arctan f(x)\biggl |_{-1}^{0^-}+\arctan f(x)\biggl |_{0^+}^{2^-}+\arctan f(x)\biggl |_{2^+}^{3}\arctan\dfrac{32}{27}-2\pi.
    \end{flalign*}
\end{solution}

\begin{example}
    计算反常积分 $\displaystyle I=\int_{-\infty}^{+\infty}|t-x|^{\frac{1}{2}}\frac{y}{(t-x)^2+y^2}\dd t.$
\end{example}
\begin{solution}
    \begin{flalign*}
        I & =\int_{-\infty}^{+\infty}|t-x|^{\frac{1}{2}}\frac{y}{(t-x)^2+y^2}\dd t
        \xlongequal[]{u=t-x}\int_{-\infty}^{+\infty}|u|^{\frac{1}{2}}\frac{y}{u^2+y^2}=2\int_{0}^{+\infty}\frac{|u|^{\frac{1}{2}}y\dd u}{u^2+y^2} \\
          & \xlongequal[]{v=\frac{u^{\frac{1}{2}}}{\sqrt[]{y}}}4\sqrt{y}\int_{0}^{+\infty}\frac{v^2\dd v}{v^4+1}
        =2\sqrt{y}\int_{0}^{+\infty}\frac{v^2+1}{v^4+1}\dd v
        =2\sqrt{y}\int_{0}^{+\infty}\frac{1+\dfrac{1}{v^2}}{v^2+\dfrac{1}{v^2}}\dd v                                                              \\
          & =2\sqrt{y}\int_{0}^{+\infty}\frac{\dd \left(v-\dfrac{1}{v}\right)}{\left(v-\dfrac{1}{v}\right)^2+2}
        =2\sqrt{y}\int_{0}^{+\infty}\cdot\frac{1}{\sqrt{2}}\arctan\left(v-\frac{1}{v}\right)\bigg |_0^{+\infty}=\sqrt{2}\pi\sqrt{y}.
    \end{flalign*}
\end{solution}

\begin{theorem}
    若 $a,b>0$, 则
    $$\int_{0}^{+\infty}f\left(ax+\frac{b}{x}\right)\dd x=\frac{1}{a}\int_{0}^{+\infty}f\left(\sqrt{t^2+4ab}\right)\dd t$$
    其中函数 $f(x)$ 的积分均有意义.
\end{theorem}
\begin{proof}[{\songti \textbf{证}}]
    令 $\displaystyle ax-\frac{b}{x}=t$, 那么 $\displaystyle x=\frac{1}{2a}\left(t+\sqrt{t^2+4ab}\right)$, $\displaystyle\dd x=\frac{t+\sqrt{t^2+4ab}}{2a\sqrt{t^2+4ab}}\dd t.$
    \begin{flalign*}
        I & =\int_{0}^{+\infty}f\left(ax+\frac{b}{x}\right)\dd x=\frac{1}{2a}\left(\int_{-\infty}^{0}+\int_{0}^{+\infty}\right)f\left(\sqrt{t^2+4ab}\right)\frac{t+\sqrt{t^2+4ab}}{\sqrt{t^2+4ab}}\dd t             \\
          & =\frac{1}{2a}\left[\int_{0}^{+\infty}f\left(\sqrt{u^2+4ab}\right)\frac{\sqrt{u^2+4ab}-u}{\sqrt{u^2+4ab}}\dd u+\int_{0}^{\infty}f\left(t^2+4ab\right)\frac{t+\sqrt{t^2+4ab}}{\sqrt{t^2+4ab}}\dd t\right] \\
          & =\frac{1}{a}\int_{0}^{+\infty}f\left(\sqrt{t^2+4ab}\right)\dd t.
    \end{flalign*}
\end{proof}
\begin{example}
    已知 $\displaystyle\int_{0}^{+\infty}\e ^{-x^2}\dd x=\frac{\sqrt{\pi}}{2}$,
    试证 $\displaystyle\int_{0}^{+\infty}\e ^{-a^2x^2-\frac{b^2}{x^2}}\dd x=\frac{\sqrt{\pi}}{2a}\e ^{-2ab}.$
\end{example}
\begin{proof}[{\songti \textbf{证}}]
    $\displaystyle I =\int_{0}^{+\infty}\e ^{-\left(ax+\frac{b}{x}\right)^2+2ab}\dd x =\e ^{2ab}\int_{0}^{+\infty}\e ^{-\left(ax+\frac{b}{x}\right)^2}\dd x=\frac{\e ^{2ab}}{a}\int_{0}^{+\infty}\e ^{-(x^2+4ab)}\dd x=\frac{\sqrt{\pi}}{2a}\e ^{-2ab}.$
\end{proof}

\begin{example}
    计算积分 $\displaystyle\int_{0}^{\frac{\pi}{2}}\cos(2nx)\ln(\cos x)\dd x.$
\end{example}
\begin{solution}
    \begin{flalign*}
        I & =\int_{0}^{\frac{\pi}{2}}\cos(2nx)\ln(\cos x)\dd x=\frac{1}{2n}\int_{0}^{\frac{\pi}{2}}\ln(\cos x)\dd \sin(2nx)
        =\frac{1}{2n}\ln(\cos x)\sin(2nx)\bigg |_0^{\frac{\pi}{2}}-\frac{1}{2n}\int_{0}^{\frac{\pi}{2}}\frac{\sin(2nx)\cdot(-\sin x)}{\cos x}\dd x                                                                                               \\
          & =\frac{1}{2n}\int_{0}^{\frac{\pi}{2}}\frac{\sin(2nx)\sin x}{\cos x}\dd x=\frac{1}{2n}\int_{0}^{\frac{\pi}{2}}\cos(2nx)\dd x-\frac{1}{2n}\int_{0}^{\frac{\pi}{2}}\frac{\cos(2n+1)}{\cos x}\dd x                                       \\
          & \xlongequal[]{x=t-\frac{\pi}{2}}\frac{(-1)^{n-1}}{2n}\int_{\frac{\pi}{2}}^{\pi}\frac{\sin(2n+1)t}{\sin t}\dd t=\frac{(-1)^{n-1}}{2n}\int_{\frac{\pi}{2}}^{\pi}\left[1+2\sum_{k=1}^{n}\cos(2kt)\right]\dd t=(-1)^{n-1}\frac{\pi}{4n}.
    \end{flalign*}
\end{solution}

\begin{example}
    设 $m,n$ 为自然数, 求 $\displaystyle\int_{0}^{1}t^n\ln^mt\dd t.$
\end{example}
\begin{solution}
    利用分部积分法获得递推公式,
    \begin{flalign*}
        I_m & =\int_{0}^{1}t^n\ln^mt\dd t=\frac{1}{n+1}\int_{0}^{1}\ln^mt\dd t^{n+1}
        =\frac{1}{n+1}\ln^mt\cdot t^{n+1}\bigg |_0^1-\frac{m}{n+1}\int_{0}^{1}t^n\ln^{m-1}t\dd t=-\frac{m}{n+1}I_{m-1} \\
            & =\left(-\frac{m}{n+1}\right)\left(-\frac{m-1}{n+1}\right)\cdots\left(-\frac{1}{n+1}\right)I_0
        =(-1)^m\frac{m!}{(n+1)^m}\int_{0}^{1}t^n\dd t=(-1)^m\frac{m!}{(n+1)^{m+1}}.
    \end{flalign*}
\end{solution}

\begin{theorem}[Euler 积分]
    $\displaystyle I=\int_{0}^{\frac{\pi}{2}}\ln\sin x\dd x=\int_{0}^{\frac{\pi}{2}}\ln \cos x\dd x=-\dfrac{\pi}{2}\ln 2.$
    \index{Euler 积分}
\end{theorem}
\begin{proof}[{\songti \textbf{证}}]
    被积函数 $ f(x)=\ln \sin x $ 在 $ x=0 $ 点的右邻域内无界, 由 L'Hospital 法则求得 $ \displaystyle \lim _{x \to 0^{+}} \frac{\ln \sin x}{x^{-\frac{1}{2}}}=0$,
    故由广义积分的比较判别法知广义积分 $ \displaystyle \int_{0}^{\frac{\pi}{2}} \ln \sin x \dd  x $ 收敛,
    同理可证: 广义积分 $ \displaystyle \int_{0}^{\frac{\pi}{2}} \ln \cos x \dd  x $ 收敛,
    令 $ x=\dfrac{\pi}{2}-t$, 则
    $$\int_{0}^{\frac{\pi}{2}} \ln \sin x \dd  x=\int_{\frac{\pi}{2}}^{0} \ln \sin \left(\frac{\pi}{2}-t\right)(-\dd  t)=\int_{0}^{\frac{\pi}{2}} \ln \cos t \dd  t=\int_{0}^{\frac{\pi}{2}} \ln \cos x \dd  x$$
    \begin{flalign*}
        I & =\int_{0}^{\frac{\pi}{2}}\ln\sin x\dd x\xlongequal[]{x=2t}\int_{0}^{\frac{\pi}{4}}2\ln\sin 2t\dd t=2\ln2\cdot\frac{\pi}{4}+2\int_{0}^{\frac{\pi}{4}}\ln\sin t\dd t+2\int_{0}^{\frac{\pi}{4}}\ln\cos t\dd t \\
          & =\frac{\pi}{2}\ln2+2\int_{0}^{\frac{\pi}{4}}\ln\sin t\dd t+2\int_{\frac{\pi}{4}}^{\frac{\pi}{2}}\ln\sin u\dd u~~\left(\text{这里 }u=\frac{\pi}{2}-t\right)=\frac{\pi}{2}\ln2+2I
    \end{flalign*}
    解方程得 $\displaystyle I=-\frac{\pi}{2}\ln2.$
\end{proof}

\begin{example}
    计算极限 $$\lim_{n\to\infty}n^3\qty(\tan\int_{0}^{\pi}\sqrt[n]{\sin x}\dd x+\sin\int_{0}^{\pi}\sqrt[n]{\sin x}\dd x).$$
\end{example}
\begin{solution}
    由 $\tan(A-B)=\dfrac{\tan A-\tan B}{1+\tan A\tan B},\sin(A-B)=\sin A\cos B+\cos A\sin B,\tan x-\sin x\sim\dfrac{x^3}{2}(x\to0)$
    \begin{flalign*}
        \text{原式} & =\lim_{n\to\infty}n^3\qty[\tan\int_{0}^{\pi}\qty(\sqrt[n]{\sin x}-1)\dd x-\sin\int_{0}^{\pi}\qty(\sqrt[n]{\sin x}-1)\dd x]=\dfrac{1}{2}\lim_{n\to\infty}\qty[n\int_{0}^{\pi}\qty(\sqrt[n]{\sin x}-1)\dd x]^3                                                                            \\
                    & =\dfrac{1}{2}\lim_{n\to\infty}\qty[\dfrac{\displaystyle\int_{0}^{\pi}\qty(\sqrt[n]{\sin x}-1)\dd x}{\dfrac{1}{n}}]^3 \xlongequal[]{\frac{1}{n}=t}\dfrac{1}{2}\qty[\int_{0}^{\pi}\lim_{t\to0^+}\dfrac{\sin^tx-1}{t}] \xlongequal[]{L'}\dfrac{1}{2}\qty[\int_{0}^{\pi}\ln(\sin x)\dd x]^3 \\
                    & =\dfrac{1}{2}\qty[\int_{0}^{\frac{\pi}{2}}\ln(\sin x)\dd x+\int_{\frac{\pi}{2}}^{\pi}\ln(\sin x)\dd x]^3=-\dfrac{(\pi \ln 2)^3}{2}.
    \end{flalign*}
\end{solution}

\begin{example}\scriptsize\linespread{0.8}
    计算积分 $\displaystyle I_n=\int_{1}^{+\infty}\frac{\dd x}{\displaystyle\prod\limits_{k=0}^{n}(x+k)}.$
\end{example}
\begin{solution}\scriptsize\linespread{0.8}
    设 $$\frac{1}{\displaystyle\prod\limits_{k=0}^{n}(x+k)}=\sum_{k=0}^{n}\frac{A_k}{x+k}$$ 其中 $A_k$ 为待定系数. 将 $\displaystyle\prod_{k=0}^{n}(x+k)$ 同乘等式两边, 然后令 $x\to-k$, 得
    \begin{flalign*}
        A_k=\frac{1}{(-k)(-k+1)\cdots(-1)\cdot1\cdot2\cdot\cdots\cdot(-k+n)}
        =(-1)^k\frac{1}{k!(n-k)!}=(-1)^k\frac{\mathrm{C}_n^k}{n!}
    \end{flalign*}
    \begin{flalign*}
        I_n=\int_{1}^{+\infty}\left[\sum_{k=0}^{n}(-1)^k\frac{\mathrm{C}_n^k}{n!}\frac{1}{x+k}\right]\dd x=\sum_{k=0}^{n}(-1)^k\frac{\mathrm{C}_n^k}{n!}\int_{1}^{+\infty}\frac{\dd x}{x+k}
        =\frac{1}{n!}\sum_{k=0}^{n}(-1)^k\mathrm{C}_n^k\ln(x+k)\bigg |_1^{+\infty}
    \end{flalign*}
    注意到
    \begin{flalign*}
        \sum_{k=0}^{n}(-1)^k\mathrm{C}_n^k\ln(x+k) & =\sum_{k=0}^{n}(-1)^k\mathrm{C}_n^k\ln\left[x\left(1+\frac{k}{x}\right)\right]=\ln x\cdot\sum_{k=0}^{n}(-1)^k\mathrm{C}_n^k+\sum_{k=0}^{n}(-1)^k\mathrm{C}_n^k\ln\left(1+\frac{k}{x}\right) \\
                                                   & =\ln x\cdot(1-1)^n+\sum_{k=0}^{n}(-1)^k\mathrm{C}_n^k\ln\left(1+\frac{k}{x}\right)\to0~~(x\to+\infty)
    \end{flalign*}
    因此 $\displaystyle I_n=\frac{1}{n!}\sum_{k=1}^{n}(-1)^{k+1}\ln(1+k).$
\end{solution}

\subsection{反常积分敛散性}

% \begin{theorem}
%     Abel 判别法: 若 $\displaystyle\int_a^{+\infty}f(x)\dd x$ 收敛, 且 $x\nearrow +\infty$ 时, $g(x)$ 单调有界, 则 $\displaystyle\int_a^{+\infty}f(x)g(x)\dd x$ 收敛.
% \end{theorem}

\subsubsection{积分的收敛域}

\begin{example}
    求积分的收敛域.
    \setcounter{magicrownumbers}{0}
    \begin{table}[H]
        \centering
        \begin{tabular}{l | l}
            (\rownumber{}) $\displaystyle\int_{a}^{+\infty}\dfrac{\e ^{-ax}}{1+x^2}\dd x.$ & (\rownumber{}) $\displaystyle\int_{0}^{+\infty}\dfrac{\sin x^q}{x^p}\dd x.$
        \end{tabular}
    \end{table}
\end{example}
\begin{solution}
    \begin{enumerate}[label=(\arabic{*})]
        \item 当 $a\geqslant 0$ 时, $\displaystyle \int_{a}^{+\infty}\dfrac{\e ^{-ax}}{1+x^2}\dd x\leqslant \int_{a}^{+\infty}\dfrac{\dd x}{1+x^2}=\arctan x\biggl |_a^{+\infty}=\dfrac{\pi}{2}-\arctan a$, 原积分收敛;
              当 $a<0$ 时, 原积分显然发散, 于是, 积分 $\displaystyle\int_{a}^{+\infty}\dfrac{\e ^{-ax}}{1+x^2}\dd x$ 的收敛域为 $a\geqslant 0$ 的一切 $a$ 值.
        \item 若 $q=0$, 则由于积分 $\displaystyle\int_{A}^{+\infty}\dfrac{\dd x}{x^p}$ 仅当 $p>1$ 时收敛, 而积分 $\displaystyle\int_{0}^{A}\dfrac{\dd x}{x^p}$ 仅当 $p\leqslant1$ 时收敛, 故积分
              $\displaystyle\int_{0}^{+\infty}\dfrac{\sin 1}{x^p}\dd x$ 对于任何的 $p$ 值及 $q=0$ 发散;\\
              若 $q\neq 0$, 则积分 $\displaystyle\int_{0}^{+\infty}\dfrac{\sin x^q}{x^p}\dd x=\int_{0}^{+\infty}x^{-p}\sin x^q\dd x$
    \end{enumerate}
\end{solution}

\subsubsection{敛散性的判断}

常用用的反常积分敛散性判别.
\setcounter{magicrownumbers}{0}
\begin{table}[H]
    \centering
    \caption{常用用的反常积分敛散性判别}
    \begin{tabular}{l l}
        (\rownumber{}) $\displaystyle \int_{a}^{+\infty} \dfrac{1}{x^{p}} \dd x\begin{cases}
        p>1,&\text{收敛} \\ p\leqslant 1,&\text{发散}
    \end{cases} a>0.$                              & (\rownumber{}) $\displaystyle \int_{a}^{b} \dfrac{1}{(x-a)^{p}} \dd x\begin{cases}
        p<1,&\text{收敛}\\ p\geqslant 1,&\text{发散}.
    \end{cases}$\\
    \midrule
    (\rownumber{}) $ \displaystyle \int_{a}^{+\infty} x^{k}\e ^{-\lambda x} \dd x\begin{cases}
        \lambda>0,&\text{收敛} \\ \lambda\leqslant 0,&\text{发散}
    \end{cases} k\geqslant 0.$ & (\rownumber{}) $ \displaystyle \int_{a}^{+\infty} \dfrac{1}{x\ln ^p x} \dd x\begin{cases}
        p>1,&\text{收敛} \\ p\leqslant 1,&\text{发散}
    \end{cases} a>1.$
    \end{tabular}
\end{table}

\begin{theorem}[反常积分标准形敛散性判定]
    将任意反常积分化为标准型 $\displaystyle  \int \frac{1}{x^{\alpha} \ln ^{\beta} x} \dd x $, 那么
    \begin{enumerate}[label=(\arabic{*})]
        \item 当 $ x \to 0 $ (瑕点), 当且仅当 $\alpha<1$ 或 $\begin{cases}
            \alpha=1\\ \beta>1
        \end{cases} $ 时收敛, 其余情况均发散;
        \item 当 $ x \to \infty $, 当且仅当 $ \alpha>1 $ 或 $\begin{cases}
            \alpha=1\\ \beta>1
        \end{cases} $ 时收敛, 其余情况均发散.
    \end{enumerate}
    \label{fcjfbzxing}
\end{theorem}

\begin{example}
    设反常积分 $\displaystyle \int_{1}^{+\infty} x^{k}\qty(\e ^{-\cos \frac{1}{x}}-\e ^{-1}) \dd x$ 收敛, 则正确的是 
    \begin{tasks}(4)
        \task $k>-1$
        \task $k<-1$
        \task $k>1$
        \task $k<1$
    \end{tasks}
\end{example}
\begin{solution}
    当 $x\to+\infty$ 时, $x^{k}\qty(\e ^{-\cos \frac{1}{x}}-\e ^{-1})\sim \dfrac{1}{2\e x^{2-k}}$, 
    因为 $\displaystyle \int_{1}^{+\infty} \dfrac{\dd x}{x^{p}} \dd x$, 当 $p\leqslant 1$ 时, 发散; 当 $p>1$ 时, 收敛, 所以 $2-k>1$, 即 $k<1$, 选 D.
\end{solution}

\begin{example}[2022 数二]
    设 $p$ 为常数, 若反常积分 $\displaystyle \int_{0}^{1} \dfrac{\ln x}{x^{p}(1+x)^{1-p}} \dd x$ 收敛, 则 $p$ 的取值范围是
    \begin{tasks}(4)
        \task $(-1,1)$
        \task $(-1,2)$
        \task $(-\infty,1)$
        \task $(-\infty,2)$
    \end{tasks}
\end{example}
\begin{solution}
    $x=0$ 为瑕点, 由定理 \ref{fcjfbzxing} 情况 (1) 知, $\beta=-1$, 那么 $p<1$; $x=1$ 也为瑕点, 由定理 \ref{fcjfbzxing} 情况 (1) 知, 注意此时化为标准形时, $x=1$ 对于 $x^{p}$ 非瑕点, 分子 $\ln x$ 需等价为 $x-1$, 分子 $x-1$ 与分母 $(1-x)^{1-p}$ 合并为 $(1-x)^{-p}$, 故当且仅当 $-p<1$, 即 $p>-1$ 时, 收敛, 综上所述, $p$ 的取值范围为 $(-1,1).$
\end{solution}

\begin{example}
    设函数 $f(x)=\begin{cases}
        \displaystyle \dfrac{1}{(x-1)^{\alpha-1}},&1<x<\e \\ 
        \displaystyle \dfrac{1}{x\ln^{\alpha+1}x},&x\geqslant \e
    \end{cases}$ 若反常积分 $\displaystyle \int_{1}^{+\infty} f(x) \dd x$ 收敛, 则 
    \begin{tasks}(4)
        \task $\alpha<-2$
        \task $\alpha>2$
        \task $-2<\alpha<2$
        \task $0<\alpha<2$
    \end{tasks}
\end{example}
\begin{solution}
    $ \displaystyle \int_{1}^{+\infty} f(x) \dd x=\int_{1}^{\e } \dfrac{1}{(x-1)^{\alpha-1}} \dd x+\int_{\e }^{+\infty} \dfrac{1}{x\ln^{\alpha+1}x} \dd x $, 当 $\alpha-1<1$, 即 $\alpha<2$ 时 $\displaystyle \int_{1}^{\e } \dfrac{1}{(x-1)^{\alpha-1}} \dd x$ 收敛, 
    $$
    \int_{\e }^{+\infty} \dfrac{1}{x\ln ^{\alpha+1}x} \dd x=\int_{\e }^{+\infty} \dfrac{\dd \ln x}{\ln ^{\alpha+1}x} = \int_{1}^{+\infty} \dfrac{1}{u^{\alpha+1}} \dd x
    $$
    则当 $\alpha>0$ 时, $\displaystyle \int_{\e }^{+\infty} \dfrac{1}{x\ln ^{\alpha+1}x} \dd x$ 收敛, 故 $0<\alpha<2$ 时原积分收敛, 选 D.
\end{solution}

\begin{example}
    判断广义积分 $\displaystyle\int_{1}^{+\infty}\dfrac{\qty(\e ^{\frac{1}{x}}-1)^\alpha}{\ln^\beta \qty(1+\dfrac{1}{x})}\dd x$ 收敛性.
\end{example}
\begin{solution}
    当 $x\to+\infty $ 时, $\e ^{\frac{1}{x}}-1\sim \dfrac{1}{x},~\ln\qty(1+\dfrac{1}{x})$, 所以 $\displaystyle \dfrac{\qty(\e ^{\frac{1}{x}}-1)^\alpha}{\ln^\beta \qty(1+\dfrac{1}{x})}\sim \dfrac{1}{x^{\alpha-\beta}}~  (x\to+\infty)$,
    由此可知, 当 $\alpha-\beta>1$ 时, 原积分收敛, 其余情况均发散.
\end{solution}

\begin{example}
    判定反常积分 $\displaystyle\int_{0}^{+\infty}\dfrac{x\dd x}{1+x^6\sin^2x}$ 的敛散性.
\end{example}
\begin{solution}
    原积分的敛散性的判定等价于判定以下级数的连续性
    $$\int_{0}^{+\infty}\dfrac{x\dd x}{1+x^6\sin^2x}=\sum_{n=0}^{+\infty}\int_{n\pi}^{(n+1)\pi}\dfrac{x\dd x}{1+x^6\sin^2x}\xlongequal[]{x-n\pi=t}\sum_{n=0}^{+\infty}\int_{0}^{\pi}\dfrac{(t+n\pi)\dd t}{1+(t+n\pi)^6\sin^2t}$$
    记 $\displaystyle a_n=\int_{0}^{\pi}\dfrac{(t+n\pi)\dd t}{1+(t+n\pi)^6\sin^2t}$ 则 $\displaystyle a_n\leqslant \int_{0}^{\pi}\dfrac{\pi(n+1)\dd t}{1+(n\pi)^6\sin^2t}=\pi(n+1)\int_{0}^{\pi}\dfrac{\dd t}{1+(n\pi)^6\sin^2t}$, 其中
    \begin{flalign*}
        \int_{0}^{\pi}\dfrac{\dd t}{1+(n\pi)^6\sin^2t} & =\int_{0}^{\pi}\dfrac{\csc^2t\dd t}{\csc^2t+(n\pi)^6}=-\int_{0}^{\pi}\dfrac{\dd \cot t}{1+(n\pi)^6+\cot^2t}                  \\
                                                       & =-\dfrac{1}{\sqrt{\pi^6n^6+1}}\arctan\qty(\dfrac{\cot t}{\sqrt{\pi^6n^6+1}})\bigg |_{0}^{\pi}=\dfrac{\pi}{\sqrt{\pi^6n^6+1}}
    \end{flalign*}
    即 $a_n\leqslant\dfrac{\pi^2(n+1)}{\sqrt{\pi^6n^6+1}}$, 由比较判别法的极限形式可知, 它与 $\dfrac{1}{n^2}$ 构成的 $p=2$ 的 $p-$级数具有相同的敛散性, 所以级数收敛, 从而可得积分也收敛.
\end{solution}

\subsubsection{无穷限反常积分的敛散性与无穷远处的极限}

\begin{example}
    设有下列命题
    \begin{enumerate}[label=(\arabic{*})]
        \item 设 $ f(x) $ 在 $ (-\infty,+\infty) $ 内连续是奇函数, 则 $\displaystyle  \int_{-\infty}^{+\infty} f(x) \dd x=0 $;
        \item 设 $ f(x) $ 在 $ (-\infty,+\infty) $ 内连续, 又 $\displaystyle \lim _{R \to+\infty} \int_{-R}^{R} f(x) \dd x $ 存在, 则 $\displaystyle \int_{-\infty}^{+\infty} f(x) \dd x $ 收敛;
        \item $\displaystyle \int_{a}^{+\infty} f(x) \dd x,~ \int_{a}^{+\infty} g(x) \dd x $ 均发散, 则 $\displaystyle  \int_{a}^{+\infty}[f(x)+g(x)] \dd x $ 可能发散, 也可能收敛;
        \item 若 $ \displaystyle \int_{-\infty}^{0} f(x) \dd x $ 与 $\displaystyle  \int_{0}^{+\infty} f(x) \dd x $ 均发散, 则不能确定 $\displaystyle \int_{-\infty}^{+\infty} f(x) \dd x $ 是否收敛.
    \end{enumerate}
    则以上命题中正确的个数是
    \begin{tasks}(4)
        \task 1
        \task 2
        \task 3
        \task 4
    \end{tasks}
\end{example}
\begin{solution}
    命题 (1) 显然错误, 不妨令 $f(x)=x$, 则 $\displaystyle\int_{-\infty}^{+\infty}x\dd x$ 发散, 一般地, 若 $f(x)$ 在区间 $[-a,a]$ 上是奇函数, 且 $\displaystyle\int_{-a}^{a}f(x)\dd x$ 收敛, 则 $\displaystyle\int_{0}^{a}f(x)\dd x$ 收敛;
    同样地, 命题 (2) 显然错误, 也令 $f(x)=x$, 则 $\displaystyle\lim_{R\to+\infty}\int_{-R}^{R}x\dd x=0$, 但 $\displaystyle\int_{-\infty}^{+\infty}x\dd x$ 发散;
    命题 (3) 正确, 对于不同名的函数的线性组合, 其组合后的收敛性与各函数的收敛性无关; 命题 (4) 错误, 对于相同的函数, 若 $ \displaystyle \int_{-\infty}^{0} f(x) \dd x $ 与 $\displaystyle  \int_{0}^{+\infty} f(x) \dd x $ 均发散,
    则 $\displaystyle \int_{-\infty}^{+\infty} f(x) \dd x $ 一定发散, 综上, 选 A.
\end{solution}

\subsection{反常积分的极限}

\begin{example}
    设 $\displaystyle \varphi (x)=\int_{0}^{x}\cos\dfrac{1}{t}\dd t$, 求 $\varphi'(0).$
\end{example}
\begin{solution}
    因为 $\displaystyle \varphi'(0)=\lim_{x\to0}\dfrac{\varphi(x)-\varphi(0)}{x}=\lim_{x\to0}\dfrac{\displaystyle\int_{0}^{x}\cos\dfrac{1}{t}\dd t}{x}$, 但不能用 L'Hospital 法则, 故
    \begin{flalign*}
        \int_{0}^{x}\cos\dfrac{1}{t}\dd t  \xlongequal[]{\frac{1}{t}=u}\int_{\frac{1}{x}}^{+\infty}\dfrac{\cos u}{u^2}\dd u=\int_{\frac{1}{x}}^{+\infty}\dfrac{1}{u^2}\dd \sin u=\dfrac{\sin u}{u^2}\biggl |_{\frac{1}{x}}^{+\infty}+\int_{\frac{1}{x}}^{+\infty}\dfrac{2\sin u}{u^3}\dd u
        =-x^2\sin\dfrac{1}{x}+\int_{\frac{1}{x}}^{+\infty}\dfrac{2\sin u}{u^3}\dd u
    \end{flalign*}
    所以
    \begin{flalign*}
        \qty|\dfrac{\displaystyle\int_{0}^x\cos\dfrac{1}{t}\dd t}{x}|\leqslant |x|\qty|\sin\dfrac{1}{x}|+\dfrac{1}{|x|}\int_{\frac{1}{x}}^{+\infty}\dfrac{2}{u^3}\dd u=|x|\qty|\sin\dfrac{1}{x}|+|x|\to0~  (x\to0)
    \end{flalign*}
    所以 $\varphi'(0)=0.$
\end{solution}

% \subsection{反常积分作为“积分和”的极限}

\begin{example}
    设 $0<a<d,~\displaystyle A_n=\dfrac{1}{n}\sum_{k=0}^{n-1}(a+kd)$, $\displaystyle G_n=\sqrt[n]{\prod\limits_{k=0}^{n-1}(a+kd)}$,
    试证 $\displaystyle\lim_{n\to\infty}\dfrac{G_n}{A_n}=\dfrac{2}{\e }$.
\end{example}
\begin{proof}[{\songti \textbf{证}}]
    由于 $\displaystyle\dfrac{G_n}{A_n}=\dfrac{\sqrt[n]{\displaystyle\prod_{k=0}^{n-1}(a+kd)}}{\displaystyle\dfrac{1}{n}\sum_{k=0}^{n-1}(a+kd)}=\dfrac{\sqrt[n]{\displaystyle\prod_{k=0}^{n-1}(a+kd)}}{a+\dfrac{1}{2}(n-1)d}
        \xlongequal[]{c=\frac{a}{d}}\dfrac{\sqrt[n]{\displaystyle\prod_{k=0}^{n-1}\dfrac{c+k}{n}}}{\dfrac{c}{n}+\dfrac{n-1}{2n}}$,
    \begin{flalign*}
        \lim_{n\to\infty}\dfrac{1}{n}\sum_{k=0}^{n-1}\ln\dfrac{c+k}{n}=\int_{0}^{1}\ln x\dd x=-1~  \lim_{n\to\infty}\qty(\dfrac{c}{n}+\dfrac{n-1}{n})=\dfrac{1}{2}
    \end{flalign*}
    所以 $\displaystyle \lim_{n\to\infty}\dfrac{G_n}{A_n}=\dfrac{2}{\e }.$
\end{proof}

\begin{theorem}[Frullani 积分定理]
    \index{Frullani 积分定理}形如 $\displaystyle\int_{0}^{+\infty}\dfrac{f(ax)-f(bx)}{x}\dd x~ (a>0,b>0)$, 其中 $f(x)$ 在 $(0,+\infty)$ 内连续, 且 $f(0^+),f(+\infty)\in\mathbb{R} $, 则有
    $$\int_{0}^{+\infty}\dfrac{f(ax)-f(bx)}{x}\dd x=(f(+\infty)-f(0^+))\ln\dfrac{a}{b}.$$
    若 $\displaystyle\lim_{x\to+\infty}f(x)$ 不存在, 但积分 $\displaystyle\int_{0}^{+\infty}\dfrac{f(x)}{x}\dd x$ 存在, 则
    $$\int_{0}^{+\infty}\dfrac{f(ax)-f(bx)}{x}\dd x=-f(0)\ln\dfrac{a}{b}.$$
\end{theorem}

\begin{example}
    求下列积分.
    \setcounter{magicrownumbers}{0}
    \begin{table}[H]
        \centering
        \begin{tabular}{l | l}
            (\rownumber{}) $\displaystyle\int_{0}^{+\infty}\dfrac{\arctan ax-\arctan bx}{x}\dd x.$ & (\rownumber{}) $\displaystyle\int_{0}^{1}\dfrac{x^{a-1}-x^{b-1}}{\ln x}\dd x.$                                  \\
            (\rownumber{}) $\displaystyle\int_{0}^{+\infty}\dfrac{1-\cos ax}{x}\cos bx\dd x.$      & (\rownumber{}) $\displaystyle\int_{0}^{+\infty}\qty(\dfrac{x}{\e ^x-\e ^{-x}}-\dfrac{1}{2})\dfrac{\dd x}{x^2}.$
        \end{tabular}
    \end{table}
\end{example}
\begin{solution}
    \begin{enumerate}[label=(\arabic{*})]
        \item 取 $f(x)=\arctan x$, 则 $f(+\infty)=\dfrac{\pi}{2},~f(0^+)=0$, 故原积分 $=\dfrac{\pi}{2}\ln\dfrac{a}{b}.$
        \item 令 $\ln x=-t$, 原式化为 $\displaystyle\int_{0}^{+\infty}\dfrac{\e ^{-bt}-\e ^{-at}}{t}\dd t=\ln\dfrac{a}{b}.$
        \item 当 $a=b$ 时, 积分化为 $$\int_{0}^{+\infty}\dfrac{\cos ax-\cos ^2ax}{x}\dd x=\int_{0}^{1}\dfrac{\cos ax-\cos ^2ax}{x}\dd x+\int_{1}^{+\infty}\dfrac{\cos ax-\cos ^2ax}{x}\dd x$$
              由于 $\displaystyle\lim_{x\to0}\dfrac{\cos ax-\cos ^2ax}{x}=\lim_{x\to0}\dfrac{1-\cos ax}{x}\xlongequal[]{\text{L'}}\lim_{x\to0}a\sin ax=0$, 故 $\displaystyle\int_{0}^{1}\dfrac{\cos ax-\cos ^2ax}{x}\dd x$ 为正常积分, 而
              $$\int_{1}^{+\infty}\dfrac{\cos ax-\cos ^2ax}{x}\dd x=\int_{1}^{+\infty}\dfrac{\cos ax}{x}\dd x-\int_{1}^{+\infty}\dfrac{\cos ^2ax}{x}\dd x$$
              由 Dirichlet 判别法, $\displaystyle\int_{1}^{+\infty}\dfrac{\cos ax}{x}\dd x$ 收敛, 又
              $$\int_{1}^{+\infty}\dfrac{\cos ^2ax}{x}\dd x=\int_{1}^{+\infty}\dfrac{\dfrac{1+\cos 2x}{2}}{x}\dd x=\int_{1}^{+\infty}\dfrac{\dd x}{2x}+\int_{1}^{+\infty}\dfrac{\cos 2ax}{2x}\dd x$$
              由 Dirichlet 判别法知 $\displaystyle\int_{1}^{+\infty}\dfrac{\cos 2ax}{2x}\dd x$ 收敛, 但 $\displaystyle\int_{1}^{+\infty}\dfrac{\dd x}{2x}$ 发散, 从而当 $a=b$ 时, 原积分发散;\\
              当 $a>b$ 时,
              \begin{flalign*}
                  I & =\int_{0}^{+\infty}\dfrac{1-\cos ax}{x}\cos bx\dd x=\int_{0}^{+\infty}\dfrac{2\sin^2\dfrac{a}{2}x}{x}\cos bx\dd x
                  =\int_{0}^{+\infty}\dfrac{\sin\dfrac{a}{2}x\qty(2\sin\dfrac{a}{2}x\cos bx)}{x}\dd x                                              \\
                    & =\int_{0}^{+\infty}\dfrac{\sin\dfrac{a}{2}x\qty[\sin\qty(\dfrac{a}{2}+b)x+\sin\qty(\dfrac{a}{2}-b)x]}{x}\dd x                \\
                    & =\int_{0}^{+\infty}\dfrac{\dfrac{1}{2}[\cos(-b)x-\cos(a+b)x]+\dfrac{1}{2}[\cos bx-\cos(a-b)x]}{x}\dd x                       \\
                    & =\dfrac{1}{2}\int_{0}^{+\infty}\dfrac{\cos bx-\cos(a+b)}{x}+\dfrac{1}{2}\int_{0}^{+\infty}\dfrac{\cos bx-\cos(a-b)x}{x}\dd x
              \end{flalign*}
              取 $f(x)=\cos x$, 则 $\displaystyle f(0^+)=1,~\lim_{x\to+\infty}\cos x$ 不存在, 但 $\displaystyle\int_{0}^{+\infty}\dfrac{f(x)}{x}\dd x=\int_{0}^{+\infty}\dfrac{\cos x}{x}\dd x$ 存在, 于是
              \begin{flalign*}
                  \int_{0}^{+\infty}\dfrac{1-\cos ax}{x}\cos bx\dd x=\dfrac{1}{2}\ln\dfrac{a+b}{b}+\dfrac{1}{2}\ln\dfrac{a-b}{b}=\dfrac{1}{2}\ln\dfrac{a^2-b^2}{b^2}=\ln\dfrac{\sqrt{a^2-b^2}}{b}
              \end{flalign*}
              若 $a<b$, $\displaystyle\int_{0}^{+\infty}\dfrac{1-\cos ax}{x}\cos bx\dd x=\ln\dfrac{\sqrt{b^2-a^2}}{b}$, 总之, 当 $a\neq b$ 时, 原积分 $=\ln\dfrac{\sqrt{|a^2-b^2|}}{b}.$
        \item 先将 $\displaystyle \dfrac{1}{x^2}\qty(\dfrac{x}{\e ^x-\e ^{-x}}-\dfrac{1}{2})$ 展开,
              \begin{flalign*}
                  I & =\dfrac{1}{x^2}\qty(\dfrac{x}{\e ^x-\e ^{-x}}-\dfrac{1}{2})=\dfrac{1}{x^2}\qty(\dfrac{x\e ^x}{\e ^{2x}-1}-\dfrac{1}{2})=\dfrac{1}{x^2}\qty[\dfrac{x(\e ^x+1)}{\e ^{2x}-1}-\dfrac{x}{\e ^{2x}-1}-\dfrac{1}{2}] \\
                    & =\dfrac{1}{x^2}\qty(\dfrac{x}{\e ^x-1}-\dfrac{x}{\e ^{2x}-1}-\dfrac{1}{2}) =\dfrac{1}{2x}\qty(\dfrac{2}{\e ^x-1}-\dfrac{2}{\e ^{2x}-1}-\dfrac{1}{x})                                                          \\
                    & =\dfrac{1}{2x}\qty(-\e ^{-x}+\e ^{-2x}+\dfrac{2}{\e ^x-1}-\dfrac{2}{x}+\e ^{-x}-\dfrac{2}{\e ^{2x}-1}+\dfrac{1}{x}-\e ^{-2x})                                                                                 \\
                    & =-\dfrac{1}{2x}\qty(\e ^{-x}-\e ^{-2x})+\dfrac{1}{x}\qty(\dfrac{1}{\e ^x-1}-\dfrac{1}{x}+\dfrac{\e ^{-x}}{2})-\dfrac{1}{x}\qty(\dfrac{1}{\e ^{2x}-1}-\dfrac{1}{2x}+\dfrac{\e ^{-2x}}{2})
              \end{flalign*}
              于是
              \begin{flalign*}
                  \int_{0}^{+\infty}\qty(\dfrac{x}{\e ^x}-\e ^{-x}-\dfrac{1}{2})\dfrac{\dd x}{x^2}= & \int_{0}^{+\infty}-\dfrac{1}{2x}\qty(\e ^{-x}-\e ^{-2x})\dd x+\int_{0}^{+\infty}\dfrac{1}{x}\qty(\dfrac{1}{\e ^x-1}-\dfrac{1}{x}+\dfrac{\e ^{-x}}{2})\dd x \\
                                                                                                    & -\int_{0}^{+\infty}\dfrac{1}{x}\qty(\dfrac{1}{\e ^{2x}-1}-\dfrac{1}{2x}+\dfrac{\e ^{-2x}}{2})
              \end{flalign*}
              在上式右端第二个积分中令 $x=2t$, 于是
              $$\int_{0}^{+\infty}\dfrac{1}{x}\qty(\dfrac{1}{\e ^x-1}-\dfrac{1}{x}+\dfrac{\e ^{-x}}{2})\dd x=\int_{0}^{+\infty}\dfrac{1}{t}\qty(\dfrac{1}{\e ^{2t}-1}-\dfrac{1}{2t}+\dfrac{\e ^{-2t}}{2})\dd t$$
              因此 $$\int_{0}^{+\infty}\qty(\dfrac{x}{\e ^x-\e ^{-x}}-\dfrac{1}{2})\dfrac{\dd x}{x^2}=-\dfrac{1}{2}\int_{0}^{+\infty}\dfrac{\e ^{-x}-\e ^{-2x}}{x}\dd x=-\dfrac{1}{2}\ln 2.$$
    \end{enumerate}
\end{solution}

\subsection{反常积分综合性问题}

\begin{example}
    设 $\displaystyle \int_{0}^{+\infty} f(x) \dd x$ 收敛, 且 $\displaystyle f(x)=\dfrac{1}{1+x^2}-\dfrac{\e ^{-x}}{1+\e ^{x}}\int_{0}^{+\infty} f(x) \dd x$, 求 $\displaystyle \int_{0}^{+\infty} f(x) \dd x$.
\end{example}
\begin{solution}
    因为 $\displaystyle \int_{0}^{+\infty} f(x) \dd x$ 收敛, 所以设 $\displaystyle \int_{0}^{+\infty} f(x) \dd x=A$, 那么 
    $$
    \int_{0}^{+\infty} f(x) \dd x=\int_{0}^{+\infty} \qty(\dfrac{1}{1+x^2}-A\cdot\dfrac{\e ^{-x}}{1+\e ^{x}}) \dd x=\dfrac{\pi}{2}-A(1-\ln 2)=A
    $$
    解得 $A=\dfrac{\pi}{2(2-\ln 2)}$.
\end{solution}

\begin{example}
    求曲线 $y=2\e ^{-x}\sin x~ (x\geqslant 0)$ 与 $x$ 轴围成的无界图形的面积 $S$.
\end{example}
\begin{solution}
    由积分的几何意义, 得
    \begin{flalign*}
        S=\lim_{A\to+\infty}\int_{0}^{A}\qty|2\e ^{-x}\sin x|\dd x=\int_{0}^{+\infty}\qty|2\e ^{-x}\sin x|\dd x=2\sum_{n=0}^{\infty}(-1)^n\int_{n\pi}^{(n+1)\pi}\e ^{-x}\sin x\dd x
    \end{flalign*}
    由于 $\displaystyle\int\e ^{-x}\sin x\dd x=-\dfrac{1}{2}\e ^{-x}(\sin x+\cos x)+C$, 所以
    $$\int_{n\pi}^{(n+1)\pi}\e ^{-x}\sin x\dd x=-\dfrac{1}{2}\e ^{-x}(\sin x+\cos x)\biggl |_{n\pi}^{(n+1)\pi}=-\dfrac{1}{2}\qty[\e ^{-(n+1)\pi}(-1)^{n+1}-\e ^{-n\pi}(-1)^n]$$
    代入以上面积计算的积分结果, 得
    \begin{flalign*}
        S & =2\sum_{n=0}^{\infty}(-1)^n\int_{n\pi}^{(n+1)\pi}\e ^{-x}\sin x\dd x=2\sum_{n=0}^{\infty}(-1)^{n+1}\qty[\e ^{-(n+1)\pi}(-1)^{n+1}-\e ^{-n\pi}(-1)^n] \\
          & =\sum_{n=0}^{\infty}\e ^{-(n+1)\pi}+\sum_{n=0}^{\infty}\e ^{-n\pi}=\dfrac{\e ^{-\pi}+1}{1-\e ^{-\pi}}=\dfrac{\e ^\pi+1}{\e ^\pi-1}.
    \end{flalign*}
\end{solution}

\begin{example}[2023 四川大学]\scriptsize\linespread{0.8}
    计算 $I=\displaystyle\lim_{x\to+\infty}x\int_{x}^{+\infty}\dfrac{|\sin t|}{t^2}\dd t$.
\end{example}
\begin{solution}\scriptsize\linespread{0.8}
    \textbf{法一: }设 $f(x)=\displaystyle\int_{x}^{+\infty}\dfrac{|\sin t|}{t^2}\dd t,g(x)=\dfrac{1}{x}$, 于是由函数型 Stolz 定理以及第一积分中值定理, 可知
    \begin{flalign*}
        I & =\lim_{x\to+\infty}\dfrac{f(x)}{g(x)}\xlongequal[]{\text{Stolz}}\lim_{x\to+\infty}\dfrac{f(x+\pi)-f(x)}{g(x+\pi)-g(x)}=\lim_{x\to+\infty}\dfrac{1}{\dfrac{1}{x+\pi}-\dfrac{1}{x}}\qty(\int_{x+\pi}^{+\infty}\dfrac{|\sin t|}{t^2}\dd t-\int_{x}^{+\infty}\dfrac{|\sin t|}{t^2}\dd t) \\
          & =\dfrac{1}{\pi}\lim_{x\to+\infty}(x+\pi)x\int_{x}^{x+\pi}\dfrac{|\sin t|}{t^2}\dd t=\dfrac{1}{\pi}\lim_{x\to+\infty}(x+\pi)x\dfrac{1}{\xi^2}\int_{}^{\pi}|\sin t|\dd t=\dfrac{2}{\pi}\lim_{x\to+\infty}\dfrac{x(x+\pi)}{\xi^2}
    \end{flalign*}
    因为 $\xi\in(x,x+\pi)$, 故 $\dfrac{x(x+\pi)}{(x+\pi)^2}<\dfrac{x(x+\pi)}{\xi^2}<\dfrac{x(x+\pi)}{x^2}$, 故由夹逼准则得原极限等于 $\dfrac{2}{\pi}.$\\
    \textbf{法二: }由题意可知,
    $$\int_{x}^{x+\pi}\dfrac{|\sin t|}{t^2}\dd t=\sum_{n=0}^{\infty}\int_{x+n\pi}^{x+(n+1)\pi}\dfrac{|\sin t|}{t^2}\dd t$$
    由第一积分中值定理可知, $\exists\xi_n\in(x+n\pi,x+(n+1)\pi)$, 使得
    $$\int_{x+n\pi}^{x+(n+1)\pi}\dfrac{|\sin t|}{t^2}\dd t=\dfrac{1}{\xi_n^2}\int_{x+n\pi}^{x+(n+1)\pi}|\sin t|\dd t=\dfrac{1}{\xi_n^2}\int_{0}^{\pi}|\sin t|\dd t=\dfrac{2}{\xi_n^2}$$
    即 $$\sum_{n=0}^{\infty}\dfrac{2}{\qty[x+(n+1)\pi]^2}<\dfrac{2}{\xi_n^2}<\sum_{n=0}^{\infty}\dfrac{2}{(x+n\pi)^2}$$
    一方面,
    \begin{flalign*}
        \sum_{n=0}^{\infty}\dfrac{2}{(x+n\pi)^2}  <\sum_{n=0}^{\infty}\dfrac{2}{(x+n\pi)(x+n\pi-\pi)}=\dfrac{2}{\pi}\sum_{n=0}^{\infty}\qty(\dfrac{1}{x+n\pi-\pi}-\dfrac{1}{x+n\pi})
        =\dfrac{2}{\pi}\qty(\dfrac{1}{x-\pi}-\dfrac{1}{x+n\pi})<\dfrac{2}{\pi}\dfrac{1}{x-\pi}
    \end{flalign*}
    另一方面,
    \begin{flalign*}
        \sum_{n=0}^{\infty}\dfrac{2}{\qty[x+(n+1)\pi]^2}  >\sum_{n=0}^{\infty}\dfrac{2}{\qty[x+(n+1)\pi][x+(n+2)\pi]}>\dfrac{2}{\pi}\sum_{n=0}^{\infty}\qty[\dfrac{1}{x+(n+1)\pi}-\dfrac{1}{x+(n+2)\pi}]
        =\dfrac{2}{\pi}\qty[\dfrac{1}{x+\pi}-\dfrac{1}{x+(n+2)\pi}]
    \end{flalign*}
    最后由夹逼准则易得原极限等于 $\dfrac{2}{\pi}.$
\end{solution}

