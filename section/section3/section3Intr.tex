\begin{flushright}
    \begin{tabular}{r|}
        \textit{“埋头苦干是第一, 发白才知智叟. 呆勤能补拙是良训, 一分辛苦一分才. ”}\\
        ——\textit{华罗庚}
    \end{tabular}
\end{flushright}

一元函数积分学是微积分的另一个重要分支, 主要研究一元函数的不定积分、定积分和积分应用. 下面简要介绍一元函数积分学的几个重要概念: 

1. 不定积分: 不定积分是积分学中的一种基本概念, 也称为原函数. 对于一元函数 $f(x)$, 其不定积分记为 $\displaystyle \int f(x) \dd x$, 表示求函数 $f(x)$ 的反导函数. 不定积分的结果是一个函数, 其导数为原函数 $f(x)$. 

2. 定积分: 定积分是积分学中的另一个重要概念, 表示函数在一个区间上的积分值. 对于一元函数 $f(x)$, 其在区间 $[a, b]$ 上的定积分表示为 $\displaystyle \int_{a}^{b} f(x) \dd x$, 表示函数 $f(x)$ 在区间 $[a, b]$ 上的面积或有符号的累积量. 定积分可以理解为曲线下的面积, 也可以用来计算函数在给定区间上的平均值等. 

3. 积分法: 积分法是求解不定积分和定积分的方法. 常见的积分法包括换元法、分部积分法、三角代换法等. 通过运用这些积分法, 可以求解各种类型的积分, 包括有理函数、三角函数、指数函数等的积分. 

4. 积分应用: 积分在物理学、工程学、经济学等领域有着广泛的应用. 例如, 积分可以用来计算曲线下的面积、求解物体的质心、计算定积分表示的总量等. 积分应用也包括微积分基本定理、积分中值定理等. 

一元函数积分学是微积分的重要组成部分, 它与微分学相辅相成, 共同构成了微积分学的理论体系. 通过学习一元函数积分学, 可以更深入地理解函数的面积、累积量和平均值等概念, 为解决实际问题提供了有力的数学工具. 