\section{特殊构型积分补充}

\subsection{Dirichlet 积分}

\begin{lemma}[Dirichlet 核]
    \label{Dirichlethe}
    在区间 $(0,\pi)$ 上定义 $D_n=\dfrac{\sin\dfrac{(2n+1)x}{2}}{2\sin\dfrac{x}{2}},~n\in\mathbb{N}_+$, 则有 $$\int_{0}^{\pi}D_n(x)\dd x=\dfrac{\pi}{2}.$$
\end{lemma}
\begin{proof}[{\songti \textbf{证}}]
    显然, $D_n$ 在 $x=0$ 处无定义, 但是 $\displaystyle\lim_{x\to0^+}D_n(x)=\dfrac{2n+1}{2}$, 因此 $D_n$ 在 $[0,\pi]$ 可积, 因为有三角恒等式
    $$2\sin\dfrac{x}{2}\qty(\dfrac{1}{2}+\sum_{k=1}^{n}\cos kx\dd x)=\sin\dfrac{(2n+1)x}{2}$$
    于是, $\displaystyle D_n=\dfrac{1}{2}+\sum_{k=1}^{n}\cos kx$, 所以, 不难得到
    $$\int_{0}^{\pi}D_n(x)\dd x=\int_{0}^{\pi}\qty(\dfrac{1}{2}+\sum_{k=1}^{n}\cos kx)\dd x=\dfrac{\pi}{2}.$$
\end{proof}

\begin{theorem}[Dirichlet 积分等式]
    $\displaystyle\int_{0}^{+\infty}\dfrac{\sin x}{x}\dd x=\dfrac{\pi}{2}.$
\end{theorem}
\begin{proof}[{\songti \textbf{证}}]
    易知该反常积分条件收敛, 根据引理 \ref{Dirichlethe} 有, $\displaystyle\int_{0}^{\pi}\dfrac{\sin\dfrac{(2n+1)\pi}{2}}{2\sin\dfrac{x}{2}}\dd x=\dfrac{\pi}{2}$, 考虑将其分母换为 $x$ 所产生的影响, 
    由 L'Hospital 法则, 有 $$f(x)=\dfrac{1}{x}-\dfrac{1}{2\sin\dfrac{x}{2}}=O(x)~~ (x\to0)$$
    因此 $f$ 在 $[0,\pi]$ 上常义可积, 由 Riemann-Lebesgue 定理, 有
    $$\lim_{n\to\infty}f(x)\sin\qty(n+\dfrac{1}{2})x\dd x=0$$
    即 $$\lim_{n\to\infty}\int_{0}^{\pi}\dfrac{\sin\qty(n+\dfrac{1}{2})x}{x}\dd x=\lim_{n\to\infty}\int_{0}^{\pi}\dfrac{\sin\qty(n+\dfrac{1}{2})x}{2\sin\dfrac{x}{2}}\dd x=\dfrac{\pi}{2}$$
    最后, 作代换 $t=\qty(n+\dfrac{1}{2})x$, 即得 $$\int_{0}^{\pi}\dfrac{\sin\qty(n+\dfrac{1}{2})x}{x}\dd x=\int_{0}^{\qty(n+\frac{1}{2})\pi}\dfrac{\sin t}{t}\dd t=\dfrac{\pi}{2}.$$
\end{proof}

\subsection{Lobachevsky 积分法}

\begin{theorem}[Lobachevsky 积分法]
    若 $f(x)$ 在 $x\in[0,+\infty)$ 范围内满足 $f(x+\pi)=f(x)$ 及 $f(\pi-x)=f(x)$, 则
    $$\int_{0}^{+\infty}f(x)\dfrac{\sin x}{x}\dd x=\int_{0}^{\frac{\pi}{2}}f(x)\dd x.$$
    当 $f(x)=1$ 时, 便是 Dirichlet 积分.
\end{theorem}

\subsection{Fresnel 积分与 Fej\texorpdfstring{$\acute{\text{e}}$}.r 积分}

\begin{theorem}[Fresnel 积分等式]
    $\displaystyle\int_{-\infty}^{+\infty}\sin x^2\dd x=\int_{-\infty}^{+\infty}\cos x^2\dd x=\sqrt{\dfrac{\pi}{2}}.$
\end{theorem}

\begin{theorem}[Fej$\acute{\text{e}}$r 积分]
    假设 $k\in\mathbb{N}^+$, 则有以下四种形式的 Fej$\acute{\text{e}}$r 积分:
    \begin{enumerate}[label=(\arabic{*})]
        \item $\displaystyle \int_{0}^{\pi}\dfrac{\sin nx}{\sin x}\dd x=\begin{cases}
                      0,   & n=2k    \\
                      \pi, & n=2k-1;
                  \end{cases}$
        \item $\displaystyle\int_{0}^{\frac{\pi}{2}}\dfrac{\sin nx}{\sin x}\dd x=\begin{cases}
                      2\displaystyle\sum_{i=1}^{k}\dfrac{(-1)^{k-1}}{2k-1}, & n=2k    \\[6pt]
                      \dfrac{\pi}{2},                                       & n=2k-1;
                  \end{cases}$
        \item $\displaystyle \int_{0}^{\pi}\qty(\dfrac{\sin nx}{\sin x})^2\dd x=n\pi;$
        \item $\displaystyle\int_{0}^{\frac{\pi}{2}}\qty(\dfrac{\sin nx}{\sin x})^2\dd x=\dfrac{n\pi  }{2}.$
    \end{enumerate}
\end{theorem}

\subsection{Laplace 积分}

\begin{theorem}[Laplace 积分等式]
    $\displaystyle \int_{0}^{+\infty}\dfrac{\cos bx}{a^2+x^2}\dd x=\dfrac{\pi}{2a}\e^{-ab},~\int_{0}^{+\infty}\dfrac{x\sin bx}{a^2+x^2}\dd x=\dfrac{\pi}{2}\e^{-ab}~~(a,b>0).$
\end{theorem}
\begin{inference}
    当 $4q>p^2$ 时, 有
    \begin{flalign*}
        \int_{-\infty}^{+\infty}\dfrac{\cos x}{x^2+px+q}\dd x & =\dfrac{2\pi}{\sqrt{4q-p^2}}\e^{-\frac{\sqrt{4q-p^2}}{2}}\cos\dfrac{p}{2}   \\
        \int_{-\infty}^{+\infty}\dfrac{\sin x}{x^2+px+q}\dd x & =-\dfrac{2\pi}{\sqrt{4q-p^2}}\e^{-\frac{\sqrt{4q-p^2}}{2}}\sin\dfrac{p}{2}.
    \end{flalign*}
\end{inference}