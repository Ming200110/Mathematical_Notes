\begin{flushright}
    \begin{tabular}{r|||}
        \textit{“概率论, 当局限在适当的范围内, 应该同样引起数学家、实验者和政治家的兴趣。”}\\
        ——\textit{德·摩根}
    \end{tabular}
\end{flushright}

概率论是数学的一个分支, 研究随机事件的发生概率和随机现象的规律性. 它是研究不确定性的一种工具和方法. 

概率论的基本概念包括随机试验、样本空间、事件、概率等. 随机试验是指具有多种可能结果的试验, 例如掷硬币、抛骰子等. 样本空间是指随机试验的所有可能结果的集合. 事件是样本空间的子集, 表示试验中我们感兴趣的一些结果. 概率是事件发生的可能性的度量, 通常用一个介于 0 和 1 之间的数来表示. 

概率论的基本原理包括古典概型、几何概型和统计概型. 古典概型适用于试验结果的数量有限且每个结果发生的可能性相等的情况, 例如抛硬币、掷骰子等. 几何概型适用于试验结果的数量无限且每个结果发生的可能性相等的情况, 例如在一条直线上选择一个点的位置. 统计概型适用于试验结果的数量无限且每个结果发生的可能性不相等的情况, 例如从一副牌中抽取一张牌. 

概率论的应用广泛, 包括统计学、金融学、物理学、工程学、生物学等领域. 它可以用于描述随机现象的规律性、进行风险评估和决策分析、进行数据分析和建模等. 概率论提供了一种量化不确定性的方法, 帮助我们理解和解决各种实际问题. 