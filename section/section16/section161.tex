\section{离散型随机变量的概率分布}

\subsection{离散型随机变量的分布函数}

\begin{definition}[分布函数]
    设离散型随机变量 $ X $ 的分布律为
    $P\left\{X=x_{k}\right\}=p_{k},~k=1,2, \cdots,$ 则 $ X $ 的分布函数为
    $$F(x)=P\{X \leqslant x\}=\sum_{x_{k} \leqslant x} p_{k}$$
    或者
    $F(x)=\begin{cases}
            0,           & x<x_{1}                 \\
            p_{1},       & x_{1} \leqslant x<x_{2} \\
            p_{1}+p_{2}, & x_{2} \leqslant x<x_{3} \\
            \vdots       & \vdots
        \end{cases}$
\end{definition}

\begin{theorem}[分布函数的性质]
    以下三条是判定一个函数是否为分布函数的充要条件.
    \begin{enumerate}[label=(\arabic{*})]
        \item 单调性: $F(x) $ 是 $ x $ 的单调不减函数, 即对于任意实数 $ x_{1}<x_{2}$, 有 $ F\left(x_{1}\right) \leqslant F\left(x_{2}\right) $;
        \item 有界性: $\displaystyle0 \leqslant F(x) \leqslant 1 ,~F(-\infty)=\lim _{x \rightarrow-\infty} F(x)=0,~F(+\infty)=\lim_{x\to+\infty}F(x)=1$;
        \item 右连续: $F(x)$ 是 $x$ 的右连续函数, 即对任意实数 $x$, 有 $F(x^+)=F(x).$
    \end{enumerate}
\end{theorem}

\subsection{二项分布}

\begin{definition}[二项分布]
    在 $ n $ 重伯努利试验中, 若用随机变量 $ X $ 表示所关心事件 $ \sj{A} $ 发生的次数, $P(\sj{A})=p~~(0<p<1)$, 则 $ X $ 的分布律为
    $$P\{X=k\}=\C_{n}^{k} p^{k}(1-p)^{n-k}, ~k=0,1, \cdots, n$$
    则称 $ X $ 服从参数为 $ n, p $ 的二项分布, 记作 $ X \sim B(n, p) $.
    \begin{enumerate}[label=(\arabic{*})]
        \item 0-1 分布可以视为 $ n=1 $ 时的二项分布;
        \item 若 $ X \sim B(n, p) $, 令 $ Y=n-X$, 则 $ Y \sim B(n, 1-p) .$
    \end{enumerate}
\end{definition}

% \subsection{离散型随机变量的性质}
% 
% \begin{example}
%     设随机变量 $X$ 的概率分布 $P\qty{X=k}=\dfrac{a}{k(k+1)},k=1,2,\cdots,$, 其中 $a$ 为常数, $X$ 的分布函数为 $F(x)$, 已知 $F(b)=\dfrac{3}{4}$, 求 $b$ 的取值范围.
% \end{example}
% \begin{solution}
%     $\displaystyle \sum_{k=1}^{+\infty}P\qty{X=k}=\sum_{k=1}^{+\infty}\dfrac{a}{k(k+1)}=a\sum_{k=1}^{+\infty}\qty(\dfrac{1}{k}-\dfrac{1}{k+1})=a\lim_{k\to\infty}\qty(1-\dfrac{1}{k+1})=1\Rightarrow a=1$
%     则, $$F(x)=\sum_{k\leqslant x}\qty(\dfrac{1}{k}-\dfrac{1}{k+1})$$
%     当 $i\leqslant x<i+1$ 时, $$F(x)=\sum_{k\leqslant i}\qty(\dfrac{1}{k}-\dfrac{1}{k+1})=1-\dfrac{1}{i+1}$$
%     则 $F(b)=\dfrac{3}{4}=1-\dfrac{1}{4}\Rightarrow 3\leqslant b<4.$
% \end{solution}
% 
% \subsection{五大离散型分布}