\begin{flushright}
    \begin{tabular}{r|||}
        \textit{“我创建了形貌随机征象的一种概率散布.。”}\\
        ——\textit{泊松}
    \end{tabular}
\end{flushright}

随机变量是一个可以随机取不同值的变量,其取值是由随机事件的结果决定的. 随机变量可以是离散的,也可以是连续的. 

随机变量的分布描述了其取不同值的概率分布情况. 常见的随机变量分布包括: 
\begin{enumerate}
    \item 离散型随机变量的分布: \begin{enumerate}
        \item 二项分布: 描述了在一次伯努利试验中成功的次数的分布. 
        \item 泊松分布: 描述了在一段时间内某事件发生次数的分布. 
        \item 几何分布: 描述了在一系列独立伯努利试验中首次成功所需的次数的分布. 
    \end{enumerate}
    \item 连续型随机变量的分布: \begin{enumerate}
        \item 正态分布: 也称为高斯分布,是最常见的连续型随机变量分布,具有钟形曲线. 
        \item 均匀分布: 描述了在一个区间内各个数值出现的概率相同的分布. 
        \item 指数分布: 描述了连续时间内某事件发生的间隔时间的分布. 
    \end{enumerate}
\end{enumerate}

除了以上列举的分布外,还有众多其他类型的随机变量分布,每种分布都有其特定的数学形式和特征. 在统计学和概率论中,随机变量及其分布是研究的重要对象,用来描述和分析随机现象的规律. 
