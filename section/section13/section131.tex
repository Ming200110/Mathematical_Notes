\section{矩阵的特征值与特征向量}

矩阵的特征值与特征向量是线性代数中的重要概念, 它们在矩阵的可对角化问题、概率统计、物理、工程和经济学等领域有广泛应用. 特征值代表了矩阵作用下的伸缩性, 而特征向量则是在线性变换下保持方向不变的向量. 

\subsection{矩阵的特征值}

\begin{theorem}[特征多项式展开定理]
    设 $\vb*{A}$ 是 $n$ 阶矩阵, 则 $\vb*{A}$ 的特征多项式为 
    $$f(\lambda)=|\lambda\vb*{E}-\vb*{A}|=\lambda^n-a_1\lambda^{n-1}+\cdots+(-1)^ka_k\lambda^{n-k}+\cdots+(-1)^na_n$$
    其中 $a_k$ 是 $\vb*{A}$ 的所有 $k$ 阶主子式之和, 即 
    $$a_k=\sum_{1\leqslant j_1<\cdots<j_k\leqslant n}\mqty|a_{j_1j_1}&\cdots&a_{j_1j_k}\\\vdots&&\vdots\\a_{j_kj_1}&\cdots&a_{j_kj_k}|$$
    特别地 $a_1=\tr\vb*{A},~a_n=\det\vb*{A}.$
    对于常见的 3 阶矩阵有 
    \begin{flalign*}
        f(\lambda)& =\lambda^3-\tr\vb*{A}\cdot\lambda^2+\qty(\mqty|a_{11}&a_{12}\\a_{21}&a_{22}|+\mqty|a_{11}&a_{13}\\a_{31}&a_{32}|+\mqty|a_{22}&a_{23}\\a_{32}&a_{33}|)\lambda-\det\vb*{A}\\ 
        & =\lambda^3-\tr\vb*{A}\cdot\lambda^2+(A_{11}+A_{22}+A_{33})\lambda-\det\vb*{A}
    \end{flalign*}
    其中 $A_{ii}$ 为 $\vb*{A}$ 的元素 $a_{ii}~(i=1,2,3)$ 所对应的代数余子式.
\end{theorem}

图 \ref{fig:kjiezzshi} 是 $a_2$ 的形象化表达.
\begin{figure}[H]
    \centering
    \includegraphics[scale=1]{figures/kjiezzshi.pdf}
    \caption{}
    \label{fig:kjiezzshi}
\end{figure}

\begin{theorem}[特征值的积与和]
    \index{特征值的积与和}设 $n$ 阶方阵 $\vb*{A}$ 的特征值为 $\lambda_1,\lambda_2,\cdots,\lambda_n$, 则 $$\det\vb*{A}=\prod_{i=1}^{n}\lambda_i,~\tr\vb*{A}=\sum_{i=1}^{n}\lambda_i.$$
\end{theorem}

\begin{example}
    设 $\vb*{A}$ 是 3 阶矩阵, $\vb*{A}$ 的特征值为 $1,2,-1$, 如果 $\vb*{B}=\vb*{A}^2-2\vb*{A}+3\vb*{E}$, 求 $|\vb*{B}|$.
\end{example}
\begin{solution}
    易求得 $\vb*{B}$ 的特征值分别是 $2,3,6$, 那么 $\det\vb*{B}=2\times3\times6=36.$
\end{solution}

\begin{theorem}[逆矩阵的特征值]
    矩阵可逆当且仅当其特征值都不等于零, 若 $\lambda$ 是可逆矩阵 $\vb*{A}$ 的特征值, 则 $\dfrac{1}{\lambda}$ 是逆矩阵 $\vb*{A}^{-1}$ 的特征值.\index{逆矩阵的特征值}
\end{theorem}

\begin{theorem}[秩一矩阵的特征值]
    若 $n$ 阶方阵 $\vb*{A}$ 满足 $\rank\vb*{A}=1$, 那么矩阵 $\vb*{A}$ 的特征值为 $\tr\vb*{A}$ 及 $n-1$ 重 0 根.\index{秩一矩阵的特征值}
\end{theorem}

\begin{example}
    已知 $\vb*{A}$ 是 3 阶矩阵, $\vb*{E}$ 是 3 阶单位矩阵, 如果 $\vb*{A},\vb*{A}-2\vb*{E},3\vb*{A}+2\vb*{E}$ 均不可逆, 求 $|\vb*{A}+\vb*{E}|.$
\end{example}
\begin{solution}
    易知矩阵 $\vb*{A}$ 的特征值分别为 $0,2,-\dfrac{2}{3}$, 那么 $\det(\vb*{A}+\vb*{E})=1\times3\times\dfrac{1}{3}=1.$
\end{solution}

\begin{theorem}[伴随矩阵的特征值 A]
    若 $\vb*{A}$ 为 $n$ 阶可逆矩阵, 且 $\lambda_i~ (i=1,2,\cdots,n)$ 是 $\vb*{A}$ 的全部特征值, 则 $\dfrac{|\vb*{A}|}{\lambda_i}$ 是 $\vb*{A}^*$ 的全部特征值; 
    若 $\vb*{A}$ 为 $n$ 阶不可逆矩阵, 则 $\vb*{A}^*$ 有 $n-1$ 个特征值为 0, 另一个特征值为 $\tr\vb*{A}^*.$
    \index{伴随矩阵的特征值}
\end{theorem}

\begin{theorem}[伴随矩阵的特征值 B]
    \label{bsjzdtzzB}\index{伴随矩阵的特征值}设 $n$ 阶方阵 $\vb*{A}$ 的特征值为 $\lambda_j~ (j=1,2,\cdots,n)$, 则 $\vb*{A}^*$ 的 $n$ 个特征值为 $\lambda_i^*$, 则
    $$\lambda_i^*=\prod_{\substack{1\leqslant j\leqslant n\\j\neq i}}\lambda_j~  (i=1,2,\cdots,n).$$
\end{theorem}

\begin{example}
    设 $ \vb*{A} $ 为 3 阶矩阵, 特征值为 $ \lambda_{1}=\lambda_{2}=1,\lambda_{3}=2 $, 
    对应的线性无关的特征向量为 $ \vb*{\alpha}_{1}, \vb*{\alpha}_{2}, \vb*{\alpha}_{3}$, 
    令 $ \vb*{P}=\left(\vb*{\alpha}_{1}-\vb*{\alpha}_{3}, \vb*{\alpha}_{2}+\vb*{\alpha}_{3}, \vb*{\alpha}_{3}\right)$, 则 $\vb*{P}^{-1} \vb*{A}^{*} \vb*{P}$ 等于
    \begin{tasks}(4)
        \task $\mqty(1&0&0\\0&3&0\\0&0&1)$
        \task $\mqty(-1&0&0\\0&3&0\\0&0&2)$
        \task $\mqty(2&0&1\\0&2&-1\\0&0&1)$
        \task $\mqty(2&0&0\\0&2&0\\1&-1&1)$
    \end{tasks}
\end{example}
\begin{solution}
    由 $\vb*{P}=(\vb*{\alpha}_{1}-\vb*{\alpha}_{3}, \vb*{\alpha}_{2}+\vb*{\alpha}_{3}, \vb*{\alpha}_{3})$, 得 $\vb*{P}=(\vb*{\alpha}_1,\vb*{\alpha}_2,\vb*{\alpha}_3)\vb*{E}_{31}(-1)\vb*{E}_{32}(1):=\vb*{B}\vb*{E}_{31}(-1)\vb*{E}_{32}(1)$, 那么
    \begin{flalign*}
        \vb*{P}^{-1}\vb*{A}^*\vb*{P}&=(\vb*{B}\vb*{E}_{31}(-1)\vb*{E}_{32}(1))^{-1}\vb*{A}^*\vb*{B}\vb*{E}_{31}(-1)\vb*{E}_{32}(1)=\vb*{E}_{32}(-1)\vb*{E}_{31}(1)\diag(2,2,1)\vb*{E}_{31}(-1)\vb*{E}_{32}(1)\\
        &=\vb*{E}_{32}(-1)\vb*{E}_{31}(1)\vb*{E}_{1}(2)\vb*{E}_{2}(2)\vb*{E}_{31}(-1)\vb*{E}_{32}(1)=\mqty(2&0&0\\0&2&0\\1&-1&1)
    \end{flalign*}
    因此选 D.
\end{solution}

\begin{example}
    已知矩阵 $\vb*{A}=\mqty(1&-1&a\\1&3&5\\0&0&2)$, 只有两个线性无关的特征向量, 求矩阵 $\vb*{A}$ 的特征值以及 $a$.
\end{example}
\begin{solution}
    由 $|\lambda\vb*{E}-\vb*{A}|=\mqty|\lambda-1&1&-a\\-1&\lambda-3&-5\\0&0&\lambda-2|=(\lambda-1)(\lambda-3)(\lambda-2)+(\lambda-2)=(\lambda-2)^3=0$, 于是特征值为 $\lambda_1=\lambda_2=\lambda_3=2$, 因为 $n-\rank(2\vb*{E}-\vb*{A})=2$, 于是 $\rank(2\vb*{E}-\vb*{A})=1$, 进而解得 $a=-5.$
\end{solution}

\begin{theorem}[特征值不等式]
    \label{tzzbds}\index{特征值不等式}若 $\vb*{A}$ 的特征值按从小到大排列, 即 $\lambda_1\leqslant \lambda_2\leqslant \cdots\leqslant \lambda_n$, 则 $$\lambda_1\vb*{x}^\top\vb*{x}\leqslant \vb*{x}^\top\vb*{Ax}\leqslant \lambda_n\vb*{x}^\top\vb*{x}.$$
\end{theorem}
\begin{proof}[{\songti \textbf{证}}]
    存在正交矩阵 $\vb*{Q}$, 令 $\vb*{x}=\vb*{Qy}$, 使得 $$f(x_1,x_2,\cdots,x_n)=\vb*{x}^\top\vb*{Ax}=\vb*{y}^\top\vb*{Q}^\top\vb*{AQy}=\lambda_1y_1^2+\lambda_2y_2^2+\cdots+\lambda_ny_n^2$$
    不妨设 $\lambda_1\leqslant \lambda_2\leqslant\cdots\leqslant\lambda_n$, 有 
    $$\lambda_1\sum_{k=1}^{n}y_k^2\leqslant \sum_{k=1}^{n}\lambda_ky_k^2\leqslant \lambda_n\sum_{k=1}^{n}y_k^2$$
    即 $$\lambda_1\vb*{x}^\top\vb*{x}\leqslant \vb*{x}^\top\vb*{Ax}\leqslant \lambda_n\vb*{x}^\top\vb*{x}.$$
\end{proof}

\begin{theorem}[行均和定理]
    \label{hangjunhedl}\index{行均和定理}若矩阵 $\vb*{A}$ 的每行元素之和均为 $k$, 则 $\vb*{A}$ 有一特征值为 $k$, 所对应特征向量为 $(1,1,\cdots,1)^\top.$
\end{theorem}

\subsubsection{特征值在求解迭代方程中的应用}

\begin{example}
    给定 3 个数列 $\qty{x_n},\qty{y_n},\qty{z_n}$ 满足 $x_1=-2,y_1=1,z_1=-1$, 且当 $n\geqslant 1$ 有
    $$x_{n+1}=3x_n-6y_n-z_n,\quad y_{n+1}=-x_n+2y_n+z_n,\quad z_{n+1}=x_n+3y_n-z_n$$
    求极限 $\displaystyle\lim_{n\to\infty}\dfrac{x_n+y_n+z_n}{3^n+5^n}.$
\end{example}
\begin{solution}
    由 $\begin{cases}
            x_{n+1}=3x_n-6y_n-z_n \\
            y_{n+1}=-x_n+2y_n+z_n \\
            z_{n+1}=x_n+3y_n-z_n
        \end{cases}\Rightarrow \mqty(x_{n+1}\\y_{n+1}\\z_{n+1})=\mqty(3&-6&-1\\-1&2&1\\1&3&-1)\mqty(x_{n}\\y_n\\z_n):=\vb*{A}\mqty(x_{n}\\y_n\\z_n)$, 于是特征方程为
    $$\lambda^3-4\lambda^2-7\lambda+10=(\lambda-1)(\lambda-5)(\lambda+2)=0\Rightarrow \lambda_1=1,\lambda_2=5,\lambda_3=-2$$
    于是 $x_n+y_n+z_n=a\cdot 1^n+b\cdot 5^n+c\cdot (-2)^n$, 且
    $$\begin{cases}
            x_1+y_1+z_1=-2+1-1=-2=a+5b-2c   \\
            x_2+y_2+z_2=-11+3+3=-6=a+25b+4c \\
            x_3+y_3+z_3=-53+19-4=-38=a+125b-8c
        \end{cases}\Rightarrow \begin{cases}
            a=0             \\
            b=-\dfrac{2}{7} \\[6pt]
            c=\dfrac{2}{7}
        \end{cases}$$
    于是 $x_n+y_n+z_n=-\dfrac{2}{7}\cdot 5^n+\dfrac{2}{7}(-1)^n\cdot 2^n$, 那么 $\displaystyle\lim_{n\to\infty}\dfrac{-\dfrac{2}{7}\cdot 5^n+\dfrac{2}{7}(-1)^n\cdot 2^n}{3^n+5^n}=-\dfrac{2}{7}$.
\end{solution}

\begin{example}[2024 数一]
    已知数列 $\qty{x_n},\qty{y_n},\qty{z_n}$ 满足 $x_0=-1,y_0=0,z_0=2$, 且 $$\begin{cases}
            x_n=-2x_{n-1}+2z_{n-1} \\
            y_n=-2y_{n-1}-2z_{n-1} \\
            z_n=-6x_{n-1}-3y_{n-1}+3z_{n-1}
        \end{cases}$$ 记 $\vb*{\alpha}_n=\mqty(x_n\\y_n\\z_n)$, 写出满足 $\vb*{\alpha}_n=\vb*{A\alpha}_{n-1}$ 的矩阵 $\vb*{A}$, 并求 $\vb*{A}^n$ 及 $x_n,y_n,z_n$.
\end{example}
\begin{solution}
    由题设知, $\mqty(x_n\\y_n\\z_n)=\mqty(-2&0&2\\0&-2&-2\\-6&-3&3)\mqty(x_{n-1}\\y_{n-1}\\z_{n-1})$, 即 $\vb*{A}=\mqty(-2&0&2\\0&-2&-2\\-6&-3&3)$, 
    由 $|\lambda\vb*{E}-\vb*{A}|=0$ 得 $\lambda(\lambda-1)(\lambda+2)=0$, 解得 $\lambda_1=0,\lambda_2=1,\lambda_3=-2$, 那么, 当 $\lambda_1=0$ 时, 对应的特征向量为 $\vb*{\xi}_1=(1,-1,1)^\top$, 
    当 $\lambda_2=1$ 时, 对应的特征向量为 $\vb*{\xi}_2=(2,-2,3)^\top$, 当 $\lambda_2=-2$ 时, 对应的特征向量为 $\vb*{\xi}_3=(1,-2,0)^\top$, 
    故存在可逆矩阵 $\vb*{P}=(\vb*{\xi}_1,\vb*{\xi}_2,\vb*{\xi}_3)=\mqty(1&2&1\\-1&-2&-2\\1&3&0)$, 使得 $\vb*{P}^{-1}\vb*{AP}=\diag(0,1,-2)=\vb*{\Lambda}$, 即 $\vb*{A}=\vb*{P\Lambda P}^{-1}$, 那么 
    $$\vb*{A}^n=\vb*{P}\vb*{\Lambda}^{n}\vb*{P}^{-1}=\mqty(-4+(-1)^{n+1}\cdot 2^{n}& -2+(-1)^{n+1}\cdot 2^{n}& 2\\4+(-1)^{n}\cdot 2^{n+1}& 2+(-1)^{n}\cdot 2^{n+1}&-2\\-6&-3&3)$$
    由 $\vb*{\alpha}_n=\vb*{A\alpha}_{n-1}$, 那么 $\vb*{\alpha}_n=\vb*{A}^n\vb*{\alpha}_0=\mqty(-4+(-1)^{n+1}\cdot 2^{n}& -2+(-1)^{n+1}\cdot 2^{n}& 2\\4+(-1)^{n}\cdot 2^{n+1}& 2+(-1)^{n}\cdot 2^{n+1}&-2\\-6&-3&3)\mqty(x_0\\y_0\\z_0)=\mqty(8+(-2)^{n}\\-8+(-2)^{n+1}\\12)$, 
    则 $x_n=8+(-2)^{n},y_n=-8+(-2)^{n+1},z_n=12.$
\end{solution}

\subsection{矩阵的特征向量}

\begin{theorem}
    设 $\vb*{A}$ 为 $n$ 阶矩阵, 则 
    \begin{enumerate}[label=(\arabic{*})]
        \item $\vb*{\alpha}$ 为 $\vb*{A}^\top$ 特征向量时, 不一定为 $\vb*{A}$ 的特征向量;
        \item $\vb*{\alpha}$ 为 $\vb*{A}^*$ 特征向量时, 不一定为 $\vb*{A}$ 的特征向量;
        \item $\vb*{\alpha}$ 为 $\vb*{A}^2$ 特征向量时, 不一定为 $\vb*{A}$ 的特征向量;
        \item $\vb*{\alpha}$ 为 $k\vb*{A}~(k\neq0)$ 特征向量时, 一定为 $\vb*{A}$ 的特征向量;
    \end{enumerate}
\end{theorem}

\begin{example}
    设矩阵 $\vb*{A}\mqty(3&2&2\\2&3&2\\2&2&3),~\vb*{P}=\mqty(0&1&0\\1&0&1\\0&0&1),~\vb*{B}=\vb*{P}^{-1}\vb*{A}^*\vb*{P}$, 求 $\vb*{B}+2\vb*{E}$ 的特征值与特征向量.
\end{example}
\begin{solution}
    设 $\vb*{A}$ 的特征值为 $\lambda$, 对应的特征向量为 $\vb*{\xi}$, 即 $\vb*{A\xi}=\lambda\vb*{\xi}$, 由于 $|\vb*{A}=7\neq0|$, 所以 $\lambda\neq0$, 于是有 
    $$\vb*{A}^*\vb*{\xi}=\dfrac{|\vb*{A}|}{\lambda}\vb*{\xi},~(\vb*{B}+2\vb*{E})\vb*{P}^{-1}\xi=\qty(\dfrac{|\vb*{A}|}{\lambda}+2)\vb*{P}^{-1}\xi$$
    即 $\dfrac{|\vb*{A}|}{\lambda}+2$ 是 $\vb*{B}+2\vb*{E}$ 的特征值, $\vb*{P}^{-1}\vb*{\xi}$ 为对应的特征向量, 
    易求得 $\vb*{A}$ 的特征值为 $\lambda_1=\lambda_2=1,\lambda_3=7$, 属于 $\lambda_{1,2}=1$ 的特征向量为 
    $$\vb*{\xi}_1=(-1,1,0)^\top,~\vb*{\xi}_2=(-1,0,1)^\top$$
    属于 $\lambda_3$ 的特征向量为 $\vb*{\xi}_3=(1,1,1)^\top$, 因此, $\vb*{B}+2\vb*{E}$ 的三个特征向量为 $9,9,3$, 属于特征值 9 的两个线性无关的特征向量为 
    $$\vb*{P}^{-1}\vb*{\xi}_1=(1,-1,0)^\top,~\vb*{P}^{-1}\vb*{\xi}_2=(-1,-1,1)^\top$$
    属于特征值 3 的特征向量为 $\vb*{P}^{-1}\vb*{\xi}_3=(0,1,1)^\top.$
\end{solution}

\begin{example}
    设 $\vb*{A}$ 是线性空间 $\vb*{P}^3$ 的线性变换, 已知
    $$\vb*{A}(1,0,0)=(5,6,-3),~\vb*{A}(0,1,0)=(-1,0,1),~\vb*{A}(0,0,1)=(1,2,1)$$
    \begin{enumerate}[label=(\arabic{*})]
        \item 求 $\vb*{A}$ 的全部特征值和特征向量;
        \item 问: 能否找到 $\vb*{P}^3$ 的一组基, 使 $\vb*{A}$ 在这组基下的矩阵为对角矩阵?说明理由.
    \end{enumerate}
\end{example}
\begin{solution}
    \begin{enumerate}[label=(\arabic{*})]
        \item $\vb*{A}$ 在基 $(1,0,0),~(0,1,0),~(0,0,1)$ 下的矩阵为
              $\begin{pmatrix}
                      5  & -1 & 1 \\
                      6  & 0  & 2 \\
                      -3 & 1  & 1
                  \end{pmatrix}$, 那么 $\vb*{A}$ 的特征多项式为
              $$f(\lambda)=|\lambda\vb*{E}-\vb*{A}|=\begin{vmatrix}
                      \lambda -5 & 1       & -1         \\
                      -6         & \lambda & -2         \\
                      -3         & -1      & \lambda -1
                  \end{vmatrix}\xlongequal[]{}(\lambda-2)^3$$
              所以 $\vb*{A}$ 的特征值为 2, 对 $\lambda=2$ 求特征向量, 解齐次方程组
              $$\left\{\begin{matrix}
                      -3x_1 & + & x_2  & - & x_3  & =0 \\
                      -6x_1 & + & 2x_2 & - & 2x_3 & =0 \\
                      3x_1  & - & x_2  & + & x_3  & =0
                  \end{matrix}\right.$$
              求得基础解系为: $(1,3,0),~(0,1,1)$, 所以 $\vb*{A}$ 的全部特征向量为 $k_1(1,3,0)+k_2(0,1,1)$, 其中 $k_1,~k_2$ 为数域 $\vb*{P}$
              中不全为零的任意数, 这些特征向量都属于特征值 2.
        \item 因为 $\vb*{A}$ 只有 2 个线性无关的特征向量, 而 $\vb*{P}^3$ 是 3 维线性空间, 所以找不到一组基, 使 $\vb*{A}$ 在这组基下的矩阵为对角矩阵.
    \end{enumerate}
\end{solution}

\begin{example}
    设 $\vb*{A}$ 是线性空间 $\vb*{P}^3$ 的线性变换, 
    已知 $$\vb*{A}(1,0,0)=(8,-6,3),~\vb*{A}(1,1,0)=(14,-10,6),~\vb*{A}(1,1,1)=(8,-4,5)$$
    求 $\vb*{A}$ 的全部特征值和特征向量, 并求 $\vb*{P}^3$ 的一组基使 $\vb*{A}$ 在这组基下的矩阵为对角矩阵.
\end{example}
\begin{solution}
    因为
    \begin{flalign*}
        \vb*{A}(1,0,0) & =(8,-6,3)=14(1,0,0)-9(1,1,0)+3(1,1,1)    \\
        \vb*{A}(1,1,0) & =(14,-10,6)=24(1,0,0)-16(1,1,0)+6(1,1,1) \\
        \vb*{A}(1,1,1) & =(8,-4,5)=12(1,0,0)-9(1,1,0)+5(1,1,1)
    \end{flalign*}
    所以 $\vb*{A}$ 在基 $(1,0,0),~(1,1,0),~(1,1,1)$ 下的矩阵为
    $\vb*{A}=\begin{pmatrix}
            14 & 24  & 12 \\
            -9 & -16 & -9 \\
            3  & 6   & 5
        \end{pmatrix}$, 
    $\vb*{A}$ 的特征多项式为
    $$f(\lambda)=|\lambda \vb*{E}-\vb*{A}|=\begin{vmatrix}
            \lambda -14 & -24         & -12        \\
            9           & \lambda +16 & 9          \\
            -3          & -6          & \lambda -5
        \end{vmatrix}\xlongequal[r_1-4r_3]{r_2+3r_3}\begin{vmatrix}
            \lambda -2 & 0          & -4\lambda +8 \\
            0          & \lambda -2 & 3\lambda -6  \\
            -3         & -6         & \lambda -5
        \end{vmatrix}=(\lambda-2)^2(\lambda+1)$$
    对 $\lambda=-1$ 解齐次方程组
    $$\left\{\begin{matrix}
            -15x_1 & - & 24x_2 & - & 12x_3 & =0 \\
            9x_1   & + & 15x_2 & + & 9x_3  & =0 \\
            -3x_1  & - & 6x_2  & - & 6x_3  & =0
        \end{matrix}\right.$$
    得基础解系: $(4,-3,1)$, 对应特征向量为 $4(1,0,0)-3(1,1,0)+(1,1,1)=(2,-2,1)$, 所以属于特征值 -1 的全部特征向量为 $k(2,-2,1)$, 
    其中 $k$ 为数域 $\vb*{P}$ 中任一非零数;\\
    对 $\lambda=2$ 解齐次方程组
    $$\left\{\begin{matrix}
            -12x_1 & - & 24x_2 & - & 12x_3 & =0 \\
            9x_1   & + & 18x_2 & + & 9x_3  & =0 \\
            -3x_1  & - & 6x_2  & - & 3x_3  & =0
        \end{matrix}\right.$$
        得基础解系 $(-1,0,1),~(0,-1,2)$, 对应于特征向量分别为
        $$-(1,0,0)+(1,1,1)=(0,1,1)\text{ 及 }-(1,1,0)+2(1,1,1)=(1,1,2)$$
        所以属于特征值 2 的全部特征向量为 $k_1(0,1,1)+k_2(1,1,2)$, 其中 $k_1,~k_2$ 为数域 $\vb*{P}$ 不全为 0 的任意性, 
        于是以 $$(2,-2,1),~(0,1,1),~(1,1,2)$$ 为基, 则 $\vb*{A}$ 在这组基下的矩阵为对角矩阵 $\mathrm{diag}(-1,2,2).$
\end{solution}

\begin{example}
    设 $\vb*{A}$ 是 3 阶矩阵, $\lambda_1,\lambda_2,\lambda_3$ 是 $\vb*{A}$ 的 3 个不同特征值, 对应的特征向量分别为 $\vb*{\alpha}_1,\vb*{\alpha}_2,\vb*{\alpha}_3$, 令 $\vb*{\beta}=\vb*{\alpha}_1+\vb*{\alpha}_2+\vb*{\alpha}_3$, 
    \begin{enumerate}[label=(\arabic{*})]
        \item 证明: $\vb*{\beta}=\vb*{\alpha}_1+\vb*{\alpha}_2+\vb*{\alpha}_3$ 不是 $\vb*{A}$ 的特征向量;
        \item 证明: $\vb*{\beta},\vb*{A\beta},\vb*{A^2\beta}$ 线性无关;
        \item 若 $\vb*{A}^3\vb*{\beta}=2\vb*{A\beta}$, 求 $\vb*{A}$ 的特征值;
        \item 在 (3) 的基础上证明: $\vb*{A\beta}$ 和 $\vb*{A}^2\vb*{\beta}$ 是方程组 $\qty(\vb*{A}^2-2\vb*{E})\vb*{x}=\vb*{0}$ 的基础解系.
    \end{enumerate}
\end{example}
\begin{solution}
    \begin{enumerate}[label=(\arabic{*})]
        \item 反证法: 假设 $\lambda$ 是 $\vb*{A}$ 的特征值, 且对应的特征向量为 $\vb*{\beta}$, 那么 $\vb*{A\beta}=\lambda\vb*{\beta}$, 即
              \begin{flalign*}
                  \vb*{A}(\vb*{\alpha}_1+\vb*{\alpha}_2+\vb*{\alpha}_3)                               & =\lambda(\vb*{\alpha}_1+\vb*{\alpha}_2+\vb*{\alpha}_3)             \\
                  \Rightarrow \vb*{A\alpha}_1+\vb*{A\alpha}_2+\vb*{A\alpha}_3                         & =\lambda\vb*{\alpha}_1+\lambda\vb*{\alpha}_2+\lambda\vb*{\alpha}_3 \\
                  \Rightarrow \lambda_1\vb*{\alpha}_1+\lambda_1\vb*{\alpha}_2+\lambda_1\vb*{\alpha}_3 & =\lambda\vb*{\alpha}_1+\lambda\vb*{\alpha}_2+\lambda\vb*{\alpha}_3 \\
              \end{flalign*}
              即 $\Rightarrow (\lambda_1-\lambda)\vb*{\alpha}_1+(\lambda_2-\lambda)\vb*{\alpha}_2+(\lambda_3-\lambda)\vb*{\alpha}_3=0$, 即 $\lambda_1=\lambda_2=\lambda_3$, 但这与题意矛盾, 故假设不成立, 
              即 $\vb*{\beta}=\vb*{\alpha}_1+\vb*{\alpha}_2+\vb*{\alpha}_3$ 不是 $\vb*{A}$ 的特征向量.
        \item 要证: $\vb*{\beta},\vb*{A\beta},\vb*{A^2\beta}$ 线性无关, 即证: $k_1\vb*{\beta}+k_2\vb*{A\beta}+k_3\vb*{A}^2\vb*{\beta}=\vb*{0}$, 其中 $k_1+k_2+k_3=0$, 为此, 则有
              $$k_1(\vb*{\alpha}_1+\vb*{\alpha}_2+\vb*{\alpha}_3)+k_2(\lambda_1\vb*{\alpha}_1+\lambda_2\vb*{\alpha}_2+\lambda_3\vb*{\alpha}_3)+k_3\qty(\lambda_1^2\vb*{\alpha}_1+\lambda_2^2\vb*{\alpha}_2+\lambda_3^2\vb*{\alpha}_3)=\vb*{0}$$
              为使等式成立, 则有 $$\begin{cases}
                  k_1+k_2\lambda_1+k_3\lambda_1^2=0 \\
                  k_1+k_2\lambda_2+k_3\lambda_2^2=0 \\
                  k_1+k_2\lambda_3+k_3\lambda_3^2=0
              \end{cases}\Rightarrow \mqty|1&\lambda_1&\lambda_1^2\\[6pt]1&\lambda_2&\lambda_2^2\\[6pt]1&\lambda_3&\lambda_3^2|$$
              该行列式为 Vandermonde 行列式, 则其值为 $(\lambda_2-\lambda_1)(\lambda_3-\lambda_1)(\lambda_3-\lambda_2)\neq0$, 故 $k_1+k_2+k_3=0$, 因此 $\vb*{\beta},\vb*{A\beta},\vb*{A^2\beta}$ 线性无关.
        \item 设矩阵 $\vb*{B}\sim\vb*{A}$, 且 $\exists\vb*{P}$ 使得 $\vb*{P}^{-1}\vb*{AP}=\vb*{B}$, 即 $\vb*{AP}=\vb*{PB}$, 不妨取 $\vb*{P}=\qty(\vb*{\beta},\vb*{A\beta},\vb*{A^2\beta})$, 因此
              $$\vb*{AP}=\qty(\vb*{A\beta},\vb*{A}^2\vb*{\beta},\vb*{A}^3\vb*{\beta})\xlongequal{\vb*{A}^3\vb*{\beta}=2\vb*{A\beta}}\qty(\vb*{A\beta},\vb*{A}^2\vb*{\beta},2\vb*{A\beta})=\qty(\vb*{\beta},\vb*{A\beta},\vb*{A^2\beta})\mqty(0&0&0\\1&0&2\\0&1&0)$$
              因此 $\vb*{B}=\mqty(0&0&0\\1&0&2\\0&1&0)$, 则特征值方程为 $|\lambda\vb*{E}-\vb*{B}|=\mqty|\lambda &0&0\\-1&\lambda&-2\\0&-1&\lambda|=\lambda\mqty|\lambda&-2\\-1&\lambda|=\lambda\qty(\lambda^2-2)=0$, 因此特征值为 $\lambda_1=0,\lambda_2=\sqrt{2},\lambda_3=-\sqrt{2}$.
        \item 为证: $\vb*{A\beta}$ 和 $\vb*{A}^2\vb*{\beta}$ 是方程组 $\qty(\vb*{A}^2-2\vb*{E})\vb*{x}=\vb*{0}$ 的基础解系, 需分别证:
              \begin{enumerate}[label=(\roman{*})]
                  \item $\vb*{A\beta}$ 和 $\vb*{A}^2\vb*{\beta}$ 是方程组 $\qty(\vb*{A}^2-2\vb*{E})\vb*{x}=\vb*{0}$ 的解;\\
                        当 $\vb*{x}=\vb*{A\beta}$ 时, 有 $\qty(\vb*{A}^2-2\vb*{E})\vb*{A\beta}=\vb*{A}^3\vb*{\beta}-2\vb*{A\beta}\xlongequal{\vb*{A}^3\vb*{\beta}=2\vb*{A\beta}}\vb*{0}$ 成立;\\
                        当 $\vb*{x}=\vb*{A}^2\vb*{\beta}$ 时, 有 $\qty(\vb*{A}^2-2\vb*{E})\vb*{A}^2\vb*{\beta}=\vb*{A}^4\vb*{\beta}-2\vb*{A}^2\vb*{\beta}=\vb*{A}\qty(\vb*{A}^3\vb*{\beta}-2\vb*{A\beta})\xlongequal{\vb*{A}^3\vb*{\beta}=2\vb*{A\beta}}\vb*{0}$ 成立;
                  \item $\vb*{A\beta}$ 和 $\vb*{A}^2\vb*{\beta}$ 线性无关;\\
                        由 (2) 可知 $\vb*{\beta},\vb*{A\beta},\vb*{A^2\beta}$ 线性无关 \textcolor{red}{(整体无关, 部分必无关; 部分相关, 整体必相关)}, 故 $\vb*{A\beta}$ 和 $\vb*{A}^2\vb*{\beta}$ 线性无关;
                  \item $S=n-\rank\qty(\vb*{A}^2-2\vb*{E})$, 其中 $S=2$.\\
                        因为 $\vb*{P}^{-1}\vb*{AP}=\vb*{\Lambda}_1$, 即 $\vb*{A}$ 能相似对角化, 故 $\vb*{P}^{-1}\qty(\vb*{A}^2-2\vb*{E})\vb*{P}$ 亦能相似对角化为矩阵 $\vb*{\Lambda}_2$, 又因为 $\qty(\vb*{A}^2-2\vb*{E})$ 的特征值为 $2,0,0$, 那么 $\rank\qty(\vb*{A}^2-2\vb*{E})=1$, 且 $n=3$, 故 $S=n-\rank\qty(\vb*{A}^2-2\vb*{E})$ 成立, 
              \end{enumerate}
              综上所述, $\vb*{A\beta}$ 和 $\vb*{A}^2\vb*{\beta}$ 是方程组 $\qty(\vb*{A}^2-2\vb*{E})\vb*{x}=\vb*{0}$ 的基础解系.
    \end{enumerate}
\end{solution}
