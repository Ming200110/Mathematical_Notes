\section{矩阵的秩}

\subsection{秩的相关不等式证明}

\begin{definition}[满秩方阵]
    对于 $n$ 阶方阵 $\vb*{A}$,若 $\rank\vb*{A}=n$,则称 $\vb*{A}$ 为满秩 (非退化) 方阵,否则称为降秩 (退化) 方阵.
\end{definition}

\begin{theorem}[矩阵与向量组秩的联系]
    设 $\vb*{\alpha}_i~ (i=1,2,\cdots,m),~\vb*{\beta}$ 是 $n$ 维列向量,则
    $$\rank(\vb*{\alpha}_1,\vb*{\alpha}_2,\cdots,\vb*{\alpha}_m)\leqslant \rank(\vb*{\alpha}_1,\vb*{\alpha}_2,\cdots,\vb*{\alpha}_m,\vb*{\beta})\leqslant \rank(\vb*{\alpha}_1,\vb*{\alpha}_2,\cdots,\vb*{\alpha}_m)+1.$$
\end{theorem}

\begin{example}
    设 $\vb*{A},~\vb*{B}$ 分别是 $f\times m,~f\times n$ 的矩阵,证明: $\rank\vb*{A}\leqslant \rank(\vb*{A},\vb*{B})\leqslant \rank\vb*{A}+n.$
\end{example}
\begin{proof}{\songti \textbf{证}}
    设 $\vb*{A}=(\vb*{\alpha}_1,\vb*{\alpha}_2,\cdots,\vb*{\alpha}_m),~\vb*{B}=(\vb*{\beta}_1,\vb*{\beta}_2,\cdots,\vb*{\beta}_n)$,那么 
    $$\rank\vb*{A}=\rank(\vb*{\alpha}_1,\vb*{\alpha}_2,\cdots,\vb*{\alpha}_m)\leqslant \rank(\vb*{\alpha}_1,\vb*{\alpha}_2,\cdots,\vb*{\alpha}_m,\vb*{\beta}_1)\leqslant\cdots\leqslant \rank(\vb*{\alpha}_1,\cdots,\vb*{\alpha}_m,\vb*{\beta}_1,\cdots,\vb*{\beta}_n)=\rank(\vb*{A},\vb*{B})$$
    又因为 
    $$\rank(\vb*{A},\vb*{B})=\rank(\vb*{\alpha}_1,\cdots,\vb*{\alpha}_m,\vb*{\beta}_1,\cdots,\vb*{\beta}_n)\leqslant \rank(\vb*{\alpha}_1,\cdots,\vb*{\alpha}_m,\vb*{\beta}_1,\cdots,\vb*{\beta}_{n-1})+1\leqslant \cdots\leqslant \rank(\vb*{\alpha}_1,\cdots,\vb*{\alpha}_m)+n$$
    综上 $\rank\vb*{A}\leqslant \rank(\vb*{A},\vb*{B})\leqslant \rank\vb*{A}+n$ 成立.
\end{proof}

\begin{theorem}
    对于 $m\times n$ 矩阵 $\vb*{A}$ 与 $\vb*{B}$,有 $\rank(\vb*{A}+\vb*{B})\leqslant \rank(\vb*{A},\vb*{B})$.
\end{theorem}
\begin{proof}[{\songti \textbf{证}}]
    不妨设 $\vb*{A}=(\vb*{\alpha}_1,\vb*{\alpha}_2,\cdots,\vb*{\alpha}_m),~\vb*{B}=(\vb*{\beta}_1,\vb*{\beta}_2,\cdots,\vb*{\beta}_m)$,则结论显然成立.
\end{proof}

\begin{theorem}
    对于 $m\times n$ 矩阵 $\vb*{A}$ 与 $m\times s$ 矩阵 $\vb*{B}$,有 $\rank(\vb*{A},\vb*{B})\leqslant \rank\vb*{A}+\rank\vb*{B}.$
\end{theorem}
\begin{proof}[{\songti \textbf{证}}]
    提示: 用极大无关组可证明.
\end{proof}

\begin{theorem}
    对于 $n$ 阶方阵 $\vb*{A},~\vb*{B}$,则 $\rank(\vb*{A}\pm\vb*{B})\leqslant \rank\vb*{A}+\rank\vb*{B}.$
\end{theorem}

\begin{theorem}
    对于 $m\times n$ 矩阵 $\vb*{A}$ 与 $n\times s$ 矩阵 $\vb*{B}$,有 $\rank(\vb*{AB})\leqslant \min\qty{\rank\vb*{A},\rank\vb*{B}}.$
\end{theorem}
\begin{proof}[{\songti \textbf{证}}]
    先设 $\vb*{A}=(\vb*{\alpha}_1,\vb*{\alpha}_2,\cdots,\vb*{\alpha}_n),~\vb*{B}=\mqty(b_{11}&\cdots&b_{1s}\\\vdots&&\vdots\\b_{n1}&\cdots&b_{ns})$,
    并且无论是 $b_{11}\vb*{\alpha}_1+\cdots+b_{n1}\vb*{\alpha}_n$ 还是 $b_{1s}\vb*{\alpha}_1+\cdots+b_{ns}\vb*{\alpha}_n$,均可由 $\vb*{\alpha}_1,\cdots,\vb*{\alpha}_n$ 线性表示,于是
    $$\rank(\vb*{AB})=\rank(b_{11}\vb*{\alpha}_1+\cdots+b_{n1}\vb*{\alpha}_n,\cdots,b_{1s}\vb*{\alpha}_1+\cdots+b_{ns}\vb*{\alpha}_n)\leqslant \rank(\vb*{\alpha}_1,\cdots,\vb*{\alpha}_n)=\rank(\vb*{A})$$
    再设 $\vb*{A}=\mqty(a_{11}&\cdots&a_{1n}\\\vdots&&\vdots\\a_{m1}&\cdots&a_{mn}),~\vb*{B}=\mqty(\vb*{\beta}_1\\ \vdots\\\vb*{\beta}_n)^\top$,同理 $\rank(\vb*{AB})\leqslant \rank\vb*{B}$,
    则 $$\rank(\vb*{AB})\leqslant \rank\vb*{A}\text{ 且 }\rank(\vb*{AB})\leqslant \rank\vb*{B}.$$
\end{proof}

\begin{theorem}
    若 $\vb*{A}$ 是 $m\times n$ 矩阵,$\vb*{B}$ 是 $n\times s$ 矩阵,且 $\vb*{AB}=\vb*{O}$,则 $\rank\vb*{A}+\rank\vb*{B}\leqslant n.$
\end{theorem}

\begin{theorem}
    对于 $m\times n$ 矩阵 $\vb*{A}$ 与 $s\times t$ 矩阵 $\vb*{B}$,有 $\rank\mqty(\vb*{A}&\vb*{O}\\\vb*{C}&\vb*{B})\geqslant \rank\vb*{A}+\rank\vb*{B}.$
\end{theorem}

\begin{example}
    设 $\vb*{A},~\vb*{B}$ 为 $n$ 阶方阵,证明:
    $$\rank(\vb*{AB}-\vb*{E})\leqslant \rank(\vb*{A}-\vb*{E})+\rank(\vb*{B}-\vb*{E})$$
    这里 $\vb*{E}$ 为 $n$ 阶单位矩阵.
\end{example}
\begin{proof}[{\songti \textbf{证}}]
    因为 $\vb*{AB}-\vb*{E}=(\vb*{A}-\vb*{E})\vb*{B}+(\vb*{B}-\vb*{E})$,于是 
    $$\rank(\vb*{AB}-\vb*{E})\leqslant \rank((\vb*{A}-\vb*{E})\vb*{B})+\rank(\vb*{B}-\vb*{E})\leqslant \rank(\vb*{A}-\vb*{E})+\rank(\vb*{B}-\vb*{E}).$$
\end{proof}

\begin{example}
    设 $\vb*{A},~\vb*{B}$ 为 $n$ 阶方阵,证明:
    \begin{enumerate}[label=(\arabic{*})]
        \item $\rank(\vb*{A}-\vb*{B})\geqslant \rank\vb*{A}-\rank\vb*{B}$;
        \item 若 $\vb*{A}$ 是可逆矩阵,则结论 (1) 中的等号成立当且仅当 $\vb*{BA}^{-1}\vb*{B}=\vb*{B}.$
    \end{enumerate}
\end{example}
\begin{proof}[{\songti \textbf{证}}]
    \begin{enumerate}[label=(\arabic{*})]
        \item 由 $\vb*{A}=(\vb*{A}-\vb*{B})+\vb*{B}$,且 $\rank\vb*{A}\leqslant \rank(\vb*{A}-\vb*{B})+\rank\vb*{B}$,移项得 $\rank(\vb*{A}-\vb*{B})\geqslant \rank\vb*{A}-\rank\vb*{B}$;
        \item 利用分块矩阵的初等行变换,得 
        $$\mqty(\vb*{A}-\vb*{B}&\vb*{O}\\\vb*{O}&\vb*{B})\to\mqty(\vb*{A}&\vb*{B}\\\vb*{B}&\vb*{B})\xrightarrow[]{\text{行}}\mqty(\vb*{A}&\vb*{B}\\\vb*{O}&\vb*{B}-\vb*{BA}^{-1}\vb*{B})\xrightarrow[]{\text{列}}\mqty(\vb*{A}&\vb*{O}\\\vb*{O}&\vb*{B}-\vb*{BA}^{-1}\vb*{B})$$
        因为初等行变化不会改变矩阵的秩,所以 
        $$\rank\mqty(\vb*{A}-\vb*{B}&\vb*{O}\\\vb*{O}&\vb*{B})=\rank\mqty(\vb*{A}&\vb*{O}\\\vb*{O}&\vb*{B}-\vb*{BA}^{-1}\vb*{B})$$
        即
        $$\rank(\vb*{A}-\vb*{B})+\rank\vb*{B}=\rank\vb*{A}+\rank(\vb*{B}-\vb*{BA}^{-1}\vb*{B})$$
        故 $\rank(\vb*{A}-\vb*{B})=\rank\vb*{A}-\rank\vb*{B}$ 当且仅当 $\rank(\vb*{B}-\vb*{BA}^{-1}\vb*{B})=0$,即 $\vb*{BA}^{-1}\vb*{B}=\vb*{B}.$
    \end{enumerate}
\end{proof}

\begin{example}
    设 $\vb*{A},~\vb*{B}$ 都是数域 $P$ 上的 $n$ 阶方阵,满足 $\vb*{AB}=\vb*{BA}$,证明:
    $$\rank(\vb*{A}+\vb*{B})\leqslant \rank\vb*{A}+\rank\vb*{B}-\rank(\vb*{AB}).$$
\end{example}
\begin{proof}[{\songti \textbf{证}}]
    利用分块矩阵的初等行变换,有
    $$\mqty(\vb*{A}&\vb*{O}\\\vb*{O}&\vb*{B})\xrightarrow[]{\text{行}}\mqty(\vb*{A}&\vb*{B}\\\vb*{O}&\vb*{B})\xrightarrow[]{\text{列}}\mqty(\vb*{A}+\vb*{B}&\vb*{B}\\\vb*{B}&\vb*{B})$$
    因为 $\vb*{AB}=\vb*{BA}$,于是
    $$\mqty(\vb*{E}&\vb*{O}\\-\vb*{B}&\vb*{A}+\vb*{B})\mqty(\vb*{A}+\vb*{B}&\vb*{B}\\\vb*{B}&\vb*{B})=\mqty(\vb*{A}+\vb*{B}&\vb*{B}\\\vb*{O}&\vb*{AB})$$
    于是,有 $$\rank\vb*{A}+\rank\vb*{B}=\rank\mqty(\vb*{A}+\vb*{B}&\vb*{B}\\\vb*{B}&\vb*{B})\geqslant \rank\mqty(\vb*{A}+\vb*{B}&\vb*{B}\\\vb*{O}&\vb*{AB})\geqslant \rank(\vb*{A}+\vb*{B})+\rank(\vb*{AB})$$
    即得证不等式.
\end{proof}

\subsection{秩的相关等式证明}

\begin{theorem}
    对于 $m\times n$ 矩阵 $\vb*{A}$ 与 $s\times t$ 矩阵 $\vb*{B}$,有 $\rank\mqty(\vb*{A}&\vb*{O}\\\vb*{O}&\vb*{B})=\rank\vb*{A}+\rank\vb*{B}.$
\end{theorem}

\begin{theorem}[秩的第一降阶定理]
    若 $\vb*{A}$ 是 $r$ 阶可逆矩阵,分块矩阵 $\vb*{M}=\mqty(\vb*{A}&\vb*{B}\\\vb*{C}&\vb*{D})$,那么 $$\rank\vb*{M}=r+\rank(\vb*{D}-\vb*{CA}^{-1}\vb*{B}).$$
\end{theorem}

\begin{example}
    设 $\vb*{A},~\vb*{D}$ 分别为 $m$ 阶与 $n$ 阶可逆矩阵,$\vb*{B},~\vb*{C}$ 分别为 $m\times n$ 与 $n\times m$ 矩阵,证明:
    $$\rank \vb*{A}-\rank\qty(\vb*{A}-\vb*{BD}^{-1}\vb*{C})=\rank \vb*{D}-\rank\qty(\vb*{D}-\vb*{CA}^{-1}\vb*{B}).$$
\end{example}
\begin{proof}[{\songti \textbf{证}}]
    利用分块矩阵的初等行变换,得
    \begin{flalign*}
        \mqty(\vb*{A}&\vb*{O}\\\vb*{O}&\vb*{D}-\vb*{CA}^{-1}\vb*{B})&\xrightarrow[]{\text{行}}\mqty(\vb*{A}&\vb*{O}\\\vb*{C}&\vb*{D}-\vb*{CA}^{-1}\vb*{B})\xrightarrow[]{\text{列}}\mqty(\vb*{A}&\vb*{B}\\\vb*{C}&\vb*{D})\\
        \mqty(\vb*{A}-\vb*{BD}^{-1}\vb*{C}&\vb*{O}\\\vb*{O}&\vb*{D})&\xrightarrow[]{\text{行}}\mqty(\vb*{A}-\vb*{BD}^{-1}\vb*{C}&\vb*{B}\\\vb*{O}&\vb*{D})\xrightarrow[]{\text{列}}\mqty(\vb*{A}&\vb*{B}\\\vb*{C}&\vb*{D})
    \end{flalign*}
    因为初等行变化不会改变矩阵的秩,于是 
    \begin{flalign*}
        \rank\vb*{A}+\rank\qty(\vb*{D}-\vb*{CA}^{-1}\vb*{B})&=\rank\mqty(\vb*{A}&\vb*{O}\\\vb*{O}&\vb*{D}-\vb*{CA}^{-1}\vb*{B})=\rank\mqty(\vb*{A}&\vb*{B}\\\vb*{C}&\vb*{D})\\
        \rank\qty(\vb*{A}-\vb*{BD}^{-1}\vb*{C})+\rank\vb*{D}&=\rank\mqty(\vb*{A}-\vb*{BD}^{-1}\vb*{C}&\vb*{O}\\\vb*{O}&\vb*{D})=\rank\mqty(\vb*{A}&\vb*{B}\\\vb*{C}&\vb*{D})
    \end{flalign*}
    得证 $\rank \vb*{A}-\rank\qty(\vb*{A}-\vb*{BD}^{-1}\vb*{C})=\rank \vb*{D}-\rank\qty(\vb*{D}-\vb*{CA}^{-1}\vb*{B}).$
\end{proof}
\begin{inference}
    \label{rankmn}
    特别地,令 $\vb*{A}=\lambda_0\vb*{E}_m,~\vb*{D}=\vb*{E}_n$,其中 $\lambda_0$ 为任意非零常数,则有
    $$m-\rank\qty(\lambda_0\vb*{E}_m-\vb*{BC})=n-\rank\qty(\lambda_0\vb*{E}_n-\vb*{CB}).$$
\end{inference}

\begin{example}
    设 $\vb*{A}$ 是 $m\times n$ 矩阵,$\vb*{B}$ 是 $n\times s$ 矩阵,且 $\rank(\vb*{AB})=\rank\vb*{B}$,证明:
    对任一 $s\times t$ 矩阵 $\vb*{C}$,有 $\rank(\vb*{ABC})=\rank(\vb*{BC}).$
\end{example}
\begin{proof}[{\songti \textbf{证}}]
    利用分块矩阵的初等行变换,得 
    $$\mqty(\vb*{ABC}&\vb*{O}\\\vb*{O}&\vb*{B})\xrightarrow[]{\text{行}}\mqty(\vb*{ABC}&\vb*{AB}\\\vb*{O}&\vb*{B})\xrightarrow[]{\text{列}}\mqty(\vb*{O}&\vb*{AB}\\-\vb*{BC}&\vb*{B})\xrightarrow[]{\text{列}}\mqty(\vb*{AB}&\vb*{O}\\\vb*{B}&\vb*{BC})$$
    因为初等行变化不会改变矩阵的秩,所以
    $$\rank\mqty(\vb*{ABC}&\vb*{O}\\\vb*{O}&\vb*{B})=\rank(\vb*{ABC})+\rank\vb*{B}=\rank\mqty(\vb*{AB}&\vb*{O}\\\vb*{B}&\vb*{BC})\geqslant \rank(\vb*{AB})+\rank(\vb*{BC})$$
    又 $\rank(\vb*{AB})=\rank\vb*{B}$,所以 $\rank(\vb*{ABC})\geqslant \rank(\vb*{BC})$,又 $\rank(\vb*{ABC})\leqslant \rank(\vb*{BC})$,
    故 $$\rank(\vb*{ABC})=\rank(\vb*{BC}).$$
\end{proof}

\subsection{秩的应用}

\begin{example}
    设 $\vb*{A},~\vb*{B}$ 分别是实数域上的 $3\times4$ 和 $4\times3$ 矩阵,且满足
    $$\vb*{AB}=\mqty(-9&2&2\\-20&5&4\\-35&7&8),~\vb*{BA}=\mqty(-14 &2x-5 &2 &6\\0 &1 &0 &0\\-15 &3x-3 &3 &6\\-32 &6x-7 &4 &14)$$
    求 $x$ 的值.
\end{example}
\begin{solution}
    由推论 \ref{rankmn},令 $\lambda_0=1$,于是  $3-\rank\qty(\vb*{E}_3-\vb*{AB})=4-\rank\qty(\vb*{E}_4-\vb*{BA})$,那么 
    \begin{flalign*}
        \rank\qty(\vb*{E}_4-\vb*{BA})&=4-3+\rank\qty(\vb*{E}_3-\vb*{AB})\\
        \rank\mqty(15 &5-2x &-2 &-6\\0 &0 &0 &0\\15 &3-3x &-2 &-6\\32 &7-6x &-4 &-13)&=1+\rank\mqty(10 &-2 &-2\\20 &-4 &-4\\35 &-7 &-7)=2
    \end{flalign*}
    解得 $x=-2.$
\end{solution}

\begin{example}
    设 $\vb*{A}$ 是 4 阶矩阵,向量 $\vb*{\alpha},~\vb*{\beta}$ 是齐次方程组 $(\vb*{A}-\vb*{E})\vb*{x}=\vb*{0}$ 的基础解系,向量 $\vb*{\gamma}$ 是齐次方程组 $(\vb*{A}+\vb*{E})\vb*{x}=\vb*{0}$ 的基础解系,求 $\qty(\vb*{A}^2-\vb*{E})\vb*{x}=\vb*{0}$ 的通解
    \begin{tasks}(2)
        \task $c_1\vb*{\alpha}+c_2\vb*{\beta}$ 其中 $c_1,~c_2$ 为任意常数
        \task $c_1\vb*{\alpha}+c_2\vb*{\gamma}$ 其中 $c_1,~c_2$ 为任意常数
        \task $c_1\vb*{\beta}+c_2\vb*{\gamma}$ 其中 $c_1,~c_2$ 为任意常数
        \task $c_1\vb*{\alpha}+c_2\vb*{\beta}+c_3\vb*{\gamma}$ 其中 $c_1,~c_2,~c_3$ 为任意常数
    \end{tasks}
\end{example}
\begin{solution}
    因为向量 $\vb*{\alpha},~\vb*{\beta}$ 是齐次方程组 $(\vb*{A}-\vb*{E})\vb*{x}=\vb*{0}$ 的基础解系,所以 $$s=n-\rank(\vb*{A}-\vb*{E})\Rightarrow 2=4-\rank(\vb*{A}-\vb*{E})\Rightarrow \rank(\vb*{A}-\vb*{E})=2$$
    同理 $\rank(\vb*{A}+\vb*{E})=3$,又因为 
    $$\rank\qty(\vb*{A}^2-\vb*{E})=\rank[(\vb*{A}+\vb*{E})(\vb*{A}-\vb*{E})]\leqslant \min\qty{\rank(\vb*{A}+\vb*{E}),\rank(\vb*{A}-\vb*{E})}\Rightarrow \rank\qty(\vb*{A}^2-\vb*{E})\leqslant 2$$
    当 $\rank\qty(\vb*{A}^2-\vb*{E})=0$ 时,$\vb*{A}^2-\vb*{E}=\vb*{O}\Rightarrow (\vb*{A}+\vb*{E})(\vb*{A}-\vb*{E})=\vb*{O}\Rightarrow \rank(\vb*{A}+\vb*{E})+\rank(\vb*{A}-\vb*{E})\leqslant 4$,而 $\rank(\vb*{A}+\vb*{E})=3,~\rank(\vb*{A}-\vb*{E})=2$,与之矛盾;
    当 $\rank\qty(\vb*{A}^2-\vb*{E})=2$ 时,说明有 2 个线性无关解,$\qty(\vb*{A}^2-\vb*{E})\vb*{x}=\vb*{0}\Rightarrow (\vb*{A}+\vb*{E})(\vb*{A}-\vb*{E})\vb*{x}=\vb*{0}$,即 $\vb*{\alpha},~\vb*{\beta},~\vb*{\gamma}$ 是方程组的三个解,
    又因为 $\vb*{\alpha},~\vb*{\beta}$ 是齐次方程组 $(\vb*{A}-\vb*{E})\vb*{x}=\vb*{0}$ 的基础解系,说明 $\vb*{\alpha},~\vb*{\beta}$ 线性无关,且 $\vb*{\alpha},~\vb*{\beta}$ 是特征值 1 对应的特征向量,$\vb*{\gamma}$ 是 -1 对应的特征向量,因此 $\vb*{\gamma}$ 与 $\vb*{\alpha}$ 或 $\vb*{\gamma}$ 线性无关,故方程组有 3 个线性无关解,矛盾;
    所以 $\rank\qty(\vb*{A}^2-\vb*{E})=1\Rightarrow s'=n-\rank\qty(\vb*{A}^2-\vb*{E})=4-1=3$,因此选 D.
\end{solution}
