\section{伴随矩阵、逆矩阵与矩阵方程}

\subsection{伴随矩阵}

\begin{definition}[伴随矩阵的定义]
    设 $ \boldsymbol{A}=\left(a_{i j}\right) $ 是 $ n $ 阶方阵 $ (n \geqslant 2)$, $A $ 的伴随矩阵 $\vb*{A}^*$ (或 $\adj\vb*{A}$) 定义为
    $$\boldsymbol{A}^{*}=\begin{pmatrix}
            A_{11}  & A_{21}  & \cdots & A_{n 1} \\
            A_{12}  & A_{22}  & \cdots & A_{n 2} \\
            \vdots  & \vdots  &        & \vdots  \\
            A_{1 n} & A_{2 n} & \cdots & A_{n n}
        \end{pmatrix},$$
    其中 $ A_{i j} $ 为 $ \boldsymbol{A} $ 的元素 $ a_{i j}~ (i, j=1,2, \cdots, n) $ 的代数余子式.
\end{definition}

伴随矩阵具有如下性质:
\setcounter{magicrownumbers}{0}
\begin{table}[H]
    \centering
    \begin{tabular}{l l}
        (\rownumber) $\displaystyle \boldsymbol{AA}^{*}=\boldsymbol{A}^{*} \boldsymbol{A}=|\boldsymbol{A}| \boldsymbol{E}$                                            & (\rownumber) $\displaystyle(\lambda \boldsymbol{A})^{*}=\lambda^{n-1} \boldsymbol{A}^{*}$                                                                                                                                                            \\
        (\rownumber) $\displaystyle \left|\boldsymbol{A}^{*}\right|=|\boldsymbol{A}|^{n-1} $                                                                          & (\rownumber) $\displaystyle\left(\boldsymbol{A}^{*}\right)^{\top}=\left(\boldsymbol{A}^{\top}\right)^{*}$                                                                                                                                            \\
        \midrule
        (\rownumber) $\displaystyle(\boldsymbol{AB})^{*}=\boldsymbol{B}^{*} \boldsymbol{A}^{*}$                                                                       & (\rownumber) $\displaystyle\text{当 } \boldsymbol{A} \text{ 可逆时, } \boldsymbol{A}^{*}=|\boldsymbol{A}| \boldsymbol{A}^{-1}$                                                                                                                       \\
        (\rownumber) $\displaystyle\text{当 } \boldsymbol{A} \text{ 可逆时, }\left(\boldsymbol{A}^{*}\right)^{-1}=\frac{1}{|\boldsymbol{A}|} \boldsymbol{A}$          & (\rownumber) $\displaystyle\text{当 } \boldsymbol{A} \text{ 可逆时, }\left(\boldsymbol{A}^{*}\right)^{-1}=\left(\boldsymbol{A}^{-1}\right)^{*} $                                                                                                     \\
        \midrule
        (\rownumber) $\displaystyle\left(\boldsymbol{A}^{*}\right)^{*}=\begin{cases}|\boldsymbol{A}|^{n-2} \boldsymbol{A}, & n>2 \\ \boldsymbol{A}, & n=2\end{cases}$ & (\rownumber) $\displaystyle\operatorname{rank} \boldsymbol{A}^{*}=\left\{\begin{array}{ll}n, & \operatorname{rank} \boldsymbol{A}=n \\ 1, & \operatorname{rank} \boldsymbol{A}=n-1 \\ 0, & \operatorname{rank} \boldsymbol{A}<n-1\end{array}\right.$ \\
    \end{tabular}
\end{table}

\begin{example}
    设矩阵 $\vb*{A}=\mqty(2&1&0\\1&2&0\\0&0&1)$, 矩阵 $\vb*{B}$ 满足 $\vb*{ABA}^*=2\vb*{BA}^*+\vb*{E}$, 求 $\qty|2\vb*{B}^\top|.$
\end{example}
\begin{solution}
    注意到 $\det\vb*{A}=3$, 且 $\vb*{A}^*\vb*{A}=|\vb*{A}|\vb*{E}$, 对 $\vb*{ABA}^*=2\vb*{BA}^*+\vb*{E}$, 右乘矩阵 $\vb*{A}$, 得
    $$3\vb*{AB}=6\vb*{B}+\vb*{A}\Rightarrow 3(\vb*{A}-2\vb*{E})\vb*{B}=\vb*{A}\Rightarrow |3(\vb*{A}-2\vb*{E})\vb*{B}|=3\Rightarrow |\vb*{B}|=\dfrac{1}{9|\vb*{A}-2\vb*{E}|}$$
    其中 $|\vb*{A}-2\vb*{E}|=\mqty|0&1&0\\1&0&0\\0&0&-1|=1$, 于是 $\vb*{B}=\dfrac{1}{9}\Rightarrow \qty|2\vb*{B}^\top|=\dfrac{8}{9}.$
\end{solution}

\begin{example}
    求矩阵 $\vb*{A}=\mqty(0&0&0&1&3\\0&0&0&-1&2\\1&1&1&0&0\\0&1&1&0&0\\0&0&1&0&0)$ 的所有代数余子式之和.
\end{example}
\begin{solution}
    记矩阵分块矩阵 $\vb*{A}=\mqty(\vb*{O}_{2\times 3}&\vb*{B}\\\vb*{C}&\vb*{O}_{3\times 2})$, 其中 $\vb*{B}=\mqty(1&3\\-1&2),~\vb*{C}=\mqty(1&1&1\\0&1&1\\0&0&1)$, 则 $\det\vb*{B}=5,~\det\vb*{C}=1$, 那么
    $$\vb*{A}^*=|\vb*{A}|\vb*{A}^{-1}=(-1)^{2\times 3}|\vb*{B}|~|\vb*{C}|\mqty(\vb*{O}_{3\times 2}&\vb*{C}^{-1}\\\vb*{B}^{-1}&\vb*{O}_{2\times 3})=\mqty(0&0&5&-5&0\\0&0&0&5&-5\\0&0&0&0&5\\2&-3&0&0&0\\1&1&0&0&0)$$
    因此矩阵 $\vb*{A}$ 的所有代数余子式之和为 $5-5+5-5+5+2-3+1+1=6.$
\end{solution}

\begin{example}
    设实矩阵 $\vb*{A}=(a_{ij})$, $A_{ij}$ 是 $a_{ij}$ 的代数余子式, $|a_{ij}|$ 与 $|A_{ij}|$ 分别表示两个表达式的绝对值, 则下列结论不正确的是
    \begin{tasks}
        \task 若 $\det\vb*{A}=1$ 且对任意 $i,j$ 均有 $a_{ij}=A_{ij}$, 则 $\vb*{A}$ 为正交矩阵
        \task 若 $\det\vb*{A}=1$ 且对任意 $i,j$ 均有 $a_{ij}=-A_{ij}$, 则 $\vb*{A}$ 为正交矩阵
        \task 若 $\vb*{A}$ 为正交矩阵且 $\det\vb*{A}=1$, 则对任意 $i,j$, 有 $|a_{ij}|=|A_{ij}|$
        \task 若 $\vb*{A}$ 为正交矩阵且 $\det\vb*{A}=-1$, 则对任意 $i,j$, 有 $|a_{ij}|=|A_{ij}|$
    \end{tasks}
\end{example}
\begin{solution}
    由 $a_{ij}=A_{ij}$, 可知, 对于 A 选项
    $$\vb*{AA}^\top=\vb*{AA}^*=\det\vb*{A}\cdot\vb*{E}=\vb*{E}\Rightarrow \vb*{A}\text{ 是正交矩阵}$$
    因此 A 正确; 那么 B 选项错误, 因为
    $$a_{ij}=-A_{ij}\Rightarrow \vb*{A}^*=-\vb*{A}^\top\not\Rightarrow \vb*{AA}^\top=\vb*{E}$$
    对于 C、D 选项, 有
    $$\vb*{AA}^\top=\vb*{E}=\vb*{A}^\top\vb*{A}\Rightarrow \vb*{A}^*\vb*{A}=\det\vb*{A}\cdot\vb*{E}=\pm\vb*{E}=\pm\vb*{A}^\top\vb*{A}\Rightarrow \vb*{A}^*=\pm\vb*{A}^\top\Rightarrow |a_{ij}|=|A_{ij}|$$
    因此 C,D 正确, 故选 B.
\end{solution}

\subsection{可逆矩阵}

可逆矩阵具有如下性质:
\begin{enumerate}[label=(\arabic{*})]
    \item 方阵 $ \boldsymbol{A} $ 可逆的充分必要条件是 $ |\boldsymbol{A}| \neq 0 $;
    \item 若 $ \boldsymbol{A} $ 可逆, 则 $ \boldsymbol{A}^{\top} $ 亦可逆, 
          且 $ \left(\boldsymbol{A}^{\top}\right)^{-1}=\left(\boldsymbol{A}^{-1}\right)^{\top}$ ;
    \item 若 $ \boldsymbol{A} $ 可逆, 则 $ \boldsymbol{A}^{-1} $ 亦可逆, 且 $ \left(\boldsymbol{A}^{-1}\right)^{-1}=\boldsymbol{A} $;
    \item 若 $ \boldsymbol{A} $ 可逆, 数 $ \lambda \neq 0 $, 则 $ \lambda \boldsymbol{A} $ 亦可逆, 且 $ (\lambda \boldsymbol{A})^{-1}=\dfrac{1}{\lambda} \boldsymbol{A}^{-1}$ ;
    \item 若 $\boldsymbol{A}$ 可逆, 则 $\left|\boldsymbol{A}^{-1}\right|=|\boldsymbol{A}|^{-1}$;
    \item 若 $\boldsymbol{A},\boldsymbol{B}$ 为同阶方阵, 且均可逆, 则 $\boldsymbol{AB}$ 亦可逆, 且 $(\boldsymbol{AB})^{-1}=\boldsymbol{B}^{-1}\boldsymbol{A}^{-1}.$
\end{enumerate}

\begin{theorem}[矩阵逆的和]
    设矩阵 $\vb*{A}$ 与 $\vb*{B}$ 均为 $n$ 阶矩阵, 则 $\qty|\vb*{A}^{-1}+\vb*{B}^{-1}|=\dfrac{|\vb*{A}+\vb*{B}|}{|\vb*{A}|\cdot|\vb*{B}|}.$
\end{theorem}
\begin{proof}[{\songti \textbf{证}}]
    $\qty|\vb*{A}^{-1}+\vb*{B}^{-1}|=\qty|\vb*{EA}^{-1}+\vb*{B}^{-1}\vb*{E}|=\qty|\vb*{B}^{-1}\vb*{BA}^{-1}+\vb*{B}^{-1}\vb*{AA}^{-1}|=\qty|\vb*{B}^{-1}(\vb*{B}+\vb*{A})\vb*{A}^{-1}|=\dfrac{|\vb*{A}+\vb*{B}|}{|\vb*{A}|\cdot|\vb*{B}|}.$
\end{proof}

\begin{theorem}[分块矩阵的逆 A]
    设 $ \vb*{A}=\left(\begin{array}{llll}\vb*{A}_{1} & & & \\ & \vb*{A}_{2} & & \\ & & \ddots & \\ & & & \vb*{A}_{m}\end{array}\right) $, 其中 $ \vb*{A}_{i}(i=1,2, \cdots, m) $ 均为可逆矩阵, 则
    $$\vb*{A}^{-1}=\left(\begin{array}{llll}
                \vb*{A}_{1}^{-1} &                  &        &                  \\
                                 & \vb*{A}_{2}^{-1} &        &                  \\
                                 &                  & \ddots &                  \\
                                 &                  &        & \vb*{A}_{m}^{-1}
            \end{array}\right).$$
    设 $\vb*{A}=\mqty(
        &                                        &           & \vb*{A}_{1}         \\
        &                                        & \vb*{A}_{2} &                \\
        & \begin{sideways}$\ddots$\end{sideways} &           &                \\
        \vb*{A}_{m}         &                                        &           &
        )$, 其中 $\vb*{A}_i(i=1,2,\cdots,m)$ 均为可逆矩阵, 则
    $$\vb*{A}^{-1}=\mqty(
        &                                        &           & \vb*{A}_{m}^{-1}         \\
        &                                        & \begin{sideways}$\ddots$\end{sideways} &                \\
        & \vb*{A}_{2}^{-1} &           &                \\
        \vb*{A}_{1}^{-1}         &                                        &           &
        ).$$
\end{theorem}

\begin{theorem}[分块矩阵的逆 B]
    设 $\vb*{A}$ 是 $m\times m$ 可逆矩阵, $\vb*{B}$ 是 $m\times n$ 矩阵, $\vb*{C}$ 是 $n\times m$ 矩阵, $\vb*{D}$ 是 $n\times n$ 可逆矩阵, 则有以下分块矩阵的求逆公式:
    $$\mqty(\vb*{A}&\vb*{B}\\\vb*{C}&\vb*{D})^{-1}=\mqty(\vb*{A}^{-1}+\vb*{A}^{-1}\vb*{BWCA}^{-1}&-\vb*{A}^{-1}\vb*{BW}\\-\vb*{WCA}^{-1}&\vb*{W})$$
    其中 $\vb*{W}=\qty(\vb*{D}-\vb*{CA}^{-1}\vb*{B})^{-1}.$
\end{theorem}
\begin{proof}[{\songti \textbf{证}}]
    为方便叙述, 记 $\vb*{W}=\qty(\vb*{D}-\vb*{CA}^{-1}\vb*{B})^{-1}$ 设 $\vb*{F}=\mqty(\vb*{A}&\vb*{B}\\\vb*{C}&\vb*{D})$ 的逆矩阵为 $\vb*{U}=\mqty(\vb*{U}_{11}&\vb*{U}_{12}\\\vb*{U}_{21}&\vb*{U}_{22})$, 其中 $\vb*{U}$ 的分块与 $\mqty(\vb*{A}&\vb*{B}\\\vb*{C}&\vb*{D})$ 分块后满足乘法原则, 
    则根据可逆矩阵的定义 $\vb*{FU}=\vb*{E}$, 下面直接计算 $\vb*{FU}$ 的分块矩阵的乘积即可, 
    \begin{flalign*}
        \mqty(\vb*{E}_{11} & \vb*{O} \\\vb*{O}&\vb*{E}_{12})=\vb*{E}=\vb*{FU}=\mqty(\vb*{A}&\vb*{B}\\\vb*{C}&\vb*{D})\mqty(\vb*{U}_{11}&\vb*{U}_{12}\\\vb*{U}_{21}&\vb*{U}_{22})=\mqty(\vb*{AU}_{11}+\vb*{BU}_{21}&\vb*{AU}_{12}+\vb*{BU}_{22}\\\vb*{CU}_{11}+\vb*{DU}_{21}&\vb*{CU}_{12}+\vb*{DU}_{22})
    \end{flalign*}
    所以, 各个矩阵小块分别对应相等得:
    \begin{equation}
        \vb*{AU}_{11}+\vb*{BU}_{21}=\vb*{E}_{11}
        \tag*{(1)}
    \end{equation}
    \begin{equation}
        \vb*{AU}_{12}+\vb*{BU}_{22}=\vb*{O}
        \tag*{(2)}
    \end{equation}
    \begin{equation}
        \vb*{CU}_{11}+\vb*{DU}_{21}=\vb*{O}
        \tag*{(3)}
    \end{equation}
    \begin{equation}
        \vb*{CU}_{12}+\vb*{DU}_{22}=\vb*{E}_{22}
        \tag*{(4)}
    \end{equation}
    由 (1) 可得: $\vb*{U}_{11}=\vb*{A}^{-1}(\vb*{E}_{11}-\vb*{BU}_{21})$, 代入 (3) 中, 有
    $$\vb*{CA}^{-1}(\vb*{E}_{11}-\vb*{BU}_{21})+\vb*{DU}_{21}=\vb*{O}$$
    化简后得: $\vb*{CA}^{-1}+\qty(\vb*{D}-\vb*{CA}^{-1}\vb*{B})\vb*{U}_{21}=\vb*{O}$, 即
    $$\vb*{U}_{21}=-\qty(\vb*{D}-\vb*{CA}^{-1}\vb*{B})\vb*{CA}^{-1}=-\vb*{WCA}^{-1}$$
    由 (4) 可得: $\vb*{U}_{12}=\vb*{C}^{-1}(\vb*{E}_{22}-\vb*{DU}_{22})$, 代入 (2) 中, 有
    $$\vb*{A}\qty(\vb*{C}^{-1}-\vb*{C}^{-1}\vb*{DU}_{22})+\vb*{BU}_{22}=\vb*{O}$$
    即
    \begin{flalign*}
        \vb*{U}_{22} & =-\qty(\vb*{B}-\vb*{AC}^{-1}\vb*{D})^{-1}\vb*{AC}^{-1}=-\qty(\vb*{B}-\vb*{AC}^{-1}\vb*{D})^{-1}\qty(\vb*{CA}^{-1})^{-1}=-\qty(\qty(\vb*{CA}^{-1})\qty(\vb*{B}-\vb*{AC}^{-1}\vb*{D}))^{-1} \\
                     & =-\qty(\vb*{CA}^{-1}\vb*{B}-\vb*{CA}^{-1}\vb*{AC}^{-1}\vb*{D})^{-1}=-\qty(\vb*{CA}^{-1}\vb*{B}-\vb*{D})^{-1}=\qty(\vb*{D}-\vb*{CA}^{-1}\vb*{B})=\vb*{W}
    \end{flalign*}
    由 (2) 可得: $\vb*{U}_{12}=-\vb*{A}^{-1}\vb*{BU}_{22}$ 将 $\vb*{U}_{22}=\vb*{W}$ 代入其中可得: $\vb*{U}_{12}=-\vb*{A}^{-1}\vb*{BW}$, 由 (3) 可得:
    $\vb*{U}_{11}=\vb*{A}^{-1}(\vb*{E}_{11}-\vb*{BU}_{21})$, 将 $\vb*{U}_{21}=-\vb*{WCA}^{-1}$ 代入其中可得:
    $$\vb*{U}_{11}=\vb*{A}^{-1}\qty(\vb*{E}_{11}+\vb*{BWCA}^{-1})=\vb*{A}^{-1}+\vb*{A}^{-1}\vb*{BWCA}^{-1}$$
    综上所述, 即得证.
\end{proof}

% \begin{example}
%     设 5 阶矩阵 $\vb*{A}=\begin{pmatrix}
%             1 & 1 & 0 & 0   & 0   \\
%             1 & 3 & 0 & 0   & 0   \\
%             0 & a & a & a^2 & a^3 \\
%             a & 1 & 0 & a   & a^2 \\
%             0 & a & 0 & 0   & a
%         \end{pmatrix}$, 求使得 $\vb*{A}$ 可逆的 $a$ 值, 并计算 $\vb*{A}^{-1}.$
% \end{example}
% \begin{solution}
%     对行列式 $|\vb*{A}|$ 进行初等变换, 令 $|\vb*{A}|\neq0$ 即可, 
%     \begin{flalign*}
%         |\vb*{A}|&=\begin{vmatrix}
%                       1 & 1 & 0 & 0   & 0   \\
%                       1 & 3 & 0 & 0   & 0   \\
%                       0 & a & a & a^2 & a^3 \\
%                       a & 1 & 0 & a   & a^2 \\
%                       0 & a & 0 & 0   & a
%                   \end{vmatrix}\xlongequal[]{c_2-c_1}
%         \begin{vmatrix}
%             1 & 0   & 0 & 0   & 0   \\
%             1 & 2   & 0 & 0   & 0   \\
%             0 & a   & a & a^2 & a^3 \\
%             a & 1-a & 0 & a   & a^2 \\
%             0 & a   & 0 & 0   & a
%         \end{vmatrix}\\
%         &=\left|\begin{array}{c:cccc}
%             1 & 0 & 0 & 0 & 0\\ \hdashline 
%             1 & 2 & 0 & 0 & 0\\
%             0 & a & a & a^2 & a^3\\
%             a & 1-a & 0 & a & a^2\\
%             0 & a & 0 & 0 &a
%            \end{array}\right|=
%            \left|\begin{array}{c:ccc}
%             2 & 0 & 0 & 0\\ \hdashline 
%             a & a & a^2 & a^3\\
%             1-a & 0 & a & a^2\\
%             a & 0 & 0 &a
%            \end{array}\right|=2a^3\neq 0
%     \end{flalign*}
%     则 $a\neq0$ 时, 矩阵 $\vb*{A}$ 可逆, 且 
%     \begin{flalign*}
%         \vb*{A}^{-1}=\dfrac{1}{|\vb*{A}|}\vb*{A}^*=
%     \end{flalign*}
% \end{solution}

\begin{example}
    求矩阵 $\vb*{A}=\mqty(2&3&5\\1&2&7\\3&4&4)$ 的逆矩阵.
\end{example}
\begin{solution}
    \textbf{法一: }因为 $\vb*{A}^{-1}=\dfrac{1}{|\vb*{A}|}\vb*{A}^*$, 于是
    \begin{flalign*}
        \vb*{A}^{-1}=\dfrac{1}{\mqty|2 & 3 & 5 \\1&2&7\\3&4&4|}\mqty(\mqty|2&7\\4&4|&-\mqty|3&5\\4&4|&\mqty|3&5\\2&7|\\
        -\mqty|1                       & 7     \\4&4|&\mqty|2&5\\3&4|&-\mqty|2&5\\1&7|\\ \mqty|1&2\\3&4|&-\mqty|2&3\\3&4|&\mqty|2&3\\1&2|)=\mqty(-20&8&11\\17&-7&-9\\-2&1&1).
    \end{flalign*}
    \textbf{法二: }将 $\begin{pNiceArray}{c:c}
            \vb*{A} & \vb*{E}
        \end{pNiceArray}$ 初等行变换为 $\begin{pNiceArray}{c:c}
            \vb*{E} & \vb*{B}
        \end{pNiceArray}$, 则 $\vb*{B}=\vb*{A}^{-1}$, 于是
    \begin{flalign*}
        \begin{pNiceArray}{c:c}
            \vb*{A} & \vb*{E}
        \end{pNiceArray} & =\begin{pNiceArray}{ccc:ccc}
                                2 & 3 & 5 & 1 & 0 & 0 \\
                                1 & 2 & 7 & 0 & 1 & 0 \\
                                3 & 4 & 4 & 0 & 0 & 1 \\
                            \end{pNiceArray}\xrightarrow[r_3-3r_1]{\substack{r_1\leftrightarrow r_2 \\r_2-2r_1}}\begin{pNiceArray}{ccc:ccc}
            1 & 2  & 7   & 0 & 1  & 0 \\
            0 & -1 & -9  & 1 & -2 & 0 \\
            0 & -2 & -17 & 0 & -3 & 1 \\
        \end{pNiceArray}\\
                                & \xrightarrow[r_3+2r_2]{\substack{r_1+r_3                          \\r_2\times(-1)}}\begin{pNiceArray}{ccc:ccc}
            1 & 0 & -10 & 0  & -2 & 1 \\
            0 & 1 & 9   & -1 & 2  & 0 \\
            0 & 0 & 1   & -2 & 1  & 1 \\
        \end{pNiceArray}\xrightarrow[r_2-9r_3]{r_1+10r_3}    \begin{pNiceArray}{ccc:ccc}
            1 & 0 & 0 & -20 & 8  & 11 \\
            0 & 1 & 0 & 17  & -7 & -9 \\
            0 & 0 & 1 & -2  & 1  & 1  \\
        \end{pNiceArray}
    \end{flalign*}
    于是 $\vb*{A}^{-1}=\mqty(-20  & 8  & 11 \\17  & -7 & -9 \\-2 & 1  & 1 \\).$\\
    \textbf{法三: }设 $\vb*{M}=\mqty(\vb*{A}&\vb*{E}\\-\vb*{E}&\vb*{O})$, 若将 $\vb*{M}$ 的前 $n$ 行施行初等行变换, 将 $\vb*{M}$ 化为分块矩阵 $\displaystyle\mqty(\vb*{B}&\vb*{C}\\\vb*{O}&\vb*{X})$, 那么 $\vb*{X}=\vb*{A}^{-1}$, 则
    \begin{flalign*}
        \mqty(\vb*{A} & \vb*{E}                                  \\-\vb*{E}&\vb*{O})&=\begin{vNiceArray}{ccc:c}
            2 & 3        & 5 &         \\
            1 & 2        & 7 & \vb*{E} \\
            3 & 4        & 4           \\ \hdottedline
            & -\vb*{E} &   & \vb*{O}
        \end{vNiceArray}\xrightarrow[r_3-3r_1]{\substack{r_1\leftrightarrow r_2\\r_2-2r_1}}\begin{pNiceArray}{ccc:ccc}
            1 & 2        & 7   & 0 & 1       & 0 \\
            0 & -1       & -9  & 1 & -2      & 0 \\
            0 & -2       & -17 & 0 & -3      & 1 \\ \hdottedline
            & -\vb*{E} &     &   & \vb*{O}
        \end{pNiceArray}\\
                      & \xrightarrow[r_3+2r_2]{\substack{r_1+r_3 \\r_2\times(-1)}}\begin{pNiceArray}{ccc:ccc}
            1 & 0        & -10 & 0  & -2      & 1 \\
            0 & 1        & 9   & -1 & 2       & 0 \\
            0 & 0        & 1   & -2 & 1       & 1 \\ \hdottedline
            & -\vb*{E} &     &    & \vb*{O}
        \end{pNiceArray}\xrightarrow[r_2-9r_3]{r_1+10r_3}\begin{pNiceArray}{ccc:ccc}
            1 & 0        & 0 & -20 & 8       & 11 \\
            0 & 1        & 0 & 17  & -7      & -9 \\
            0 & 0        & 1 & -2  & 1       & 1  \\ \hdottedline
            & -\vb*{E} &   &     & \vb*{O}
        \end{pNiceArray}
    \end{flalign*}
    故得 $\vb*{A}^{-1}=\mqty(-20  & 8  & 11 \\17  & -7 & -9 \\-2 & 1  & 1 \\).$\\
    \textbf{法四: (该方法仅对三阶矩阵有效) }第一步, 将矩阵前两列复制一份在原矩阵的右侧, 将矩阵前两行复制一份再矩阵的下侧, 再将子矩阵 $\mqty(a_{11}&a_{12}\\a_{21}&a_{22})$ 复制到原矩阵的右下角, 即
    $$\mqty({\color{red} 2}&{\color{red} 3}&{\color{red} 5}\\{\color{orange} 1}&{\color{orange} 2}&{\color{orange} 7}\\3&4&4)\Rightarrow
        \mqty({\color{red} 2}&{\color{red} 3}&{\color{red} 5}&{\color{red} 2}&{\color{red} 3}\\{\color{orange} 1}&{\color{orange} 2}&{\color{orange} 7}& {\color{orange} 1}&{\color{orange} 2}\\3&4&4&3&4)\Rightarrow
        \mqty({\color{red} 2}&{\color{red} 3}&{\color{red} 5}&{\color{red} 2}&{\color{red} 3}\\{\color{orange} 1}&{\color{orange} 2}&{\color{orange} 7}& {\color{orange} 1}&{\color{orange} 2}\\3&4&4&3&4\\{\color{red} 2}&{\color{red} 3}&{\color{red} 5}\\{\color{orange} 1}&{\color{orange} 2}&{\color{orange} 7})\Rightarrow
        \mqty({\color{red} 2}&{\color{red} 3}&{\color{red} 5}&{\color{red} 2}&{\color{red} 3}\\{\color{orange} 1}&{\color{orange} 2}&{\color{orange} 7}& {\color{orange} 1}&{\color{orange} 2}\\3&4&4&3&4\\{\color{red} 2}&{\color{red} 3}&{\color{red} 5}&{\color{red} 2}&{\color{red} 3}\\{\color{orange} 1}&{\color{orange} 2}&{\color{orange} 7}&{\color{orange} 1}&{\color{orange} 2}):=\vb*{B}$$
    第二步, 删去 $\vb*{B}$ 矩阵的第一行、第一列, 得到 $4\times 4$ 大小的 $\vb*{C}$ 矩阵, 即 $\vb*{C}=\mqty(2&7&1&2\\4&4&3&4\\3&5&2&3\\2&7&1&2)$;
    第三步, 计算 $\vb*{C}$ 的二阶子式, 得到矩阵 $\vb*{A}$ 的伴随矩阵 $\vb*{A}^*$, 即
    $$\vb*{A}^*=\mqty(2\times4-7\times4& 4\times5-4\times3& 3\times7-5\times2\\7\times3-1\times4& 4\times2-3\times5& 5\times1-2\times7\\1\times4-2\times3& 3\times3-4\times2& 2\times2-3\times1)=\mqty(-20  & 8  & 11 \\17  & -7 & -9 \\-2 & 1  & 1)$$
    最后一步, 选择 $\vb*{A}^*$ 的第 $i$ 行 (或列) 与 $\vb*{A}$ 的第 $i$ 列 (或行) 相乘, 即为 $|\vb*{A}|$, 此题 $|\vb*{A}|=1$, 故 $\vb*{A}^{-1}=\mqty(-20  & 8  & 11 \\17  & -7 & -9 \\-2 & 1  & 1).$
\end{solution}

\begin{example}
    设矩阵 $\vb*{A}=\mqty(1&3&0&0\\0&4&6&0\\0&0&7&9\\0&0&0&10),~\vb*{B}=(2\vb*{A}+\vb*{E})(\vb*{A}+2\vb*{E})^{-1}$, 则 $|\vb*{B}-2\vb*{E}|$ 中所有元素的代数余子式之和为多少.
\end{example}
\begin{solution}
    由 $\vb*{B}=(2\vb*{A}+\vb*{E})(\vb*{A}+2\vb*{E})^{-1}$, 可知
    \begin{flalign*}
        \qty(\vb*{B}-2\vb*{E})^{-1} & =\qty((2\vb*{A}+\vb*{E})(\vb*{A}+2\vb*{E})^{-1}-2\vb*{E})^{-1}=\qty((2\vb*{A}+\vb*{E})(\vb*{A}+2\vb*{E})^{-1}-2(\vb*{A}+2\vb*{E})(\vb*{A}+2\vb*{E})^{-1})^{-1}             \\
                                    & =\qty((-3\vb*{E})(\vb*{A}+2\vb*{E})^{-1})^{-1}=\qty(-3(\vb*{A}+2\vb*{E})^{-1})^{-1}=-\dfrac{1}{3}(\vb*{A}+2\vb*{E})                                                        \\
                                    & =-\dfrac{1}{3}\mqty(3                                                                                                                                          & 3 & 0 & 0 \\0&6&6&0\\0&0&9&9\\0&0&0&12)=\mqty(-1&-1&0&0\\0&-2&-2&0\\0&0&-3&-3\\0&0&0&-4)
    \end{flalign*}
    因为所有代数余子式之和等于这个伴随矩阵所有元素之和, 故先求出它的伴随矩阵, 再计算伴随矩阵各个元素相加, 
    $|\vb*{B}-2\vb*{E}|=\dfrac{1}{24}$, 于是
    $(\vb*{B}-2\vb*{E})^*=|\vb*{B}-2\vb*{E}|(\vb*{B}-2\vb*{E})^{-1}=\dfrac{1}{24}\mqty(-1&-1&0&0\\0&-2&-2&0\\0&0&-3&-3\\0&0&0&-4)$, 
    因此 $|\vb*{B}-2\vb*{E}|$ 中所有元素的代数余子式之和为 $-\dfrac{2}{3}.$
\end{solution}

\begin{example}
    设矩阵 $\vb*{A}$ 的伴随矩阵 $\vb*{A}^*=\begin{pmatrix}
            1 & 0  & 0 & 0 \\
            0 & 1  & 0 & 0 \\
            1 & 0  & 1 & 0 \\
            0 & -3 & 0 & 8
        \end{pmatrix}$, 且 $\vb*{ABA}^{-1}=\vb*{BA}^{-1}+3\vb*{E}$, 其中 $\vb*{E}$ 为 $4$ 阶单位矩阵, 求矩阵 $\vb*{B}.$
    \label{ABA}
\end{example}
\begin{solution}
    因为 $\vb*{ABA}^{-1}=\vb*{BA}^{-1}+3\vb*{E}\Rightarrow \vb*{B}=\qty(\vb*{A}-\vb*{E})^{-1}\cdot 3\vb*{A}=3\qty[\vb*{A}^{-1}\qty(\vb*{A}-\vb*{E})]^{-1}=3\qty(\vb*{A}^{-1}\vb*{A}-\vb*{A}^{-1})^{-1}=3\qty(\vb*{E}-\dfrac{1}{|\vb*{A}|}\vb*{A}^*)^{-1}$, 
    且 $|\vb*{A}^*|=8$, 又 $|\vb*{A}^*|=|\vb*{A}|^{n-1}$, 故 $|\vb*{A}|=2$, 所以
    \begin{flalign*}
        \vb*{B}=3\qty(\begin{pmatrix}
                          1 &   &   &   \\
                            & 1 &   &   \\
                            &   & 1 &   \\
                            &   &   & 1
                      \end{pmatrix}-\begin{pmatrix}
                                        \dfrac{1}{2} & 0             & 0            & 0 \\
                                        0            & \dfrac{1}{2}  & 0            & 0 \\
                                        \dfrac{1}{2} & 0             & \dfrac{1}{2} & 0 \\
                                        0            & -\dfrac{3}{2} & 0            & 4
                                    \end{pmatrix})^{-1}=3\begin{pmatrix}
                                                             \dfrac{1}{2}  & 0            & 0            & 0  \\
                                                             0             & \dfrac{1}{2} & 0            & 0  \\
                                                             -\dfrac{1}{2} & 0            & \dfrac{1}{2} & 0  \\
                                                             0             & \dfrac{3}{2} & 0            & -3
                                                         \end{pmatrix}^{-1}=\begin{pmatrix}
                                                                                6 & 0 & 0 & 0  \\
                                                                                0 & 6 & 0 & 0  \\
                                                                                6 & 0 & 6 & 0  \\
                                                                                0 & 3 & 0 & -1
                                                                            \end{pmatrix}
    \end{flalign*}
\end{solution}

\begin{example}
    设 $3$ 阶矩阵 $\vb*{A}=\begin{pmatrix}
            1 & 1 & 0 \\
            0 & 1 & 1 \\
            1 & 1 & 2
        \end{pmatrix}$, 求 $\displaystyle \qty|\qty(\dfrac{1}{4}\vb*{A}^2)^{-1}-\vb*{A}^*|$ 及 $\displaystyle \qty[\qty(\dfrac{1}{4}\vb*{A}^2)^{-1}-\vb*{A}^*]^*.$
\end{example}
\begin{solution}
    $\displaystyle|\vb*{A}|=\begin{vmatrix}
            1 & 1 & 0 \\
            0 & 1 & 1 \\
            1 & 1 & 2
        \end{vmatrix}=2+1-1=2$, 又因为
    \begin{flalign*}
        \begin{pNiceArray}{c:c}
            \vb*{A} & \vb*{E}
        \end{pNiceArray}
         & =\begin{pNiceArray}{ccc:ccc}
                1 & 1 & 0 & 1 & 0 & 0 \\
                0 & 1 & 1 & 0 & 1 & 0 \\
                1 & 1 & 2 & 0 & 0 & 1 \\
            \end{pNiceArray}\xrightarrow[r_1-r_2]{r_3-r_1}\begin{pNiceArray}{ccc:ccc}
                                                              1 & 0 & -1 & 1  & -1 & 0 \\
                                                              0 & 1 & 1  & 0  & 1  & 0 \\
                                                              0 & 0 & 2  & -1 & 0  & 1 \\
                                                          \end{pNiceArray}                                             \\
         & \xrightarrow[]{}2\begin{pNiceArray}{ccc:ccc}
                                1 & 0 & -1 & 1             & -1 & 0            \\
                                0 & 1 & 1  & 0             & 1  & 0            \\
                                0 & 0 & 1  & -\dfrac{1}{2} & 0  & \dfrac{1}{2} \\
                            \end{pNiceArray}\xrightarrow[r_2-r_3]{r_1+r_3}2\begin{pNiceArray}{ccc:ccc}
                                                                               1 & 0 & 0 & \dfrac{1}{2}  & -1 & \dfrac{1}{2}  \\[6pt]
                                                                               0 & 1 & 0 & \dfrac{1}{2}  & 1  & -\dfrac{1}{2} \\[6pt]
                                                                               0 & 0 & 1 & -\dfrac{1}{2} & 0  & \dfrac{1}{2}  \\
                                                                           \end{pNiceArray}
    \end{flalign*}
    于是 $\displaystyle \vb*{A}^*=\begin{pmatrix}
            1  & -2 & 1  \\
            1  & 2  & -1 \\
            -1 & 0  & 1
        \end{pmatrix}$, 因为 $\vb*{A}^{-1}=\dfrac{1}{|\vb*{A}|}\vb*{A}^*$, 所以 $2\vb*{A}^{-1}=\vb*{A}^*$, $$\displaystyle\qty(\dfrac{1}{4}\vb*{A}^2)^{-1}=4\qty(\vb*{A}^2)^{-1}=4\dfrac{1}{|\vb*{A}||\vb*{A}|}\qty(\vb*{A}^2)^*=\qty(\vb*{A}^2)^*$$
    所以
    \begin{flalign*}
        \qty|\qty(\dfrac{1}{4}\vb*{A}^2)^{-1}-\vb*{A}^*|=\qty|\qty(\vb*{A}^2)^*-\vb*{A}^*|=\qty|\vb*{A}^*\qty(\vb*{A}^*-\vb*{E}_3)|=\qty|\vb*{A}^*|\qty|\vb*{A}^*-\vb*{E}_3|=-2
    \end{flalign*}
    所以
    \begin{flalign*}
        \qty[\qty(\dfrac{1}{4}\vb*{A}^2)^{-1}-\vb*{A}^*]^* & =\qty[\vb*{A}^*\qty(\vb*{A}^*-\vb*{E}_3)]^*=\qty(\vb*{A}^*-\vb*{E}_3)^*\qty(\vb*{A}^*)^* \\
                                                           & =\begin{pmatrix}
                                                                  0  & -2 & 1  \\
                                                                  1  & 1  & -1 \\
                                                                  -1 & 0  & 0
                                                              \end{pmatrix}^*\begin{pmatrix}
                                                                                 1  & -2 & 1  \\
                                                                                 1  & 2  & -1 \\
                                                                                 -1 & 0  & 1
                                                                             \end{pmatrix}^*=\begin{pmatrix}
                                                                                                 -3 & -4 & 3  \\
                                                                                                 3  & 0  & -1 \\
                                                                                                 -1 & 2  & -1
                                                                                             \end{pmatrix}^*=\begin{pmatrix}
                                                                                                                 2 & 2  & 4  \\
                                                                                                                 4 & 6  & 6  \\
                                                                                                                 6 & 10 & 12
                                                                                                             \end{pmatrix}
    \end{flalign*}
\end{solution}

\subsection{矩阵方程}

\begin{example}
    设矩阵 $\vb*{A}=\mqty(1&1&-1\\-1&1&1\\1&-1&1)$, 其伴随矩阵记作 $\vb*{A}^*$, 求满足
    $$\vb*{A}^*\vb*{X}\qty(\dfrac{1}{2}\vb*{A}^*)^*=8\vb*{A}^{-1}\vb*{X}+4\vb*{E}$$ 的矩阵 $\vb*{X}.$
\end{example}
\begin{solution}
    $\det \vb*{A}=\mqty|1&1&-1\\-1&1&1\\1&-1&1|=4$, 且 $\vb*{A}\vb*{A}^*=|\vb*{A}|\vb*{E}=4\vb*{E}$, 于是给矩阵方程同时左乘 $\vb*{A}$, 得
    \begin{flalign*}
        \vb*{A}\vb*{A}^*\vb*{X}\qty(\dfrac{1}{2}\vb*{A}^*)^* =8\vb*{A}\vb*{A}^{-1}\vb*{X}+4\vb*{A}\vb*{E} \Rightarrow4\vb*{E}\vb*{X}\qty(\dfrac{1}{2}\vb*{A}^*)^*=8\vb*{E}\vb*{X}+4\vb*{A}\vb*{E} \Rightarrow\vb*{X}\qty(\dfrac{1}{2}\vb*{A}^*)^*                 =2\vb*{X}+\vb*{A}
    \end{flalign*}
    并且 $\qty(\dfrac{1}{2}\vb*{A}^*)^*=\qty(\dfrac{1}{2})^{3-1}\qty(\vb*{A}^*)^*=\dfrac{1}{4}\qty|\vb*{A}^*|^*=\dfrac{1}{4}|\vb*{A}|^{3-2}\vb*{A}=\vb*{A}$, 以及 $\det(\vb*{A}-2\vb*{E})= -4\neq0$, 于是 $\vb*{A}-2\vb*{E}$ 可逆,
    而 $(\vb*{A}-2\vb*{E})^{-1}=\mqty(-1&1&-1\\-1&-1&1\\1&-1&-1)^{-1}=-\dfrac{1}{2}\mqty(1&1&0\\0&1&1\\1&0&1)$, 那么
    $$\vb*{X}=-\dfrac{1}{2}\mqty(1&1&-1\\-1&1&1\\1&-1&1)\mqty(1&1&0\\0&1&1\\1&0&1)=\mqty(0&-1&0\\0&0&-1\\-1&0&0).$$
\end{solution}

\begin{example}
    $\vb*{A},~\vb*{B}$ 是数域 $K$ 上的 $n$ 阶已知方阵, $\det \vb*{A}=\dfrac{1}{2},~\det\vb*{B}=\dfrac{1}{3}$, 求解关于 $\vb*{X}$ 的矩阵方程
    $$\vb*{X}+\qty(\qty(\vb*{A}^\top\vb*{B})^*\vb*{A}^\top)^{-1}+\vb*{BA}^{-1}\vb*{B}=\vb*{X}\qty(\vb*{B}^\top\qty(\vb*{AB}^\top)^{-1}\vb*{A}^2)^{-1}(\vb*{A}+2\vb*{B}).$$
\end{example}
\begin{solution}
    注意到 $\qty(\vb*{A}^\top\vb*{B})^*=\qty|\vb*{A}^\top\vb*{B}|\qty(\vb*{A}^\top\vb*{B})^{-1}=\qty|\vb*{A}^\top\vb*{B}|\vb*{B}^{-1}\qty(\vb*{A}^\top)^{-1}$,
    那么 $$\qty(\qty(\vb*{A}^\top\vb*{B})^*\vb*{A}^\top)^{-1}=\qty(\qty|\vb*{A}^\top\vb*{B}|\vb*{B}^{-1}\vb*{E})^{-1}=\dfrac{1}{\qty|\vb*{A}^\top\vb*{B}|}\vb*{B}=6\vb*{B}$$
    并且 $\qty(\vb*{B}^\top\qty(\vb*{AB}^\top)^{-1}\vb*{A}^2)^{-1}=\qty(\vb*{B}^\top\qty(\vb*{B}^\top)^{-1}\vb*{A}^{-1}\vb*{A}^2)^{-1}=\vb*{A}^{-1}$, 于是
    $$\vb*{X}\qty(\vb*{B}^\top\qty(\vb*{AB}^\top)^{-1}\vb*{A}^2)^{-1}(\vb*{A}+2\vb*{B})=\vb*{X}\vb*{A}^{-1}\qty(\vb*{A}+2\vb*{B})=\vb*{X}+2\vb*{XA}^{-1}\vb*{B}$$
    那么 \begin{flalign*}
        \vb*{X}=\dfrac{1}{2}\qty(6\vb*{B}+\vb*{BA}^{-1}\vb*{B})\qty(\vb*{A}^{-1}\vb*{B})^{-1}=\dfrac{1}{2}\qty(6\vb*{B}+\vb*{BA}^{-1}\vb*{B})\qty(\vb*{B}^{-1}\vb*{A})=3\vb*{A}+\dfrac{1}{2}\vb*{B}
    \end{flalign*}
\end{solution}

\begin{example}[2016 浙江大学]
    设矩阵 $\vb*{A}=\begin{pmatrix}
            a & b & c \\
            d & e & f \\
            h & x & y
        \end{pmatrix}$ 的逆矩阵 $\vb*{A}^{-1}=\begin{pmatrix}
            -1 & -2 & -1 \\
            -2 & 1  & 0  \\
            0  & -3 & 1
        \end{pmatrix}$, 矩阵 $\vb*{B}=\begin{pmatrix}
            a-2b & b-3  & -c \\
            d-2e & e-3f & -f \\
            h-2x & x-3y & -y
        \end{pmatrix}$, 求矩阵 $\vb*{X}$ 使之满足
    $$\vb*{X}+\qty(\vb*{B}\qty(\vb*{A}^{\top}\vb*{B}^2)^{-1}\vb*{A}^{\top})^{-1}=\vb*{X}\qty(\vb*{A}^2\qty(\vb*{B}^{\top}\vb*{A})^{-1}\vb*{B}^{\top})^{-1}(\vb*{A}+\vb*{B}).$$
\end{example}
\begin{solution}
    先证 $\vb*{B}$ 是可逆矩阵, 为此, 对行列式 $|\vb*{B}|$ 依次将第 2 列减去第 3 列的 3 倍, 第 1 列加上第 2 列的 2 倍, 再按第 1 行拆项, 得
    \begin{flalign*}
        |\vb*{B}| & =\begin{vmatrix}
                         a-2b & b-3  & -c \\
                         d-2e & e-3f & -f \\
                         h-2x & x-3y & -y
                     \end{vmatrix}\xlongequal[c_1+2c_2]{c_2-3c_3}\begin{vmatrix}
                                                                     a+6c-6 & b+3c-3 & -c \\
                                                                     d      & e      & -f \\
                                                                     h      & x      & -y
                                                                 \end{vmatrix}=-|\vb*{A}|+3(c-1)\begin{vmatrix}
                                                                                                    2 & 1 & 0  \\
                                                                                                    d & e & -f \\
                                                                                                    h & x & -y
                                                                                                \end{vmatrix} \\
                  & =1-3(c-1)(2A_{11}-A_{12})
    \end{flalign*}
    其中 $A_{11},~A_{12}$ 是矩阵 $\vb*{A}$ 的第 1 行的前两个元素的代数余子式, 另一方面, 易知 $|\vb*{A}^{-1}|=-1$, 所以 $|\vb*{A}|=-1$, 于是, 有
    $\vb*{A}^*=|\vb*{A}|\vb*{A}^{-1}=\mqty(1&2&1\\2&-1&0\\0&3&1)$, 由此可知 $A_{11}=1,~A_{12}=2$, 于是 $|\vb*{B}|=1$, 表明 $\vb*{B}$ 是可逆矩阵, 因此
    $$\qty(\vb*{B}\qty(\vb*{A}^{\top}\vb*{B}^2)^{-1}\vb*{A}^{\top})^{-1}=\vb*{B},~\vb*{X}\qty(\vb*{A}^2\qty(\vb*{B}^{\top}\vb*{A})^{-1}\vb*{B}^{\top})^{-1}=\vb*{A}^{-1}$$
    故原方程可化简为 $\vb*{X}+\vb*{B}=\vb*{XA}^{-1}(\vb*{A}+\vb*{B})\Rightarrow \vb*{X}(\vb*{A}^{-1}\vb*{B})=\vb*{B}$, 所以
    $$\vb*{X}=\vb*{B}(\vb*{A}^{-1}\vb*{B})^{-1}=\vb*{A}.$$
\end{solution}

\begin{example}
    设矩阵 $\vb*{A}=\begin{pmatrix}
            1 & -1 & 0  & 0  \\
            0 & 1  & -1 & 0  \\
            0 & 0  & 1  & -1 \\
            0 & 0  & 0  & 1
        \end{pmatrix}$, $\vb*{B}=\begin{pmatrix}
            2 & 1 & 3 & 4 \\
            0 & 2 & 1 & 3 \\
            0 & 0 & 2 & 1 \\
            0 & 0 & 0 & 2
        \end{pmatrix}$, 求解方程
    $$\vb*{X}\qty(\vb*{E}_n-\vb*{B}^{-1}\vb*{A})^{\top}\vb*{B}^{\top}=\vb*{E}_n.$$
\end{example}
\begin{solution}
    $\vb*{E}_n=\vb*{X}\qty(\vb*{B}\qty(\vb*{E}_n-\vb*{B}^{-1}\vb*{A}))^{\top}=\vb*{X}\qty(\vb*{BE}_n-\vb*{BB}^{-1}\vb*{A})^{\top}=\vb*{X}\qty(\vb*{B}-\vb*{A})^{\top}\Rightarrow \vb*{X}=\qty(\qty(\vb*{B}-\vb*{A})^{\top})^{-1}$,
    \begin{flalign*}
        \vb*{X} & =\qty(\qty(\begin{pmatrix}
                                 2 & 1 & 3 & 4 \\
                                 0 & 2 & 1 & 3 \\
                                 0 & 0 & 2 & 1 \\
                                 0 & 0 & 0 & 2
                             \end{pmatrix}-\begin{pmatrix}
                                               1 & -1 & 0  & 0  \\
                                               0 & 1  & -1 & 0  \\
                                               0 & 0  & 1  & -1 \\
                                               0 & 0  & 0  & 1
                                           \end{pmatrix})^{\top})^{-1}
        =\begin{pNiceArray}{cc:cc}
             1 & 0 & 0 & 0 \\
             2 & 1 & 0 & 0 \\ \hdottedline
             3 & 2 & 1 & 0 \\
             4 & 3 & 2 & 1
         \end{pNiceArray}^{-1}                                                                                                         \\
                & =\begin{pNiceArray}{c:c}
                       \begin{pmatrix}
                1 & 0 \\
                2 & 1
            \end{pmatrix}^{-1}                & O                                   \\ \hdottedline
                       -\begin{pmatrix}
                1 & 0 \\
                2 & 1
            \end{pmatrix}^{-1}\begin{pmatrix}
                3 & 2 \\
                4 & 3
            \end{pmatrix}\begin{pmatrix}
                1 & 0 \\
                2 & 1
            \end{pmatrix}^{-1} & \begin{pmatrix}
                1 & 0 \\
                2 & 1
            \end{pmatrix}^{-1}
                   \end{pNiceArray}=\begin{pmatrix}
                                        1  & 0  & 0  & 0 \\
                                        -2 & 1  & 0  & 0 \\
                                        1  & -2 & 1  & 0 \\
                                        0  & 1  & -2 & 1
                                    \end{pmatrix}.
    \end{flalign*}
\end{solution}

\begin{example}
    设 $\vb*{A}=\mqty(1& 2& 2\\ 2& 5& 4\\ 2& 4& 5),~\vb*{B}=\mqty(1&1\\1&0),~\vb*{C}=\begin{pmatrix}
            1 & 1 &   &   &   \\
              & 1 & 1 &   &   \\
              &   & 1 & 1 &   \\
              &   &   & 1 & 1 \\
              &   &   &   & 1
        \end{pmatrix}$, 求 $\vb*{X}$ 使 $\vb*{X}\mqty(\vb*{O}&\vb*{B}\\\vb*{A}&\vb*{O})=\vb*{C}.$
\end{example}
\begin{solution}
    易得 $\vb*{A},~\vb*{B}$ 都可逆, 且 $\mqty(\vb*{O}&\vb*{B}\\\vb*{A}&\vb*{O})^{-1}=\mqty(\vb*{O}&\vb*{A}^{-1}\\\vb*{B}^{-1}&\vb*{O})$, 所以 $\vb*{X}=\vb*{C}\mqty(\vb*{O}&\vb*{A}^{-1}\\\vb*{B}^{-1}&\vb*{O})$, 且
    $$\vb*{C}=\begin{pNiceArray}{ccc:cc}
            1 & 1 &   &   &   \\
            & 1 & 1 &   &   \\
            &   & 1 & 1 &   \\ \hdottedline
            &   &   & 1 & 1 \\
            &   &   &   & 1
        \end{pNiceArray}=\begin{pNiceArray}{ccc:cc}
            1 & 1 & 0 & 0 & 0 \\
            0 & 1 & 1 & 0 & 0 \\
            0 & 0 & 1 & 1 & 0 \\ \hdottedline
            0 & 0 & 0 & 1 & 1 \\
            0 & 0 & 0 & 0 & 1
        \end{pNiceArray}=\mqty(\vb*{C}_1&\vb*{C}_2\\\vb*{O}&\vb*{C}_3)$$
    并且 $\vb*{A}^{-1}=\mqty(9& -2& -2 \\-2& 1& 0 \\-2& 0& 1),~\vb*{B}^{-1}=\mqty(1& -1\\0& 1)$, 那么
    $$\vb*{X}=\mqty(\vb*{C}_2\vb*{B}^{-1}& \vb*{C}_1\vb*{A}^{-1}\\\vb*{C}_3\vb*{B}^{-1}&\vb*{O})=\begin{pNiceArray}{cc:ccc}
            0 & 0  & 7  & -1 & -2 \\
            0 & 0  & -4 & 1  & 1  \\
            1 & -1 & -2 & 0  & 1  \\ \hdottedline
            1 & 0  & 0  & 0  & 0  \\
            0 & 1  & 0  & 0  & 0
        \end{pNiceArray}.$$
\end{solution}

\begin{example}
    设 4 阶矩阵 $\vb*{A}$ 的伴随矩阵 $\vb*{A}^*=\begin{pmatrix}
            1 & 0  & 0 & 0 \\
            0 & 1  & 0 & 0 \\
            1 & 0  & 1 & 0 \\
            0 & -3 & 0 & 8
        \end{pmatrix}$, 求解矩阵方程 $\vb*{AXA}^{-1}=\vb*{XA}^{-1}+3\vb*{E}_4$, 并计算 $\vb*{X}^*\vb*{A}.$
\end{example}
\begin{solution}
    同例题 \ref{ABA} 的解法, 解得 $\vb*{X}=\begin{pmatrix}
            6 & 0 & 0 & 0  \\
            0 & 6 & 0 & 0  \\
            6 & 0 & 6 & 0  \\
            0 & 3 & 0 & -1
        \end{pmatrix}$, 由于 $$\vb*{X}^*\vb*{A}=\vb*{X}^*\dfrac{1}{|\vb*{A}|^{4-2}}\qty(\vb*{A}^*)^*=\dfrac{1}{4}\vb*{X}^*\qty(\vb*{A}^*)^*=\dfrac{1}{4}\qty(\vb*{A}^*\vb*{X})^*$$
    \begin{flalign*}
        \vb*{X}^*\vb*{A} & =\dfrac{1}{4}\qty(\begin{pmatrix}
                                                     1 & 0  & 0 & 0 \\
                                                     0 & 1  & 0 & 0 \\
                                                     1 & 0  & 1 & 0 \\
                                                     0 & -3 & 0 & 8
                                                 \end{pmatrix}\begin{pmatrix}
                                                                  6 & 0 & 0 & 0  \\
                                                                  0 & 6 & 0 & 0  \\
                                                                  6 & 0 & 6 & 0  \\
                                                                  0 & 3 & 0 & -1
                                                              \end{pmatrix})^*=\dfrac{1}{4}\begin{pNiceArray}{cc:cc}
                                                                                           6  & 0 & 0 & 0  \\
                                                                                           0  & 6 & 0 & 0  \\ \hdottedline
                                                                                           12 & 0 & 6 & 0  \\
                                                                                           0  & 6 & 0 & -8 \\
                                                                                       \end{pNiceArray}^*            \\
                         & =\begin{pNiceArray}{c:c}
                                \begin{vmatrix}
                6 & 0  \\
                0 & -8
            \end{vmatrix}\begin{pmatrix}
                6 & 0 \\
                0 & 6
            \end{pmatrix}^* & O                            \\ \hdottedline
                                -\begin{pmatrix}
                6 & 0  \\
                0 & -8
            \end{pmatrix}^*\begin{pmatrix}
                12 & 0 \\
                0  & 6
            \end{pmatrix}\begin{pmatrix}
                6 & 0 \\
                0 & 6
            \end{pmatrix}^*
                                & \begin{vmatrix}
                6 & 0 \\
                0 & 6
            \end{vmatrix}\begin{pmatrix}
                6 & 0  \\
                0 & -8
            \end{pmatrix}^*
                            \end{pNiceArray}=\begin{pmatrix}
                                                 -72 & 0   & 0   & 0  \\
                                                 0   & -72 & 0   & 0  \\
                                                 144 & 0   & -72 & 0  \\
                                                 0   & -54 & 0   & 54
                                             \end{pmatrix}.
    \end{flalign*}
\end{solution}

\begin{example}
    设 $4$ 阶矩阵 $\vb*{A}=\begin{pmatrix}
            1 & 2 & 0  & 0 \\
            1 & 3 & 0  & 0 \\
            0 & 0 & 0  & 2 \\
            0 & 0 & -1 & 0
        \end{pmatrix}$, 求解 $\qty[\qty(\dfrac{1}{2}\vb*{A})^*]^{-1}\vb*{XA}^{-1}=2\vb*{AX}+12\vb*{E}_4.$
\end{example}
\begin{solution}
    因为 $|\vb*{A}|=\begin{vmatrix}
            1 & 2 \\
            1 & 3
        \end{vmatrix}
        \begin{vmatrix}
            0  & 2 \\
            -1 & 0
        \end{vmatrix}=2$, $\qty[\qty(\dfrac{1}{2}\vb*{A})^*]^{-1}=\qty[\qty|\dfrac{1}{2}\vb*{A}|\qty(\dfrac{1}{2}\vb*{A})^{-1}]^{-1}=8\cdot \dfrac{1}{2}\vb*{A}=4\vb*{A}$,
    因此原方程变成 $$4\vb*{AXA}^{-1}=2\vb*{AX}+12\vb*{E}_4\Rightarrow 2\vb*{AXA}^{-1}=\vb*{AX}+6\vb*{E}_4\Rightarrow \vb*{X}=6\vb*{A}^{-1}\qty(2\vb*{A}^{-1}-\vb*{E}_4)^{-1}=6\qty(2\vb*{E}_4-\vb*{A})^{-1}$$
    \begin{flalign*}
        \vb*{X} & =6\qty(\begin{pmatrix}
                             2 &   &   &   \\
                               & 2 &   &   \\
                               &   & 2 &   \\
                               &   &   & 2
                         \end{pmatrix}-\begin{pmatrix}
                                           1 & 2 & 0  & 0 \\
                                           1 & 3 & 0  & 0 \\
                                           0 & 0 & 0  & 2 \\
                                           0 & 0 & -1 & 0
                                       \end{pmatrix})^{-1}=6\begin{pNiceArray}{cc:cc}
                                                                1  & -2 & 0 & 0  \\
                                                                -1 & -1 & 0 & 0  \\ \hdottedline
                                                                0  & 0  & 2 & -2 \\
                                                                0  & 0  & 1 & 2
                                                            \end{pNiceArray}^{-1}                                    \\
                & =6\begin{pNiceArray}{c:c}
                        \begin{pmatrix}
                1  & -2 \\
                -1 & -1
            \end{pmatrix}^{-1} & O                  \\ \hdottedline
                        O                  & \begin{pmatrix}
                2 & -2 \\
                1 & 2
            \end{pmatrix}^{-1}
                    \end{pNiceArray}=6\begin{pNiceArray}{c:c}
                                          -\dfrac{1}{3} \begin{pmatrix}
                -1 & 2 \\
                1  & 1
            \end{pmatrix} & O                             \\ \hdottedline
                                          O                             & \dfrac{1}{6} \begin{pmatrix}
                2  & 2 \\
                -1 & 2
            \end{pmatrix}
                                      \end{pNiceArray} \\
                & =\begin{pNiceArray}{c:c}
                       -2 \begin{pmatrix}
                -1 & 2 \\
                1  & 1
            \end{pmatrix} & O                \\ \hdottedline
                       O                  & \begin{pmatrix}
                2  & 2 \\
                -1 & 2
            \end{pmatrix}
                   \end{pNiceArray}=\begin{pmatrix}
                                        2  & -4 & 0  & 0 \\
                                        -2 & -2 & 0  & 0 \\
                                        0  & 0  & 2  & 2 \\
                                        0  & 0  & -1 & 2
                                    \end{pmatrix}.
    \end{flalign*}
\end{solution}