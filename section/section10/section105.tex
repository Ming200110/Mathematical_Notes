\section{矩阵的等价标准形}

% \subsection{等价标准形}
% 
% \begin{example}
%     设 $\vb*{A}=\mqty(1&2&3\\2&1&2\\3&3&5\\1&-1&-1\\4&2&4)$,求可逆矩阵 $\vb*{P},~\vb*{Q}$ 使 $\vb*{PAQ}$ 为 $\vb*{A}$ 的等价标准形.
% \end{example}
% \begin{solution}
%     由矩阵的初等变换,
%     \begin{flalign*}
%         \mqty(\vb*{A} & \vb*{E}_5 \\\vb*{E}_3&\vb*{O})=\left(\begin{array}{c:c}
%                 \begin{matrix}
%                     1 & 2  & 3  \\
%                     2 & 1  & 2  \\
%                     3 & 3  & 5  \\
%                     1 & -1 & -1 \\
%                     4 & 2  & 4
%                 \end{matrix} & \vb*{E}_5 \\ \hdashline
%                 \vb*{E}_3      & \vb*{O}
%             \end{array}\right)=
%     \end{flalign*}
% \end{solution}

\subsection{矩阵分解}

\begin{example}
    设 $\vb*{A}$ 是 $n$ 阶方阵,证明: 存在一 $n$ 阶可逆矩阵 $\vb*{B}$ 及一个 $n$ 阶等幂矩阵 $\vb*{C}$,使得 $\vb*{A}=\vb*{BC}.$
\end{example}
\begin{proof}[{\songti \textbf{证}}]
    设 $\rank\vb*{A}=r$,则存在 $n$ 阶可逆矩阵 $\vb*{P}$ 和 $\vb*{Q}$ 使得,
    $$\vb*{A}=\vb*{P}\mqty(\vb*{E}_r&\vb*{O}\\\vb*{O}&\vb*{O})\vb*{Q}=(\vb*{PQ})\vb*{Q}^{-1}\mqty(\vb*{E}_r&\vb*{O}\\\vb*{O}&\vb*{O})\vb*{Q}$$
    令 $\vb*{B}=\vb*{PQ},~\vb*{C}=\vb*{Q}^{-1}\mqty(\vb*{E}_r&\vb*{O}\\\vb*{O}&\vb*{O})\vb*{Q}$,并且
    $$\vb*{C}^2=\vb*{Q}^{-1}\mqty(\vb*{E}_r&\vb*{O}\\\vb*{O}&\vb*{O})\vb*{Q}\vb*{Q}^{-1}\mqty(\vb*{E}_r&\vb*{O}\\\vb*{O}&\vb*{O})\vb*{Q}=C$$
    故得证.
\end{proof}

\begin{example}
    一个矩阵称为行 (列) 满秩矩阵,如果它的行 (列) 向量组是线性无关的,
    \begin{enumerate}[label=(\arabic{*})]
        \item 证明: 如果一个 $m\times n$ 矩阵 $\vb*{A}$ 的秩为 $r$,那么有 $m\times r$ 的列满秩矩阵 $\vb*{B}$ 和 $r\times n$ 的行满秩矩阵 $\vb*{C}$,使得 $\vb*{A}=\vb*{BC}$,我们称 $\vb*{A}=\vb*{BC}$ 为矩阵 $\vb*{A}$ 的满秩分解表达式;
        \item 利用 (1) 求矩阵 $\vb*{A}=\mqty(1&2&0&1&1&10\\3&6&1&4&2&36\\2&4&0&2&2&27\\6&12&1&7&5&73)$ 的满秩分解表达式.
    \end{enumerate}
\end{example}
\begin{solution}
    \begin{enumerate}[label=(\arabic{*})]
        \item 因为 $\rank \vb*{A}=r$,所以存在 $m$ 阶可逆矩阵 $\vb*{P}$ 和 $n$ 阶可逆矩阵 $\vb*{Q}$,使得
              $$\vb*{A}=\vb*{P}\mqty(\vb*{E}_r&\vb*{O}\\\vb*{O}&\vb*{O})\vb*{Q}=\vb*{P}\mqty(\vb*{E}_r\\\vb*{O})(\vb*{E}_r,\vb*{O})\vb*{Q}$$
              令 $\vb*{B}=\vb*{P}\mqty(\vb*{E}_r\\\vb*{O}),~\vb*{C}=(\vb*{E}_r,\vb*{O})\vb*{Q}$,则 $\vb*{B}$ 是 $m\times r$ 的列满秩矩阵,$\vb*{C}$ 是 $r\times n$ 的行满秩矩阵,且 $\vb*{A}=\vb*{BC}.$
        \item 用初等行变换和初等列变换将矩阵 $\vb*{A}$ 化为等价标准形,然后求逆即可,具体地,有
              \begin{flalign*}
                  \vb*{A} & \xrightarrow[\substack{r_3-2r_1 \\r_4-6r_1}]{r_2-3r_1}\mqty(1&2&0&1&1&10\\0&0&1&1&-1&6\\0&0&0&0&0&7\\0&0&1&1&-1&13)\xrightarrow[\substack{r_4-r_3\\r_3\times\frac{1}{7}}]{r_4-r_2}\mqty(1&2&0&1&1&10\\0&0&1&1&-1&6\\0&0&0&0&0&1\\0&0&0&0&0&0)\\
                          & \xrightarrow[\substack{c_5-c_1  \\c_6-10c_1}]{\substack{c_2-2c_1\\c_4-c_1}}\mqty(1&0&0&0&0&0\\0&0&1&1&-1&6\\0&0&0&0&0&1\\0&0&0&0&0&0)\xrightarrow[\substack{c_5+c_3\\c_6-6c_3}]{c_4-c_3}\mqty(1&0&0&0&0&0\\0&0&1&0&0&0\\0&0&0&0&0&1\\0&0&0&0&0&0)\xrightarrow[c_3\leftrightarrow c_6]{c_2\leftrightarrow c_3}\mqty(\vb*{E}_3&\vb*{O}\\\vb*{O}&\vb*{O})_{4\times6}
              \end{flalign*}
              用矩阵表示为
              $$\vb*{L}_4\vb*{L}_3\vb*{L}_2\vb*{L}_1\vb*{A}\vb*{R}_1\vb*{R}_2\vb*{E}_{23}\vb*{E}_{36}=\mqty(\vb*{E}_3&\vb*{O}\\\vb*{O}&\vb*{O})_{4\times6}$$
              其中 $$\vb*{L}_1=\mqty(1\\-3&1\\-2&&1\\-6&&&1),\vb*{L}_2=\mqty(1\\&1\\&&1\\&-1&&1),\vb*{L}_3=\mqty(1\\&1\\&&1\\&&-1&1),\vb*{L}_4=\mqty(1\\&1\\&&\dfrac{1}{7}\\&&&1)$$
              $$\vb*{R}_1=\mqty(1&-2&0&-1&-1&-10\\&1\\&&1\\&&&1\\&&&&1\\&&&&&1),~\vb*{R}_2=\mqty(1\\&1\\&&1&-1&1&-6\\&&&1\\&&&&1\\&&&&&1)$$
              % $$\vb*{E}_{23}=\mqty(\dmat{1\\&&1\\&1,1,1,1}) ,~\vb*{E}_{36}=\mqty(1\\&1\\&&&&&1\\&&&1\\&&&&1\\&&1)$$
              于是,有 $$\vb*{A}=\qty(\vb*{L}_1^{-1}\vb*{L}_2^{-1}\vb*{L}_3^{-1}\vb*{L}_4^{-1}\mqty(\vb*{E}_3\\\vb*{O}))\qty(\mqty(\vb*{E}_3,\vb*{O})\vb*{E}_{36}\vb*{E}_{23}\vb*{R}_2^{-1}\vb*{R}_1^{-1})=\vb*{BC}$$
              那么 $$\vb*{B}=\vb*{L}_1^{-1}\vb*{L}_2^{-1}\vb*{L}_3^{-1}\vb*{L}_4^{-1}\mqty(\vb*{E}_3\\\vb*{O})=\mqty(1&0&0\\3&1&0\\2&0&7\\6&1&7)$$
              $$\vb*{C}=\mqty(\vb*{E}_3,\vb*{O})\vb*{E}_{36}\vb*{E}_{23}\vb*{R}_2^{-1}\vb*{R}_1^{-1}=\mqty(1&2&0&1&1&10\\0&0&1&1&-1&6\\0&0&0&0&0&1).$$
    \end{enumerate}
\end{solution}

\subsection{矩阵方程}

\begin{example}
    设矩阵 $\vb*{A}=\mqty(1&1&-1\\-1&1&1\\1&-1&1)$,其伴随矩阵记作 $\vb*{A}^*$,求满足
    $$\vb*{A}^*\vb*{X}\qty(\dfrac{1}{2}\vb*{A}^*)^*=8\vb*{A}^{-1}\vb*{X}+4\vb*{E}$$ 的矩阵 $\vb*{X}.$
\end{example}
\begin{solution}
    $\det \vb*{A}=\mqty|1&1&-1\\-1&1&1\\1&-1&1|=4$,且 $\vb*{A}\vb*{A}^*=|\vb*{A}|\vb*{E}=4\vb*{E}$,于是给矩阵方程同时左乘 $\vb*{A}$,得
    \begin{flalign*}
        \vb*{A}\vb*{A}^*\vb*{X}\qty(\dfrac{1}{2}\vb*{A}^*)^* =8\vb*{A}\vb*{A}^{-1}\vb*{X}+4\vb*{A}\vb*{E} \Rightarrow4\vb*{E}\vb*{X}\qty(\dfrac{1}{2}\vb*{A}^*)^*=8\vb*{E}\vb*{X}+4\vb*{A}\vb*{E} \Rightarrow\vb*{X}\qty(\dfrac{1}{2}\vb*{A}^*)^*                 =2\vb*{X}+\vb*{A}
    \end{flalign*}
    并且 $\qty(\dfrac{1}{2}\vb*{A}^*)^*=\qty(\dfrac{1}{2})^{3-1}\qty(\vb*{A}^*)^*=\dfrac{1}{4}\qty|\vb*{A}^*|^*=\dfrac{1}{4}|\vb*{A}|^{3-2}\vb*{A}=\vb*{A}$,以及 $\det(\vb*{A}-2\vb*{E})= -4\neq0$,于是 $\vb*{A}-2\vb*{E}$ 可逆,
    而 $(\vb*{A}-2\vb*{E})^{-1}=\mqty(-1&1&-1\\-1&-1&1\\1&-1&-1)^{-1}=-\dfrac{1}{2}\mqty(1&1&0\\0&1&1\\1&0&1)$,那么
    $$\vb*{X}=-\dfrac{1}{2}\mqty(1&1&-1\\-1&1&1\\1&-1&1)\mqty(1&1&0\\0&1&1\\1&0&1)=\mqty(0&-1&0\\0&0&-1\\-1&0&0).$$
\end{solution}

\begin{example}
    $\vb*{A},~\vb*{B}$ 是数域 $K$ 上的 $n$ 阶已知方阵,$\det \vb*{A}=\dfrac{1}{2},~\det\vb*{B}=\dfrac{1}{3}$,求解关于 $\vb*{X}$ 的矩阵方程
    $$\vb*{X}+\qty(\qty(\vb*{A}^\top\vb*{B})^*\vb*{A}^\top)^{-1}+\vb*{BA}^{-1}\vb*{B}=\vb*{X}\qty(\vb*{B}^\top\qty(\vb*{AB}^\top)^{-1}\vb*{A}^2)^{-1}(\vb*{A}+2\vb*{B}).$$
\end{example}
\begin{solution}
    注意到 $\qty(\vb*{A}^\top\vb*{B})^*=\qty|\vb*{A}^\top\vb*{B}|\qty(\vb*{A}^\top\vb*{B})^{-1}=\qty|\vb*{A}^\top\vb*{B}|\vb*{B}^{-1}\qty(\vb*{A}^\top)^{-1}$,
    那么 $$\qty(\qty(\vb*{A}^\top\vb*{B})^*\vb*{A}^\top)^{-1}=\qty(\qty|\vb*{A}^\top\vb*{B}|\vb*{B}^{-1}\vb*{E})^{-1}=\dfrac{1}{\qty|\vb*{A}^\top\vb*{B}|}\vb*{B}=6\vb*{B}$$
    并且 $\qty(\vb*{B}^\top\qty(\vb*{AB}^\top)^{-1}\vb*{A}^2)^{-1}=\qty(\vb*{B}^\top\qty(\vb*{B}^\top)^{-1}\vb*{A}^{-1}\vb*{A}^2)^{-1}=\vb*{A}^{-1}$,于是
    $$\vb*{X}\qty(\vb*{B}^\top\qty(\vb*{AB}^\top)^{-1}\vb*{A}^2)^{-1}(\vb*{A}+2\vb*{B})=\vb*{X}\vb*{A}^{-1}\qty(\vb*{A}+2\vb*{B})=\vb*{X}+2\vb*{XA}^{-1}\vb*{B}$$
    那么 \begin{flalign*}
        \vb*{X}=\dfrac{1}{2}\qty(6\vb*{B}+\vb*{BA}^{-1}\vb*{B})\qty(\vb*{A}^{-1}\vb*{B})^{-1}=\dfrac{1}{2}\qty(6\vb*{B}+\vb*{BA}^{-1}\vb*{B})\qty(\vb*{B}^{-1}\vb*{A})=3\vb*{A}+\dfrac{1}{2}\vb*{B}
    \end{flalign*}
\end{solution}

\begin{example}[2016 浙江大学]
    设矩阵 $\vb*{A}=\begin{pmatrix}
            a & b & c \\
            d & e & f \\
            h & x & y
        \end{pmatrix}$ 的逆矩阵 $\vb*{A}^{-1}=\begin{pmatrix}
            -1 & -2 & -1 \\
            -2 & 1  & 0  \\
            0  & -3 & 1
        \end{pmatrix}$,矩阵 $\vb*{B}=\begin{pmatrix}
            a-2b & b-3  & -c \\
            d-2e & e-3f & -f \\
            h-2x & x-3y & -y
        \end{pmatrix}$,求矩阵 $\vb*{X}$ 使之满足
    $$\vb*{X}+\qty(\vb*{B}\qty(\vb*{A}^{\top}\vb*{B}^2)^{-1}\vb*{A}^{\top})^{-1}=\vb*{X}\qty(\vb*{A}^2\qty(\vb*{B}^{\top}\vb*{A})^{-1}\vb*{B}^{\top})^{-1}(\vb*{A}+\vb*{B}).$$
\end{example}
\begin{solution}
    先证 $\vb*{B}$ 是可逆矩阵,为此,对行列式 $|\vb*{B}|$ 依次将第 2 列减去第 3 列的 3 倍,第 1 列加上第 2 列的 2 倍,再按第 1 行拆项,得
    \begin{flalign*}
        |\vb*{B}| & =\begin{vmatrix}
                         a-2b & b-3  & -c \\
                         d-2e & e-3f & -f \\
                         h-2x & x-3y & -y
                     \end{vmatrix}\xlongequal[c_1+2c_2]{c_2-3c_3}\begin{vmatrix}
                                                                     a+6c-6 & b+3c-3 & -c \\
                                                                     d      & e      & -f \\
                                                                     h      & x      & -y
                                                                 \end{vmatrix}=-|\vb*{A}|+3(c-1)\begin{vmatrix}
                                                                                                    2 & 1 & 0  \\
                                                                                                    d & e & -f \\
                                                                                                    h & x & -y
                                                                                                \end{vmatrix} \\
                  & =1-3(c-1)(2A_{11}-A_{12})
    \end{flalign*}
    其中 $A_{11},~A_{12}$ 是矩阵 $\vb*{A}$ 的第 1 行的前两个元素的代数余子式,另一方面,易知 $|\vb*{A}^{-1}|=-1$,所以 $|\vb*{A}|=-1$,于是,有
    $\vb*{A}^*=|\vb*{A}|\vb*{A}^{-1}=\mqty(1&2&1\\2&-1&0\\0&3&1)$,由此可知 $A_{11}=1,~A_{12}=2$,于是 $|\vb*{B}|=1$,表明 $\vb*{B}$ 是可逆矩阵,因此
    $$\qty(\vb*{B}\qty(\vb*{A}^{\top}\vb*{B}^2)^{-1}\vb*{A}^{\top})^{-1}=\vb*{B},~\vb*{X}\qty(\vb*{A}^2\qty(\vb*{B}^{\top}\vb*{A})^{-1}\vb*{B}^{\top})^{-1}=\vb*{A}^{-1}$$
    故原方程可化简为 $\vb*{X}+\vb*{B}=\vb*{XA}^{-1}(\vb*{A}+\vb*{B})\Rightarrow \vb*{X}(\vb*{A}^{-1}\vb*{B})=\vb*{B}$,所以
    $$\vb*{X}=\vb*{B}(\vb*{A}^{-1}\vb*{B})^{-1}=\vb*{A}.$$
\end{solution}

\begin{example}
    设矩阵 $\vb*{A}=\begin{pmatrix}
            1 & -1 & 0  & 0  \\
            0 & 1  & -1 & 0  \\
            0 & 0  & 1  & -1 \\
            0 & 0  & 0  & 1
        \end{pmatrix}$,$\vb*{B}=\begin{pmatrix}
            2 & 1 & 3 & 4 \\
            0 & 2 & 1 & 3 \\
            0 & 0 & 2 & 1 \\
            0 & 0 & 0 & 2
        \end{pmatrix}$,求解方程
    $$\vb*{X}\qty(\vb*{E}_n-\vb*{B}^{-1}\vb*{A})^{\top}\vb*{B}^{\top}=\vb*{E}_n.$$
\end{example}
\begin{solution}
    $\vb*{E}_n=\vb*{X}\qty(\vb*{B}\qty(\vb*{E}_n-\vb*{B}^{-1}\vb*{A}))^{\top}=\vb*{X}\qty(\vb*{BE}_n-\vb*{BB}^{-1}\vb*{A})^{\top}=\vb*{X}\qty(\vb*{B}-\vb*{A})^{\top}\Rightarrow \vb*{X}=\qty(\qty(\vb*{B}-\vb*{A})^{\top})^{-1}$,
    \begin{flalign*}
        \vb*{X} & =\qty(\qty(\begin{pmatrix}
                                 2 & 1 & 3 & 4 \\
                                 0 & 2 & 1 & 3 \\
                                 0 & 0 & 2 & 1 \\
                                 0 & 0 & 0 & 2
                             \end{pmatrix}-\begin{pmatrix}
                                               1 & -1 & 0  & 0  \\
                                               0 & 1  & -1 & 0  \\
                                               0 & 0  & 1  & -1 \\
                                               0 & 0  & 0  & 1
                                           \end{pmatrix})^{\top})^{-1}
        =\begin{pNiceArray}{cc:cc}
             1 & 0 & 0 & 0 \\
             2 & 1 & 0 & 0 \\ \hdottedline
             3 & 2 & 1 & 0 \\
             4 & 3 & 2 & 1
         \end{pNiceArray}^{-1}                                                                                                         \\
                & =\begin{pNiceArray}{c:c}
                       \begin{pmatrix}
                1 & 0 \\
                2 & 1
            \end{pmatrix}^{-1}                & O                                   \\ \hdottedline
                       -\begin{pmatrix}
                1 & 0 \\
                2 & 1
            \end{pmatrix}^{-1}\begin{pmatrix}
                3 & 2 \\
                4 & 3
            \end{pmatrix}\begin{pmatrix}
                1 & 0 \\
                2 & 1
            \end{pmatrix}^{-1} & \begin{pmatrix}
                1 & 0 \\
                2 & 1
            \end{pmatrix}^{-1}
                   \end{pNiceArray}=\begin{pmatrix}
                                        1  & 0  & 0  & 0 \\
                                        -2 & 1  & 0  & 0 \\
                                        1  & -2 & 1  & 0 \\
                                        0  & 1  & -2 & 1
                                    \end{pmatrix}.
    \end{flalign*}
\end{solution}

\begin{example}
    设 $\vb*{A}=\mqty(1& 2& 2\\ 2& 5& 4\\ 2& 4& 5),~\vb*{B}=\mqty(1&1\\1&0),~\vb*{C}=\begin{pmatrix}
            1 & 1 &   &   &   \\
              & 1 & 1 &   &   \\
              &   & 1 & 1 &   \\
              &   &   & 1 & 1 \\
              &   &   &   & 1
        \end{pmatrix}$,求 $\vb*{X}$ 使 $\vb*{X}\mqty(\vb*{O}&\vb*{B}\\\vb*{A}&\vb*{O})=\vb*{C}.$
\end{example}
\begin{solution}
    易得 $\vb*{A},~\vb*{B}$ 都可逆,且 $\mqty(\vb*{O}&\vb*{B}\\\vb*{A}&\vb*{O})^{-1}=\mqty(\vb*{O}&\vb*{A}^{-1}\\\vb*{B}^{-1}&\vb*{O})$,所以 $\vb*{X}=\vb*{C}\mqty(\vb*{O}&\vb*{A}^{-1}\\\vb*{B}^{-1}&\vb*{O})$,且
    $$\vb*{C}=\begin{pNiceArray}{ccc:cc}
            1 & 1 &   &   &   \\
            & 1 & 1 &   &   \\
            &   & 1 & 1 &   \\ \hdottedline
            &   &   & 1 & 1 \\
            &   &   &   & 1
        \end{pNiceArray}=\begin{pNiceArray}{ccc:cc}
            1 & 1 & 0 & 0 & 0 \\
            0 & 1 & 1 & 0 & 0 \\
            0 & 0 & 1 & 1 & 0 \\ \hdottedline
            0 & 0 & 0 & 1 & 1 \\
            0 & 0 & 0 & 0 & 1
        \end{pNiceArray}=\mqty(\vb*{C}_1&\vb*{C}_2\\\vb*{O}&\vb*{C}_3)$$
    并且 $\vb*{A}^{-1}=\mqty(9& -2& -2 \\-2& 1& 0 \\-2& 0& 1),~\vb*{B}^{-1}=\mqty(1& -1\\0& 1)$,那么
    $$\vb*{X}=\mqty(\vb*{C}_2\vb*{B}^{-1}& \vb*{C}_1\vb*{A}^{-1}\\\vb*{C}_3\vb*{B}^{-1}&\vb*{O})=\begin{pNiceArray}{cc:ccc}
            0 & 0  & 7  & -1 & -2 \\
            0 & 0  & -4 & 1  & 1  \\
            1 & -1 & -2 & 0  & 1  \\ \hdottedline
            1 & 0  & 0  & 0  & 0  \\
            0 & 1  & 0  & 0  & 0
        \end{pNiceArray}.$$
\end{solution}

\begin{example}
    设 4 阶矩阵 $\vb*{A}$ 的伴随矩阵 $\vb*{A}^*=\begin{pmatrix}
            1 & 0  & 0 & 0 \\
            0 & 1  & 0 & 0 \\
            1 & 0  & 1 & 0 \\
            0 & -3 & 0 & 8
        \end{pmatrix}$,求解矩阵方程 $\vb*{AXA}^{-1}=\vb*{XA}^{-1}+3\vb*{E}_4$,并计算 $\vb*{X}^*\vb*{A}.$
\end{example}
\begin{solution}
    同例题 \ref{ABA} 的解法,解得 $\vb*{X}=\begin{pmatrix}
            6 & 0 & 0 & 0  \\
            0 & 6 & 0 & 0  \\
            6 & 0 & 6 & 0  \\
            0 & 3 & 0 & -1
        \end{pmatrix}$,由于 $$\vb*{X}^*\vb*{A}=\vb*{X}^*\dfrac{1}{|\vb*{A}|^{4-2}}\qty(\vb*{A}^*)^*=\dfrac{1}{4}\vb*{X}^*\qty(\vb*{A}^*)^*=\dfrac{1}{4}\qty(\vb*{A}^*\vb*{X})^*$$
    \begin{flalign*}
        \vb*{X}^*\vb*{A} & =\dfrac{1}{4}\qty(\begin{pmatrix}
                                                     1 & 0  & 0 & 0 \\
                                                     0 & 1  & 0 & 0 \\
                                                     1 & 0  & 1 & 0 \\
                                                     0 & -3 & 0 & 8
                                                 \end{pmatrix}\begin{pmatrix}
                                                                  6 & 0 & 0 & 0  \\
                                                                  0 & 6 & 0 & 0  \\
                                                                  6 & 0 & 6 & 0  \\
                                                                  0 & 3 & 0 & -1
                                                              \end{pmatrix})^*=\dfrac{1}{4}\begin{pNiceArray}{cc:cc}
                                                                                           6  & 0 & 0 & 0  \\
                                                                                           0  & 6 & 0 & 0  \\ \hdottedline
                                                                                           12 & 0 & 6 & 0  \\
                                                                                           0  & 6 & 0 & -8 \\
                                                                                       \end{pNiceArray}^*            \\
                         & =\begin{pNiceArray}{c:c}
                                \begin{vmatrix}
                6 & 0  \\
                0 & -8
            \end{vmatrix}\begin{pmatrix}
                6 & 0 \\
                0 & 6
            \end{pmatrix}^* & O                            \\ \hdottedline
                                -\begin{pmatrix}
                6 & 0  \\
                0 & -8
            \end{pmatrix}^*\begin{pmatrix}
                12 & 0 \\
                0  & 6
            \end{pmatrix}\begin{pmatrix}
                6 & 0 \\
                0 & 6
            \end{pmatrix}^*
                                & \begin{vmatrix}
                6 & 0 \\
                0 & 6
            \end{vmatrix}\begin{pmatrix}
                6 & 0  \\
                0 & -8
            \end{pmatrix}^*
                            \end{pNiceArray}=\begin{pmatrix}
                                                 -72 & 0   & 0   & 0  \\
                                                 0   & -72 & 0   & 0  \\
                                                 144 & 0   & -72 & 0  \\
                                                 0   & -54 & 0   & 54
                                             \end{pmatrix}.
    \end{flalign*}
\end{solution}

\begin{example}
    设 $4$ 阶矩阵 $\vb*{A}=\begin{pmatrix}
            1 & 2 & 0  & 0 \\
            1 & 3 & 0  & 0 \\
            0 & 0 & 0  & 2 \\
            0 & 0 & -1 & 0
        \end{pmatrix}$,求解 $\qty[\qty(\dfrac{1}{2}\vb*{A})^*]^{-1}\vb*{XA}^{-1}=2\vb*{AX}+12\vb*{E}_4.$
\end{example}
\begin{solution}
    因为 $|\vb*{A}|=\begin{vmatrix}
            1 & 2 \\
            1 & 3
        \end{vmatrix}
        \begin{vmatrix}
            0  & 2 \\
            -1 & 0
        \end{vmatrix}=2$,$\qty[\qty(\dfrac{1}{2}\vb*{A})^*]^{-1}=\qty[\qty|\dfrac{1}{2}\vb*{A}|\qty(\dfrac{1}{2}\vb*{A})^{-1}]^{-1}=8\cdot \dfrac{1}{2}\vb*{A}=4\vb*{A}$,
    因此原方程变成 $$4\vb*{AXA}^{-1}=2\vb*{AX}+12\vb*{E}_4\Rightarrow 2\vb*{AXA}^{-1}=\vb*{AX}+6\vb*{E}_4\Rightarrow \vb*{X}=6\vb*{A}^{-1}\qty(2\vb*{A}^{-1}-\vb*{E}_4)^{-1}=6\qty(2\vb*{E}_4-\vb*{A})^{-1}$$
    \begin{flalign*}
        \vb*{X} & =6\qty(\begin{pmatrix}
                             2 &   &   &   \\
                               & 2 &   &   \\
                               &   & 2 &   \\
                               &   &   & 2
                         \end{pmatrix}-\begin{pmatrix}
                                           1 & 2 & 0  & 0 \\
                                           1 & 3 & 0  & 0 \\
                                           0 & 0 & 0  & 2 \\
                                           0 & 0 & -1 & 0
                                       \end{pmatrix})^{-1}=6\begin{pNiceArray}{cc:cc}
                                                                1  & -2 & 0 & 0  \\
                                                                -1 & -1 & 0 & 0  \\ \hdottedline
                                                                0  & 0  & 2 & -2 \\
                                                                0  & 0  & 1 & 2
                                                            \end{pNiceArray}^{-1}                                     \\
                & =6\begin{pNiceArray}{c:c}
                        \begin{pmatrix}
                1  & -2 \\
                -1 & -1
            \end{pmatrix}^{-1} & O                  \\ \hdottedline
                        O                  & \begin{pmatrix}
                2 & -2 \\
                1 & 2
            \end{pmatrix}^{-1}
                    \end{pNiceArray}=6\begin{pNiceArray}{c:c}
                                          -\dfrac{1}{3} \begin{pmatrix}
                -1 & 2 \\
                1  & 1
            \end{pmatrix} & O                             \\ \hdottedline
                                          O                             & \dfrac{1}{6} \begin{pmatrix}
                2  & 2 \\
                -1 & 2
            \end{pmatrix}
                                      \end{pNiceArray} \\
                & =\begin{pNiceArray}{c:c}
                       -2 \begin{pmatrix}
                -1 & 2 \\
                1  & 1
            \end{pmatrix} & O                \\ \hdottedline
                       O                  & \begin{pmatrix}
                2  & 2 \\
                -1 & 2
            \end{pmatrix}
                   \end{pNiceArray}=\begin{pmatrix}
                                        2  & -4 & 0  & 0 \\
                                        -2 & -2 & 0  & 0 \\
                                        0  & 0  & 2  & 2 \\
                                        0  & 0  & -1 & 2
                                    \end{pmatrix}.
    \end{flalign*}
\end{solution}

