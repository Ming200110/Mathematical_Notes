\section{多元函数极值}

多元函数极值的判定方法有很多种, 包括但不限于条件极值和非条件极值的情况. 这些方法的选择依赖于具体问题的性质和约束条件. 例如, Lagrange 乘数法是一种常用的求解条件极值问题的方法, 而对于无条件极值, 可以通过分析函数的一阶偏导数来判断是否存在极值. 

\subsection{二元函数的极值}

\begin{theorem}[二元函数取极值的必要条件]
    可偏导的二元函数 $f(x,y)$ 在 $(a,b)$ 取极值的必要条件是
    $$f'_x(a,b)=0~  f'_y(a,b)=0$$
    称点 $(a,b)$ 为 $f(x,y)$ 的驻点.
    \index{二元函数取极值的必要条件}
\end{theorem}

\begin{theorem}[二元函数取极值的充分条件]
    \index{二元函数取极值的充分条件}若 $f(x,y)$ 在 $(a,b)$ 处二阶偏导函数连续, $(a,b)$ 是 $f(x,y)$ 的驻点, 令
    $$A=f''_{xx}(a,b),~  B=f''_{xy}(a,b),~  C=f''_{yy}(a,b)$$
    \begin{enumerate}[label=(\arabic{*})]
        \item 当 $\Delta=B^2-AC<0,~A>0$ 时, $f(a,b)$ 为极小值;
        \item 当 $\Delta=B^2-AC<0,~A<0$ 时, $f(a,b)$ 为极大值;
        \item 当 $\Delta=B^2-AC>0$ 时, $f(a,b)$ 不是 $f$ 的极值;
    \end{enumerate}
\end{theorem}

\begin{example}[2009 数二]
    设函数 $z=f(x,y)$ 的全微分为 $\dd z=x\dd x+y\dd y$, 则点 $(0,0)$
    \begin{tasks}(2)
        \task 不是 $f(x,y)$ 的连续点
        \task 不是 $f(x,y)$ 的极值点
        \task 是 $f(x,y)$ 的极大值点
        \task 是 $f(x,y)$ 的极小值点
    \end{tasks}
\end{example}
\begin{solution}
    $\displaystyle A=\pdv[2]{z}{x}=1,~B=\pdv{z}{x}{y}=\pdv{z}{y}{x}=0,~C=\pdv[2]{z}{y}=1$, 又在点 $(0,0)$ 处, $B^2-AC<0,~A>0$, 故 $(0,0)$ 为函数 $z=f(x,y)$ 的一个极小值点, 选 D.
\end{solution}

\begin{example}[2009 数一]
    求二元函数 $f(x, y)=x^{2}\left(2+y^{2}\right)+y \ln y $ 的极值.
\end{example}
\begin{solution}
    令 $\begin{cases}
        f_x'(x,y)=2x\qty(2+y^2)=0\\
        f_y'(x,y)=2x^2y+\ln y+1=0
    \end{cases}$, 解得驻点 $\qty(0,\dfrac{1}{\e})$, 因此 
    \begin{flalign*}
        A=f_{xx}''\qty(0,\dfrac{1}{\e})=\eval{4+2y^2}_{\qty(0,\frac{1}{\e})}=4+\dfrac{2}{\e^2},~B=f_{xy}''(x,y)=\eval{4xy}_{\qty(0,\frac{1}{\e})}=0,~C=f_{yy}''(x,y)=\eval{2x^2+\dfrac{1}{y}}_{\qty(0,\frac{1}{\e})}=\e
    \end{flalign*}
    则 $B^2-AC=-4\e-\dfrac{2}{\e}<0$, 又 $A>0$, 从而 $f\qty(0,\dfrac{1}{\e})$ 是 $f(x,y)$ 的极小值, 极小值为 $f\qty(0,\dfrac{1}{\e})=-\dfrac{1}{\e}.$
\end{solution}

\begin{example}
    求函数 $f(x,y)=3(x-2y)^2+x^3-8y^3$ 的极值, 并证明 $f(0,0)=0$ 不是 $f(x,y)$ 的极值.
\end{example}
\begin{solution}
    由 $\displaystyle \left\{\begin{matrix}
            f'_x= & 3x^2   & +6x  & -12y=0 \\
            f'_y= & -24y^2 & +24y & -12x=0
        \end{matrix}\right.\Rightarrow \text{ 驻点 }\begin{cases}
            P_1(-4,2) \\
            P_2(0,0)
        \end{cases}$, 且
    $$A=\pdv[2]{f}{x}=6x+6~  B=\pdv{f}{x}{y}=-12~  \pdv[2]{f}{y}=-48y+24$$
    在 $P_1$ 处 $A=-18,~B=-12,~C=-72,~\Delta=-1152<0$, 且 $A<0$, 所以 $f(-4,2)=64$ 为极大值;
    在 $P_2$ 处 $A=6,~B=-12,~C=24,~\Delta=0$, 所以不能用 $\Delta$ 来证明 $f(0,0)$ 不是极值, 
    下面用极值的定义来判断, 任取 $ (0,0) $ 的去心邻域
    $$U_{\delta}^{\circ}=\left\{(x, y) \mid 0<\sqrt{x^{2}+y^{2}}<\delta\right\}$$
    \begin{enumerate}[label=(\arabic{*})]
        \item 在 $ y=0 $ 上, 取 $ \left(x_{n}, y_{n}\right)=\left(\dfrac{1}{n}, 0\right)\left(n \in \mathbb{N}^{*}\right) $, 
              则当 $ n $ 充分大时, 显然有 $ \left(x_{n}, y_{n}\right) \in U_{\delta}^{\circ} $, 且
              $$f\left(x_{n}, y_{n}\right)=f\left(\dfrac{1}{n}, 0\right)=\dfrac{1}{n^{2}}\left(3+\dfrac{1}{n}\right)>0$$
        \item 在 $ x=k y(0<k<2) $ 上, 有 $ f(k y, y)=\left(k^{3}-8\right) y^{2}\left(y-\dfrac{3(2-k)}{4+2 k+k^{2}}\right) $, 
              故取 $ y=\dfrac{4(2-k)}{4+2 k+k^{2}}(>0) $ 时 $ f(k y, y)=\left(k^{3}-8\right) y^{2} \dfrac{2-k}{4+2 k+k^{2}}<0 $, 即取
              $$\left(x_{k}, y_{k}\right)=\left(\dfrac{4 k(2-k)}{4+2 k+k^{2}}, \dfrac{4(2-k)}{4+2 k+k^{2}}\right)$$
              时有
              \begin{flalign*}
                  f\left(x_{k}, y_{k}\right) & =f\left(\dfrac{4 k(2-k)}{4+2 k+k^{2}}, \dfrac{4(2-k)}{4+2 k+k^{2}}\right)                                                    \\
                                             & =\left(k^{3}-8\right) \dfrac{16(2-k)^{3}}{\left(4+2 k+k^{2}\right)^{3}}=-\dfrac{16(2-k)^{4}}{\left(4+2 k+k^{2}\right)^{2}}<0
              \end{flalign*}
              又因为
              $$\lim _{k \to 2^{-}}\left(x_{k}, y_{k}\right)=\lim _{k \to 2^{-}}\left(\dfrac{4 k(2-k)}{4+2 k+k^{2}}, \dfrac{4(2-k)}{4+2 k+k^{2}}\right)=(0,0)$$
    \end{enumerate}
    所以当 $ k $ 小于 $2$ 且充分接近 $2$ 时, $\left(x_{k}, y_{k}\right) \in U_{\delta}^{\circ} $.
    由上述 (1) 和 (2) 可得, 在 $ P_{2}(0,0) $ 的任意小邻域 $ U_{\delta}^{\circ} $ 内, 既存在点 $ \left(x_{n}, y_{n}\right) $, 
    使得 $ f\left(x_{n}, y_{n}\right)>0 $, 也存在点 $ \left(x_{k}, y_{k}\right) $, 使得 $ f\left(x_{k}, y_{k}\right)<0 $, 故 $ f(0,0)=0 $ 不是极值.
\end{solution}

\begin{theorem}[Hesse 矩阵与极值定理]
    设 $ P_{0} $ 为稳定点, 若在 $ P_{0} $ 处 Hesse 矩阵
    $$\vb*{H}(P_0)=\mqty(f_{x_{1} x_{1}}^{\prime \prime} & f_{x_{1} x_{2}}^{\prime \prime} & \cdots & f_{x_{1} x_{n}}^{\prime \prime} \\
        f_{x_{2} x_{1}}^{\prime \prime} & f_{x_{2} x_{2}}^{\prime \prime} & \cdots & f_{x_{2} x_{n}}^{\prime \prime} \\
        \vdots & \vdots & & \vdots \\
        f_{x_{n} x_{1}}^{\prime \prime} & f_{x_{n} x_{2}}^{\prime \prime} & \cdots & f_{x_{n} x_{n}}^{\prime \prime})$$
    为正定的, 则 $ f $ 在 $ P_{0} $ 处取极小值; 若 $ \boldsymbol{H}\left(P_{0}\right) $ 为负定的, 则 $ f $ 在 $ P_{0} $ 处取极大值; 
    若 $ \boldsymbol{H}\left(P_{0}\right) $ 为不定的, 则 $ f $ 在 $ P_{0} $ 处无极值. 关于正定性的介绍可阅读定理 \ref{Hurwitztheorem}.
    \index{Hesse 矩阵与极值定理}
\end{theorem}

\begin{example}
    求函数 $f(x,y)=x^4+y^4-(x+y)^2$ 的极值.
\end{example}
\begin{solution}
    令 $\begin{cases}
        f_x'(x,y)=4x^3-2(x+y)=0\\
        f_y'(x,y)=4y^3-2(x+y)=0
    \end{cases}$ 解得驻点 $P_1(0,0),~P_2(1,1),~P_3(-1,-1)$, 又 
    $$\displaystyle \pdv[2]{f}{x}=12x^2-2,~\pdv[2]{f}{y}=12y^2-2,~\pdv{f}{x}{y}=-2,~\pdv{f}{y}{x}=-2$$
    由 Hesse 矩阵 $\vb*{H}(P_1)=\mqty(-2&-2\\-2&-2),~\vb*{H}(P_2)=\mqty(10&-2\\-2&10),~\vb*{H}(P_3)=\mqty(10&-2\\-2&10)$ 且 $\vb*{H}(P_1)$ 不定, 故 $(0,0)$ 不是极值点; 
    $\vb*{H}(P_2)$ 正定, 故 $(1,1)$ 为极小值点; $\vb*{H}(P_3)$ 正定, 故 $(-1,-1)$ 为极小值点, 
    综上, $f$ 的极小值为 $f(1,1)=f(-1,-1)=-2.$
\end{solution}

\begin{example}
    求 $f(x,y)=x^3-x^2y+\dfrac{5}{6}y^2+\dfrac{1}{3}y^3$ 的极值.
\end{example}
\begin{solution}
    因为 $\begin{cases}
        \displaystyle \pdv{f}{x}=3x^2-2xy=0\\ [6pt]
        \displaystyle \pdv{f}{y}=-x^2+\dfrac{5}{3}y+y^2=0
    \end{cases}$ 解得驻点为 $P_1(0,0), P_2(-2,-3), P_3\qty(0,-\dfrac{5}{3})$, 且 $f''_{xx}=6x-2y, f''_{yy}=\dfrac{5}{3}+2y, f''_{xy}=-2x$, 于是 
    $$
    \vb*{H}(P_1)=\mqty(0&0\\0&\dfrac{5}{3}) \text{不定}, \vb*{H}(P_2)=\mqty(-6&4\\4&-\dfrac{13}{3})\text{负定}, \vb*{H}(P_3)=\mqty(\dfrac{10}{3}&0\\0&-\dfrac{5}{3})\text{不定}
    $$
    故 $f(x,y)$ 有极大值, 无极小值, $f_{\text{极大}}=f(-2,-3)=-8+12+\dfrac{15}{2}-9=\dfrac{5}{2}.$
\end{solution}

\begin{example}[2004 数一]
    设 $z=z(x,y)$ 是由 $x^2-6xy+10y^2-2yz-z^2+18=0$ 确定的函数, 求 $z=z(x,y)$ 的极值点与极值.
\end{example}
\begin{solution}
    在 $x^2-6xy+10y^2-2yz-z^2+18=0$ 两边对 $x,~y$ 求偏导, 得 $\begin{cases}
        \displaystyle 2x-6y-2y\pdv{z}{x}-2z\pdv{z}{x}=0\\ 
        \displaystyle -6x+20y-2z-2y\pdv{z}{y}-2z\pdv{z}{y}=0
    \end{cases}$ 令 $\begin{cases}
        z'_x=0\\ z'_y=0
    \end{cases}$, 得 $\begin{cases}
        x=3y\\ z=y 
    \end{cases}$ 代入 $x^2-6xy+10y^2-2yz-z^2+18=0$ 得 $\begin{cases}
        x=9\\ y=3 \\ z=3
    \end{cases}\text{或 } \begin{cases}
        x=-9\\ y=-3 \\ z=-3
    \end{cases}$ 并且 
    \begin{flalign*}
        0=&2-2y\pdv[2]{z}{x}-2\qty(\pdv{z}{x})^2-2z\pdv[2]{z}{x} \\ 
        0=&-6-2\pdv{z}{x}-2y\pdv{z}{x}{y}-2\pdv{z}{y}\pdv{z}{x}-2z\pdv{z}{x}{y}\\ 
        0=&20-2\pdv{z}{y}-2\pdv{z}{y}-2y\pdv[2]{z}{y}-2\qty(\pdv{z}{y})^2-2z\pdv[2]{z}{y}
    \end{flalign*}
    于是 $\vb*{H}(P_1)=\eval{\mqty(z''_{xx}&z''_{xy}\\z''_{yx}&z''_{yy})}_{(9,3,3)}=\mqty(\dfrac{1}{6}&-\dfrac{1}{2}\\[6pt]-\dfrac{1}{2}&\dfrac{5}{3})$ 正定, 故点 $(9,3)$ 是 $z(x,y)$ 的极小值点, 极小值为 $z(9,3)=3$, 
    同理可得 $\vb*{H}(P_2)=\mqty(-\dfrac{1}{6}&\dfrac{1}{2}\\[6pt]\dfrac{1}{2}&-\dfrac{5}{3})$ 负定, 从而点 $(-9,-3)$ 是 $z(x,y)$ 的极大值点, 极大值为 $z(-9,-3)=-3.$
\end{solution}

\begin{example}[2012 数一]
    求函数 $f(x,y)=x\e^{-\frac{x^2+y^2}{2}}$ 的极值.
\end{example}
\begin{solution}
    \textbf{法一: }令 $\begin{cases}
        f_x'(x,y)=\e^{-\frac{x^2+y^2}{2}}-x^2\e^{-\frac{x^2+y^2}{2}}=0\\[6pt]
        f_y'(x,y)=-xy \e^{-\frac{x^2+y^2}{2}}=0
    \end{cases}$ 解得驻点 $P_1(1,0),~P_2(-1,0)$, 且 
    $$\pdv[2]{f}{x}=-3x\e^{-\frac{x^2+y^2}{2}}+x^3\e^{-\frac{x^2+y^2}{2}},~\pdv[2]{f}{y}=-x\e^{-\frac{x^2+y^2}{2}}+xy^2\e^{-\frac{x^2+y^2}{2}},~\pdv{f}{x}{y}=-y\e^{-\frac{x^2+y^2}{2}}+x^2y\e^{-\frac{x^2+y^2}{2}}=\pdv{f}{y}{x}$$
    由 Hesse 矩阵 $\vb*{H}(P_1)=\mqty(-2\e^{-\frac{1}{2}}&0\\0&-\e^{-\frac{1}{2}}),~\vb*{H}(P_2)=\mqty(2\e^{-\frac{1}{2}}&0\\0&\e^{-\frac{1}{2}})$, $\vb*{H}(P_1)$ 负定, 故 $(1,0)$ 为极大值点; $\vb*{H}(P_2)$ 正定, 故 $(-1,0)$ 为极大值点, 
    因此 $f$ 的极大值为 $f(1,0)=\e^{-\frac{1}{2}}$; 极小值为 $f(-1,0)=-\e^{-\frac{1}{2}}$.\\
    \textbf{法二: }在驻点 $ (1,0) $ 处, 
    $$A=\left.\frac{\partial^{2} f}{\partial x^{2}}\right|_{(1,0)}=-2 \e^{-\frac{1}{2}},~ B=\left.\frac{\partial^{2} f}{\partial x \partial y}\right|_{(1,0)}=0,~ C=\left.\frac{\partial^{2} f}{\partial y^{2}}\right|_{(1,0)}=-\e^{-\frac{1}{2}}  $$
    由于 $ A C-B^{2}=2 \e^{-1}>0 $, 且 $ A<0$, 故 $ (1,0) $ 为极大值点, $f(1,0)=\e^{-\frac{1}{2}} $ 为极大值; 在驻点 $ (-1,0) $ 处, 
    $$A=\left.\frac{\partial^{2} f}{\partial x^{2}}\right|_{(-1,0)}=2 \e^{-\frac{1}{2}},~ B=\left.\frac{\partial^{2} f}{\partial x \partial y}\right|_{(-1,0)}=0,~ C=\left.\frac{\partial^{2} f}{\partial y^{2}}\right|_{(-1,0)}=\e^{-\frac{1}{2}}$$
    由于 $ A C-B^{2}=2 \e^{-1}>0, A>0 $, 故 $ (-1,0) $ 为极小值点,  $f(-1,0)=-\e^{-\frac{1}{2}} $ 为极小值.
\end{solution}

\begin{example}[2015 数二]
    已知函数 $f(x,y)$ 满足 $f''_{xy}(x,y)=2(y+1)\e^x,~f'_x(x,0)=(x+1)\e^x,~f(0,y)=y^2+2y$, 求 $f(x,y)$ 的极值.
\end{example}
\begin{solution}
    方程 $f''_{xy}(x,y)=2(y+1)\e^x$ 两边对 $y$ 求积分, 得 $f'_x=\e^x(y+1)^2+\varphi(x)$, 并令 $y=0$, 与 $f'_x(x,0)=(x+1)\e^x$ 对比得 $\varphi(x)=x\e ^x$, 那么 
    $f'_x(x,y)=\e^x(y+1)^2+x\e^x$, 两边再对 $x$ 求积分, 得 $$f(x,y)=(y+1)^2\e ^x+(x-1)\e ^x+\psi(y)$$
    又 $f(0,y)=y^2+2y$, 得 $\psi(y)=0$, 则 $f(x,y)=(y+1)^2\e ^x+(x-1)\e ^x$, 
    因此 $\begin{cases}
        f'_x=(y+1)^2\e ^x+x\e ^x=0\\ 
        f'_y=2(y+1)\e ^x=0
    \end{cases}$ 得 $x=0,~y=-1$, 且 $f''_{xx}=(y+1)^2\e ^x+(x+1)\e ^x,~f''_{xy}=2(y+1)\e ^x,~f''_{yy}=2$, 则 $\vb*{H}(0,-1)=\eval{\mqty(z''_{xx}&z''_{xy}\\z''_{yx}&z''_{yy})}_{(0,-1)}$ 正定, 
    于是 $f(x,y)$ 在 $(0,-1)$ 处取得极小值, $f(0,-1)=-1.$
\end{solution}

\begin{example}[2023 数一]
    求函数 $f(x,y)=\qty(y-x^2)\qty(y-x^3)$ 的极值.
\end{example}
\begin{solution}
    \textbf{法一: }令 $\begin{cases}
            f_x'(x,y)=5x^4-3x^2y-2xy=0 \\
            f_y'(x,y)=2y-x^3-x^2=0
        \end{cases}$ 解得驻点为 $P_1(0,0),~P_2(1,1),~P_3\qty(\dfrac{2}{3},\dfrac{10}{27})$, 且
    $$\pdv[2]{f}{x}=20x^3-6xy-2y,~\pdv[2]{f}{y}=2,~\pdv{f}{x}{y}=-3x^2-2x=\pdv{f}{y}{x}$$
    由 Hesse 矩阵得 $\vb*{H}(P_1)=\mqty(0&0\\0&2)$ 为不定, 故 $(0,0)$ 非极值点; $\vb*{H}(P_2)=\mqty(12&-5\\-5&2)$ 为不定, 故 $(1,1)$ 非极值点; $\vb*{H}(P_3)=\mqty(\dfrac{100}{27}&-\dfrac{8}{3}\\[6pt]-\dfrac{8}{3}&2) $ 为正定, 故 $\qty(\dfrac{2}{3},\dfrac{10}{27})$ 为极小值点, 且
    函数的极小值为 $f\qty(\dfrac{2}{3},\dfrac{10}{27})=-\dfrac{4}{27^2}.$\\
    \textbf{法二: }
    计算二阶偏导以及二阶混合偏导, 有
    \begin{flalign*}
        A & =\frac{\partial^{2} f}{\partial x^{2}}=-2 y-6 x y+20 x^{3}                                            \\
        B & =\frac{\partial^{2} f}{\partial x \partial y}=-2 x-3 x^{2}, C=\frac{\partial^{2} f}{\partial y^{2}}=2
    \end{flalign*}
    \begin{enumerate}[label=(\roman{*})]
        \item 当驻点为 $ (x, y)=(0,0) $ 时, 有 $ A=0, B=0, C=2 $, 此时 $ A C-B^{2}=0  $, 多元函数取得极值的充分条件失效, 由
              $f(x, y)=\left(y-x^{2}\right)\left(y-x^{3}\right)$ 可知 $ f(0,0)=0$, 取 $ y=2 x^{2}  $, 当 $ x $ 为充分小的正数时, 
              $f(x, y)=x^{4}(2-x)>0 $;
              取 $ y=2 x^{3}  $, 当 $ x $ 为充分小的正数时, 
              $f(x, y)=x^{5}(2 x-1)<0 $, 
              由极值点的定义可知, $ (0,0) $ 点不是 $ f(x, y) $ 的极值点;
        \item 当驻点为 $ (x, y)=\left(\dfrac{2}{3}, \dfrac{10}{27}\right) $ 时, 有
              $$A=\frac{100}{27}, B=-\frac{8}{3}, C=2 $$
              则 $ A C-B^{2}=\dfrac{100}{27} \cdot 2-\left(-\dfrac{8}{3}\right)^{2}=\dfrac{8}{27}>0 $ 且 $ A=\dfrac{100}{27}>0 $, 
              故 $ f(x, y) $ 在 $ \left(\dfrac{2}{3}, \dfrac{10}{27}\right) $ 取得极小值;
        \item 当驻点为 $ (x, y)=(1,1) $ 时, 有
              $$A=12, B=-5, C=2$$
              此时 $ A C-B^{2}=12 \cdot 2-(-5)^{2}=-1 $, 所以 $ f(x, y) $ 在 $ (1,1) $ 处不取极值, 
              综上可知, $ f(x, y) $ 仅有一个极小值点 $ \left(\dfrac{2}{3}, \dfrac{10}{27}\right) $ 且取得极小值为
              $$f\left(\frac{2}{3}, \frac{10}{27}\right)=\left(\frac{10}{27}-\frac{4}{9}\right)\left(\frac{10}{27}-\frac{8}{27}\right)=-\frac{4}{729} .$$
    \end{enumerate}
\end{solution}

\subsection{条件极值}

二元函数 $ z=f(x, y) $ 在条件 $ \varphi(x, y)=0 $ 下的极值.
构造 Lagrange 函数:
$$F(x, y, \lambda)=f(x, y)+\lambda \varphi(x, y)$$
得方程组
$$\begin{cases}
    \dfrac{\partial F}{\partial x}=\dfrac{\partial f}{\partial x}+\lambda \dfrac{\partial \varphi}{\partial x}=0 \\[6pt]
\dfrac{\partial F}{\partial y}=\dfrac{\partial f}{\partial y}+\lambda \dfrac{\partial \varphi}{\partial y}=0 \\[6pt]
\dfrac{\partial F}{\partial \lambda}=\varphi(x, y)=0
\end{cases}$$
解得满足方程组所有点 $ (x, y) $, 即为可能的极值点, 然后再逐一判定.

\subsubsection{单项链等法}

当目标函数 $f(x,y,z)=mx^ay^bz^c~~(mabc\neq0)$ 时, 可构造出相等的项, 将其放在等号一边使其连等, 消除 $\lambda$.

\begin{example}
    设 $x,~y,~z>0$, 求 $f(x,y,z)=8xyz$, 在条件 $\dfrac{x^2}{a^2}+\dfrac{y^2}{b^2}+\dfrac{z^2}{c^2}=1~~(a,b,c>0)$ 下的最大值.
\end{example}
\begin{solution}
    $F(x,y,z,\lambda)=8xyz+\lambda\qty(\dfrac{x^2}{a^2}+\dfrac{y^2}{b^2}+\dfrac{z^2}{c^2}-1)$, 那么 
    $$\begin{cases}
        F'_x=8 y z + \dfrac{2 \lambda x}{a^{2}}=0 \xrightarrow{\text{乘 }x} -8xyz=\dfrac{2\lambda x^2}{a^2}\\
        F'_y=8 x z + \dfrac{2 \lambda y}{b^{2}}=0 \xrightarrow{\text{乘 }y} -8xyz=\dfrac{2\lambda y^2}{b^2}\\
        F'_z=8 x y + \dfrac{2 \lambda z}{c^{2}}=0 \xrightarrow{\text{乘 }z} -8xyz=\dfrac{2\lambda z^2}{c^2}\\
        F'_\lambda=\dfrac{x^{2}}{a^{2}}+ \dfrac{y^{2}}{b^{2}} +\dfrac{z^{2}}{c^{2}}  -1=0
    \end{cases}$$
    由此可得 $\dfrac{x^2}{a^2}=\dfrac{y^2}{b^2}=\dfrac{z^2}{c^2}=\dfrac{1}{3}\Rightarrow\begin{cases}
        x=\dfrac{a}{\sqrt{3}}\\ y=\dfrac{b}{\sqrt{3}}\\ z=\dfrac{c}{\sqrt{3}} 
    \end{cases}$, 故 $f_{max}=\dfrac{8abc}{3\sqrt{3}}.$
\end{solution}

\subsubsection{齐次构造法}

如果目标函数 $f$ 是齐次的, 同时约束条件可以转化成齐次函数 $g(x,y,z)=c$ 的形式.
若此时目标函数 $f(x,y,z)$ 为 $m$ 次齐次函数, $g(x,y,z)$ 为 $n$ 次齐次函数 $(m,n\neq0)$, 
则可以构造出这个式子: $f(x,y,z)=-\dfrac{cn}{m}\lambda$, 此时再求出 $\lambda$ 的值即可求出目标函数 $f(x,y,z)$ 的最值.

\begin{example}[2010 数三]
    求函数 $u=xy+2yz$ 在约束条件 $x^2+y^2+z^2=10$ 下的最大值和最小值.
\end{example}
\begin{solution}
    令 $F(x,y,z,\lambda)=xy+2yz+\lambda\qty(x^2+y^2+z^2-10)$, 解方程组 
    $\begin{cases}
        F'_x=2 \lambda x + y=0\\ 
        F'_y=2 \lambda y + x + 2 z=0\\ 
        F'_z=2 \lambda x + 2y=0\\ 
        F'_\lambda=x^{2} + y^{2} + z^{2} - 10=0
    \end{cases}$ 那么 $$\dfrac{x}{2}\cdot F'_x+\dfrac{y}{2}\cdot F'_y+\dfrac{z}{2}\cdot F'_z=0\Rightarrow xy+2yz=-10\lambda$$
    由于 $x,y,z$ 全为 $0$ 显然不成立, 因此 $\begin{vmatrix}
        2\lambda&1&0 \\ 1&2\lambda&2 \\ 0&2&2\lambda
    \end{vmatrix}=0\Rightarrow \lambda=0,\pm\dfrac{\sqrt{5}}{2}$, 故 $u_{min}=-5\sqrt{5},~u_{max}=5\sqrt{5}.$
\end{solution}

% \begin{example}[2008 数二]
%     求函数 $u=x^2+y^2+z^2$ 在约束条件 $z=x^2+y^2$ 和 $x+y+z=4$ 下的最大值和最小值.
% \end{example}
% \begin{solution}
%     $F(x,y,z,\lambda,\mu)=x^2+y^2+z^2+\lambda\qty(x^2+y^2-z)+\mu\qty(x+y+z-4)$, 那么 
%     $\begin{cases}
%         F'_x=2 \lambda x + \mu + 2 x =0 \\ 
%         F'_y=2 \lambda y + \mu + 2 y =0 \\ 
%         F'_z=- \lambda + \mu + 2 z   =0 \\ 
%         F'_\lambda=x^2+y^2-z=0 \\ 
%         F'_\mu=x+y+z-4=0
%     \end{cases}$
% \end{solution}

\subsubsection{多约束条件}

\begin{example}[2019 数一]
    已知曲线 $C:\begin{cases}
        x^2+2y^2-z=6\\
        4x+2y+z=30
    \end{cases}$ 求 $C$ 上的点到 $xOy$ 坐标平面的距离的最大值.
\end{example}
\begin{solution}
    设 $P(x,y,z)\in C$, 到 $xOy$ 坐标平面的距离为 $d=|z|$, 则目标函数 $f(x,y,z)=z^2$, 约束函数为 $x^2+2y^2-z-6=0$ 及 $4x+2y+z-30=0$, 则构造 Lagrange 函数
    $$F(x,y,z,\lambda,\mu)=z^2+\lambda\qty(x^2+2y^2-z-6)+\mu\qty(4x+2y+z-30)=0$$
    所以 $\begin{cases}
        F_x'=2\lambda x+4\mu=0\\
        F_y'=4\lambda y+2\mu=0\\
        F_z'=2z-\lambda+\mu=0\\
        F_\lambda'=x^2+2y^2-z-6=0\\
        F_\mu'=4x+2y+z-30=0
    \end{cases}\Rightarrow \begin{cases}
        x=4\\
        y=1\\
        z=12
    \end{cases}\text{ 或 }\begin{cases}
        x=-8\\
        y=-2\\
        z=66
    \end{cases}$ 得驻点 $(4,1,12)$ 和 $(-8,-2,66)$, 故曲线 $C$ 上的点到 $xOy$ 坐标平面的距离的最大值 $d_{max}=66.$
\end{solution}

\subsubsection{三角代换}

\begin{example}
    求 $f(x,y)=3x^2-2xy+2y^2$ 在单位圆周 $x^2+y^2=1$ 上的最大值和最小值.
\end{example}
\begin{solution}
    令 $x=\cos\theta,~y=\sin\theta$, 那么 $f(\theta)=3\cos^2\theta+2\sin^2\theta-2\sin \theta\cos\theta=\dfrac{1}{2}\qty[5-\sqrt{5}\sin(2\theta-\varphi)]$, 其中 $\varphi=\arctan\dfrac{1}{2}$, 
    由 $\theta\in[0,2\pi]$ 可知 $2\theta-\varphi\in[-\varphi,4\pi-\varphi]$, 所以 $\sin(2\theta-\varphi)\in[-1,1]$, 故 $f_{max}=\dfrac{5+\sqrt{5}}{2},~f_{min}=\dfrac{5-\sqrt{5}}{2}.$
\end{solution}

\subsection{多元函数的最值问题}

求连续函数在有界闭域 $D$ 上最值的步骤:
\begin{enumerate}[label=(\arabic{*})]
    \item 求 $D$ 内的驻点和不可导点;
    \item 用求极值的方法求 $D$ 的边界上最值的可疑点;
    \item 比较这些点的函数值的大小.
\end{enumerate}

\begin{example}
    求 $f(x,y)=x^2+2y^2-x^2y^2$ 在 $D=\qty{(x,y)|x^2+y^2 \leqslant 4, x\geqslant 0, y\geqslant 0}$ 上的最大值和最小值.
\end{example}
\begin{solution}
    $\begin{cases}
        \displaystyle \pdv{f}{x}=2x-2xy^2=0 \\[6pt]
        \displaystyle \pdv{f}{y}=4y-2x^2y=0
    \end{cases}$ 解得 $P\qty(\sqrt{2},1)~(x,y\geqslant 0)$, $f''_{xx}=2-2y^2, f''_{yy}=4-2x^2, f''_{xy}=-4xy$, 那么 $\vb*{H}(P)=\mqty(0&-4\sqrt{2}\\ -4\sqrt{2}&0)$ 不定, 所以 $P$ 点不是极值点, 从而也非最值点, 
    当 $x=0~(0\leqslant y\leqslant 2)$ 时, $f_{min}=0, f_{max}=8$, 当 $y=0~(0\leqslant x\leqslant 2)$ 时, $f_{min}=0, f_{max}=4$,\\ 
    考虑边界情况, 令 $F=x^2+2y^2-x^2y^2+\lambda\qty(x^2+y^2-4)$, 那么 
    $$
    \begin{cases}
        F'_x=2x-2xy^2+2\lambda x=0 \\ 
        F'_y=4y-2x^2y+2\lambda y=0 \\ 
        F'_\lambda=x^2+y^2-4=0
    \end{cases}\Rightarrow \begin{cases}
        x=\dfrac{\sqrt{10}}{2} \\[6pt]
        y=\dfrac{\sqrt{6}}{2}
    \end{cases}
    $$ 即 $f\qty(\dfrac{\sqrt{10}}{2},\dfrac{\sqrt{6}}{2})=\dfrac{7}{4}$, 故所求的最小值为 $m=0$, 最大值为 $M=8.$
\end{solution}

\begin{example}[2005 数二]
    已知函数 $z=f(x,y)$ 的全微分 $\dd z=2x\dd x-2y\dd y$, 并且 $f(1,1)=2$, 求 $f(x,y)$ 在椭圆域 $D=\qty{(x,y)\Bigg| x^2+\dfrac{y^2}{4}\leqslant  1 }$ 上的最大值和最小值.
\end{example}
\begin{solution}
    由 $\dd z=2x\dd x-2y\dd y$ 知, $\displaystyle\pdv{f}{x}=2x,~\pdv{f}{y}=-2y$, 那么 $f(x,y)=x^2-y^2+C$, 又 $f(1,1)=2$, 求得 $C=2$, 因此 $f(x,y)=x^2-y^2+2$, 那么 $f''_{xx}=2,~f''_{xy}=0,~f''_{yy}=-2$, 
    令 $f'_x=0,~f'_y=0$, 得可疑极值点 $P(0,0)$, 所以 $\vb*{H}(P)=\mqty(2&0\\0&-2)$ 不定, $P(0,0)$ 不是极值点, 从而也非最值点, \\ 
    再考虑边界情况, $x^2+\dfrac{y^2}{4}=1$ 上的情形, 作 Lagrange 函数 $F(x,y,\lambda)=x^2-y^2+2+\lambda\qty(x^2+\dfrac{y^2}{4}-1)$, 那么 
    $$\begin{cases}
        F'_x=2 x (\lambda + 1)=0\\ 
        F'_y=\dfrac{y (\lambda - 4)}{2}=0\\ 
        F'_\lambda=x^2+\dfrac{y^2}{4}-1=0
    \end{cases}$$ 解得 $\begin{cases}
        x=0\\ y=\pm2
    \end{cases}\text{或 }\begin{cases}
        x=\pm 1\\ y=0
    \end{cases}$ 代入 $f(x,y)$, 得 $f(0,\pm 2)=-2,~f(\pm1,0)=3$, 综上, $z=f(x,y)$ 在区域 $D$ 上的最大值为 $3$, 最小值为 $-2$.
\end{solution}

\begin{example}
    设 $I(a,b)=\displaystyle \int_{0}^{2\pi} (a\cos x-2b\sin x)^2 \dd x$, 在 $I(a,b)\leqslant 4\pi$, 求使得 $a^2+4b^2-2a-b\leqslant k$ 成立的 $k$ 的最小值.
\end{example}
\begin{solution}
    由题意, 
    $$
    I(a,b)=\int_{0}^{2\pi} \qty(a^2\cos^2x+4b^2\sin^2x-4ab \sin x\cos x) \dd x=\pi a^2-4\pi b^2
    $$
    即 $a^2+4b^2\leqslant 4$,令 $f(a,b)=a^2+4b^2-2a-b$, 即求 $f(a,b)$ 在闭区域 $a^2+4b^2\leqslant 4$ 上的最大值, 当 $a^2+4b^2=\pi$ 时, 构造 Lagrange 函数 $F=a^2+4b^2-2a-b+\lambda\qty(a^2+4b^2-4)$, 那么 
    $$
    \begin{cases}
        F_a'=2a-2+2\lambda a=0 \\
        F_b'=8b-1+8\lambda b=0 \\ 
        F_\lambda'=a^2+4b^2-4=0
    \end{cases}\Rightarrow \begin{cases}
        a=\dfrac{8}{\sqrt{17}} \\[6pt]
        b=\dfrac{1}{\sqrt{17}}
    \end{cases}\text{或}
    \begin{cases}
        a=-\dfrac{8}{\sqrt{17}} \\[6pt]
        b=-\dfrac{1}{\sqrt{17}}
    \end{cases}
    $$
    当 $a^2+4b^2<4$ 时, $\begin{cases}
        \displaystyle \pdv{f}{a}=2a-2=0\\ 
        \displaystyle \pdv{f}{b}=8b-1=0
    \end{cases}\Rightarrow \begin{cases}
        a=1\\ 
        b=\dfrac{1}{8}
    \end{cases}$, 那么 $f\qty(\dfrac{8}{\sqrt{17}},\dfrac{1}{\sqrt{17}})=4-\sqrt{17}, f\qty(-\dfrac{8}{\sqrt{17}},-\dfrac{1}{\sqrt{17}})=4+\sqrt{17}, f\qty(1,\dfrac{1}{8})=-\dfrac{17}{16}$, 于是 $f_{max}=4+\sqrt{17}$, 故 $k_{min}=4+\sqrt{17}.$
\end{solution}