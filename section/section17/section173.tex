\section{多维混合型随机变量}

\subsection{二维混合型随机变量的分布}

\begin{theorem}[二维混合型随机变量函数的分布]
    设二维随机变量 $ (X, Y) $, 其中离散型随机变量 $ X $ 的分布律为 $ P\left\{X=x_{i}\right\}=p_{i}(i=1,2, \cdots) $, 
    $ Y $ 为连续型随机变量, 则 $ (X, Y) $ 的函数 $ Z=g(X, Y) $ 分布函数为
    $$F_{Z}(z)=P\{Z \leqslant z\}=P\{g(X, Y) \leqslant z\} =\sum_{i=1}^{\infty} P\left\{X=x_{i}\right\} P\left\{g(X, Y) \leqslant z \mid X=x_{i}\right\}$$
\end{theorem}

\begin{example}
    设随机变量 $X,Y$ 相互独立, 且 $$P\qty{X=0}=P\qty{X=1}=\dfrac{1}{2},~P\qty{Y\leqslant x}=x,~0<x\leqslant 1$$
    求 $Z=XY$ 的分布函数.
\end{example}
\begin{solution}
    由题意可知 $Y\in(0,1]$, 故 $Z=XY=[0,1]$, 当 $z<0$ 时, $F_Z(z)=0$; 当 $z\geqslant 1$ 时, $F_Z(z)=1$;
    当 $0\leqslant z<1$ 时, 
    \begin{flalign*}
        F_Z(z) & =P\qty{Z\leqslant z}=P\qty{X=0}P\qty{Z\leqslant z\mid X=0}+P\qty{X=1}P\qty{Z\leqslant z\mid X=1}                                                   \\
               & =\dfrac{1}{2}P\qty{XY\leqslant z\mid X=0}+\dfrac{1}{2}P\qty{XY\leqslant z\mid X=1}=\dfrac{1}{2}P\qty{0\leqslant z}+\dfrac{1}{2}P\qty{Y\leqslant z} \\
               & =\dfrac{1}{2}+\dfrac{1}{2}z=\dfrac{1}{2}(1+z)
    \end{flalign*}
    因此 $F_Z(z)=\begin{cases}
            0,                 & z<0            \\
            \dfrac{1}{2}(1+z), & 0\leqslant z<1 \\[6pt]
            1,                 & z\geqslant1.
        \end{cases}$
\end{solution}

\begin{example}
    设随机变量 $X$ 与 $Y$ 相互独立, 且 $X$ 的分布为 $\begin{NiceArray}{c|cc}
            X & -1          & 1           \\ \hline
            P & \frac{1}{2} & \frac{1}{2}
        \end{NiceArray}$, $Y$ 服从 $N(0,1)$ 分布, 记 $Z=XY$, 求 $Z$ 的分布函数 $F_Z(z).$
\end{example}
\begin{solution}
    根据全概率公式, 
    \begin{flalign*}
        F_Z(z) & =P\qty{Z\leqslant z}=P\qty{XY\leqslant z}=P\qty{X=-1}P\qty{XY\leqslant z\mid X=-1}+P\qty{X=1}P\qty{XY\leqslant z\mid X=1}                           \\
               & =\dfrac{1}{2}P\qty{-Y\leqslant z\mid X=-1}+\dfrac{1}{2}P\qty{Y\leqslant z\mid X=1}=\dfrac{1}{2}P\qty{Y\geqslant -z}+\dfrac{1}{2}P\qty{Y\leqslant z} \\
               & =\dfrac{1}{2}\qty[1-P\qty{Y<-z}]+\dfrac{1}{2}\varPhi(z)=\dfrac{1}{2}\qty[1-\varPhi(-z)]+\dfrac{1}{2}\varPhi(z)                                      \\
               & =\dfrac{1}{2}\qty[1-1+\varPhi(z)]+\dfrac{1}{2}\varPhi(z)=\varPhi(z)
    \end{flalign*}
\end{solution}

\begin{example}[2008 数一]
    设随机变量 $X$ 和 $Y$ 相互独立, $X$ 概率分布为 $P\qty{X=i}=\dfrac{1}{3}~~(i=-1,0,1)$, $Y$ 的概率密度为 $f_Z(z)=\begin{cases}
            1, & 0\leqslant y\leqslant 1 \\
            0, & \text{其他}
        \end{cases}$, 记 $Z=X+Y.$
    \begin{enumerate}[label=(\arabic{*})]
        \item 求 $P\qty{Z\leqslant \dfrac{1}{2}\biggl | X=0}$;
        \item 求 $Z$ 的概率密度 $f_Z(z).$
    \end{enumerate}
\end{example}
\begin{solution}
    \begin{enumerate}[label=(\arabic{*})]
        \item $\displaystyle P\qty{Z\leqslant \dfrac{1}{2}\biggl |X=0}=P\qty{X+Y\leqslant \dfrac{1}{2}\biggl |X=0}=P\qty{Y\leqslant \dfrac{1}{2}\biggl |X=0}=P\qty{Y\leqslant \dfrac{1}{2}}=\int_{0}^{\frac{1}{2}}f_Y(y)\dd y=\dfrac{1}{2}$.
        \item 记 $Z$ 的分布函数为 $F_Z(z)$ 则
              \begin{flalign*}
                  F_Z(z) & =P\qty{Z\leqslant z}=P\qty{X+Y\leqslant z}=\sum_{i=-1}^{1}P\qty{X=i,X+Y\leqslant z}                  \\
                         & =\sum_{i=-1}^{1}P\qty{X=i,Y\leqslant z-i}=\sum_{i=-1}^{1}P\qty{X=i}P\qty{Y\leqslant z-i}             \\
                         & =\dfrac{1}{3}P\qty{Y\leqslant z+1}+\dfrac{1}{3}P\qty{Y\leqslant z}+\dfrac{1}{3}P\qty{Y\leqslant z-1}
              \end{flalign*}
              \begin{enumerate}[label=(\roman{*})]
                  \item 当 $z<-1$ 时, $F_Z(z)=0$;
                  \item 当 $-1\leqslant z<0$ 时, $F_Z(z)=\dfrac{1}{3}P\qty{Y\leqslant z+1}=\dfrac{z+1}{3}$;
                  \item 当 $0\leqslant z<1$ 时, $F_Z(z)=\dfrac{1}{3}P\qty{Y\leqslant z+1}+\dfrac{1}{3}P\qty{Y\leqslant z}=\dfrac{1}{3}+\dfrac{1}{3}z=\dfrac{z+1}{3}$;
                  \item 当 $1\leqslant z<2$ 时, $F_Z(z)=\dfrac{1}{3}P\qty{Y\leqslant z+1}+\dfrac{1}{3}P\qty{Y\leqslant z}+\dfrac{1}{3}P\qty{Y\leqslant z+1}=\dfrac{1}{3}+\dfrac{1}{3}+\dfrac{z-1}{3}=\dfrac{z+1}{3}$;
                  \item 当 $z\geqslant 2$ 时, $F_Z(z)=1.$
              \end{enumerate}
              于是 $F_Z(z)=\begin{cases}
                      0,              & z<-1            \\
                      \dfrac{z+1}{3}, & -1\leqslant z<2 \\[6pt]
                      1,              & z\geqslant 2
                  \end{cases}$, 那么 $f_Z(z)=F_Z'(z)=\begin{cases}
                      \dfrac{1}{3}, & -1\leqslant z<2 \\
                      0,            & \text{其他}
                  \end{cases}$
    \end{enumerate}
\end{solution}

\subsection{随机变量与函数性质}

\begin{example}[2009 数一]
    设随机变量 $ X $ 与 $ Y $ 相互独立, 且 $ X $ 服从标准正态分布 $ N(0,1)$, $ Y $ 的概率分布为 $ P\{Y=0\}=   P\{Y=1\}=\dfrac{1}{2} $, 
    记 $ F_{Z}(z) $ 为随机变量 $ Z=X Y $ 的分布函数, 则函数 $ F_{Z}(z) $ 的间断点个数为 (\quad).
    \begin{tasks}(4)
        \task 0
        \task 1
        \task 2
        \task 3
    \end{tasks}
\end{example}
\begin{solution}
    由于 $x,~y$ 相互独立, 因此
    \begin{flalign*}
        F_{Z}(z) & =P\{x y \leqslant z\} =  P\{x y \leqslant z \mid y=0\} P\{y=0\}+P\{x y \leqslant z \mid y=1\} P\{y=1\}                                           \\
                 & =  \frac{1}{2}\qty[P\{x y \leqslant z \mid y=0\}+P\{x y \leqslant z \mid y=1\}]=\dfrac{1}{2}\qty[P\qty{x\cdot 0\leqslant z}+P\qty{x\leqslant z}]
    \end{flalign*}
    若 $z<0$, 则 $F_Z(z)=\dfrac{1}{2}\varPhi(z)$; 若 $z\geqslant 0$, 则 $F_Z(z)=\dfrac{1}{2}[1+\varPhi(z)]$, 所以 $z=0$ 为间断点, 故有一个间断点, 因此选 B.
\end{solution}
