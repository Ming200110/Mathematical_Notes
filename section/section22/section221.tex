\section{矩估计法}

设总体 $ X $ 的分布形式已知, $\theta $ 为总体的待估参数, $X_{1}, X_{2}, \cdots, X_{n} $ 为从总体 $ X $ 中抽取的样本, 
如果总体 $ X $ 的数学期望 $ E(X) $ 存在, 那么 $ E(X) $ 是 $ \theta $ 的函数. 例如, 在泊松分布总体 $ \pi(\lambda) $ 中, 
样本一阶矩 $ E(X)=\lambda$; 在指数分布总体 $ X \sim e(\lambda) $ 中, $E(X)=\dfrac{1}{\lambda}$.
由于 $ X_{1}, X_{2}, \cdots, X_{n} $ 相互独立且与总体 $ X $ 同分布, 由大数定理知, 当 $ n $ 越来越大时, $\displaystyle\bar{X}=\dfrac{1}{n} \sum_{i=1}^{n} X_{i} $
依概率收玫到 $ E(X)=h(\theta) .$
要估计 $ \theta $, 令
$$E(X)=\frac{1}{n} \sum_{i=1}^{n} X_{i}=\bar{X}$$
解方程 $h(\theta)=\bar{X}$, 可求出 $ \theta$, 此种方法所得的估计 $ \hat{\theta} $ 称为未知参数 $ \theta $ 的矩估计.

\begin{example}
    设总体 $ X \sim U(a, b)$, $ a, b $ 未知, $X_{1}, X_{2}, \cdots, X_{n} $ 为来自总体 $ X $ 的样本, $  x_{1} ,  x_{2}, \cdots, x_{n} $ 是样本值, 试求 $ a, b $ 的矩估计量和矩估计值.
\end{example}
\begin{solution}
    令 $\begin{cases}
            \displaystyle E(X)=\dfrac{1}{n}\sum_{i=1}^{n}X_i=\bar{X} \\
            \displaystyle E\qty(X^2)=\dfrac{1}{n}\sum_{i=1}^{n}X_i^2
        \end{cases}$, 且 $X\sim U(a,b)$, 那么
    $$E(X)=\dfrac{a+b}{2},~E\qty(X^2)=D(X)+E^2(X)=\dfrac{(b-a)^2}{12}+\dfrac{(a+b)^2}{4}$$
    即 $\begin{cases}
            a+b =2\bar{X} \\
            \displaystyle b-a=2\sqrt{3}\sqrt{\dfrac{1}{n}\sum_{i=1}^{n}X_i^2-\bar{X}^2}
        \end{cases}$ 并且 $\displaystyle\dfrac{1}{n}\sum_{i=1}^{n}X_i^2-\bar{X}^2=\dfrac{1}{n}\sum_{i=1}^{n}\qty(X_i-\bar{X})^2$, 
    解得矩估计量和矩估计值分别为
    $$\begin{cases}
            \displaystyle \hat{a}=\bar{X}-\sqrt{\dfrac{3}{n}\sum_{i=1}^{n}\qty(X_i-\bar{X})^2} \\
            \displaystyle \hat{b}=\bar{X}+\sqrt{\dfrac{3}{n}\sum_{i=1}^{n}\qty(X_i-\bar{X})^2},
        \end{cases}
        \begin{cases}
            \displaystyle \hat{a}=\bar{x}-\sqrt{\dfrac{3}{n}\sum_{i=1}^{n}\qty(x_i-\bar{x})^2} \\
            \displaystyle \hat{b}=\bar{x}+\sqrt{\dfrac{3}{n}\sum_{i=1}^{n}\qty(x_i-\bar{x})^2}.
        \end{cases}$$
\end{solution}

\begin{example}
    设总体 $X$ 的分布函数为 $F(x)=\begin{cases}
            1-\e ^{-(x-\theta)^2}, & x\geqslant \theta \\
            0,                     & x<\theta
        \end{cases}(\theta>0\text{ 为未知数})$, $X_1,X_2,\cdots,X_n$ 为来自总体 $X$ 的简单随机样本, $\displaystyle\bar{X}=\dfrac{1}{n}\sum_{i=1}^{n}X_i$, 求 $\theta$ 的矩估计量 $\bar\theta$.
\end{example}
\begin{solution}
    总体 $ X $ 的密度函数为 $f(x)=
        \begin{cases}
            2(x-\theta) \mathrm{e}^{-(x-\theta)^{2}}, & x \geqslant \theta \\ 0, & x<\theta
        \end{cases}$, 那么
    \begin{flalign*}
        E(X) & =\int_{0}^{+\infty} x \cdot 2(x-\theta) \mathrm{e}^{-(x-\theta)^{2}} \dd  x=2 \int_{\theta}^{+\infty}[(x-\theta)+\theta] \cdot(x-\theta) \mathrm{e}^{-(x-\theta)^{2}} \dd (x-\theta)                                                      \\
             & =2 \int_{0}^{+\infty} x(x+\theta) \mathrm{e}^{-x^{2}} \dd  x=2 \int_{0}^{+\infty} x^{2} \mathrm{e}^{-x^{2}} \dd  x+\theta \int_{0}^{+\infty} \mathrm{e}^{-x^{2}} \dd \left(x^{2}\right)                                                   \\
             & \xlongequal{x^{2}=t} 2 \int_{0}^{+\infty} t \mathrm{e}^{-t} \cdot \frac{1}{2 \sqrt{t}} \dd  t+\theta=\int_{0}^{+\infty} t^{\frac{1}{2}} \mathrm{e}^{-t} \dd  t+\theta=\Gamma\left(\frac{1}{2}+1\right)+\theta=\frac{\sqrt{\pi}}{2}+\theta
    \end{flalign*}
    令 $ E(X)=\bar{X} $ 得参数 $ \theta $ 的矩估计量为 $ \displaystyle\hat{\theta}=\bar{X}-\frac{\sqrt{\pi}}{2} .$
\end{solution}

\begin{example}
    设总体 $ X $ 的概率密度为
    $f(x)=\begin{cases}
            2(x-\theta) \mathrm{e}^{-(x-\theta)^{2}}, & x>\theta \\ 0, & x \leqslant \theta
        \end{cases}$
    $\left(X_{1}, X_{2}, \cdots, X_{n}\right)$ 为来自总体 $ X $ 的简单随机样本.
    \begin{enumerate}[label=(\arabic{*})]
        \item 求参数 $ \theta $ 的矩估计量;
        \item 设 $ U=\min \left\{X_{1}, X_{2}, \cdots, X_{n}\right\} $, 求 $ E(U) .$
    \end{enumerate}
\end{example}
\begin{solution}
    \begin{enumerate}[label=(\arabic{*})]
        \item 与上题类似, 有
              \begin{flalign*}
                  E(X) & =\int_{\theta}^{+\infty} x \cdot 2(x-\theta) \mathrm{e}^{-(x-\theta)^{2}} \dd  x \xlongequal{x-\theta=t} \int_{0}^{+\infty}(t+\theta) \cdot 2 t \mathrm{e}^{-t^{2}} \dd  t                                                                                                         \\
                       & =\int_{0}^{+\infty} 2 t^{2} \mathrm{e}^{-t^{2}} \dd  t+\theta \int_{0}^{+\infty} 2 t \mathrm{e}^{-t^{2}} \dd  t=\int_{0}^{+\infty}\left(t^{2}\right)^{\frac{1}{2}} \mathrm{e}^{-t^{2}} \dd \left(t^{2}\right)+\theta \int_{0}^{+\infty} \mathrm{e}^{-t^{2}} \dd \left(t^{2}\right) \\
                       & =\Gamma\left(\frac{1}{2}+1\right)+\theta=\frac{1}{2} \Gamma\left(\frac{1}{2}\right)+\theta=\theta+\frac{\sqrt{\pi}}{2}
              \end{flalign*}
              由 $ E(X)=\bar{X} $ 得参数 $ \theta $ 的矩估计量为 $\displaystyle\hat{\theta}=\bar{X}-\frac{\sqrt{\pi}}{2}$.
        \item 总体 $ X $ 的分布函数为 $F(x)=P\{X \leqslant x\}$, 

              当 $ x<\theta $ 时, $F(x)=0$; \\
              当 $ x \geqslant \theta $ 时, $\displaystyle F(x)=\int_{\theta}^{x} 2(x-\theta) \mathrm{e}^{-(x-\theta)^{2}} \dd  x=1-\mathrm{e}^{-(x-\theta)^{2}} $, 即
              $$F(x)=\begin{cases}
                      0,                              & x<\theta           \\
                      1-\mathrm{e}^{-(x-\theta)^{2}}, & x \geqslant \theta
                  \end{cases}$$
              $U $ 的分布函数为
              \begin{flalign*}
                  F_{U}(x) & =P\{U \leqslant x\}=1-P\{U>x\}=1-P\left\{X_{1}>x\right\} P\left\{X_{2}>x\right\} \cdots P\left\{X_{n}>x\right\} \\
                           & =1-[P\{X>x\}]^{n}=1-[1-F(x)]^{n}=\begin{cases}
                                                                  0,                               & x<\theta           \\
                                                                  1-\mathrm{e}^{-n(x-\theta)^{2}}, & x \geqslant \theta
                                                              \end{cases}
              \end{flalign*}
              $U$ 的密度函数为
              $$f_{U}(x)=\begin{cases}
                      0,                                           & x \leqslant \theta \\
                      2 n(x-\theta) \mathrm{e}^{-n(x-\theta)^{2}}, & x>\theta
                  \end{cases}$$
              则
              \begin{flalign*}
                  E(U) & =\int_{\theta}^{+\infty} 2 n x(x-\theta) \mathrm{e}^{-n(x-\theta)^{2}} \dd  x=\int_{\theta}^{+\infty}\left[\frac{1}{\sqrt{n}}\left[n(x-\theta)^{2}\right]^{\frac{1}{2}}+\theta\right] \mathrm{e}^{-n(x-\theta)^{2}} \dd\left[n(x-\theta)^{2}\right] \\
                       & =\int_{0}^{+\infty}\left(\frac{1}{\sqrt{n}} t^{\frac{1}{2}}+\theta\right) \mathrm{e}^{-t} \dd  t=\frac{1}{\sqrt{n}} \cdot \Gamma\left(\frac{1}{2}+1\right)+\theta=\frac{\sqrt{\pi n}}{2 n}+\theta
              \end{flalign*}
    \end{enumerate}
\end{solution}