\section{估计量优良性的判定标准}

虽然总体分布中的参数是确定的, 但是对于同一个参数, 可以有许多不同的点估计. 
在这些估计中, 我们自然地希望挑选一个最“优”的点估计, 因此, 有必要建立评价估计量优劣的标准. 下面介绍几个常用的标准:无偏性、有效性和一致性.

\subsection{无偏性}

\begin{definition}[无偏性]
    \index{无偏性}
    若 $ \hat{\theta}=\hat{\theta}\left(X_{1}, X_{2}, \cdots, X_{n}\right) $ 是参数 $ \theta $ 的估计量, 
    其数学期望 $ E\qty(\hat{\theta}) $ 存在, 且 $ E\qty(\hat{\theta})=\theta $, 则称 $ \hat{\theta} $ \textit{是} $ \theta $ \textit{的无偏估计量}.
\end{definition}

\begin{example}
    已知 $E(X)=\mu, D(X)=\sigma^{2} $ 存在, $ X_{1}, X_{2}, \cdots, X_{n} $ 为来自总体 $ X $ 的样本, 试判定:\label{wpgjllti}
    \begin{enumerate}[label=(\arabic{*})]
        \item $\bar{X} $ 是否为 $ \mu $ 的无偏估计量;
        \item $\displaystyle\frac{1}{n} \sum_{i=1}^{n}\left(X_{i}-\bar{X}\right)^{2} $ 是否为 $ \sigma^{2} $ 的无偏估计量.
    \end{enumerate}
\end{example}
\begin{solution}
    由定义可知, 
    \begin{enumerate}[label=(\arabic{*})]
        \item 因为总体 $ E(X)=\mu, D(X)=\sigma^{2}, X_{1}, X_{2}, \cdots, X_{n} $ 为来自总体 $ X $ 的样本, 则
              $$E\left(X_{i}\right)=\mu, D\left(X_{i}\right)=\sigma^{2}, \quad i=1,2, \cdots, n $$
              所以 $\displaystyle E\qty(\bar{X})=E\qty(\dfrac{1}{n}\sum_{i=1}^{n}X_i)=\dfrac{1}{n}\sum_{i=1}^{n}E(X_i)=\mu$, 
              所以 $\bar{X} $ 为 $ \mu $ 的无偏估计量.
        \item 因为 $\displaystyle D\qty(\bar{X})=D\qty(\dfrac{1}{n}\sum_{i=1}^{n}X_i)=\dfrac{1}{n^2}\sum_{i=1}^{n}D(X_i)=\dfrac{\sigma^2}{n}$, 所以
              $$E\qty(\bar{X}^2)=D\qty(\bar{X})+E^2\qty(\bar{X})=\dfrac{\sigma^2}{n}+\mu^2,~E\qty(X_i^2)=D(X_i)+E^2(X_i)=\sigma^2+\mu^2$$
              \begin{flalign*}
                  E\qty[\dfrac{1}{n}\sum_{i=1}^{n}\qty(X_i-\bar{X})^2] & =E\qty(\dfrac{1}{n}\sum_{i=1}^{n}X_i^2-\bar{X}^2)=\dfrac{1}{n}\sum_{i=1}^{n}E\qty(X_i^2)-E\qty(\bar{X}^2) \\
                                                                       & =\dfrac{1}{n}\sum_{i=1}^{n}\qty(\mu^2+\sigma^2)-\qty(\dfrac{\sigma^2}{n}+\mu^2)=\dfrac{n-1}{n}\sigma^2
              \end{flalign*}
              所以 $\displaystyle\dfrac{1}{n}\sum_{i=1}^{n}\qty(X_i-\bar{X})^2$ 不是 $\sigma^2$ 的无偏估计量.
    \end{enumerate}
\end{solution}

由例题 \ref{wpgjllti} 可以看出, 如果在 $ \displaystyle\frac{1}{n} \sum_{i=1}^{n}\left(X_{i}-\bar{X}\right)^{2} $ 前面乘以系数
$ \displaystyle\frac{n}{n-1} $, 就修正成为 $ \sigma^{2} $ 的无偏估计量.
由此也解释了在定义样本方差时, 我们之所以选择 $\displaystyle S^{2}=\frac{1}{n-1} \sum_{i=1}^{n}\left(X_{i}-\bar{X}\right)^{2}$, 
而不是 $\displaystyle \frac{1}{n} \sum_{i=1}^{n}\left(X_{i}-\bar{X}\right)^{2} $ 形式, 是因为 $ S^{2} $ 是 $ \sigma^{2} $ 的无偏估计量.

\subsection{有效性}

同一个参数可以有多个无偏估计量, 那么选择哪一个为好呢? 设参数 $ \theta $ 有两个无偏估计量 $ \hat{\theta}_{1} $ 和 $ \hat{\theta}_{2}$, 
在样本容量 $ n $ 相同的情况下, 若 $ \hat{\theta}_{1} $ 的观测值都集中在 $ \theta $ 的真值附近, 
而 $\hat{\theta}_{2}$ 的观测值较远离 $\theta$ 的真值, 很显然 $\hat{\theta}_{1}$ 作为 $\theta$ 的估计更合适.
即 $\hat{\theta}_{1}$ 的方差较 $\hat{\theta}_{2}$ 的方差小, 我们认为 $\hat{\theta}_{1}$ 较 $\hat{\theta}_{2}$ 要好, 由此有如下的定义.

\begin{definition}[有效性]
    \index{有效性}
    设 $ \hat{\theta}_{1}=\hat{\theta}_{1}\left(X_{1}, X_{2}, \cdots, X_{n}\right) $ 与 $ \hat{\theta}_{2}=\hat{\theta}_{2}\left(X_{1}, X_{2}, \cdots, X_{n}\right)$
    都是参数 $ \theta $ 的无偏估计量, 若 $ D\left(\hat{\theta}_{1}\right) \leqslant D\left(\hat{\theta}_{2}\right) $, 则称 $ \hat{\theta}_{1} $ \textit{较} $ \hat{\theta}_{2} $ \textit{更有效}.
\end{definition}

% \begin{example}
%     设 $X_{1}, X_{2}, X_{3}, X_{4}$ 是来自均值为 $\theta$ 的指数分布总体的样本, 其中 $\theta$ 未知, 设有估计量
%     \begin{flalign*}
%         T_{1} & =\dfrac{1}{6}\left(X_{1}+X_{2}\right)+\dfrac{1}{3}\left(X_{3}+X_{4}\right), \\
%         T_{2} & =\dfrac{X_{1}+2 X_{2}+3 X_{3}+4 X_{4}}{5},                                  \\
%         T_{3} & =\dfrac{X_{1}+X_{2}+X_{3}+X_{4}}{4} .
%     \end{flalign*}
%     \begin{enumerate}[label=(\arabic{*})]
%         \item 指出 $T_{1}, T_{2}, T_{3}$ 中哪几个是 $\theta$ 的无偏估计量;
%         \item 在上述 $\theta$ 的无偏估计中指出哪一个较为有效.
%     \end{enumerate}
% \end{example}
% \begin{solution}
    
% \end{solution}

\subsection{一致性}

无偏性和有效性都是在假设样本容量 $n$ 固定的条件下讨论的.
由于估计量是样本的函数,它依赖样本容量 $n$, 自然地, 我们希望一个好的估计量, 当 $n$ 越来越大时, 它与参数的真值几乎一致, 
这就是估计量的一致性或称之为相合性.

\begin{definition}[一致性]
    \index{一致性}
    设 $ \hat{\theta}\left(X_{1}, X_{2}, \cdots, X_{n}\right) $ 为参数 $ \theta $ 的估计量, 当 $ n \to \infty $ 时, $\hat{\theta}\left(X_{1}, X_{2}, \cdots, X_{n}\right) $ 概率收敛于
    $ \theta $, 即对于任意 $ \varepsilon>0 $, 有
    $$\lim _{n \to \infty} P\qty{\qty|\hat{\theta}-\theta|<\varepsilon}=1$$
    则称 $ \hat{\theta} $ \textit{为} $ \theta $ \textit{的相合估计量}.
\end{definition}

