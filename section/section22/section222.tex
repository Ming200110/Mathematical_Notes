% \section{区间估计}

% \subsection{矩估计量}

% % \begin{example}
% %     设总体 $X$ 的概率密度为 $$f(x;\theta)=\begin{cases}
% %         \dfrac{\theta ^2}{x^3}\mathrm{e}^{-\frac{\theta }{x} }&,x>0  \\
% %        0&,\text{其他}
% %        \end{cases}$$
% %        其中 $\theta$ 为未知参数且大于零, $X_1,X_2,\cdots,X_n$ 为来自总体 $X$ 的简单随机样本, 
% %        求 $\theta$ 的矩估计量.
% % \end{example}

% \begin{example}
%     设总体 $X$ 的概率密度为
%     $$f(x;\theta)=\begin{cases}
%             \dfrac{6x}{\theta^3}(\theta-x) & ,0<x<\theta  \\
%             0                              & ,\text{其他}
%         \end{cases}$$
%     $X_1,X_2,\cdots,X_n$ 为来自总体 $X$ 的简单随机样本, 求 $\theta$ 的矩估计量 $\hat{\theta} $ 以及求 $\hat{\theta} $ 的方差 $D\qty(\hat{\theta} ).$
% \end{example}
% \begin{solution}
%     由 $\displaystyle E(X)=\int_{0}^{\theta}xf(x)\dd x$ 得, 
%     $$E(X)=\int_{0}^{\theta}\dfrac{6x^2}{\theta^3}(\theta-x)\dd x=\dfrac{\theta}{2}$$
%     记 $\displaystyle\bar{X}=\dfrac{1}{n}\sum_{i=1}^{n}X_i$, 令 $\dfrac{\theta}{2}=\bar{X}$, 得 $\theta$ 的矩估计量为 $\hat{\theta}=2\bar{X}$;
%     根据方差与期望的关系 $$D(X)=E(X^2)-E^2(X)$$
%     且 $\displaystyle E(X^2)=\int_{0}^{\theta}x^2f(x)\dd x=\int_{0}^{\theta}\dfrac{6x^3}{\theta^3}(\theta-x)\dd x=\dfrac{3\theta^2}{10}$, 
%     那么 $D(X)=\dfrac{\theta^2}{20}$, 所以 $\hat{\theta}=2\bar{X}$ 的方差为
%     $$D\qty(\hat{\theta})=D\qty(2\bar{X})=4D(\bar{X})=\dfrac{4}{n}D(X)=\dfrac{\theta^2}{5n}.$$
% \end{solution}

% \subsection{置信区间}

% \begin{theorem}[单个正态总体均值和方差的置信区间]
%     设总体 $X\sim N(\mu,\sigma^2)$, 常见的置信区间有: 
%     \begin{enumerate}[label=(\arabic{*})]
%         \item 当 $\sigma^2$ 已知时, $\mu$ 的置信度为 $1-\alpha$ 的置信区间为 $\qty(\bar{X}-U_{\frac{\alpha}{2}}\dfrac{\sigma}{\sqrt{n}},\bar{X}+U_{\frac{\alpha}{2}}\dfrac{\sigma}{\sqrt{n}})$.
%         \item 当 $\sigma^2$ 未知时, $\mu$ 的置信度为 $1-\alpha$ 的置信区间为 $\qty(\bar{X}-t_{\frac{\alpha}{2}}(n-1)\dfrac{S}{\sqrt{n}},\bar{X}+t_{\frac{\alpha}{2}}(n-1)\dfrac{S}{\sqrt{n}})$.
%         \item 当 $\mu$ 已知时, $\sigma^2$ 的置信度为 $1-\alpha$ 的置信区间为 $\qty(\dfrac{\sum_{i=1}^{n}(X_i-n)^2}{\chi^2_{\frac{\alpha}{2}}(n)},\dfrac{\sum_{i=1}^{n}(X_i-n)^2}{\chi^2_{1-\frac{\alpha}{2}}(n)}).$
%         \item 当 $\mu$ 未知时, $\sigma^2$ 的置信度为 $1-\alpha$ 的置信区间为 $\qty(\dfrac{(n-1)S^2}{\chi^2_{\frac{\alpha}{2}}(n)},\dfrac{(n-1)S^2}{\chi^2_{1-\frac{\alpha}{2}}(n)})$, $\sigma$ 的置信区间为 $\qty(\sqrt{\dfrac{(n-1)S^2}{\chi^2_{\frac{\alpha}{2}}(n)}},\sqrt{\dfrac{(n-1)S^2}{\chi^2_{1-\frac{\alpha}{2}}(n)}})$.
%     \end{enumerate}
% \end{theorem}

% \begin{theorem}[两个正态总体均值差的置信区间]
%     $\qty(\bar{X}-\bar{Y}-u_{\frac{\alpha}{2}}\sqrt{\dfrac{\sigma_1^2}{n_1}+\dfrac{\sigma_2^2}{n_2}},\bar{X}-\bar{Y}+u_{\frac{\alpha}{2}}\sqrt{\dfrac{\sigma_1^2}{n_1}+\dfrac{\sigma_2^2}{n_2}}).$
% \end{theorem}

\section{极大似然估计}

极大 (最大) 似然估计法是参数估计使用的最广泛的方法, 最早由德国数学家 Gauss 在 1821 年提出,
但是此法一般归功于英国统计学家 Fisher, 因为 Fisher 于 1922 年再次提出了这个思想, 并且证明了这种方法的一些性质,
从而使得极大似然法得到更普遍的应用.

求极大似然估计的步骤:
\begin{enumerate}[label=(\arabic{*})]
    \item 如果假设一组独立同分布的样本 $ \boldsymbol{x}=\left(x_{1}, \ldots, x_{N}\right) $ 来自于参数总体 $ P_{\theta} $, 且密度函数为 $ f\left(x_{i} \mid \theta\right) $, 那么样本的联合分布函数为
          $$
              f(\boldsymbol{x} \mid \theta)=\prod_{i=1}^{N} f\left(x_{i} \mid \theta\right)
          $$
    \item 现在, 将未知参数 $ \theta $ 视为变量, $ x $ 为给定的样本, 由于对数函数为单调函数, 因而可以将联合分布函数取对数, 得到对数似然函数,
          $$
              L(\theta \mid \boldsymbol{x})=\ln f(\boldsymbol{x} \mid \theta)=\sum_{i=1}^{N} \ln f\left(x_{i} \mid \theta\right)
          $$
    \item 极大似然估计即找到一个 $ \hat{\theta} $ 使得对数似然函数最大化
          $$
              \hat{\theta}=\arg \max _{\theta} L(\theta \mid \boldsymbol{x})
          $$
          从而我们得到了极大似然估计量 $ \hat{\theta} $.
\end{enumerate}

\begin{example}
    设总体 $X$ 的概率分布为 $P\qty{X=1}=\dfrac{1-\theta}{2},P\qty{X=2}=P\qty{X=3}=\dfrac{1+\theta}{4}$, 利用来自总体的样本值 $1,3,2,2,1,3,1,2$ 可得 $\theta$ 的极大似然估计值为 (\quad).
    \begin{tasks}(4)
        \task $\dfrac{1}{4}$
        \task $\dfrac{3}{8}$
        \task $\dfrac{1}{2}$
        \task $\dfrac{5}{8}$
    \end{tasks}
\end{example}
\begin{solution}
    样本值中 $1$ 出现三次, $2$ 出现三次, $3$ 出现两次, 那么似然函数
    $$
        L(\theta)=\qty(\dfrac{1-\theta}{2})^3\cdot \qty(\dfrac{1+\theta}{4})^3\cdot \qty(\dfrac{1+\theta}{4})^2=\qty(\dfrac{1-\theta}{2})^3\qty(\dfrac{1+\theta}{4})^5
    $$
    那么
    $$
        \ln L(\theta)=3\ln(1-\theta)+5\ln(1+\theta)-13\ln 2
    $$
    令 $$
        \dv{\ln L(\theta)}{\theta}=\dfrac{5}{1+\theta}-\dfrac{3}{1-\theta}=0
    $$
    解得 $\theta=\dfrac{1}{4}$, 那么 $\theta$ 的极大似然估计值为 $\hat{\theta}=\dfrac{1}{4}$, 故选 A.
\end{solution}

\begin{example}
    设 $X_i,~i=1,2,\cdots,n$ 为来自总体 $X\sim B(N,p)~~(0<p<1)$ 的简单随机样本, 求 $p$ 的极大似然估计量.
\end{example}
\begin{solution}
    似然函数为 $\displaystyle L(p)=\prod_{i=1}^{n} \mathrm{C}_{N}^{x_{i}} \cdot p^{x_{i}} \cdot(1-p)^{N-x_{i}} $, 两边取对数, 得
    $$\ln L(p)=\sum_{i=1}^{n} \ln C_{N}^{x_{i}}+\sum_{i=1}^{n} x_{i} \ln p+\sum_{i=1}^{n}\left(N-X_{i}\right) \ln (1-p)$$
    令 $$\frac{\mathrm{d}}{\mathrm{d} p} \ln L(p)=\frac{1}{p} \sum_{i=1}^{n} x_{i}-\frac{1}{1-p}\left(n N-\sum_{i=1}^{n} x_{i}\right)=0$$
    解得 $\displaystyle p=\frac{\displaystyle \sum_{i=1}^{n} x_{i}}{n N}=\frac{\bar{X}}{N} $, 故 $ p $ 的最大似然估计量为 $ \hat{p}=\dfrac{\bar{X}}{N} $,
    其中 $ \displaystyle\bar{X}=\dfrac{1}{n} \sum_{i=1}^{n} X_{i} .$
\end{solution}

\begin{example}
    设总体 $X\sim F(x;\theta)=\begin{cases}
            0, & x<1 \\ \theta, & 1\leqslant x<2 \\ 2\theta, & 2\leqslant x<3 \\ 1,&x\geqslant 3
        \end{cases}(0<\theta<\dfrac{1}{2})$, 样本值为 1,1,3,2,1,2,3,3, 求 $\theta$ 的矩估计和极大似然估计.
\end{example}
\begin{solution}
    对于离散型变量 $P\qty{X=x_0}=F(x_0)-\displaystyle \lim_{x \to x_0^-}F(x)$, 那么
    \begin{flalign*}
        P\qty{X=1} & = F(1)-\lim_{x\to 1^-}F(x)=\theta-0=\theta       \\
        P\qty{X=2} & = F(2)-\lim_{x\to 2^-}F(x)=2\theta-\theta=\theta \\
        P\qty{X=3} & = F(3)-\lim_{x\to 3^-}F(x)=1-2\theta
    \end{flalign*}
    故 $X$ 的分布律为 $X\sim \mqty(1& 2& 3\\ \theta & \theta & 1-2\theta)$,
    $$
    EX=\theta+2\theta+3-6\theta=3-3\theta, \quad \bar{X}=\dfrac{1\times 3+3\times 3+2\times 2}{8}=2
    $$
    所以 $\hat{\theta}=\dfrac{1}{3}$,
    $$
    L(\theta)=\theta^3\cdot\theta^2\cdot (1-2\theta)^3\Rightarrow \ln L(\theta)=5\theta+3\ln(1-2\theta)
    $$
    令 $\displaystyle \hat{\theta}=\arg \max _{\theta} \ln L(\theta)$, 解得 $\hat{\theta}=\dfrac{5}{12}$.
\end{solution}

\begin{example}
    设总体概率函数如下, $x_1,x_2,\cdots,x_n$ 是样本, 试求未知参数的极大似然估计:
    \begin{enumerate}[label=(\arabic{*})]
        \item $\displaystyle p(x;\theta)=\sqrt{\theta}x^{\sqrt{\theta}-1},0<x<1,\theta>0;$
        \item $\displaystyle p(x;\theta)=\theta c^\theta x^{-(\theta+1)},x>c>0,\theta>1.$
    \end{enumerate}
\end{example}
\begin{solution}
    \begin{enumerate}[label=(\arabic{*})]
        \item 似然函数 $\displaystyle L(\theta)=\left( \sqrt{\theta }\right) ^{n}\left( \prod ^{n}_{i=1}x_{i}\right) ^{\sqrt{\theta }-1}$, 取对数得
              $$\ln L\left( \theta \right) =\dfrac{n}{2}\ln \theta +\left( \sqrt{\theta }-1\right) \sum ^{n}_{i=1}\ln x_{i}$$
              将 $\ln L(\theta)$ 关于 $\theta$ 求导, 并令其值为 $0$ 即得到似然函数
              $$\dfrac{\partial \ln L\left( \theta \right) }{\partial \theta }=\dfrac{n}{2\theta }+\dfrac{1}{2\sqrt{\theta }}\sum ^{n}_{i=1}x_{i}=0$$
              解得 $\displaystyle\hat{\theta }=\left( \dfrac{1}{n}\sum ^{n}_{i=1}\ln x_{i}\right) ^{-2}$, 并且
              $$ \dfrac{\partial ^{2}\ln L\left( \theta \right) }{\partial \theta ^{2}}\bigg | _{\theta=\hat{\theta }}= \left( -\dfrac{n}{2\theta ^{2}}-\dfrac{1}{4\theta ^{3/2}}\sum ^{n}_{i=1}\ln x_{i}\right) \bigg | _{\theta=\hat{\theta }}=-\dfrac{3}{4n^{3}}\left( \sum ^{n}_{i=1}\ln x_{i}\right) ^{4}<0$$
              所以 $\hat{\theta}$ 是 $\theta$ 的极大似然估计.
        \item 似然函数 $\displaystyle L(\theta)=\theta ^{n}c^{n\theta }\left( \prod ^{n}_{i=1}x_{i}\right) ^{-\left( \theta +1\right) }$, 取对数得
              $$\ln L\left( \theta \right) =n\ln \theta +n\theta \ln c-\left( \theta +1\right) \sum ^{n}_{i=1}\ln x_{i}$$
              将 $\ln L(\theta)$ 关于 $\theta$ 求导, 并令其值为 $0$ 即得到似然函数
              $$\dfrac{\partial \ln L\left( \theta \right) }{\partial \theta }=\dfrac{n}{\theta }+n\ln c-\sum ^{n}_{i=1}\ln x_{i}=0$$
              解得 $\displaystyle \hat{\theta }=\left[ \dfrac{1}{n}\sum ^{n}_{i=1}\left( \ln x_{i}-\ln c\right) \right] ^{-1}$,
              并且 $\displaystyle \dfrac{\partial ^{2}\ln L\left( \theta \right) }{\partial \theta ^{2}}=\dfrac{-n}{\theta ^{2}} <0$, 所以 $\hat{\theta}$ 是 $\theta$ 的极大似然估计.
    \end{enumerate}
\end{solution}

\begin{example}[2018 数一]
    设总体 $X$ 的概率密度为 $$
        f(x;\sigma)=\dfrac{1}{2\sigma}\e ^{-\frac{|x|}{\sigma}}, -\infty<x<+\infty
    $$
    其中 $\sigma\in(0,+\infty)$ 为未知参数, $X_1, X_2, \cdots ,X_n$ 为来自总体 $X$ 的简单随机样本, 记 $\sigma$ 的极大似然估计值为 $\hat{\sigma}$,
    \begin{enumerate}[label=(\arabic{*})]
        \item 求 $\hat{\sigma}$;
        \item 求 $E \hat{\sigma}$ 和 $D \hat{\sigma}$.
    \end{enumerate}
\end{example}
\begin{solution}
    \begin{enumerate}[label=(\arabic{*})]
        \item 设 $x_1, x_2, \cdots ,x_n$ 为样本观测值, 似然函数为 $$
                  L(\sigma)=\prod_{i=1}^{n}f(x_i;\sigma)=\dfrac{1}{2^{n}\sigma^{n}}\exp\qty(-\dfrac{1}{\sigma}\sum_{i=1}^{n} |x_i|)
              $$
              两边取对数 $$
                  \ln L(\sigma)=-n\ln 2-n\ln \sigma-\dfrac{1}{\sigma}\sum_{i=1}^{n} |x_i|
              $$
              对 $\sigma$ 求导得 $$
                  \dv{\ln L(\sigma)}{\sigma}=-\dfrac{n}{\sigma}+\dfrac{1}{\sigma^2}\sum_{i=1}^{n} |x_i|=0
              $$
              解得 $\sigma=\displaystyle \dfrac{1}{n}\sum_{i=1}^{n} |x_i|$, 所以 $\hat{\sigma}=\displaystyle \dfrac{1}{n}\sum_{i=1}^{n} |X_i|$.
        \item 由 (1) 得
              \begin{flalign*}
                  E \hat{\sigma} & =\dfrac{1}{n}\sum_{i=1}^{n}E|X_i|=E|X|=\int_{-\infty}^{+\infty}|x|\cdot f(x;\sigma)\dd x=\int_{-\infty}^{+\infty}\dfrac{|x|}{2\sigma}\e ^{-\frac{|x|}{\sigma}}\dd x \\
                                 & =\dfrac{1}{\sigma}\int_{0}^{+\infty}x\e ^{-\frac{x}{\sigma}}\dd x\xlongequal{\frac{x}{\sigma}=t}\sigma\int_{0}^{+\infty}t\e ^{-t}\dd t=\sigma \Gamma(2)=\sigma
              \end{flalign*}
              又因为
              \begin{flalign*}
                  E|X|^2 & =EX^2=\int_{-\infty}^{+\infty}x^2\cdot f(x;\sigma)\dd x=\int_{-\infty}^{+\infty}x^2\cdot \dfrac{1}{2\sigma}\e ^{-\frac{|x|}{\sigma}}\dd x                                \\
                         & =\dfrac{1}{\sigma}\int_{0}^{+\infty}x^2\e ^{-\frac{x}{\sigma}}\dd x\xlongequal{\frac{x}{\sigma}=t}\sigma^2\int_{0}^{+\infty}t^2\e ^{-t}\dd t=\sigma^2\Gamma(3)=2\sigma^2
              \end{flalign*}
              因此  $D|X|=E|X|^2=E^2|X|=\sigma^2$,所以 $\displaystyle D \hat{\sigma}=\dfrac{1}{n^2}\sum_{i=1}^{n} D|X_i|=\dfrac{1}{n}D|X|=\dfrac{\sigma^2}{n}.$
    \end{enumerate}
\end{solution}

\begin{example}[2019 数一]
    设总体 $X$ 的概率密度为 $f\qty(x;\sigma^2)=\begin{cases}
            \dfrac{A}{\sigma}\e ^{-\frac{(x-\mu)^2}{2\sigma^2}}, & x\geqslant \mu \\ 0,&x<\mu
        \end{cases}$ 其中 $\mu$ 是已知参数, $\sigma>0$ 是未知参数, $A$ 是常数, $X_1, X_2, \cdots, X_n$ 是来自总体 $X$ 的简单随机样本,
    \begin{enumerate}[label=(\arabic{*})]
        \item 求 $A$;
        \item 求 $\sigma^2$ 的极大似然估计量.
    \end{enumerate}
\end{example}
\begin{solution}
    \begin{enumerate}[label=(\arabic{*})]
        \item $ \displaystyle \int_{-\infty}^{+\infty} f\qty(x;\sigma^2) \dd x =\int_{\mu}^{+\infty} \dfrac{A}{\sigma}\e ^{-\frac{(x-\mu)^2}{2\sigma^2}} \dd x \xlongequal{\frac{x-\mu}{\sqrt{2}\sigma}=t}\sqrt{2}A\int_{0}^{+\infty} \e ^{-t^2} \dd t=\sqrt{\dfrac{\pi}{2}}A=1$,解得 $A=\sqrt{\dfrac{2}{\pi}}$.
        \item 设 $x_1, x_2, \cdots ,x_n$ 为 $X_1, X_2, \cdots X_n$ 的观测值, 则似然函数 $$
                  L\qty(x_1, x_2, \cdots ,x_n;\sigma^2)=\prod_{i=1}^{n}f\qty(x_i;\sigma^2)=\qty(\dfrac{2}{\pi})^{\frac{n}{2}}\qty(\sigma^2)^{-\frac{n}{2}}\exp\qty[-\dfrac{1}{2\sigma^2}\sum_{i=1}^{n} (x_i-\mu)^2]
              $$
              两边取对数, 得 $$
                  \ln L=\dfrac{n}{2}\ln\dfrac{2}{\pi}-\dfrac{n}{2}\ln\sigma^2-\dfrac{1}{2\sigma^2}\sum_{i=1}^{n} (x_i-\mu)^2
              $$
              对 $\sigma^2$ 求导, 得 $$
                  \dv{\ln L}{\sigma^2}=-\dfrac{n}{2\sigma^2}+\dfrac{1}{2\qty(\sigma^2)^2}\sum_{i=1}^{n} (x_i-\mu)^2
              $$
              令 $\displaystyle \dv{\ln L}{\sigma^2}=0$, 解得 $\sigma^2$ 的极大似然估计值为 $\displaystyle \hat{\sigma^2}=\dfrac{1}{n}\sum_{i=1}^{n} (x_i-\mu)^2$, 则 $\sigma^2$ 的极大似然估计量为 $\displaystyle \hat{\sigma^2}=\dfrac{1}{n}\sum_{i=1}^{n} (X_i-\mu)^2$.
    \end{enumerate}
\end{solution}

\begin{example}[2020 数一]
    设某种元件的使用寿命 $T$ 的分布函数为 $$F(t)=\begin{cases}
            1-\mathrm{e}^{-\qty(\frac{t}{\theta})^m} & ,t\geqslant 0 \\
            0                                        & ,\text{其他}
        \end{cases}$$
    其中 $m,~\theta$ 为参数且大于 0,
    \begin{enumerate}[label=(\arabic{*})]
        \item 求概率 $P\qty{T>t}$ 与 $P\qty{T>s+t~|~T>s}$, 其中 $s,t>0$;
        \item 任取 $n$ 个这种元件做寿命试验, 测得它们的寿命分别为 $t_1,t_2,\cdots,t_n$, 若 $m$ 已知, 求 $\theta$ 的极大似然估计值 $\hat\theta.$
    \end{enumerate}
\end{example}
\begin{solution}
    \begin{enumerate}[label=(\arabic{*})]
        \item 由 $P\qty{T>t}=1-P\qty{T\leqslant t}$, 所以 $P\qty{T>t}=1-F(t)=\mathrm{e}^{-\qty(\frac{t}{\theta})^m}$,
              并且 $$P\qty{T>s+t~|~T>s}=\dfrac{P\qty{T>s+t,T>s}}{P\qty{T>s}}=\dfrac{P\qty{T>s+t}}{P\qty{T>s}}$$
              所以 $P\qty{T>s+t~|~T>s}=\mathrm{e}^{-\qty(\frac{s+t}{\theta})^m+\qty(\frac{s}{\theta})^m},~s,t>0$;
        \item 由 $f(t)=F'(t)$ 可得概率密度函数 $$f(t)=\begin{cases}
                      \dfrac{mt^{m-1}}{\theta^m}\mathrm{e}^{-\qty(\frac{t}{\theta})^m} & ,t\geqslant 0 \\
                      0                                                                & ,t<0
                  \end{cases}$$
              对数似然函数为
              \begin{flalign*}
                  \ln\qty[L(\theta)] & =\ln\qty[\prod_{i=1}^{n}f(t_i;\theta)]=\ln\qty[\prod_{i=1}^{n}\dfrac{mt_i^{m-1}}{\theta^m}\mathrm{e}^{-\qty(\frac{t_i}{\theta})^m}]=\ln\qty[\dfrac{m^n(t_1t_2\cdots t_n)^{m-1}}{\theta^{mn}}\cdot\mathrm{e}^{-\frac{1}{\theta^m}\sum\limits_{i=1}^{n}t_i^m}] \\
                                     & =n\ln m+(m-1)\ln\prod_{i=1}^{n}t_i-mn\ln\theta-\dfrac{1}{\theta^m}\sum_{i=1}^{n}t_i^m
              \end{flalign*}
              求导得 $\displaystyle\dv{\ln\qty[L(\theta)]}{\theta}=-\dfrac{mn}{\theta}+m\dfrac{1}{\theta^{m+1}}\sum_{i=1}^{n}t_i^m=0$, 解得 $\displaystyle\dfrac{1}{\theta^m}\sum_{i=1}^{n}t_i^m=n$,
              那么 $\theta$ 的极大似然估计值为 $\hat\theta=\sqrt[m]{\dfrac{1}{n}\displaystyle\sum_{i=1}^{n}t_i^m}.$
    \end{enumerate}
\end{solution}

\begin{example}
    设随机变量 $X$ 与 $Y$ 相互独立, $X$ 的概率密度 $f(x)$ 满足 $f'(x)+\dfrac{x}{\sigma^2}f(x)=0~(\sigma>0)$, $Y$ 的分布律为 $P\qty{Y=-1}=P\qty{Y=1}=\dfrac{1}{2}$, $Z=XY$.
    \begin{enumerate}[label=(\arabic{*})]
        \item 求 $Z$ 的概率密度 $f(z)$;
        \item $X$ 与 $Y$ 是否相关,说明理由;
        \item $Z_1,Z_2,\cdots,Z_n$, 为总体 $Z$ 的简单随机样本,求 $\sigma^2$ 的极大似然估计量 $\hat{\sigma^2}$.
    \end{enumerate}
\end{example}
\begin{solution}
    \begin{enumerate}[label=(\arabic{*})]
        \item 由微分方程 $f'(x)+\dfrac{x}{\sigma^2}f(x)=0~(\sigma>0)$ 知 $f(x)=C\e ^{-\frac{x^2}{2\sigma^2}}$, 又因为
              $$\int_{-\infty}^{+\infty} f(x) \dd x=1\Rightarrow C=\qty(2\int_{0}^{+\infty} \e ^{-\frac{x^2}{2\sigma^2}} \dd x)^{-1}=\qty[\sqrt{2}\sigma\Gamma\qty(\dfrac{1}{2})]^{-1}=\dfrac{1}{\sqrt{2\pi}\sigma}$$
              于是 $X$ 的概率密度 $f(x)=\dfrac{1}{\sqrt{2\pi}\sigma}\e ^{-\frac{x^2}{2\sigma^2}}$, 即 $X\sim N(0,1)$,那么 $Z$ 的分布函数为
              \begin{flalign*}
                  F_Z(z) & =P\qty{Z\leqslant z}=P\qty{XY\leqslant z}                                                                                                                                                      \\
                         & =P\qty{XY\leqslant z, Y=-1}+P\qty{XY\leqslant z, Y=1}=P\qty{Y=-1}P\qty{X\geqslant -z}+P\qty{Y=1}P\qty{X\leqslant z}                                                                            \\
                         & =\dfrac{1}{2}\qty(1-P\qty{X<-z})+\dfrac{1}{2}P\qty{X\leqslant z}=\dfrac{1}{2}\qty[1-\varPhi\qty(-\dfrac{z}{\sigma})]+\dfrac{1}{2}\varPhi\qty(\dfrac{z}{\sigma})=\varPhi\qty(\dfrac{z}{\sigma})
              \end{flalign*}
              其中 $\varPhi(x)$ 为 $N(0,1)$ 的分布函数, 故
              $$f_Z(z)=F_Z'(z)=\dfrac{1}{\sqrt{2\pi}\sigma}\e ^{-\frac{z^2}{2\sigma^2}}.$$
        \item $EX=0$,又因为 $E\qty(X^2Y)=E\qty(X^2)EY$,且 $EY=-1\times \frac{1}{2}+1\times \frac{1}{2}=0$,所以
              $$
                  \cov(X,Z)=E(XZ)-EX\cdot EZ=E\qty(X^2Y)-EX\cdot E\qty(XY)=EX^2 EY-EX\cdot EX\cdot EY=0
              $$
              故 $X$ 与 $Y$ 不相关.
        \item 似然函数为
              $$
                  L\qty(\sigma^2)=\prod_{i=1}^{n} f_Z(z_i)=(2\pi)^{-\frac{n}{2}}\cdot \qty(\sigma^2)^{-\frac{n}{2}}\cdot\exp\qty(-\dfrac{1}{2\sigma^2}\sum_{i=1}^{n} z_i^2)
              $$
              两边取对数,得
              $$
                  \ln L\qty(\sigma^2)=-\dfrac{n}{2}\ln (2\pi) -\dfrac{n}{2}\ln\qty(\sigma^2) -\dfrac{1}{2\sigma^2}\sum_{i=1}^{n} z_i^2
              $$
              令 $\displaystyle \dv{\sigma^2}\ln\qty(\sigma^2)=0$, 解得 $\sigma^2=\dfrac{1}{n}\displaystyle \sum_{i=1}^{n} z_i^2$, 故 $\sigma^2$ 的极大似然估计量为 $\hat{\sigma^2}=\dfrac{1}{n}\displaystyle \sum_{i=1}^{n} Z_i^2.$
    \end{enumerate}
\end{solution}