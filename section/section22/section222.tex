% \section{区间估计}

% \subsection{矩估计量}

% % \begin{example}
% %     设总体 $X$ 的概率密度为 $$f(x;\theta)=\begin{cases}
% %         \dfrac{\theta ^2}{x^3}\mathrm{e}^{-\frac{\theta }{x} }&,x>0  \\
% %        0&,\text{其他}
% %        \end{cases}$$
% %        其中 $\theta$ 为未知参数且大于零,$X_1,X_2,\cdots,X_n$ 为来自总体 $X$ 的简单随机样本,
% %        求 $\theta$ 的矩估计量.
% % \end{example}

% \begin{example}
%     设总体 $X$ 的概率密度为
%     $$f(x;\theta)=\begin{cases}
%             \dfrac{6x}{\theta^3}(\theta-x) & ,0<x<\theta  \\
%             0                              & ,\text{其他}
%         \end{cases}$$
%     $X_1,X_2,\cdots,X_n$ 为来自总体 $X$ 的简单随机样本,求 $\theta$ 的矩估计量 $\hat{\theta} $ 以及求 $\hat{\theta} $ 的方差 $D\qty(\hat{\theta} ).$
% \end{example}
% \begin{solution}
%     由 $\displaystyle E(X)=\int_{0}^{\theta}xf(x)\dd x$ 得,
%     $$E(X)=\int_{0}^{\theta}\dfrac{6x^2}{\theta^3}(\theta-x)\dd x=\dfrac{\theta}{2}$$
%     记 $\displaystyle\bar{X}=\dfrac{1}{n}\sum_{i=1}^{n}X_i$,令 $\dfrac{\theta}{2}=\bar{X}$,得 $\theta$ 的矩估计量为 $\hat{\theta}=2\bar{X}$;
%     根据方差与期望的关系 $$D(X)=E(X^2)-E^2(X)$$
%     且 $\displaystyle E(X^2)=\int_{0}^{\theta}x^2f(x)\dd x=\int_{0}^{\theta}\dfrac{6x^3}{\theta^3}(\theta-x)\dd x=\dfrac{3\theta^2}{10}$,
%     那么 $D(X)=\dfrac{\theta^2}{20}$,所以 $\hat{\theta}=2\bar{X}$ 的方差为
%     $$D\qty(\hat{\theta})=D\qty(2\bar{X})=4D(\bar{X})=\dfrac{4}{n}D(X)=\dfrac{\theta^2}{5n}.$$
% \end{solution}

% \subsection{置信区间}

% \begin{theorem}[单个正态总体均值和方差的置信区间]
%     设总体 $X\sim N(\mu,\sigma^2)$,常见的置信区间有: 
%     \begin{enumerate}[label=(\arabic{*})]
%         \item 当 $\sigma^2$ 已知时,$\mu$ 的置信度为 $1-\alpha$ 的置信区间为 $\qty(\bar{X}-U_{\frac{\alpha}{2}}\dfrac{\sigma}{\sqrt{n}},\bar{X}+U_{\frac{\alpha}{2}}\dfrac{\sigma}{\sqrt{n}})$.
%         \item 当 $\sigma^2$ 未知时,$\mu$ 的置信度为 $1-\alpha$ 的置信区间为 $\qty(\bar{X}-t_{\frac{\alpha}{2}}(n-1)\dfrac{S}{\sqrt{n}},\bar{X}+t_{\frac{\alpha}{2}}(n-1)\dfrac{S}{\sqrt{n}})$.
%         \item 当 $\mu$ 已知时,$\sigma^2$ 的置信度为 $1-\alpha$ 的置信区间为 $\qty(\dfrac{\sum_{i=1}^{n}(X_i-n)^2}{\chi^2_{\frac{\alpha}{2}}(n)},\dfrac{\sum_{i=1}^{n}(X_i-n)^2}{\chi^2_{1-\frac{\alpha}{2}}(n)}).$
%         \item 当 $\mu$ 未知时,$\sigma^2$ 的置信度为 $1-\alpha$ 的置信区间为 $\qty(\dfrac{(n-1)S^2}{\chi^2_{\frac{\alpha}{2}}(n)},\dfrac{(n-1)S^2}{\chi^2_{1-\frac{\alpha}{2}}(n)})$,$\sigma$ 的置信区间为 $\qty(\sqrt{\dfrac{(n-1)S^2}{\chi^2_{\frac{\alpha}{2}}(n)}},\sqrt{\dfrac{(n-1)S^2}{\chi^2_{1-\frac{\alpha}{2}}(n)}})$.
%     \end{enumerate}
% \end{theorem}

% \begin{theorem}[两个正态总体均值差的置信区间]
%     $\qty(\bar{X}-\bar{Y}-u_{\frac{\alpha}{2}}\sqrt{\dfrac{\sigma_1^2}{n_1}+\dfrac{\sigma_2^2}{n_2}},\bar{X}-\bar{Y}+u_{\frac{\alpha}{2}}\sqrt{\dfrac{\sigma_1^2}{n_1}+\dfrac{\sigma_2^2}{n_2}}).$
% \end{theorem}

\section{极大似然估计}

极大似然估计法是参数估计使用的最广泛的方法,最早由德国数学家 Gauss 在 1821 年提出,
但是此法一般归功于英国统计学家 Fisher,因为 Fisher 于 1922 年再次提出了这个思想,并且证明了这种方法的一些性质,
从而使得极大似然法得到更普遍的应用.

\begin{example}
    设 $X_i,~i=1,2,\cdots,n$ 为来自总体 $X\sim B(N,p)~~(0<p<1)$ 的简单随机样本,求 $p$ 的极大似然估计量.
\end{example}
\begin{solution}
    似然函数为 $\displaystyle L(p)=\prod_{i=1}^{n} \mathrm{C}_{N}^{x_{i}} \cdot p^{x_{i}} \cdot(1-p)^{N-x_{i}} $,两边取对数,得
    $$\ln L(p)=\sum_{i=1}^{n} \ln C_{N}^{x_{i}}+\sum_{i=1}^{n} x_{i} \ln p+\sum_{i=1}^{n}\left(N-X_{i}\right) \ln (1-p)$$
    令 $$\frac{\mathrm{d}}{\mathrm{d} p} \ln L(p)=\frac{1}{p} \sum_{i=1}^{n} x_{i}-\frac{1}{1-p}\left(n N-\sum_{i=1}^{n} x_{i}\right)=0$$
    解得 $\displaystyle p=\frac{\displaystyle \sum_{i=1}^{n} x_{i}}{n N}=\frac{\bar{X}}{N} $,故 $ p $ 的最大似然估计量为 $ \hat{p}=\dfrac{\bar{X}}{N} $,
    其中 $ \displaystyle\bar{X}=\dfrac{1}{n} \sum_{i=1}^{n} X_{i} .$
\end{solution}

\begin{example}
    设总体概率函数如下,$x_1,x_2,\cdots,x_n$ 是样本,试求未知参数的极大似然估计:
    \begin{enumerate}[label=(\arabic{*})]
        \item $\displaystyle p(x;\theta)=\sqrt{\theta}x^{\sqrt{\theta}-1},0<x<1,\theta>0;$
        \item $\displaystyle p(x;\theta)=\theta c^\theta x^{-(\theta+1)},x>c>0,\theta>1.$
    \end{enumerate}
\end{example}
\begin{solution}
    \begin{enumerate}[label=(\arabic{*})]
        \item 似然函数 $\displaystyle L(\theta)=\left( \sqrt{\theta }\right) ^{n}\left( \prod ^{n}_{i=1}x_{i}\right) ^{\sqrt{\theta }-1}$,取对数得
              $$\ln L\left( \theta \right) =\dfrac{n}{2}\ln \theta +\left( \sqrt{\theta }-1\right) \sum ^{n}_{i=1}\ln x_{i}$$
              将 $\ln L(\theta)$ 关于 $\theta$ 求导,并令其值为 $0$ 即得到似然函数
              $$\dfrac{\partial \ln L\left( \theta \right) }{\partial \theta }=\dfrac{n}{2\theta }+\dfrac{1}{2\sqrt{\theta }}\sum ^{n}_{i=1}x_{i}=0$$
              解得 $\displaystyle\widehat{\theta }=\left( \dfrac{1}{n}\sum ^{n}_{i=1}\ln x_{i}\right) ^{-2}$,并且
              $$ \dfrac{\partial ^{2}\ln L\left( \theta \right) }{\partial \theta ^{2}}\bigg | _{\theta=\widehat{\theta }}= \left( -\dfrac{n}{2\theta ^{2}}-\dfrac{1}{4\theta ^{3/2}}\sum ^{n}_{i=1}\ln x_{i}\right) \bigg | _{\theta=\widehat{\theta }}=-\dfrac{3}{4n^{3}}\left( \sum ^{n}_{i=1}\ln x_{i}\right) ^{4}<0$$
              所以 $\widehat{\theta}$ 是 $\theta$ 的极大似然估计.
        \item 似然函数 $\displaystyle L(\theta)=\theta ^{n}c^{n\theta }\left( \prod ^{n}_{i=1}x_{i}\right) ^{-\left( \theta +1\right) }$,取对数得
              $$\ln L\left( \theta \right) =n\ln \theta +n\theta \ln c-\left( \theta +1\right) \sum ^{n}_{i=1}\ln x_{i}$$
              将 $\ln L(\theta)$ 关于 $\theta$ 求导,并令其值为 $0$ 即得到似然函数
              $$\dfrac{\partial \ln L\left( \theta \right) }{\partial \theta }=\dfrac{n}{\theta }+n\ln c-\sum ^{n}_{i=1}\ln x_{i}=0$$
              解得 $\displaystyle \widehat{\theta }=\left[ \dfrac{1}{n}\sum ^{n}_{i=1}\left( \ln x_{i}-\ln c\right) \right] ^{-1}$,
              并且 $\displaystyle \dfrac{\partial ^{2}\ln L\left( \theta \right) }{\partial \theta ^{2}}=\dfrac{-n}{\theta ^{2}} <0$,所以 $\widehat{\theta}$ 是 $\theta$ 的极大似然估计.
    \end{enumerate}
\end{solution}

\begin{example}[2020 数一]
    设某种元件的使用寿命 $T$ 的分布函数为 $$F(t)=\begin{cases}
            1-\mathrm{e}^{-\qty(\frac{t}{\theta})^m} & ,t\geqslant 0 \\
            0                                        & ,\text{其他}
        \end{cases}$$
    其中 $m,~\theta$ 为参数且大于 0,
    \begin{enumerate}[label=(\arabic{*})]
        \item 求概率 $P\qty{T>t}$ 与 $P\qty{T>s+t~|~T>s}$,其中 $s,t>0$;
        \item 任取 $n$ 个这种元件做寿命试验,测得它们的寿命分别为 $t_1,t_2,\cdots,t_n$,若 $m$ 已知,求 $\theta$ 的极大似然估计值 $\hat\theta.$
    \end{enumerate}
\end{example}
\begin{solution}
    \begin{enumerate}[label=(\arabic{*})]
        \item 由 $P\qty{T>t}=1-P\qty{T\leqslant t}$,所以 $P\qty{T>t}=1-F(t)=\mathrm{e}^{-\qty(\frac{t}{\theta})^m}$,
              并且 $$P\qty{T>s+t~|~T>s}=\dfrac{P\qty{T>s+t,T>s}}{P\qty{T>s}}=\dfrac{P\qty{T>s+t}}{P\qty{T>s}}$$
              所以 $P\qty{T>s+t~|~T>s}=\mathrm{e}^{-\qty(\frac{s+t}{\theta})^m+\qty(\frac{s}{\theta})^m},~s,t>0$;
        \item 由 $f(t)=F'(t)$ 可得概率密度函数 $$f(t)=\begin{cases}
                      \dfrac{mt^{m-1}}{\theta^m}\mathrm{e}^{-\qty(\frac{t}{\theta})^m} & ,t\geqslant 0 \\
                      0                                                                & ,t<0
                  \end{cases}$$
              对数似然函数为
              \begin{flalign*}
                  \ln\qty[L(\theta)] & =\ln\qty[\prod_{i=1}^{n}f(t_i;\theta)]=\ln\qty[\prod_{i=1}^{n}\dfrac{mt_i^{m-1}}{\theta^m}\mathrm{e}^{-\qty(\frac{t_i}{\theta})^m}]=\ln\qty[\dfrac{m^n(t_1t_2\cdots t_n)^{m-1}}{\theta^{mn}}\cdot\mathrm{e}^{-\frac{1}{\theta^m}\sum\limits_{i=1}^{n}t_i^m}] \\
                                     & =n\ln m+(m-1)\ln\prod_{i=1}^{n}t_i-mn\ln\theta-\dfrac{1}{\theta^m}\sum_{i=1}^{n}t_i^m
              \end{flalign*}
              求导得 $\displaystyle\dv{\ln\qty[L(\theta)]}{\theta}=-\dfrac{mn}{\theta}+m\dfrac{1}{\theta^{m+1}}\sum_{i=1}^{n}t_i^m=0$,解得 $\displaystyle\dfrac{1}{\theta^m}\sum_{i=1}^{n}t_i^m=n$,
              那么 $\theta$ 的极大似然估计值为 $\hat\theta=\sqrt[m]{\dfrac{1}{n}\displaystyle\sum_{i=1}^{n}t_i^m}.$
    \end{enumerate}
\end{solution}

