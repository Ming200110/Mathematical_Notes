\section{假设检验}

\subsection{假设检验的基本思想}

\begin{example}
    某厂生产一种袋装白糖,每袋白糖的净重是一个随机变量,服从正态分布 $ N\left(\mu, \sigma^{2}\right)$,
    机器正常工作时,均值是 $0.5$ (单位: $ \mathrm{kg}$),标准差是 $ 0.01 \mathrm{~kg}$,某天随机地抽取 $5$ 袋白糖,称得净重为
    $$\begin{array}{lllll}
            0.54 & 0.58 & 0.47 & 0.49 & 0.52
        \end{array}$$
    问当日机器是否正常工作?
\end{example}
\begin{solution}
    由题意知,方差 $ \sigma^{2} $ 已知,$\mu $ 未知,要判断机器是否正常工作,就是要判断该日生产的白糖净重的均值是否为 $ 0.5 \mathrm{~kg} $,
    即检验假设“$\mu=0.5$”是否正确. 因此,提出两个相互对立的假设:\\
    原假设 $ H_{0}: \mu=\mu_{0}=0.5 $,备择假设 $ H_{1}: \mu \neq 0.5 .$\\
    如果假设 $ H_{0}: \mu=\mu_{0}=0.5 $ 为真,那么机器正常工作; 如果假设 $ H_{1}: \mu \neq 0.5 $ 为真,则机器工作不正常.\\
    在假设 $ H_{0}: \mu=\mu_{0}=0.5 $ 条件下,统计量 $\displaystyle Z=\frac{\bar{X}-\mu_{0}}{\sigma / \sqrt{n}} \sim N(0,1) $,
    由标准正态分布的分位点的定义 (见图 \ref{alphafweidian}),知 $ P_{\mu_{0}}\left\{\left|\frac{\bar{X}-\mu_{0}}{\sigma / \sqrt{n}}\right| \geqslant z_{\alpha / 2}\right\}=\alpha $,
    若给定 $ \alpha=0.05$,查表知 $ z_{0.025}=1.96$,代人样本数据: $ n=5, \bar{x}=0.52 $,
    则 $$\displaystyle z=\frac{\bar{x}-\mu_{0}}{\sigma / \sqrt{n}}=\frac{0.52-0.5}{0.01 / \sqrt{5}}=4.47>1.96 .$$

    这说明小概率事件发生了,所以应该拒绝假设 $ H_{0}: \mu=\mu_{0}=0.5 $,接受备择假设 $ H_{1}: \mu \neq 0.5 $,即机器工作不正常.
\end{solution}

\begin{definition}[接收域和拒绝域]
    上述例题中,若 $ z $ 取值在区间 $ (-1.96,1.96) $ 范围内,则接受假设 $ H_{0} $,即 $ |z|<z_{\alpha / 2} $ 称为\textit{接受域},
    而 $ |z| \geqslant z_{\alpha / 2} $ 称为\textit{拒绝域},$z_{\alpha / 2} $ 称为\textit{临界值},$\displaystyle Z=\frac{\bar{X}-\mu_{0}}{\sigma / \sqrt{n}} $ 称为\textit{检验统计量}.
\end{definition}

在根据样本作推断时,由于样本的随机性,难免会做出错误的决定.

\begin{definition}[第一类错误]
    当原假设 $ H_{0} $ 为真时,而做出拒绝 $ H_{0} $ 的判断,称为犯\textit{第一类错误} (\textit{拒真错误})
\end{definition}

\begin{definition}[第二类错误]
    当原假设 $ H_{0} $ 不真时,而作出接受 $ H_{0} $ 的判断,称为犯\textit{第二类错误} (\textit{取伪错误}).
\end{definition}

\begin{definition}[显著性水平]
    在实际应用中,控制犯第一类错误的概率,使其不大于一个较小的正数 $ \alpha(0<\alpha<1)$,称 $ \alpha $ 为检验的\textit{显著性水平}.
\end{definition}

\begin{definition}[双边备择假设与双边假设检验]
    形如 $ H_{1}: \mu \neq \mu_{0} $ 的假设,表示 $ \mu $ 可能大于 $ \mu_{0} $,也可能小于 $ \mu_{0}$,称为\textit{双边备择假设};
    形如 $ H_{0}: \mu=\mu_{0} $ 的假设,称为\textit{双边假设检验}.
\end{definition}

在实际中,有时只关心均值是否减小. 例如某机器的生产效率问题,此时我们应该关注的是生产时间,时间越短越好.
对于采用新工艺来提高生产效率,生产时间是否显著缩短的问题,需要考虑如下假设检验:
原假设 $ H_{0}: \mu \geqslant \mu_{0}$,备择假设 $ H_{1}: \mu<\mu_{0} .$

\begin{definition}[单边检验]
    形如 $ H_{0}: \mu \geqslant \mu_{0} $ 的假设称为\textit{左边检验},类似的形如 $ H_{0}: \mu \leqslant \mu_{0} $ 的假设称为\textit{右边检验}.
    左边检验和右边检验统称为\textit{单边检验}.
\end{definition}

\subsection{假设检验的解题步骤}

\begin{enumerate}[label=(\arabic{*})]
    \item 确定原假设 $H_0$ 与备择假设 $H_1$;
    \item 选择合适的检验统计量,在原假设成立的条件下求其发布;
    \item 根据显著性水平 $\alpha$,再原假设成立的条件下确定临界值和拒绝域;
    \item 判断是否落入拒绝域,落入拒绝域则拒绝 $H_0$,否则拒绝 $H_1$.
\end{enumerate}