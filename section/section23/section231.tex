\section{命题逻辑}

\subsection{命题及其表示法}

在数理逻辑中,我们将使用大写字母 $A,B,\cdots,P,Q,\cdots$ 或用带下标的大写字母或用数字,如 $A_i,[12]$ 等表示命题.

\begin{definition}[命题标识符]
    表示命题的符号称为命题标识符.
\end{definition}

\begin{definition}[真值]
    一个命题,总是具有一个“值”,称为真值,真值只有“真”和“假”两种,记作 True (真) 和 False (假),分别用符号 $\vb*{T}$ 和 $\vb*{F}$ 表示.
\end{definition}

\subsection{联结词}

\begin{definition}[原子命题和复合命题]
    不包含任何联结词的命题叫做原子命题,至少包含一个联结词的命题称作复合命题.
\end{definition}

\begin{definition}[否定]
    设 $P$ 为一命题,$P$ 的否定是一个新的命题,记作 $\neg P$. 若 $P$ 为 $\vb*{T}$,$\neg P$ 为 $\vb*{F}$; 若 $P$ 为 $\vb*{F}$,$\neg P$ 为 $\vb*{T}$.
\end{definition}

\begin{definition}[合取]
    两个命题 $P$ 和 $Q$ 的合取是一个复合命题,记作 $P \wedge Q$. 当且仅当 $P,Q$ 同时为 $\vb*{T}$ 时,$P\wedge Q$ 为 $\vb*{T}$,在其他情况下,$P\wedge Q$ 的真值都是 $\vb*{F}$.
\end{definition}

\begin{definition}[析取]
    两个命题 $P$ 和 $Q$ 的析取是一个复合命题,记作 $P\vee Q$. 当且仅当 $P,Q$ 同时为 $\vb*{F}$ 时,$P\vee Q$ 为 $\vb*{F}$,在其他情况下,$P\vee Q$ 的真值都是 $\vb*{T}$.
\end{definition}

\begin{definition}[条件]
    给定两个命题 $P$ 和 $Q$,其条件命题是一个复合命题,记作 $P\to Q$,读作“如果 $P$,那么 $Q$”或“若 $P$ 则 $Q$”. 当且仅当 $P$ 的真值为 $\vb*{T}$,$Q$ 的真值为 $\vb*{F}$ 时,$P\to Q$ 的真值为 $\vb*{F}$,否则 $P\to Q$ 的真值为 $\vb*{T}$,我们称 $P$ 为前件,$Q$ 为后件.
\end{definition}

\begin{definition}[双条件]
    给定两个命题 $P$ 和 $Q$,其复合命题 $P \rightleftarrows Q$ 称作双条件命题,读作“$P$ 当且仅当 $Q$”,当 $P$ 和 $Q$ 的真值相同时,$P \rightleftarrows Q$ 的真值为 $\vb*{T}$,否则 $P \rightleftarrows Q$ 的真值为 $\vb*{F}$.
\end{definition}

\begin{table}[H]
    \centering
    \caption{}
    \begin{tabular}{c | c | c | c | c | c | c}
        $P$       & $Q$       & $\neg P$  & $P \wedge Q$ & $P\vee Q$ & $P\to Q$  & $P \rightleftarrows Q$ \\
        \midrule
        $\vb*{T}$ & $\vb*{T}$ & $\vb*{F}$ & $\vb*{T}$    & $\vb*{T}$ & $\vb*{T}$ & $\vb*{T}$              \\
        \midrule
        $\vb*{T}$ & $\vb*{F}$ & $\vb*{F}$ & $\vb*{F}$    & $\vb*{T}$ & $\vb*{F}$ & $\vb*{F}$              \\
        \midrule
        $\vb*{F}$ & $\vb*{T}$ & $\vb*{T}$ & $\vb*{F}$    & $\vb*{T}$ & $\vb*{T}$ & $\vb*{F}$              \\
        \midrule
        $\vb*{F}$ & $\vb*{F}$ & $\vb*{T}$ & $\vb*{F}$    & $\vb*{F}$ & $\vb*{T}$ & $\vb*{T}$              \\
    \end{tabular}
\end{table}

\begin{example}
    以符号形式写出下列命题:
    \begin{enumerate*}[label=(\arabic{*})]
        \item 如果天不下雪和我有时间,那么我将去镇上.
        \item 我将去镇上,仅当我有时间时.
        \item 天不下雪.
        \item 天下雪,那么我不去镇上.
    \end{enumerate*}
\end{example}
\begin{solution}
    令 $P:$ 天下雪,$Q:$ 我有时间,$R:$ 我将去镇上,那么
    \begin{enumerate*}[label=(\arabic{*})]
        \item $(\neg P\wedge Q)\to R$.
        \item $R\to Q$.
        \item $\neg P$.
        \item $P\to \neg R$.
    \end{enumerate*}
\end{solution}

\subsection{命题公式与翻译}

\begin{definition}[合式公式 (wff)]
    命题演算的合式公式 (wff),规定为:
    \begin{enumerate}[label=(\arabic{*})]
        \item 单个命题变元本身是一个合式公式.
        \item 如果 $A$ 是合式公式,那么 $\neg A$ 是合式公式.
        \item 如果 $A$ 和 $B$ 是合式公式,那么 $(A\wedge B),(A\vee B),(A\to B),(A \rightleftarrows B)$ 都是合式公式.
        \item 当且仅当能够有限次地使用 (1)、(2) 和 (3) 所得到的包含命题变元,联结词和括号的符号串是合式公式.
    \end{enumerate}
\end{definition}

\begin{definition}[联结词运算优先级]
    我们规定联结词运算的优先次序为 $\neg,\wedge,\vee,\to,\rightleftarrows$ (非、与、或、若则、仅当).
\end{definition}

\begin{example}
    判断下列公式哪些是合式公式,哪些不是合式公式.
    \begin{enumerate*}[label=(\arabic{*})]
        \item $(Q\to (R\wedge S))$.
        \item $(P\rightleftarrows (R\to S))$.
        \item $((\neg P\to Q)\to (Q\to P))$.
        \item $(RS\to P)$.
        \item $((P\to(Q\to R))\to((P\to Q)\to(P\to R)))$.
    \end{enumerate*}
\end{example}
\begin{solution}
    \begin{enumerate*}[label=(\arabic{*})]
        \item 不是合式公式,因为没有规定运算符次序.
        \item 是合式公式.
        \item 是合式公式.
        \item 不是合式公式,因为 $R$ 和 $S$ 之间缺少联结词.
        \item 是合式公式.
    \end{enumerate*}
\end{solution}

\begin{example}
    对下列各式用指定的公式进行代换.
    \begin{enumerate}[label=(\arabic{*})]
        \item $((A\to B)\vee (B\to A))$,用 $B$ 代换 $A$.
        \item $(((A\to B)\to B)\to A)$,用 $(A\to C)$ 代换 $A$,用 $((B\wedge C)\to A)$ 代换 $B$.
    \end{enumerate}
\end{example}
\begin{solution}
    \begin{enumerate*}[label=(\arabic{*})]
        \item $((B\to B)\vee (B\to B))$.
        \item $((((A\to C)\to ((B\wedge C)\to A))\to ((B\wedge C)\to A))\to (A\to C))$.
    \end{enumerate*}
\end{solution}

\begin{example}
    下列几个式子中有哪几个是别的式子经过代换得到的.
    \begin{enumerate}[label=(\arabic{*})]
        \item $(P\to(Q\to P))$.
        \item $((((P\to Q)\wedge (R\to S))\wedge (P\vee R))\to (Q\vee S))$.
        \item $(Q\to ((P\to P)\to Q))$.
        \item $(P\to ((P\to (Q\to P))\to P))$.
        \item $((((R\to S)\wedge (Q\to P))\wedge (R\vee Q))\to (S\vee P))$.
    \end{enumerate}
\end{example}
\begin{solution}
    (1) 是由 (3) 式进行代换得到,在 (3) 中用 $Q$ 代换 $P$,$(P\to P)$ 代换 $Q$.
    (4) 是由 (1) 式进行代换得到,在 (1) 中用 $P\to(Q\to P)$ 代换 $Q$.
    (5) 是由 (2) 式进行代换得到,在 (2) 中用 $R$ 代换 $P$,用 $S$ 代换 $Q$,用 $Q$ 代换 $R$,用 $P$ 代换 $S$.
\end{solution}

\subsection{真值表与等价公式}

\begin{definition}[真值表]
    在命题公式中,对于分量指派真值的各种可能组合,就确定了这个命题公式的各种真值情况,把它汇列成表,就是命题公式的真值表.
\end{definition}

\begin{example}
    给出 $(P\wedge Q)\vee (\neg P\wedge \neg Q)$ 的真值表.
\end{example}
\begin{table}[H]
    \centering
    \caption{}
    \begin{tabular}{c | c | c | c | c | c | c}
        $P$       & $Q$       & $\neg P$  & $\neg Q$  & $P\wedge Q$ & $\neg P\wedge \neg Q$ & $(P\wedge Q)\vee (\neg P\wedge \neg Q)$ \\
        \midrule
        $\vb*{T}$ & $\vb*{T}$ & $\vb*{F}$ & $\vb*{F}$ & $\vb*{T}$   & $\vb*{F}$             & $\vb*{T}$                               \\
        \midrule
        $\vb*{T}$ & $\vb*{F}$ & $\vb*{F}$ & $\vb*{T}$ & $\vb*{F}$   & $\vb*{F}$             & $\vb*{F}$                               \\
        \midrule
        $\vb*{F}$ & $\vb*{T}$ & $\vb*{T}$ & $\vb*{F}$ & $\vb*{F}$   & $\vb*{F}$             & $\vb*{F}$                               \\
        \midrule
        $\vb*{F}$ & $\vb*{F}$ & $\vb*{T}$ & $\vb*{T}$ & $\vb*{F}$   & $\vb*{T}$             & $\vb*{T}$                               \\
    \end{tabular}
\end{table}

\begin{example}
    给出 $\neg (P\wedge Q)\rightleftarrows (\neg P \vee \neg Q)$ 的真值表.
\end{example}
\begin{table}[H]
    \centering
    \caption{}
    \label{eishinCeremony}
    \begin{tabular}{c | c | c | c | c | c | c | c}
        $P$       & $Q$       & $\neg P$  & $\neg Q$  & $P\wedge Q$ & $\neg (P\wedge Q)$ & $\neg P\vee \neg Q$ & $\neg (P\wedge Q)\rightleftarrows (\neg P \vee \neg Q)$ \\
        \midrule
        $\vb*{T}$ & $\vb*{T}$ & $\vb*{F}$ & $\vb*{F}$ & $\vb*{T}$   & $\vb*{F}$          & $\vb*{F}$           & $\vb*{T}$                                               \\
        \midrule
        $\vb*{T}$ & $\vb*{F}$ & $\vb*{F}$ & $\vb*{T}$ & $\vb*{F}$   & $\vb*{T}$          & $\vb*{T}$           & $\vb*{T}$                                               \\
        \midrule
        $\vb*{F}$ & $\vb*{T}$ & $\vb*{T}$ & $\vb*{F}$ & $\vb*{F}$   & $\vb*{T}$          & $\vb*{T}$           & $\vb*{T}$                                               \\
        \midrule
        $\vb*{F}$ & $\vb*{F}$ & $\vb*{T}$ & $\vb*{T}$ & $\vb*{F}$   & $\vb*{T}$          & $\vb*{T}$           & $\vb*{T}$                                               \\
    \end{tabular}
\end{table}

由表 \ref{eishinCeremony} 可以看出,有一类公式无论命题变元作何种指派,其真值永为真 (假),我们把这类公式记为 $\vb*{T}(\vb*{F})$.

\begin{definition}[命题等价]
    给定两个命题公式 $A$ 和 $B$,设 $P_1,P_2,\cdots,P_n$ 为所有出现于 $A$ 和 $B$ 中的原子变元,若给 $P_1,P_2,\cdots,P_n$ 任一组真值指派,$A$ 和 $B$ 的真值都相同,则称 $A$ 和 $B$ 是等价的或逻辑相等,记作 $A\Leftrightarrow B$.
\end{definition}

\begin{example}
    试用真值表说明 $\neg P\vee Q\Leftrightarrow P\to Q$ 和 $P\leftrightarrows Q\Leftrightarrow (P\wedge Q)\vee (\neg P\wedge \neg Q)$ (可作为结论使用).
    \label{negpveeq}
\end{example}
\begin{table}[H]
    \centering
    \caption{}
    \begin{tabular}{c | c | c | c | c | c | c | c | c | c}
        $P$       & $Q$       & $\neg P$  & $\neg Q$  & $\neg P\vee Q$ & $P\to Q$  & $P\leftrightarrows Q$ & $P\wedge Q$ & $\neg P\wedge \neg Q$ & $(P\wedge Q)\vee (\neg P\wedge \neg Q)$ \\
        \midrule
        $\vb*{T}$ & $\vb*{T}$ & $\vb*{F}$ & $\vb*{F}$ & $\vb*{T}$      & $\vb*{T}$ & $\vb*{T}$             & $\vb*{T}$   & $\vb*{F}$             & $\vb*{T}$                               \\
        \midrule
        $\vb*{T}$ & $\vb*{F}$ & $\vb*{F}$ & $\vb*{T}$ & $\vb*{F}$      & $\vb*{F}$ & $\vb*{F}$             & $\vb*{F}$   & $\vb*{F}$             & $\vb*{F}$                               \\
        \midrule
        $\vb*{F}$ & $\vb*{T}$ & $\vb*{T}$ & $\vb*{F}$ & $\vb*{T}$      & $\vb*{T}$ & $\vb*{F}$             & $\vb*{F}$   & $\vb*{F}$             & $\vb*{F}$                               \\
        \midrule
        $\vb*{F}$ & $\vb*{F}$ & $\vb*{T}$ & $\vb*{T}$ & $\vb*{T}$      & $\vb*{T}$ & $\vb*{T}$             & $\vb*{F}$   & $\vb*{T}$             & $\vb*{T}$                               \\
    \end{tabular}
\end{table}

常用的命题等价定理.

\setcounter{magicrownumbers}{0}
\begin{table}[H]
    \centering
    \begin{tabular}{c l | c l}
        对合律                     & (\rownumber{}) $\neg\neg P\Leftrightarrow P$                                               & 幂等律                  & (\rownumber{}) $P\vee P\Leftrightarrow P,P\wedge P\Leftrightarrow P$                         \\
        \midrule
        \multirow{2}{*}{结合律}    & (\rownumber{}) $(P\vee Q)\vee R\Leftrightarrow P\vee (Q\vee R)$                            & \multirow{2}{*}{交换律} & (\rownumber{}) $P\vee Q\Leftrightarrow Q\vee P$                                              \\
                                   & (\rownumber{}) $(P\wedge Q)\wedge R\Leftrightarrow P\wedge (Q\wedge R)$                    &                         & (\rownumber{}) $P\wedge Q\Leftrightarrow Q\wedge P$                                          \\
        \midrule
        \multirow{2}{*}{分配律}    & (\rownumber{}) $P\vee (Q\wedge R)\Leftrightarrow (P\vee Q)\wedge(P\vee R)$                 & \multirow{2}{*}{吸收律} & (\rownumber{}) $P\vee(P\wedge Q)\leftrightarrow P$                                           \\
                                   & (\rownumber{}) $P\wedge (Q\vee R)\Leftrightarrow (P\wedge Q)\vee(P\wedge R)$               &                         & (\rownumber{}) $P\wedge(P\vee Q)\leftrightarrow P$                                           \\
        \midrule
        \multirow{2}{*}{德·摩根律} & (\rownumber{}) $\neg(P\vee Q)\Leftrightarrow \neg P\wedge \neg Q$                          & 同一律                  & (\rownumber{}) $P\vee \vb*{F}\Leftrightarrow P,P\wedge \vb*{T}\Leftrightarrow P$             \\
                                   & (\rownumber{}) $\neg(P\wedge Q)\Leftrightarrow \neg P\vee \neg Q$                          & 零律                    & (\rownumber{}) $P\vee \vb*{T}\Leftrightarrow \vb*{T},P\wedge \vb*{F}\Leftrightarrow \vb*{F}$ \\
        \midrule
        否定律                     & (\rownumber{}) $P\vee \neg P\Leftrightarrow \vb*{T},P\wedge \neg P\Leftrightarrow \vb*{F}$
    \end{tabular}
\end{table}

\begin{example}
    证明: $(P\wedge Q)\vee (P\wedge \neg Q)\Leftrightarrow P.$
\end{example}
\begin{proof}[{\songti \textbf{证}}]
    $(P\wedge Q)\vee (P\wedge \neg Q)\Leftrightarrow P\wedge(Q\vee\neg Q)\Leftrightarrow P\wedge \vb*{T}\Leftrightarrow P.$
\end{proof}

\begin{example}
    证明: $((P\vee Q)\wedge\neg(\neg P\wedge (\neg Q\vee \neg R)))\vee (\neg P\wedge \neg Q)\vee (\neg P\wedge \neg R)\Leftrightarrow \vb*{T}.$
\end{example}
\begin{proof}[{\songti \textbf{证}}]
    原式左边 $\Leftrightarrow ((P\vee Q)\wedge\neg(\neg P \wedge \neg(Q\wedge R)))\vee \neg(P\vee Q)\vee\neg(P\vee R)
        \Leftrightarrow ((P\vee Q)\wedge(P \vee(Q\wedge R)))\vee \neg(P\vee Q)\vee \neg(P\vee R)
        \Leftrightarrow ((P\vee Q)\wedge ((P\vee Q)\wedge (P\vee R)))\vee \neg((P\vee Q)\wedge (P\vee R))
        \Leftrightarrow ((P\vee Q)\wedge (P\vee R))\vee \neg ((P\vee Q)\wedge (P\vee R))\Leftrightarrow \vb*{T}.$
\end{proof}

\begin{example}
    证明下列等价式:
    \begin{enumerate}[label=(\arabic{*})]
        \item $A\to(B\to A)\Leftrightarrow \neg A\to(A\to\neg B)$.
        \item $\neg (A\leftrightarrows B)\Leftrightarrow (A\vee B)\wedge\neg(A\wedge B)$.
        \item $\neg(A\to B)\Leftrightarrow A\wedge \neg B$.
        \item $A\to(B\vee C)\Leftrightarrow (A\wedge \neg B)\to C$.
    \end{enumerate}
\end{example}
\begin{proof}[{\songti \textbf{证}}]
    由例题 \ref{negpveeq} 的结论可得:
    \begin{enumerate}[label=(\arabic{*})]
        \item $A\to(B\to A)\Leftrightarrow \neg A\vee (\neg B\vee A)\Leftrightarrow A\vee(\neg A\vee \neg B)\Leftrightarrow \neg A\to(A\to\neg B)$.
        \item $\neg (A\leftrightarrows B)\Leftrightarrow \neg((A\wedge B)\vee(\neg A\wedge \neg B))\Leftrightarrow \neg((A\wedge B)\vee\neg(A\vee B))\Leftrightarrow (A\vee B)\wedge\neg(A\wedge B)$.
        \item $\neg(A\to B)\Leftrightarrow \neg(\neg A\vee B)\Leftrightarrow A\wedge \neg B$.
        \item $A\to(B\vee C)\Leftrightarrow \Leftrightarrow \neg A\vee(B\vee C)\Leftrightarrow \neg(A\wedge \neg B)\vee C\Leftrightarrow (A\wedge \neg B)\to C$.
    \end{enumerate}
\end{proof}

\subsection{重言式与蕴含式}

\begin{definition}[重言式]
    给定一命题公式,若无论对分量作怎样的指派,其对应的真值永为 $\vb*{T}$,则称该命题公式为重言式或永真公式.
\end{definition}

\begin{definition}[矛盾式]
    给定一命题公式,若无论对分量作怎样的指派,其对应的真值永为 $\vb*{F}$,则称该命题公式为矛盾式或永假公式.
\end{definition}

\subsection{其他联结词}

\subsection{对偶与范式}

\subsection{推理理论}
